\documentclass[a4paper,11pt]{amsart} 
\usepackage[francais]{babel}
\usepackage{epsfig}
\usepackage{mathtime}
%\usepackage{epsfig}
\usepackage[oztex,hideboxes]{boxedeps}

\newcommand{\dessin}[4]{
\begin{figure}[htb]
\centering
\HideDisplacementBoxes 
\BoxedEPSF{#1 scaled #2}
%\epsfig{file=#1, height= #2mm}
\caption{#3}
\label{#4}
\end{figure}}

\newtheorem{remarque}{Remarque} \newcommand{\AAA}{{\mathcal A}} 
\newcommand{\BB}{{\mathcal B}} \newcommand{\CC}{{\mathcal C}} 
\newcommand{\DD}{{\mathcal D}} \newcommand{\EE}{{\mathcal E}} 
\newcommand{\FF}{{\mathcal F}} \newcommand{\GG}{{\mathcal G}} 
\newcommand{\HH}{{\mathcal H}} \newcommand{\II}{{\mathcal I}} 
\newcommand{\JJ}{{\mathcal J}} \newcommand{\KK}{{\mathcal K}} 
\newcommand{\LL}{{\mathcal L}} \newcommand{\MM}{{\mathcal M}} 
\newcommand{\NN}{{\mathcal N}} \newcommand{\OO}{{\mathcal O}} 
\newcommand{\PP}{{\mathcal P}} \newcommand{\QQ}{{\mathcal Q}} 
\newcommand{\RR}{{\mathcal R}} \newcommand{\SSS}{{\mathcal S}} 
\newcommand{\TT}{{\mathcal T}} \newcommand{\UU}{{\mathcal U}} 
\newcommand{\VV}{{\mathcal V}} \newcommand{\WW}{{\mathcal W}} 
\newcommand{\XX}{{\mathcal X}} \newcommand{\ZZ}{{\mathcal Z}} 
\newcommand{\bbR}{{\mathbb R}} \newcommand{\bbD}{{\mathbb D}} 
\newcommand{\bbO}{{\mathbb O}} \newcommand{\bbS}{{\mathbb S}} 
\newcommand{\bbE}{{\mathbb E}} \newcommand{\bbN}{{\mathbb N}} 
\newcommand{\bbM}{{\mathbb M}} \newcommand{\bbV}{{\mathbb V}} 
\newcommand{\bbC}{{\mathbb K}} \newcommand{\bbF}{{\mathbb F}} 
\newcommand{\bbP}{{\mathbb P}}
%%%%%%%%%%%%%%%%%%%%%%%%%%%%%%%%%
% Rapport fait le : 11\01\2001

\begin{document}
\title{\'ELIMINATION DES PARTIES CACH\'EES} 
\author{Q. QUADRAT}

\maketitle \tableofcontents
%%%%%%%%%%%%%%%%%%%%%%%
\vfill \vspace{0.5cm}\hrule
\section{Introduction}
\hrule\vspace{0.5cm}
%%%%%%%%%%%%%%%%%%%%%%

L'un des probl\`emes importants lors d'une projection d'une sc\`ene graphique, 
qu'elle soit en 2D ou 3D, est d'\'eliminer les parties cach\'ees 
de la sc\`ene.  Cet expos\'e va pr\'esenter deux 
algorithmes pour r\'esoudre ce probl\`eme.  
D'une part, l'algorithme de z-buffer 
qui affiche les pixels les plus proches de l'observateur dans une 
projection parall\`ele,
d'autre part, un algorithme de partitionnement binaire de 
l'espace, qui permet d'afficher les objets les plus \'eloign\'es 
avant d'afficher les plus proches.

\newpage
%%%%%%%%%%%%%%%%%%
\vfill \vspace{0.5cm}\hrule
\section{Sc\`ene 3D}
\hrule\vspace{0.5cm}
%%%%%%%%%%%%%%%%%%
Une sc\`ene 3D not\'ee $\SSS$, est d\'efinie par un ensemble d'objets 
$\{\OO_1,\OO_2,\cdots,\OO_n \}$ dans l'espace \`a trois dimensions.  
La surface ext\'erieure d'un objet $\OO$, appel\'ee bord, est not\'ee 
$\partial\OO$.  On se limite \`a des sc\`enes dont les objets sont 
poly\`edraux.  Le bord d'un poly\`edre est donc une union de faces qui 
sont des polygones : $$\partial\OO=\{\partial\OO^1, 
\partial\OO^2,\cdots, \partial\OO^n\}.$$ Un polygone peut \^etre 
triangul\'e c'est \`a dire que chaque polygone est une union de 
triangles : $$\partial\OO_0^k=\{t_0^{k,1},\cdots,t_0^{k,l}\}.$$ En fin 
de compte pour visualiser une sc\`ene nous avons besoin de 
repr\'esenter uniquement les bords des objets.  L'ensemble des bords des 
objets d'une sc\`ene sera appele bord de la sc\`ene $\partial\SSS$ qui 
sera qu'un ensemble de triangles dans l'espace 3D~: 
$$\partial\SSS=\{t_0^{k,l}\}.$$ Un triangle, est une surface 
color\'ee, 
d\'efini par ses trois sommets~: $$t=(p_1,p_2,p_3),$$ o\`u chaque 
sommet est d\'efini par le quadruplet~: $p=(x,y,z,c),$ o\`u $x,y,z$ 
sont les coordonn\'ees du sommet et $c$ sa couleur donc  un triangle est 
d\'efini par une matrice de quatres lignes et trois colonnes.  On peut 
ainsi repr\'esenter une sc\`ene par un ensemble de telles matrices.

\dessin{scene.epsf}{1000}{Exemple de sc\`ene 3D}{Web}

De m\^eme on aura besoin de sc\`enes 2D polygonales.  Dans ce cas, 
chaque objet est un polygone.  Le bord d'un objet est une union de 
faces o\`u chaque face est un segment.  Un segment est d\'efini par 
ses deux extr\'emit\'es donc une matrice 2 colonnes 3 lignes.

%%%%%%%%%%%%%%%%%%%%%%%%%%%
\vfill \newpage \vspace{0.5cm}\hrule
\section{Rappel de g\'eom\'etrie et introduction au calcul matriciel}
\hrule\vspace{0.5cm}
%%%%%%%%%%%%%%%%%%%%%%%%%%%

%%%%%%%%%%%%%%%%%%%%%%%%%%%%%%%%
\subsection{Calcul matriciel}
%%%%%%%%%%%%%%%%%%%%%%%%%%%%%%%%%
Les matrices sont des tableaux de nombres r\'eels.  On dira que la 
matrice est $m\times n$ si elle a $m$ lignes et $n$ colonnes.  Sur des 
tableaux de tailles compatibles, on d\'efinit une addition et une 
multiplication~\cite{FOLE95}.
\begin{itemize}
\item \emph{Addition matricielle} : Soit $A$ et $B$ deux matrices 
$m\times n$, la somme des matrices $A$ et $B$ est la matrice $C$ 
$m\times n$ d\'efinie par~: $$C_{ij}=A_{ij}+B_{ij}, \;\;\; i=1\cdots 
m,\; j=1\cdots n.$$
%
\item \emph{Multiplication matricielle} : Soit $A$ une matrice 
$m\times n$ et $B$ une matrice $n\times p$, le produit des matrices 
$A$ et $B$ est la matrice $C$ $m\times n$ d\'efinie par~: 
$$C_{ik}=\sum_{j=1}^n A_{ij}B_{jk}, \;\;\; i=1\cdots m,\; k=1\cdots 
p.$$ \item \emph{Multiplication par un scalaire} : Soit $A$ une 
matrice $m\times n$ et $\lambda$ un nombre r\'eels, le produit de $C=A 
\lambda = \lambda A$ est d\'efini par~: $$C_{ij}=\lambda A_{ij},\;\;\; 
i=1\cdots m,\; j=1\cdots n.$$
\end{itemize}
%
\begin{remarque} Le calcul matriciel a de bonnes propri\'et\'es.  En 
parti\-culier les matrices $m\times m$ munies de ces deux op\'erations 
a une structure d'anneau.
\end{remarque}
%----------------------------
\subsection{G\'eom\'etrie 2D}
%-----------------------------
On consid\`ere un plan muni de son syst\`eme de coordonn\'ees 
$(O,x,y)$.  Un point $P$ est d\'efini par ses coordonn\'es $(x,y)$ que 
l'on notera matriciellement~: $$P=\begin{bmatrix}x\\y\end{bmatrix}.$$

On d\'efinit sur ces points les transformations g\'eometriques 
suivantes qui \`a $P\;(x,y)$ associe $P'\;(x',y')$ ~:
\begin{itemize}
\item\emph{Translation} : la translation de vecteur $T$ est la 
transformation d\'efinie par $P'=P+T$.\vspace{0.2cm}
%
\item\emph{Rotation} : la rotation d'angle $\theta$ est la 
transformation d\'efinie par~: $$x'=x\cos \theta -y \sin 
\theta,\hspace{0.5cm}y'=x \sin \theta +y \cos \theta.$$ Sous forme 
matricielle nous avons~: $P'=R P$ avec $$R=\begin{bmatrix}\cos\theta & 
-\sin\theta \\ \sin\theta & \cos\theta\end{bmatrix}.$$\vspace{0.1cm}
%
\item\emph{Homoth\'etie} : l'homoth\'etie de rapport $h$ est la 
transformation d\'efinie par~: $$x'=hx,\hspace{0.5cm}y'=hy.$$ Sous 
forme matricielle nous avons~: $P'=hP$.\vspace{0.2cm}
%
\item\emph{Perspective} : la perspective $V$ de centre $O$ et de plan 
de projection $\PP$ (d'\'equation $y=1$) est d\'efinie par 
$x'=x/y,\;\;y'=1).$ $P'$ est l'intersection de $OP$ avec $\PP$.  Elle 
est d\'efinie lorsque $P\not= 0.$ Elle ne s'exprimera en terme de 
calcul matriciel que dans le cadre de la g\'eom\'etrie 
projective~\cite{FOLE95}.
\end{itemize}
%---------------------------------------
\subsection{G\'eom\'etrie projective 2D}
%---------------------------------------
Pour unifier les probl\`emes apparaissant en g\'eo\-m\'etries, on 
introduit la g\'eom\'etrie projective.  On se place dans un espace \`a 
trois dimensions.  Au lieu de raisonner avec les points d'un espace 
2D, on raisonne avec les \emph{rayons} (droites passant par l'origine) 
d'un espace 3D. On peut voir alors la g\'eom\'etrie 2D comme la 
g\'eom\'etrie des intersections des rayons avec un plan.  Un rayon est 
la classes de points proportionnels entre eux.  Deux points $P$ de 
coordonn\'ees $(X,Y,Z)$ et $P'$ de coordonn\'ees $(X',Y',Z')$ sont 
\'equivalents $(P\sim P')$ si seulement si $P'=\lambda P$ o\`u 
$\lambda$ est un nombre r\'eel.  Lorsque $Z\not= 0$ un repr\'esentant 
d'un rayon est $(x=X/Z,y=Y/Z,1)$.  C'est l'intersection du rayon avec 
le plan $Z=1$.  Les points $Q=(x,y,1)$ du plan $Z=1$ sont les points 
de la g\'eom\'etrie 2D du paragraphe pr\'ec\'edant.  En g\'eom\'etrie 
projective les transformations du paragraphe pr\'ec\'edant 
s'\'ecrivent comme des produits matriciels $3\times 3$. 
(Figure~\ref{Pp})~\cite{FOLE95}.  
%\vspace{0.1cm}

%
\dessin{espace3D.epsf}{800}{G\'eom\'etrie projective 2D}{Pp}
%

Aux quatre op\'erations vues pr\'ec\'edemment sont respectivement 
associ\'ees une matrice~: la translation $T$, la rotation $R$, 
l'homoth\'etie $H$ et la perspective $V$.

$$T=\begin{bmatrix}1&0&T_x\\0&1&T_y\\0&0&1\end{bmatrix},\hspace{0.5cm} 
R=\begin{bmatrix}\cos\theta&-\sin\theta&0\\
\sin\theta&\cos\theta&0\\0&0&1\end{bmatrix},\hspace{0.5cm}$$ 
$$H=\begin{bmatrix}h&0&0\\0&h&0\\0&0&h\end{bmatrix},\hspace{0.5cm} 
V=\begin{bmatrix}1&0&0\\0&1&0\\0&1&0\end{bmatrix}.$$

V\'erifions que la derni\`ere op\'eration r\'ealise bien la 
perspective~: 
$$\begin{bmatrix}1&0&0\\0&1&0\\0&1&0\end{bmatrix}\begin{bmatrix}x
\\y\\1\end{bmatrix}= 
\begin{bmatrix}x\\y\\y\end{bmatrix}\sim 
\begin{bmatrix}x/y\\1\\1\end{bmatrix}.$$
%---------------------------------------
\subsection{G\'eom\'etrie projective 3D}
%---------------------------------------
De nouveau, on d\'efinit une g\'eom\'etrie 3D projective en 
consid\'erant les rayons dans un espace 4D. Les coordonn\'ees 
homog\`enes seront les quadruplets $(X,Y,Z,T)$ et on peut associer 
\`aux op\'erations une matrice.  Sont associ\'es par une matrice~: la 
translation de vecteur $(T_x,T_y,T_z)$, la rotation d'angle $\theta$ 
et d'axe $z$, l'homoth\'etie de rapport $h$.

$$T=\begin{bmatrix}1&0&0&T_x\\0&1&0&T_y\\0&0&1&T_z\\0&0&0&1\end{bmatrix},
\; 
R=\begin{bmatrix}\cos\theta&-\sin\theta&0&0\\
\sin\theta&\cos\theta&0&0\\0&0&1&0\\0&0&0&1\end{bmatrix},\; 
H=\begin{bmatrix}h&0&0&0\\0&h&0&0\\0&0&h&0\\0&0&0&1\end{bmatrix}.$$

Nous nous interessons \`a deux sortes de perspective dans un espace 
3D~:
\begin{itemize}
\item\emph{projection perspective} de centre $O$ sur $Z=1$ (figure 
\ref{Pp})~: 
$$V=\begin{bmatrix}1&0&0&0\\0&1&0&0\\0&0&1&0\\0&0&1&0\end{bmatrix}.$$ 
\item\emph{projection parall\`ele} \`a $(Oz)$ sur le plan $Z=1$ (figure 
\ref{pa})~: 
$$W=\begin{bmatrix}1&0&0&0\\0&1&0&0\\0&0&0&1\\0&0&0&1\end{bmatrix}.$$

\dessin{projectionPara.epsf}{800}{perspective parall\`ele}{pa}

\end{itemize}
\begin{remarque}
Les transformations g\'eom\'etriques d\'efinies par des matrices 
conservent les alignements.  C'est \`a dire une droite est 
transform\'ee en une droite, un plan en un plan et l'ordre
des points sur une droite est pr\'eserv\'e.
\end{remarque}
\dessin{M.epsf}{800}{Transformation du volume canonique par la matrice $M$}{M}

Il est plus fa\c cile de faire une projection parall\`ele qu'une 
projection perspective.  On peut ramener une projection perspective 
\`a une projection parall\`ele en utilisant la transformation 
g\'eom\'etrique d\'efinie par la matrice suivante~: 
$$M=\begin{bmatrix}1&0&0&0\\0&1&0&0\\0&0&\frac{1}{1-m}&\frac{-m}{1-m} 
\\0&0&1&0\end{bmatrix}\;\text{avec } 1>m>0. $$
L'inverse de cette matrice est :
$$M^{-1}=\frac{1-m}{m}\begin{bmatrix}1&0&0&0\\0&1&0&0\\0&0&0&\frac{m}{1-m}\\
0&0&-1&\frac{1}{1-m}\end{bmatrix}\;\text{avec } 1>m>0.$$

Dans la figure~\ref{M} on montre les vues de c\^ot\'e du volume canonique avant (a) et apr\'es 
(b) l'application de la matrice $M$ (voir \cite{FOLE95}).

Cette transformation a les propri\'et\'es suivantes~:
\begin{itemize}
\item Les points du plan $z=1$ ne sont pas modifi\'es.  Pour preuve : 
$$M \begin{bmatrix}x\\y\\1\\1\end{bmatrix}= 
\begin{bmatrix}x\\y\\1\\1\end{bmatrix}.$$
\item Le plan $z = m$ est envoy\'e sur le plan $z=0$ : 
$$M\begin{bmatrix}x\\y\\m\\1\end{bmatrix}= 
\begin{bmatrix}x\\y\\0\\m\end{bmatrix}.$$
\item L'origine $O$ est envoy\'e \`a l'infini dans la 
direction $x=0,\;y=0$ : 
$$M \begin{bmatrix}0\\0\\0\\1\end{bmatrix}= 
\begin{bmatrix}0\\0\\ \frac{-m}{1-m}\\0\end{bmatrix}.$$ 
C'est math\'ematiquement incorrect, mais si on divise tous les \'el\'ements 
par $0,$ le centre de projection va \`a l'infini.
\end{itemize}
\dessin{pertopara.epsf}{800}{Sch\'ema}{ptp}

On peut r\'esumer ces transformations par la figure \ref{ptp}. On 
v\'erifie la formule $$V=M^{-1}W M=W M.$$
Les deux m\'ethodes suivantes sont \'equivalentes : 
\begin{itemize}
\item
\'eliminer les surfaces 
cach\'ees avec l'\oe il \`a l'origine et faire la perspective de 
centre $0$,
\item transformer la sc\`ene par $M$, faire l'\'elimination pour
la projection parall\`ele et faire la projection parall\`ele voir \cite{FOLE95}.
\end{itemize}

%%%%%%%%%%%%%%%%%%%%%%%%%%%%%%%%%%%%%%%%%%%%%%%%
\vfill \newpage
\vspace{0.5cm}\hrule
\section{L'algorithme de Partitionnement binaire de l'espace}
\hrule\vspace{0.5cm}
%%%%%%%%%%%%%%%%%%%%%%%%%%%%%%%%%%%%%%%%%%%%%%%%
% R\'ef\'erences : \cite{FOLE95},\cite{ZEN},\cite{FAQ},\cite{WEB},\cite{GEO}.

\subsection{Partition binaire d'un ensemble}
Un \emph{graphe orient\'e} $\GG=(N,A)$ est d\'efini comme un ensemble 
fini de \emph{noeuds} $N$ et d'\emph{arcs} $A$, o\`u un arc $a$ est 
une paire ordonn\'ee de noeuds $(n_1,n_2)$.  On dira que $n_1$ est 
\emph{pr\'ed\'ecesseur} de $n_2$ et que $n_2$ est un \emph{successeur} 
de $n_1$.

Un \emph{arbre} est un graphe qui poss\'ede un unique noeud n'ayant 
aucun pr\'ed\'ecesseur, appel\'e \emph{racine}, et tel que tout noeud 
autre que la racine a un pr\'ed\'ec\'esseur unique.

Soit un ensemble $A$ on appelle \emph{coupure} une 
application $\alpha$ qui \`a $A$ associe deux ensembles non vides 
$A^+$, $A^-$ tels que $A=A_\alpha^{+} \cup A_\alpha^{-}$.

Une partition binaire $\PP$ d'un ensemble $A$ est une partition 
obtenue par une suite r\'ecursive de coupures.  $A$ est coup\'e 
\'eventuellement par $\alpha$.  $A_\alpha^{+}$ est \'eventuellement 
coup\'ee par $\beta$ ($A_\alpha^+=A_{\alpha^+\beta^+}\cup 
A_{\alpha^+\beta^-}$) et $A_\alpha^{-}$ est \'eventuellement coup\'e 
par $\gamma \cdots$ (figure \ref{coupure}).

%----------------------------------------------------------------
\dessin{coupure.epsf}{800}{Exemple de coupures : elles ne sont pas 
forcement droites !}{coupure}
%

Dans un espace \`a n dimensions.  Les coupures sont des hyperplans 
(voir \cite{FAQ}, \cite{GEO}, \cite{WEB}). c'est \`a dire des sous espaces \`a $n-1$ dimensions.  Par exemple 
dans un espace 3D l'hyperplan est un plan et pour un espace 2D, c'est 
une droite.

\subsection{Repr\'esentation d'une sc\`ene 2D par un BSP}
Une sc\`ene 2D polygonales est d\'efinie par les segments des bords 
des objets qui la composent.  Pour construire l'arbre BSP associ\'e 
\`a la sc\`ene, on utilise, comme coupure, les droites d\'efinies par 
les faces des objets.  Chaque noeud de l'arbre est associ\'e \`a une 
coupure d\'efinie par un segment orient\'e et repr\'esent\'e par une 
matrice $3\times 2$ contenant les 2 coordonn\'ees et la couleur des 
extr\'emit\'es du segment.  Chaque coupure coupe l'espace en deux.  Le 
demi-espace positif est celui qui contient la normale au segment 
orient\'e faisant un angle de $\pi/2$ avec celui-ci.  L'autre 
demi-espace sera dit n\'egatif. 

Soit une sc\`ene $\SSS$ (figure \ref{scene}) constitu\'ee deux objets : un 
triangle et un rectangle.  Orientons toutes les faces (fl\`eches 
noires). On d\'efinit, ici, une sc\`ene comme un ensemble de segments.

%
\dessin{orientation.epsf}{800}{Sc\`ene orient\'ee}{scene}
%

Les segments constituant ces deux figures appartiennent aux coupures 
respectives~: $\alpha, \delta, \varepsilon, \gamma, \xi, \beta, \nu.$ 
Par exemple prenons $\alpha$ l'hypoth\'enuse du triangle comme premier 
segment.  Par la suite, nous ne distingueront plus coupures et 
segments : ils porteront le m\^eme nom.  Donc $\SSS=(\alpha, \delta, 
\varepsilon, \gamma, \xi, \beta, \nu)$.  Coupons $\SSS$ par $\alpha,$ 
notre sc\`ene est \`a pr\'esent form\'ee de deux demi-espaces~: 
$A_{\alpha^{+}}$ et $A_{\alpha^{-}}$.  La construction d'un arbre se 
fera selon le principe suivant.  Si $\alpha$ est la racine de l'arbre 
alors $A_{\alpha^{+}}$ sera la branche gauche et $A_{\alpha^{-}}$ la 
branche droite.  On va donc former un arbre dont les noeuds sont les 
faces des objets de la sc\`ene et les branches des ensembles.  Les 
feuilles sont les ensembles finaux de la partition.

Dans notre exemple, nous aurons donc $A_{\alpha^{-}}=(\beta, \nu)$ et 
$A_{\alpha^{-}}=(\delta, \varepsilon, \gamma,\xi)$.  Pour savoir par 
exemple si $\beta \in A_{\alpha^{-}},$ voir annexe 1 sous-section 1.


%
\dessin{partition.epsf}{800}{Partitionnement d'une sc\`ene 2D}{partit}
%

L'algorithme de construction de l'arbre sera \cite{FAQ}, \cite{GEO}, 
\cite{WEB} :

\vspace{0.5cm}
\centerline{\fbox{$\text{bsp}(\SSS)=
\begin{cases}
\bullet & \text{ si } \SSS = \text{nil}\;, \\
\SSS & \text{ si } \text{ card(}\SSS)=1\;, \\
(\text{t\^ete}(\SSS)\,\text{bsp} (\SSS^{+})\,\text{bsp(}\SSS^{-})) & \text{ 
sinon\;.} \\
\end{cases}$}}
\vspace{0.5cm}

Avec $\SSS^{+}$ la sc\`ene gauche et $\SSS^{-}$ la sc\`ene droite par rapport \`a une coupure.

En d\'eveloppant la r\'ecursion sur la sc\`ene de l'exemple, on obtient :
\begin{align*}
(\alpha \; \delta \; \varepsilon \; \gamma \; \xi \; \beta \; \nu\;) 
&=(\alpha \; (\delta \; \varepsilon \; \gamma \; \xi\;) \; (\beta \; 
\nu\;)\;) \\
&=(\alpha \; (\delta \; (\varepsilon \; \gamma \; \xi\;) \;\bullet \;)\; (\beta 
\;\bullet \; \nu\;) \;)\\
&=(\alpha \; (\delta \; (\varepsilon \; \bullet (\gamma \; \xi\;)\; \bullet 
\;)\; (\beta \; 
\bullet \; \nu)\; ) \\
&=(\alpha \; (\delta \;(\varepsilon \; \bullet \; (\gamma \; \bullet 
\xi\;)\;)\; \bullet\; )\; (\beta \; \bullet \; \nu)\;).
\end{align*}

Ce qui correspond \`a l'arbre de la figure~\ref{arbre}.
%--------
\dessin{arbre.epsf}{800}{Arbre BSP final}{arbre}
%--------


%%%%%%%%%%%%%%%%%%%

\subsection{Algorithme du peintre}
Nous appelons \emph{algorithme du peintre} un algorithme r\'ecursif 
de parcourt un arbre, qui affiche les objets des plus \'eloign\'es au plus
pr\'es de l'oeil, ce qui permetrra ne de pas voir les objets normalement
cach\'es \cite{FAQ}.

Soit $\alpha$ une coupure et $A$ un ensemble repr\'esent\'e par un 
arbre binaire tel que : $A=(\alpha,A_{\alpha^-},A_{\alpha^{+}})$ et 
$P$ la position de l'oeil, o\`u $P \in A$.  La fonction peintre est 
d\'efinie r\'ecursivement par~:

\vspace{0.1cm}
\centerline{\fbox{$\text{peintre }(A, P) =
\begin{cases}
\text{peintre } (A_{\alpha^{-}},P), \text{afficher } \alpha, 
\text{peintre } (A_{\alpha^{+}},P) & \text{si } P\in A_{\alpha^{+}}, 
 \\
\text{peintre } (A_{\alpha^{+}},P), \text{afficher } \alpha, 
\text{peintre } (A_{\alpha^{-}},P) & \text{si } P \in A_{\alpha^{-}}, 
\\
\text{peintre } (nil,P) = \bullet & \text{si } A = nil.
\end{cases}$}}
\vspace{0.5cm}

Pour savoir si $P \in A_{\alpha^{+}}$ ou $P \in A_{\alpha^{-}}$, se 
reporter \`a l'annexe 1, sous-section~1.

Dans notre exemple l'oeil $P$ est situ\'ee dans l'espace 
$A_{\alpha^{+}\delta^{+}\varepsilon^{+}}$.  Nous allons appliquer 
peintre($\alpha, P$).  Comme $P \in \alpha^{+},$ nous affichons en 
premier, la branche $A_{\alpha^{-}}$ puis $\alpha$ puis 
$A_{\alpha^{+}}.$ Dans la brache $A_{\alpha^{-}}$ nous r\'eapliquons 
l'algorithme du peintre.  Comme $\beta$ est plus \'eloign\'e que $\nu$ 
et comme on affiche pas les feuilles, on va donc afficher en premier 
$\beta,$ vient ensuite $\nu$ puis $\alpha$.  Dans la branche 
$A_{\alpha^{+}},$ comme $P \in \delta^{+}$ on affiche en premier 
$\delta$.  On arrive \`a la branche $\varepsilon$, on applique 
peintre($\varepsilon, P$) c'est \`a dire on va afficher la branche 
$A_{\alpha^{+}\delta^{+}\epsilon^{-}}$ puis $\varepsilon$. Comme $\xi$
est plus \'eloign\'e que $\gamma$ on affiche en premier $\xi$ puis 
$\gamma$ et enfin $\varepsilon$. En conclusion, nous allons afficher
dans l'orde : $\beta, \nu, \alpha, \delta, \xi, \gamma, \varepsilon.$
(Figure~\ref{arbre2}).

\dessin{parcours.epsf}{600}{Ordre de parcours}{arbre2}

Pour simplifier l'\'ecriture, nous noterons $\alpha$ pour
peintre($\alpha$, $P$) et $\dot{\alpha}$ pour afficher $\alpha$
dans le d\'eveloppement de r\'ecursion du peintre, on obtient ainsi~:

\begin{align*}
\alpha &=\alpha^- \dot{\alpha}\alpha^+ \\
       &=\backslash\!\!\!\beta^{+}\dot{\beta}\beta^{-}\dot{\alpha}\alpha^+ \\
       &=\dot{\beta}\backslash\!\!\!\nu^{+}\dot{\nu}\backslash\!\!\!\nu^{-}\dot{\alpha}\alpha^+ \\
       &=\dot{\beta}\dot{\nu}\dot{\alpha}\backslash\!\!\!\delta^{-}\dot{\delta}\delta^{+} \\
       &=\dot{\beta}\dot{\nu}\dot{\alpha}\dot{\delta}\varepsilon^-\dot{\varepsilon}\backslash\!\!\!\varepsilon^{+}\\
       &=\dot{\beta}\dot{\nu}\dot{\alpha}\dot{\delta}\gamma^{-}\dot{\gamma}\backslash\!\!\!\gamma^{+}\dot{\varepsilon}\\       
       &=\dot{\beta}\dot{\nu}\dot{\alpha}\dot{\delta}\backslash\!\!\!\xi^{+}\dot{\xi}\backslash\!\!\!\xi^-\dot{\gamma}\dot{\varepsilon}\\   
       &=\dot{\beta}\dot{\nu}\dot{\alpha}\dot{\delta}\dot{\xi}\dot{\gamma}\dot{\varepsilon}
\end{align*}

Nous obtenons ainsi le parours de l'arbre donn\'e dans la figure~\ref{arbre2}.

Ainsi, nous venons d'afficher tous les objets des plus ``\'eloign\'es'' aux 
plus ``proches''. Nous avons ainsi \'elimin\'e toutes les parties cach\'ees.
Il ne reste plus qu'\`a effectuer les projections perspectives des faces 
lorsqu'on affiche une face de la sc\`ene pour pouvoir la visualiser sur l'\'ecran 
(figure \ref{ecran}).
\dessin{peintre.epsf}{700}{Projection sur l'\'ecran}{ecran}

La construction de l'arbre BSP associ\'e \`a une sc\`ene 3D est 
analogue \`a la construction 2D. Les coupures sont, maintenant, les 
plans d\'efinis par les faces des objets. L'algorithme du peintre reste
identique.

%\section{Exemple en Caml d'\'elimination de parties cach\'ees}
%%%%%%%%%%%%%%%%%%%%
\vfill \newpage \vspace{0.5cm}\hrule
\section{Conclusion}
\hrule\vspace{0.5cm}
%%%%%%%%%%%%%%%%%%%%
\subsection{BSP}
L'arbre BSP est un arbre  o\`u les noeuds repr\'esentent les 
polygones d'une sc\`ene 2D ou 3D.

L'arbre BSP est utilis\'e, pour la gestion dans un 
moteur 3D, d'un monde, constitu\'e de nombreux polygones et en 
grande partie statique (Quake, Doom, Flight Simulator etc...).  
Il permet un ordre d'affichage des polygones \'eliminant les parties 
cach\'ees \'evitant ainsi l'utilisation de z-buffer.

Un arbre BSP a plusieurs avantages~:
\begin{itemize}
\item L'arbre donne \`a peu de frais un ordre exact d'affichage 
des polygones gr\^ace \`a l'algorithme du peintre.
\item L'arbre permet de s\'electionner efficacement les faces visibles dans 
le c\^one de vision de la cam\'era.
\item L'arbre permet d'effectuer des tests de collisions sur un nombre 
minimal de polygones.
\end{itemize}

Mais il a aussi quelques inconv\'enients~:
\begin{itemize}
\item La construction de l'arbre engendre un grand nombre de 
polygones.  En effet, de nombreux polygones vont appara\^itre \`a la 
suite de d\'ecoupage. 
\item Une fois construit, l'arbre ne peut pas 
(ou tr\`es difficilement) \^etre modifi\'e. 
\item Un arbre BSP est dur \`a d\'ebuguer.
\end{itemize}


\subsection{Zbuffer}
Algorithme standard 
pour l'obtention rapide d'images de qualit\'e moyenne.  Application \`a la 
conception et fabrication assist\'ee par ordinateur (station de travail 
graphique), \`a la r\'ealit\'e virtuelle et au jeu.


Les avantages du z-buffer sont :
\begin{itemize}
\item L'algorithme est facilement implantable au niveau logiciel et 
mat\'eriel.
\item Il peut \^etre 
optimis\'e en utilisant exclusivement des entiers, ouen utilisant des variantes 
de l'algorithme de Bresenham pour le trac\'e de segments.
\item L'algorithme est facilement pipelinable et parall\'elisable.
\item Gestion simple des recoupements entre objets.  
\item Ex\'ecution en un temps en $o(n)$ du nombre de facettes donc une bonne 
scalabilit\'e.
\end{itemize}

Les inconv\'enients du Z-buffer sont :
\begin{itemize}
\item  Les deux zones tampon peuvent avoir des 
tailles tr\`es importantes (plusieurs Mo dans le cas de grands \'ecrans). 
\item La d\'ecomposition de chaque surface \'el\'ementaire en pixels n\'ecessite une 
puissance de calcul importante (de l'ordre de 50 Mips pour 100000 
facettes/s) et une vitesse d'acc\'es m\'emoire tr\`es \'elev\'ee. 
\item  On n'a pas de gestion intrins\`eque des r\'eflexions et des 
transmissions.  
\item Des difficult\'es existent pour implanter la mod\'elisation de ph\'enomnes optiques complexes.  
\end{itemize}

%%%%%%%%%%%%%%%%%%%%%%%%%%%%%%%%%%%%%%%%%%%%%%%%
\vfill\newpage\vspace{0.5cm}\hrule
\section{Annexes}
\hrule\vspace{0.5cm}
%%%%%%%%%%%%%%%%%%%%%%%%%%%%%%%%%%%%%%%%%%%%%%%%
\subsection{D\'etermination de l'appartenance d'un point \`a un 
demi-espace}
On rappelle que le \emph{produit scalaire} de deux vecteurs 
$\overrightarrow{v_{1}}(a_{1},b_{1})$ et 
$\overrightarrow{v_{2}}(a_{2},b_{2}),$ est la constante   
$$\overrightarrow{v_{1}}.\overrightarrow{v_{2}}=a_{1}a_{2}+b_{1}b_{2}.$$
Soit $M(a_{1},b_{1})$ et $N(a_{2},b_{2})$ deux points.  Ces deux 
points d\'efinissent le vecteur 
$\overrightarrow{MN}=(a_{2}-a_{1},b_{2}-b_{1})$.  Le vecteur normal 
$\overrightarrow{n}$ de $\overrightarrow{MN}$ est~:
$$\overrightarrow{n}=(b_{1}-b_{2},a_{2}-a_{1}).$$ Tous les points 
$P(x,y)$ appartenant \`a la droite $\alpha =(MN)$ v\'erifient
$\overrightarrow{OP}. \overrightarrow{n} = k,$ avec 
$k$ une constante et $O$ l'origine du rep\`ere (figure \ref{scal}). 

\dessin{scalaire.epsf}{700}{D\'etermination de l'appartenance d'un point \`a un 
demi-espace}{scal}

Dans notre exemple nous avons~: \\
$$\overrightarrow{OP}.\overrightarrow{n} = 
x(b_{1}-b_{2})+y(a_{2}-a_{1})=k=a_{1}(b_{1}-b_{2})+b_{1}(a_{2}-a_{1}).$$ 

\vspace{0.5cm}
\centerline{\fbox{$P\in 
\begin{cases}
A_{\alpha^{+}} & \text{Si } \overrightarrow{OP}.\overrightarrow{n} > k, \\
A_{\alpha^{-}} & \text{Si } \overrightarrow{OP} . \overrightarrow{n} < 
k, \\
\alpha & \text{Si }\overrightarrow{OP}. \overrightarrow{n} = k.
\end{cases}$}}
\vspace{0.5cm}

\subsection{Partition d'une sc\`ene 2D en deux sc\`enes par une coupure}

On veut savoir si un segment $\beta$ appartient \`a l'ensemble positif 
ou n\'egatif d'une coupure.  On se donne d'abord $P_{1}, P_{2}$ les 
sommets du segment $\beta$.  Ensuite, on applique deux produits 
scalaires~: un entre le vecteur $\overrightarrow{v_{\alpha}}$ de la 
coupure $\alpha$ et $\overrightarrow{OP_{1}}$, puis un autre produit 
scalaire avec $\overrightarrow{v_{\alpha}}$ et 
$\overrightarrow{OP_{2}}$.

\begin{itemize}
\item Si les deux produits scalaires sont plus grands que la 
constante, alors $\beta \in A_{\alpha^{+}}.$ \item Si les deux 
produits scalaires sont plus petits que la constante, alors $\beta \in 
A_{\alpha^{-}}$ (Exemple 1).  \item Si un des produits scalaires est 
plus grand et que l'autre est plus petit, alors on recommence la 
m\^eme op\'eration sur les segments $[P_{1}P']$ et $[P' P_{2}],$ o\` u 
$P'$ est l'intersection entre $\alpha$ et $\beta$. (Exemple 2).
\end{itemize}
\dessin{testEnsemble1.epsf}{800}{Exemple1}{EX1}
\dessin{testEnsemble2.epsf}{800}{Exemple 2}{EX1}
%%%%%%%%%%%%%%%%%%%%%%%%%%%%%%%%%%%%%%%%%%%%%%%%%%%%%%%%
\vfill \newpage \vspace{0.5cm}\hrule
\begin{thebibliography}{999999999}
\hrule\vspace{0.5cm}
%%%%%%%%%%%%%%%%%%%%%%%%%%%%%%%%%%%%%%%%%%%%%%%%%%%%%%%
\bibitem[FOLE95]{FOLE95} J. Foley, A. van Dam, S. Feiner, J. Hughes, R. 
Phillips, \emph{Introduction \`a l'infographie}, Addison-Wesley 
France, 1995.

\bibitem[ABVA97]{ZEN} M. Abvash, \emph{Zen de la programmation graphique}
International Thomson Publishing France, 1997.

\bibitem[EBEV02]{ENG} D. Ebevly, \emph{3D Game Engine and Design} Morgan
Kaufman Publishers, 2002.

\bibitem[ROSS86]{ROSS86} J.R. Rossignac and A.A.G. Requicha, 
\emph{Depth-Buffering Display Techniques for Constructive Solid 
Geometry} CG \& A,6(9), Septembre 1986, 29-39.

\bibitem[CATM74]{CATM74} E,A Catmull, \emph{Subdivision Algorithm for Computer 
Display of Curved Surfaces} Ph.D. Thesis, Report UTEC-CSc-74-133, 
Computer Science Departement, University of Utah, Salt Lake City, 
D\'ecembre 1974.

\bibitem[ATHE81]{ATH} P.R. Atherton, \emph{A Method of Interactive 
Visualisation of CAD Surface Models on a Color Video Display} SIGGRAPH 
81, 279-287.

\bibitem[faqs.org]{FAQ} 
\emph{wwww.faqs.org$\backslash$faqs$\backslash$graphics
$\backslash$bsptree-faq$\backslash$index.html}

\bibitem[geocities.com]{GEO} 
\emph{wwww.geocities.com$\backslash$SiliconValley$\backslash$
2151$\backslash$bsp.html}

\bibitem[cs.wpi.edu]{WEB} 
\emph{wwww.cs.wpi.edu$\backslash$\~matt$\backslash$courses
$\backslash$cs563$\backslash$talks$\backslash$bsp$\backslash$bsp.html}

\end{thebibliography}
\end{document}