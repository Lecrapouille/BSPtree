\documentclass[a4paper,11pt]{amsart}

\usepackage[francais]{babel}
%\usepackage[pdftex]{graphicx}
%\uspackage{pstricks}
\usepackage{graphicx}
\newtheorem{remarque}{Remarque} \newcommand{\AAA}{{\mathcal A}}
\newcommand{\BB}{{\mathcal B}} \newcommand{\CC}{{\mathcal C}}
\newcommand{\DD}{{\mathcal D}} \newcommand{\EE}{{\mathcal E}}
\newcommand{\FF}{{\mathcal F}} \newcommand{\GG}{{\mathcal G}}
\newcommand{\HH}{{\mathcal H}} \newcommand{\II}{{\mathcal I}}
\newcommand{\JJ}{{\mathcal J}} \newcommand{\KK}{{\mathcal K}}
\newcommand{\LL}{{\mathcal L}} \newcommand{\MM}{{\mathcal M}}
\newcommand{\NN}{{\mathcal N}} \newcommand{\OO}{{\mathcal O}}
\newcommand{\PP}{{\mathcal P}} \newcommand{\QQ}{{\mathcal Q}}
\newcommand{\RR}{{\mathcal R}} \newcommand{\SSS}{{\mathcal S}}
\newcommand{\TT}{{\mathcal T}} \newcommand{\UU}{{\mathcal U}}
\newcommand{\VV}{{\mathcal V}} \newcommand{\WW}{{\mathcal W}}
\newcommand{\XX}{{\mathcal X}} \newcommand{\ZZ}{{\mathcal Z}}
\newcommand{\bbR}{{\mathbb R}} \newcommand{\bbD}{{\mathbb D}}
\newcommand{\bbO}{{\mathbb O}} \newcommand{\bbS}{{\mathbb S}}
\newcommand{\bbE}{{\mathbb E}} \newcommand{\bbN}{{\mathbb N}}
\newcommand{\bbM}{{\mathbb M}} \newcommand{\bbV}{{\mathbb V}}
\newcommand{\bbC}{{\mathbb K}} \newcommand{\bbF}{{\mathbb F}}
\newcommand{\bbP}{{\mathbb P}}

\newcommand{\dessin}[4]{
\begin{figure}[htb]
\centering
\includegraphics[scale= #2]{#1}
\caption{#3}
\label{#4}
\end{figure}}

\newcommand{\dessinsscaption}[2]{
\begin{figure}[htb]
\centering
\includegraphics[scale= #2]{#1}
\end{figure}}

\title{Mouvement horizontal}
\author{QUADRAT Quentin}

\begin{document}
\maketitle
\section{Mouvement horizontal}

Pour mod\'eliser les d\'eplacements du v\'ehicule dans le plan
horizontal, on repr\'esente le v\'ehicule par une barre de demi
longueur $l$ et de masse ponctuelle $M$ \`a laquelle sont accroch\'ees
deux roues de masse $m$.  Dans un rep\`ere fixe $(Oxy)$ (voir figure
\ref{direction}), on note~:
\begin{itemize}
\item $x(t)$ et $y(t)$ la position du centre de gravit\'e de la voiture,
\item $\phi(t)$ l'angle de la carcasse de la carcasse avec l'axe $(Ox)$,
\item $\psi(t)$ l'angle des roues avec l'axe $(Ox)$.
\end{itemize}

\dessin{direction.epsf}{0.6}{Mod\'elisation de la voiture (vue de dessus)}{direction}

On note :
\begin{itemize}
\item $a(t)$ acc\'el\'eration donn\'ee par le joueur \`a la voiture
(la puissance du moteur),
\item $\xi(t)$ le changement de direction donn\'e par le joueur au v\'ehicule.
\end{itemize}

%L'\'energie cin\'etique de la voiture vaut alors : $$\frac{1}{2}(2m+M)(\dot{x}^2+\dot{y}^2)+ml\dot{\phi}^2\;.$$
%
%L'\'energie potentielle du moteur vaut : $$-a\cos \phi\ x - a\sin \phi\ y\;.$$
%
%L'\'energie potentielle de d\'eformation des pneus lors d'un changement de direction est don\'e par: $$-\frac{\xi^2}{2}\;.$$
%
%L'action \`a minimiser est donc :
%$$\AAA=1/2\int \left(\frac{1}{2}(2m+M)(\dot{x}^2+\dot{y}^2)+ml\dot{\phi}^2 +
%a\cos \phi\ x + a\sin \phi\ y -\frac{k(\psi-\phi)^2}{2}\right) dt $$

La vitesse de rotation du v\'ehicule autour de son centre de gravite est proportionnelle au changement de direction $\xi$ indiqu\'ee.
%Le moteur exerce une force d'intensit\'e $a$ choisie par le joueur
%dans la direction $\phi$.
On suppose que lors d'un changement de direction, l'\'elasticit\'e des pneus --- adh\'erents sur le sol --- produit un couple de rotation proportionel \`a $\xi$.

En n\'egligeant les frottements de l'air, on obtient les \'equations du mouvement en appliquant le principe fondamental de la dynamique :
$$(2m+M)\ddot{x} = a\cos\phi\;,$$
$$(2m+M)\ddot{y} = a\sin\phi\;,$$
$$\dot{\phi} = \xi\;.$$


\section{Discr\'etisation des \'equations diff\'erentielles}
On peut discr\'etiser ces \'equations diff\'erentielles~:
$$x(t+h) = 2x(t)-x(t-h)+h^2\left(\frac{a(t)\cos\phi(t)}{2m+M}\right)\,$$
$$y(t+h) = 2y(t)-y(t-h)+h^2\left(\frac{a(t)\sin\phi(t)}{2m+M}\right)\,$$
$$\phi(t+h) = \phi(t)+h\xi$$
o\`u $h$ d\'esigne le pas de discr\'etisation en temps.

\end{document}
