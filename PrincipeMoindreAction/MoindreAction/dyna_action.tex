\documentclass[a4paper,11pt]{amsart}

\usepackage[francais]{babel}
%\usepackage[pdftex]{graphicx}
%\uspackage{pstricks}
\usepackage{graphicx}

\newcommand{\dessin}[4]{
\begin{figure}[htb]
\centering
\includegraphics[scale= #2]{#1}
\caption{#3}
\label{#4}
\end{figure}}

\newtheorem{remarque}{Remarque} \newcommand{\AAA}{{\mathcal A}}
\newcommand{\BB}{{\mathcal B}} \newcommand{\CC}{{\mathcal C}}
\newcommand{\DD}{{\mathcal D}} \newcommand{\EE}{{\mathcal E}}
\newcommand{\FF}{{\mathcal F}} \newcommand{\GG}{{\mathcal G}}
\newcommand{\HH}{{\mathcal H}} \newcommand{\II}{{\mathcal I}}
\newcommand{\JJ}{{\mathcal J}} \newcommand{\KK}{{\mathcal K}}
\newcommand{\LL}{{\mathcal L}} \newcommand{\MM}{{\mathcal M}}
\newcommand{\NN}{{\mathcal N}} \newcommand{\OO}{{\mathcal O}}
\newcommand{\PP}{{\mathcal P}} \newcommand{\QQ}{{\mathcal Q}}
\newcommand{\RR}{{\mathcal R}} \newcommand{\SSS}{{\mathcal S}}
\newcommand{\TT}{{\mathcal T}} \newcommand{\UU}{{\mathcal U}}
\newcommand{\VV}{{\mathcal V}} \newcommand{\WW}{{\mathcal W}}
\newcommand{\XX}{{\mathcal X}} \newcommand{\ZZ}{{\mathcal Z}}
\newcommand{\bbR}{{\mathbb R}} \newcommand{\bbD}{{\mathbb D}}
\newcommand{\bbO}{{\mathbb O}} \newcommand{\bbS}{{\mathbb S}}
\newcommand{\bbE}{{\mathbb E}} \newcommand{\bbN}{{\mathbb N}}
\newcommand{\bbM}{{\mathbb M}} \newcommand{\bbV}{{\mathbb V}}
\newcommand{\bbC}{{\mathbb K}} \newcommand{\bbF}{{\mathbb F}}
\newcommand{\bbP}{{\mathbb P}}

\title{Principe de la moindre action}
\author{QUADRAT Quentin}
  \thanks{{\tt Page web} : www.epita.fr/$\sim$quadra\_q}
  \thanks{{\tt Email :} quadra\_q@epita.fr}

\begin{document}

\maketitle

\section{D\'efinition}
On appelle action $\AAA$ d'un syst\`eme m\'ecanique l'int\'egrale
le long du mouvement de la diff\'erence de son \'energie cin\'etique et
de son \'energie potentielle:
$$\AAA(x()) = \int (\EE_c(x(t)) - \EE_p(x(t))dt\;,$$
o\`u  $\EE_c(x(t))$ designe l'\'energie cin\'etique, $\EE_p(x(t))$
l'\'energie potentielle et $t\mapsto x(t)$ la trajectoire du syst\`eme.

Le principe de la moindre action nous dit que la trajectoire
du syst\`eme est celle qui minimise l'action.

Pour trouver cette trajectoire on calcule la variation de
l'action $\delta \AAA$ associ\'ee \`a la variation de la trajectoire
$\delta x$ et on d\'etermine les conditions qui assurent que
$\delta \AAA$ soit nul quelque soit $\delta x$.

\section{Exemple d'application}
Prenons le cas de deux masses, de poids respectifs $m_1$ et $m_2$,
accroch\'ees l'une \`a l'autre par un ressort de force $F =
-kl$ ou $l$ d\'esigne l'allongement du ressort ($|x_1(t)-x_2(t)|$)
avec  $x_1(t)$ et $x_2(t)$ les positions \`a l'instant $t$ des deux masses.

\begin{align*}
\delta \AAA =& 1/2\int (m_1(\dot{x_1}+\delta \dot{x}_1)^2+
 m_2( \dot{x}_2+\delta \dot{x}_2)^2-
 k(x_2+\delta x_2-x_1-\delta x_1)^2) dt\\
 &-1/2\int (m_1\dot{x}_1^2+m_2\dot{x}_2^2-k(x_2-x_1)^2)dt \;,
\end{align*}
$$\delta \AAA = \int m_1\dot{x}_1\delta \dot{x}_1+m_2\dot{x}_2 \delta
\dot{x}_2-k(x_2-x_1) (\delta x_2-\delta x_1)+o(\| x_1-x_2\|)\;,$$

Par int\'egration par partie on obtient :
$$\delta \AAA = \int -m_1\ddot{x_1}\delta x_1 -m_2\ddot{x_2}\delta x_2
- k(x_2-x_1)(\delta x_2-\delta x_1) + o(\| x_1-x_2\|)\;,$$
car on
suppose que les variations des trajectoires sont nulles aux
extr\'emites.

Finalement on trouve :
$$\delta \AAA =\int (-m_1\ddot{x_1}+k(x_2-x_1))\delta x_1 + \int
(-m_2\ddot{x_2}+k(x_1-x_2))\delta x_2 + o(\| x_1-x_2\|)\;.$$

L'action doit \^etre minimale c.a.d. $$\delta A = 0,\quad\forall
\delta x_1,\;\forall \delta x_2,$$

donc :
$$-m_1\ddot{x_1}+k(x_2-x_1) = 0,\quad-m_2\ddot{x_2}+k(x_1-x_2) = 0\;.$$

On obtient les m\^emes \'equations que celles en appliquant la loi
fondamentale de la dynamique $m\vec{\gamma} = \vec{F}$.

\end{document}
