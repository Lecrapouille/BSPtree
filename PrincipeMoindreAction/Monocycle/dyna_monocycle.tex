\documentclass[a4paper,11pt]{amsart}

\usepackage[francais]{babel}
%\usepackage[pdftex]{graphicx}
%\uspackage{pstricks}
\usepackage{graphicx}

\newcommand{\dessin}[4]{
\begin{figure}[htb]
\centering
\includegraphics[scale= #2]{#1}
\caption{#3}
\label{#4}
\end{figure}}

\newtheorem{remarque}{Remarque} \newcommand{\AAA}{{\mathcal A}}
\newcommand{\BB}{{\mathcal B}} \newcommand{\CC}{{\mathcal C}}
\newcommand{\DD}{{\mathcal D}} \newcommand{\EE}{{\mathcal E}}
\newcommand{\FF}{{\mathcal F}} \newcommand{\GG}{{\mathcal G}}
\newcommand{\HH}{{\mathcal H}} \newcommand{\II}{{\mathcal I}}
\newcommand{\JJ}{{\mathcal J}} \newcommand{\KK}{{\mathcal K}}
\newcommand{\LL}{{\mathcal L}} \newcommand{\MM}{{\mathcal M}}
\newcommand{\NN}{{\mathcal N}} \newcommand{\OO}{{\mathcal O}}
\newcommand{\PP}{{\mathcal P}} \newcommand{\QQ}{{\mathcal Q}}
\newcommand{\RR}{{\mathcal R}} \newcommand{\SSS}{{\mathcal S}}
\newcommand{\TT}{{\mathcal T}} \newcommand{\UU}{{\mathcal U}}
\newcommand{\VV}{{\mathcal V}} \newcommand{\WW}{{\mathcal W}}
\newcommand{\XX}{{\mathcal X}} \newcommand{\ZZ}{{\mathcal Z}}
\newcommand{\bbR}{{\mathbb R}} \newcommand{\bbD}{{\mathbb D}}
\newcommand{\bbO}{{\mathbb O}} \newcommand{\bbS}{{\mathbb S}}
\newcommand{\bbE}{{\mathbb E}} \newcommand{\bbN}{{\mathbb N}}
\newcommand{\bbM}{{\mathbb M}} \newcommand{\bbV}{{\mathbb V}}
\newcommand{\bbC}{{\mathbb K}} \newcommand{\bbF}{{\mathbb F}}
\newcommand{\bbP}{{\mathbb P}}

\title{Etude du mouvement vertical d'un monocycle}
\author{QUADRAT Quentin}

\begin{document}
\maketitle

\section{Pr\'esentation du mod\'ele}
Le v\'ehicule est mod\'elis\'e en 2D, par une carcasse de masse
ponctuelle $M_v$ accroch\'ee \`a une roue (de rayon $r$ et de masse
$M_r$) par un ressort. On note $u(t)$ la l'altitude du sol par rapport
\`a au rep\'ere, $y(t)$ est l'altitude de la carcasse, $z(t)$
l'alongement du ressort, et $y(t)+z(t)$, l'altitude de la roue. On
note $g$ la gravit\'e (figure \ref{dyna}).

\dessin{dynavoit.epsf}{0.6}{Le monocycle}{dyna}

\section{Calcul des \'equations du mouvement}
Les forces qui sont en jeux sont : la pesanteur des masses (roue et
carcasse), la r\'epulsion du sol sur la roue et la force du ressort.

L'\'energie cin\'etique de la roue (not\'ee $\EE_r^c$) est :
$\EE_r^c = 1/2M_r(\dot{y}+\dot{z})^2\;.$

L'\'energie cin\'etique de la voiture (not\'ee $\EE_v^c$) est :
$\EE_v^c = 1/2M_v\dot{y}^2\;.$

L'\'energie ressort (not\'ee $\EE_r$) est :
$\EE_r = 1/2kz^2;.$

L'\'energie potentiel de la voiture (not\'ee $\EE_v^p$) est :
$\EE_v^p = M_vgy\;.$

L'\'energie potentiel de la roue (not\'ee $\EE_r^p$) est :
$\EE_r^p = M_rg(y+z)\;.$

L'\'energie de reaction du sol (not\'ee $\EE_s$) est : $\EE_s =
1/2((u-(y+z-r))^+)^2\;,$ c'est \`a dire que $\EE_s$ vaut $1/2(u-(y+z-r))^2$
quand $u-(y+z-r) > 0$, sinon il vaut $0$.

\begin{align*}
\AAA(x()) =& \int (\;1/2M_r(\dot{y}+\dot{z})^2 + 1/2M_v\dot{y}^2 -1/2kz^2 -M_vgy - M_rg(y+z)\\
&- 1/2(u-(y+z-r))^+)^2\;)dt\;
\end{align*}

Comme dans le rapport expliquant le principe de la moindre action, on
trouve un syst\`eme d'\'equation diff\'erentielle o\`u les inconnues
sont l'altitude de la roue et de la carcasse :
\begin{align}
  &(\ddot{z}+\ddot{y})M_r+\ddot{y}M_v + g(M_v+M_r) - (u-(y+z-r))^+ = 0\;,\label{equ1}\\
  &(\ddot{z}+\ddot{y})M_r+M_rg+kz-(u-(y+z-r))^+ = 0\;.\label{equ2}
\end{align}
En notant : $w=y+z$, (\ref{equ1})-(\ref{equ2}) donne :
$$\ddot{y}M_v+gM_v-k(w-y)=0.$$
(\ref{equ2}) donne :
$$\ddot{w}M_r+gM_r+k(w-y)-(u-(w-r))^+ = 0$$

\section{Discr\'etisation des \'equations diff\'erentielles}
Pour calculer les trajectoires des deux corps, nous pouvons approximer
les \'equations diff\'erentielles (\ref{equ1}) et (\ref{equ2}) par les
\'equations r\'ecurrentes, o\`u $h$ d\'esigne le pas de discr\'etisation~:

\begin{align*}
& y(t+h)=2y(t)-y(t-h)+gh^2-\frac{k(w-y)h^2}{M_v}\;,\\
& w(t+h)=2w(t)-w(t-h)+\frac{h^2}{M_r}(u(t)-w(t)+r)^+-\frac{h^2k(w(t)-y(t))}{M_r}-h^2g\;.
\end{align*}

\end{document}
