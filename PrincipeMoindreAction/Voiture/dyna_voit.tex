\documentclass[a4paper,11pt]{amsart}
%\usepackage[francais]{babel}
%\usepackage[pdftex]{graphicx}
%%\uspackage{pstricks}
\usepackage{graphicx}
\newtheorem{remarque}{Remarque} \newcommand{\AAA}{{\mathcal A}}
\newcommand{\BB}{{\mathcal B}} \newcommand{\CC}{{\mathcal C}}
\newcommand{\DD}{{\mathcal D}} \newcommand{\EE}{{\mathcal E}}
\newcommand{\FF}{{\mathcal F}} \newcommand{\GG}{{\mathcal G}}
\newcommand{\HH}{{\mathcal H}} \newcommand{\II}{{\mathcal I}}
\newcommand{\JJ}{{\mathcal J}} \newcommand{\KK}{{\mathcal K}}
\newcommand{\LL}{{\mathcal L}} \newcommand{\MM}{{\mathcal M}}
\newcommand{\NN}{{\mathcal N}} \newcommand{\OO}{{\mathcal O}}
\newcommand{\PP}{{\mathcal P}} \newcommand{\QQ}{{\mathcal Q}}
\newcommand{\RR}{{\mathcal R}} \newcommand{\SSS}{{\mathcal S}}
\newcommand{\TT}{{\mathcal T}} \newcommand{\UU}{{\mathcal U}}
\newcommand{\VV}{{\mathcal V}} \newcommand{\WW}{{\mathcal W}}
\newcommand{\XX}{{\mathcal X}} \newcommand{\ZZ}{{\mathcal Z}}
\newcommand{\bbR}{{\mathbb R}} \newcommand{\bbD}{{\mathbb D}}
\newcommand{\bbO}{{\mathbb O}} \newcommand{\bbS}{{\mathbb S}}
\newcommand{\bbE}{{\mathbb E}} \newcommand{\bbN}{{\mathbb N}}
\newcommand{\bbM}{{\mathbb M}} \newcommand{\bbV}{{\mathbb V}}
\newcommand{\bbC}{{\mathbb K}} \newcommand{\bbF}{{\mathbb F}}
\newcommand{\bbP}{{\mathbb P}}

\newcommand{\dessin}[4]{
\begin{figure}[htb]
\centering
\includegraphics[scale= #2]{#1}
\caption{#3}
\label{#4}
\end{figure}}

\title{Etude du mouvement vertical d'une voiture}
\author{QUADRAT Quentin}
  \thanks{{\tt Page web} : www.epita.fr/$\sim$quadra\_q}
  \thanks{{\tt Email :} quadra\_q@epita.fr}

\begin{document}
\maketitle

\section{Pr\'esentation du mod\'ele}
On veut \'etudier le mouvement vertical d'une voiture dont le 
mouvement horizontal est donn\'e par le triplet $(x(t),y(t),\phi 
(t))$, o\`u $\phi (t)$ est la direction du v\'ehicule. Celui-ci est 
mod\'elis\'e en 3D, par une carcasse repr\'esent\'ee
par une plaque (de longueur $2L$, de largeur $2l$, de masse
ponctuelle $M$ et de moment d'inertie $I_{\theta}$ et $I_{\alpha}$)
\`a laquelle sont accroch\'ees quatre roues (de rayon
$r$ et de masse $m$) par des ressorts (de rigidit\'e $k$).

On note  $I_{\alpha}$ le moment 
d'inertie par rapport \`a l'axe longitudinal de la voiture (roulis),
$I_{\theta}$ le moment d'inertie 
par rapport \`a l'axe transversal de la voiture (tangage), $g$ la gravit\'e, 
$u(x,y)$ l'altitude du sol, $z(t)$ l'altitude du centre de gravit\'e 
de la carcasse, $z_1(t) \ldots z_4(t)$ les
allongements des quatre ressorts de la suspension de la voiture, $\theta$ 
l'angle de tangage et $\alpha$ l'angle de roulis.

\dessin{Voiture.epsf}{0.6}{Voiture}{voiture}

\section{Calcul des \'energies cin\'etiques}
De la premi\`ere roue :
$ 1/2m(\dot{z}+L\dot{\theta}+\dot{z}_1+l\dot{\alpha})^2 \;.$

De la deuxi\`eme roue :
$ 1/2m(\dot{z}+L\dot{\theta}+\dot{z}_2-l\dot{\alpha})^2 \;.$

De la troisi\`eme roue :
$ 1/2m(\dot{z}-L\dot{\theta}+\dot{z}_3-l\dot{\alpha})^2 \;.$

De la quatri\`eme roue :
$ 1/2m(\dot{z}-L\dot{\theta}+\dot{z}_4+l\dot{\alpha})^2 \;.$

De la caracasse :
$1/2(M\dot{z}^2 +I_{\theta}\dot{\theta}^2 +I_{\alpha}\dot{\alpha}^2)\;.$


\section{Calcul des \'energies potentielles}
Des quatre roues : 
$ mg(4z+z_1+z_2+z_3+z_4)\;.$

Des quatre ressorts :
$ 1/2k(z^2_1+z^2_2+z^2_3+z^2_4)\;.$

De la caracasse :
$ Mgz\;.$

Soit :
\begin{align*} 
R_1 &= (u(x+L\cos\phi+l\sin\phi,y+L\sin\phi-l\cos\phi)-(z+z_1+L\theta+l\alpha-r))^+\;;\\
R_2 &= (u(x+L\cos\phi-l\sin\phi,y+L\sin\phi+l\cos\phi)-(z+z_2+L\theta-l\alpha-r))^+\;;\\
R_3 &= (u(x-L\cos\phi-l\sin\phi,y-L\sin\phi+l\cos\phi)-(z+z_3-L\theta-l\alpha-r))^+\;;\\
R_4 &= (u(x-L\cos\phi+l\sin\phi,y-L\sin\phi-l\cos\phi)-(z+z_4-L\theta+l\alpha-r))^+\;.
\end{align*}
 
L'\'en\'ergie de la r\'eaction du sol sur la premi\`eme roue est alors
de $1/2R_1^2\;,$ celle de la deuxi\`eme roue : $1/2R_2^2\;,$ celle de
la troisi\`eme : $1/2R_3^2\,$ et enfin la quatri\`eme : $1/2R_4^2\;.$

\section{Principe de la moindre action}

On trouve :
\begin{align*}
\delta\AAA = \int &
                      M\dot{z}\delta \dot{z} 
                      + I_{\theta}\dot{\theta}\delta\dot{\theta} +
                        I_{\alpha}\dot{\alpha}\delta\dot{\alpha} -2Mg\delta z \\
                  &   -2mg(4\delta z + \delta z_{1} + \delta z_{2} 
                          +\delta z_{3} + \delta z_{4}) \\
                  &   -k(z_{1}\delta z_{1} + z_{2}\delta z_{2}
                      +z_{3}\delta z_{3} + z_{4}\delta z_{4}) \\
                  &   +m(\dot{z}+\dot{z}_1+L\dot{\theta}+l\dot{\alpha})
                      (\delta\dot{z}+\delta\dot{z}_1+L\delta\dot{\theta}+l\delta\dot{\alpha}) 
                      \\ 
                  &   +m(\dot{z}+\dot{z}_2+L\dot{\theta}-l\dot{\alpha})
                      (\delta\dot{z}+\delta\dot{z}_2+L\delta\dot{\theta}-l\delta\dot{\alpha})
                      \\
                  &   +m(\dot{z}+\dot{z}_3-L\dot{\theta}-l\dot{\alpha})
                      (\delta\dot{z}+\delta\dot{z}_3-L\delta\dot{\theta}-l\delta\dot{\alpha})
                      \\
                  &   +m(\dot{z}+\dot{z}_4-L\dot{\theta}+l\dot{\alpha})
                      (\delta\dot{z}+\delta\dot{z}_4-L\delta\dot{\theta}+l\delta\dot{\alpha})\\
                  &   +R_1(\delta z+\delta z_1+L\delta\theta+l\delta\alpha-r)\\
                  &   +R_2(\delta y+\delta y_2+L\delta\theta-l\delta\alpha-r)\\
                  &   +R_3(\delta y+\delta y_3-L\delta\theta-l\delta\alpha-r)\\
                  &   +R_4(\delta y+\delta y_4-L\delta\theta+l\delta\alpha-r) 
\end{align*}

La variation de l'action apr\'es int\'egration par partie vaut :
\begin{align*}
\delta\AAA = \int &
                      -M\ddot{z}\delta z 
                      - I_{\theta}\ddot{\theta}\delta\theta -
                        I_{\alpha}\ddot{\alpha}\delta\alpha -2Mg\delta z \\
                  &   -2mg(4\delta z + \delta z_{1} + \delta z_{2} 
                          +\delta z_{3} + \delta z_{4}) \\
                  &   -kz_{1}\delta z_{1} -kz_{2}\delta z_{2}
                      -kz_{3}\delta z_{3} -kz_{4}\delta z_{4} \\
%avant de faire l'IPP
%                     &  +4m\dot{z}\delta\dot{z} + m\dot{z}(\delta\dot{z}_1
%                                                + \delta\dot{z}_2
%                                                + \delta\dot{z}_3
%                                                + \delta\dot{z}_4)
%                                                \\
%                    &   + m(\dot{z}_1\delta\dot{z}_1
%                        + \dot{z}_2\delta\dot{z}_2
%                        + \dot{z}_3\delta\dot{z}_3
%                        + \dot{z}_4\delta\dot{z}_4)\\
%                    &   + (\dot{z}_1 + \dot{z}_2 + \dot{z}_3 + \dot{z}_4
%                          ) m\delta\dot{z}\\
%                    &   + (\dot{z}_1+\dot{z}_2-\dot{z}_3-\dot{z}_4)
%                          mL\delta\dot{\theta} \\
%                    &   + (\dot{z}_1-\dot{z}_2-\dot{z}_3+\dot{z}_4)
%                          ml\delta\dot{\alpha} \\
%                    &   + (\delta\dot{z}_1+\delta\dot{z}_2
%                          -\delta\dot{z}_3-\delta\dot{z}_4)
%                                                mL\dot{\theta}\\
%                    &   +4mL^2\dot{\theta}\delta\dot{\theta} 
%                        +4ml^2\dot{\alpha}\delta\dot{\alpha}\\
%                    &   +(\delta\dot{z}_1-\delta\dot{z}_2
%                        -\delta\dot{z}_3+\delta\dot{z}_4)
%                        ml\dot{\alpha}\\
%END
                    &   -4m\ddot{z}\delta z - m\ddot{z}(\delta z_1
                                               +\delta z_2
                                               +\delta z_3
                                               +\delta z_4)
                                               \\
                   &   - m\ddot{z}_1\delta z_1
                       - m\ddot{z}_2\delta z_2
                       - m\ddot{z}_3\delta z_3
                       - m\ddot{z}_4\delta z_4\\
                   &   - (\ddot{z}_1 + \ddot{z}_2 + \ddot{z}_3 + \ddot{z}_4
                         ) m\delta z\\
                   &   - (\ddot{z}_1+\ddot{z}_2-\ddot{z}_3-\ddot{z}_4)
                         mL\delta\theta \\
                   &   - (\ddot{z}_1-\ddot{z}_2-\ddot{z}_3+\ddot{z}_4)
                         ml\delta\alpha \\
                   &   - (\delta z_1+\delta z_2
                         -\delta z_3-\delta z_4) mL\ddot{\theta}\\
                   &   -4mL^2\ddot{\theta}\delta \theta 
                       -4ml^2\ddot{\alpha}\delta \alpha\\
                   &   -(\delta z_1-\delta z_2
                       -\delta z_3+\delta z_4) ml\ddot{\alpha}\\
                   &   +R_1(\delta z + \delta z_1 + L\delta\theta +
                                               l\delta\alpha)\\
                   &   +R_2(\delta z + \delta z_2 + L\delta\theta -
                                               l\delta\alpha)\\
                   &   +R_3(\delta z + \delta z_3 - L\delta\theta -
                                               l\delta\alpha)\\
                   &   +R_4(\delta z + \delta z_4 - L\delta\theta + l\delta\alpha)\;.
\end{align*}

On trouve un syst\`eme d'\'equation diff\'erentielle :
\begin{align}
\label{eqv01} \ddot{z}+\frac{m(\ddot{z}_1+\ddot{z}_2+\ddot{z}_3+\ddot{z}_4)}{M+4m} & = \frac{R_1+R_2+R_3+R_4}{M+4m}-2g\;. \\ 
\label{eqv02} kz_1+m(\ddot{z}_1+\ddot{z}+L\ddot{\theta}+l\ddot{\alpha}) & = -2mg +
R_1 \;. \\ 
\label{eqv03} kz_2+m(\ddot{z}_2+\ddot{z}+L\ddot{\theta}-l\ddot{\alpha}) & = -2mg +
R_2 \;. \\ 
\label{eqv04} kz_3+m(\ddot{z}_3+\ddot{z}-L\ddot{\theta}-l\ddot{\alpha}) & = -2mg +
R_3 \;. \\ 
\label{eqv05} kz_4+m(\ddot{z}_4+\ddot{z}-L\ddot{\theta}+l\ddot{\alpha}) & = -2mg +
R_4 \;. \\ 
\label{eqv06} m(\ddot{z}_1+\ddot{z}_2-\ddot{z}_3-\ddot{z}_4+4L\ddot{\theta})+\frac{I_{\theta}\ddot{\theta}}{L} 
& = R_1+R_2-R_3-R_4\;.\\ 
\label{eqv07}m(\ddot{z}_1-\ddot{z}_2-\ddot{z}_3+\ddot{z}_4+4l\ddot{\alpha})+\frac{I_{\alpha} \ddot{\alpha}}{l}
& = R_1-R_2-R_3+R_4 \;.
\end{align}

En faisant (\ref{eqv02}) plus (\ref{eqv03}) moins
(\ref{eqv04}) moins (\ref{eqv05}) moins (\ref{eqv06}), on obtient~:
\begin{align}
\label{eqv08} I_{\theta}\ddot{\theta} = Lk(z_1+z_2-z_3-z_4)\;.
\end{align}

En faisant (\ref{eqv02}) plus (\ref{eqv06}) moins
(\ref{eqv03}) moins (\ref{eqv04}) moins (\ref{eqv07}), on obtient~:
\begin{align}
\label{eqv09} I_{\alpha}\ddot{\alpha} = lk(z_1-z_2-z_3+z_4)\;.
\end{align}

En faisant (\ref{eqv01}) moins (\ref{eqv02}) moins (\ref{eqv03}) moins
(\ref{eqv04}) moins (\ref{eqv05}), on obtient~:
\begin{align}
\label{eqv10} M\ddot{z}=-2Mg+k(z_1+z_2+z_3+z_4)\;.
\end{align}

%En faisant (\ref{eqv04}) moins (\ref{eqv05}) on obtient~:
%$$\ddot{z}_3-\ddot{z}_4=\frac{R_3-R_4+k(z_4-z_3)}{m}+2l\ddot{\alpha}\;.$$

Soit :
$$F_1=\frac{k(z_1+z_2+z_3+z_4)}{M}\;,$$
$$F_2=\frac{L^2k(z_1+z_2-z_3-z_4)}{I_{\theta}}\;,$$
$$F_3=\frac{l^2k(z_1-z_2-z_3+z_4)}{I_{\alpha}}$$

En utilisant les \'equations (\ref{eqv08}), (\ref{eqv09}) et 
(\ref{eqv10}); (\ref{eqv02}) 
s'\'ecrit :
$$\ddot{z}_1 = 
\frac{R_1-kz_1}{m}-F_1-F_2-F_3\;.$$

De m\^eme, on trouve :
\begin{align*}
\ddot{z}_2 &= 
\frac{R_2-kz_2}{m}-F_1-F_2+F_3\;,\\
\ddot{z}_3 &= 
\frac{R_3-kz_3}{m}-F_{1}+F_2+F_3\;,\\
\ddot{z}_4 &= 
\frac{R_4-kz_4}{m}-F_1+F_2-F_3\;.
\end{align*}

\section{Discr\'etrisation des \'equations}

Pour calculer les trajectoires des corps, nous pouvons approximer
les \'equations diff\'erentielles par des
\'equations r\'ecurrentes, o\`u $h$ d\'esigne le pas de discr\'etisation en temps~:

\begin{align*}
z_1(t+h) &= 2z_1(t)-z_1(t-h)+h^2\left 
(\frac{R_1-kz_1(t)}{m}-F_1(t)-F_2(t)-F_3(t) \right ) \;,\\
z_2(t+h) &= 2z_2(t)-z_2(t-h)+h^2\left 
(\frac{R_2-kz_2}{m}-F_1(t)-F_2(t)+F_3(t)  \right )\;,\\
z_3(t+h) &= 2z_3(t)-z_3(t-h)+h^2\left 
(\frac{R_3-kz_3(t)}{m}-F_1(t)+F_2(t)+F_3(t)  \right )\;,\\
z_4(t+h) &= 2z_4(t)-z_4(t-h)+h^2\left 
(\frac{R_4-kz_4(t)}{m}-F_1(t)+F_2(t)-F_3(t)  \right )\;,\\
z(t+h) &= 2z(t)-z(t-h)+h^2\left (
F_1(t)-2g\right )\;,\\
\alpha (t+h) &= 2\alpha (t)-\alpha (t-h)+ \frac{h^2 F_3(t)}{l}\;,\\
\theta (t+h) &= 2\theta (t)-\theta (t-h)+ \frac{h^2 F_2(t)}{L}\;.
\end{align*}

\end{document}
