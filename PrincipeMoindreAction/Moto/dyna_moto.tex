\documentclass[a4paper,11pt]{amsart}

\usepackage[francais]{babel}
%\usepackage[pdftex]{graphicx}
%\uspackage{pstricks}
\usepackage{graphicx}

\newcommand{\dessin}[4]{
\begin{figure}[htb]
\centering
\includegraphics[scale= #2]{#1}
\caption{#3}
\label{#4}
\end{figure}}

\newtheorem{remarque}{Remarque} \newcommand{\AAA}{{\mathcal A}}
\newcommand{\BB}{{\mathcal B}} \newcommand{\CC}{{\mathcal C}}
\newcommand{\DD}{{\mathcal D}} \newcommand{\EE}{{\mathcal E}}
\newcommand{\FF}{{\mathcal F}} \newcommand{\GG}{{\mathcal G}}
\newcommand{\HH}{{\mathcal H}} \newcommand{\II}{{\mathcal I}}
\newcommand{\JJ}{{\mathcal J}} \newcommand{\KK}{{\mathcal K}}
\newcommand{\LL}{{\mathcal L}} \newcommand{\MM}{{\mathcal M}}
\newcommand{\NN}{{\mathcal N}} \newcommand{\OO}{{\mathcal O}}
\newcommand{\PP}{{\mathcal P}} \newcommand{\QQ}{{\mathcal Q}}
\newcommand{\RR}{{\mathcal R}} \newcommand{\SSS}{{\mathcal S}}
\newcommand{\TT}{{\mathcal T}} \newcommand{\UU}{{\mathcal U}}
\newcommand{\VV}{{\mathcal V}} \newcommand{\WW}{{\mathcal W}}
\newcommand{\XX}{{\mathcal X}} \newcommand{\ZZ}{{\mathcal Z}}
\newcommand{\bbR}{{\mathbb R}} \newcommand{\bbD}{{\mathbb D}}
\newcommand{\bbO}{{\mathbb O}} \newcommand{\bbS}{{\mathbb S}}
\newcommand{\bbE}{{\mathbb E}} \newcommand{\bbN}{{\mathbb N}}
\newcommand{\bbM}{{\mathbb M}} \newcommand{\bbV}{{\mathbb V}}
\newcommand{\bbC}{{\mathbb K}} \newcommand{\bbF}{{\mathbb F}}
\newcommand{\bbP}{{\mathbb P}}

\title{Etude du mouvement vertical d'une moto (tangage)}
\author{QUADRAT Quentin}

\begin{document}
\maketitle

\section{Pr\'esentation du mod\'ele}
Le v\'ehicule est mod\'elis\'e en 2D, par une carcasse repr\'esent\'e
par une barre de demi longueur $l$ et de masse ponctuelle $M$ \`a
laquelle sont accroch\'ees deux roues (de rayon $R$ et de masse $m$)
par des ressorts. On note $u(t)$ la l'altitude du sol, $y(t)$ est
l'altitude de la carcasse, $y_1(t)$ et $y_2(t)$ les allongements des
deux ressorts, $\theta$ le degr\'e d'inclinaison du v\'ehicule.  On
note $g$ la gravit\'e, $\theta$ le degr\'e de penchement du v\'ehicule
(figure \ref{dyna}).

\dessin{dyna.epsf}{0.5}{Mod\'elisation de la voiture}{dyna}

\section{Calcul des \'equations du mouvement}
Les forces qui sont en jeux sont : la pesanteur des masses (roue et
carcasse), la r\'epulsion du sol sur les roues et la force des ressorts.

L'\'energie cin\'etique de la voiture est : $$\frac{M\dot{y}^2}{2}\;.$$

L'\'energie potentielle de la voiture  est : $$Mgy\;.$$

L'\'energie cin\'etique verticale de la roue de devant est
$(\dot{y_2}+l\cos\theta\dot{\theta}+\dot{y})^2\;, $
%On approximera ($\theta$ petit) l'\'energie cin\'etique verticale de la roue de devant par:
que l'on approxime en faisant l'hypoth\'ese $\theta$ petit par :
$$1/2m(\dot{y}_1+\dot{y}+l\dot{\theta})^2\;.$$

De m\^eme, l'\'energie cin\'etique verticale de la roue de derri\`ere est :
$$1/2m(\dot{y}_2+\dot{y}-l\dot{\theta})^2\;.$$


L'\'energie potentielle due \`a la pesanteur des deux roues est :
$$mg(2y+y_2+y_1)\;.$$

L'\'energie potentielle du ressort de la roue avant est :
$$1/2ky_1^2\;.$$


L'\'energie potentielle du ressort de la roue arri\`ere est :
$$1/2ky_2^2\;.$$

L'\'energie potentielle de r\'eaction du sol sur la roue de devant est
:
$$1/2([u(x+l)-(y_1+y+l\theta-R)]^+)^2\;,$$
o\`u $A^+$ d\'esigne la partie positive de $A$.

L'\'energie potentielle de r\'eaction du sol sur la roue de derri\`ere est :
$$1/2([u(x-l)-(y_2+y-l\theta-R)]^+)^2\;.$$


L'action \`a minimiser vaut donc :
\begin{align*}
\AAA=1/2\int &\big\{ M\dot{y}^2+m(\dot{y}_1+\dot{y}+l\dot{\theta})^2+
m(\dot{y}_2+\dot{y}-l\dot{\theta})^2 \\
&-ky_1^2-ky_2^2-2Mgy-2mg(2y+y_2+y_1) \\
&-([u(x+l)-(y_1+y+l\theta-R)]^+)^2\\
&-([u(x-l)-(y_2+y-l\theta-R)]^+)^2 \big\} dt\;
\end{align*}

On trouve un syst\`eme
d'\'equation diff\'erentielle o\`u les inconnues sont : trois
altitudes (une pour la carcasse, une pour chaque roue)
et enfin l'inclinaison de la carcasse ($\theta$).

On note :
$$R_1 =[u(x+l)-(y_1+y+l\theta-R)]^+\;,$$
$$R_2 =[u(x-l)-(y_2+y-l\theta-R)]^+\;,$$

On a :
\begin{align}
(M+2m)\ddot{y}+m\ddot{y}_1+m\ddot{y}_2 & = -2g(2m+M)+R_1+R_2\label{eqy}\;,\\
m(\ddot{y}_1+\ddot{y}+l\ddot{\theta}) &  = -ky_1-gm+R_1\label{eq1}\;,\\
m(\ddot{y}_2+\ddot{y}-l\ddot{\theta}) & = -ky_2-gm+R_2\label{eq2}\;,\\
m(\ddot{y}_1-\ddot{y}_2+2l\ddot{\theta}) & = R_1-R_2\label{eqo}\;.
\end{align}

En faisant  (\ref{eqo}) plus (\ref{eq2}) moins (\ref{eq1}) on obtient
$0 =k(y_1-y_2)$
et donc $y_1=y_2.$

En faisant  (\ref{eqy}) moins (\ref{eq1}) moins (\ref{eq2}) on obtient :
\begin{equation}
M\ddot{y}=-gM+2ky_1\;.\label{eqyy}
\end{equation}

L'equation (\ref{eqo}) donne alors :
\begin{equation}
2ml\ddot{\theta}= R_1-R_2\;. \label{eqoo}
\end{equation}

Puis, (\ref{eq1}) moins $\frac{m}{M}$(\ref{eqyy}) moins $\frac{1}{2}$(\ref{eqoo})
donne :
\begin{equation}
m\ddot{y}_1 = -ky_1(1+\frac{2m}{M})+\frac{R_1+R_2}{2}\;.
\end{equation}

%Enfin, (\ref{eq2}) moins $\frac{m}{M}$(\ref{eqyy}) moins $\frac{1}{4}$(\ref{eqoo})
%donne :
%\begin{equation}
%m\ddot{y}_2 = \frac{k(y_1-y_2)}{2}+R_2+\frac{mg(2m+M)}{M}\;.
%\end{equation}

Finalement, on obtient, le syst\`eme alg\'ebriquo-dif\'erentiel suivant :
\begin{align*}
%M\ddot{y} &=-gM+2ky_1\;,\\
%m\ddot{y}_1 &= -ky_1(1+\frac{2m}{M})+\frac{R_1+R_2}{2}\;,\\
%m\ddot{y}_2 &= -ky_2(1+\frac{2m}{M})+\frac{R_1+R_2}{2}\;,\\
%2ml\ddot{\theta} &= (R_1+R_2)\;.
\ddot{y} &=-2g+\frac{2ky_1}{M}\;,\\
\ddot{y}_1 &= -ky_1(\frac{1}{m}+\frac{2}{M})+\frac{R_1+R_2}{2m}\;,\\
y_2 &= y_1\;,\\
\ddot{\theta} &= \frac{R_1+R_2}{2ml}\;.
\end{align*}

%En notant :
%$$P=\begin{bmatrix}
%M+2m & m & m & 0 \\
%m & m & 0 & ml \\
%m & 0 & m & -ml \\
%0 & m & -m & 2ml \\
%\end{bmatrix}\;,$$
%le syst\`eme differentiel s'\'ecrit :
%$$\begin{bmatrix}
%\ddot{y} \\ \ddot{y_1} \\ \ddot{y_2} \\ \ddot{\theta}
%\end{bmatrix}
%= P^{-1}
%\begin{bmatrix}
%-g(2m+M) + R_1+R_2 \\
%-gm -ky_1+R_1\\

%-gm -ky_2+R_2\\
%R_1+R_2\\
%\end{bmatrix}\;.
%$$
%\subsection{Discr\'etisation des \'equations differentielles}

\section{Discr\'etisation des \'equations diff\'erentielles}
Pour calculer les trajectoires des corps, nous pouvons approximer
les \'equations diff\'erentielles par des
\'equations r\'ecurrentes, o\`u $h$ d\'esigne le pas de discr\'etisation en temps~:

\begin{align*}
y(t+h) &=2y(t)-y(t-h)+h^2\left(-2g+\frac{2ky_1}{M}\right)\;,\\
y_1(t+h) &=2y_1(t)-y_1(t-h)+h^2\left(-ky_1(\frac{1}{m}+\frac{2}{M})+\frac{R_1+R_2}{2m}\right)\;,\\
\theta(t+h) &=2\theta(t)-\theta(t-h)+h^2\left(\frac{R_1+R_2}{2ml}\right)\;.
\end{align*}

\end{document}
