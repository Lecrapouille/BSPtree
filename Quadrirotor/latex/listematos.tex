%=======================================================================%
\newpage
\part{Annexes C}

%====================================================================
\section{Liste du mat\'eriel}\label{matos}

\begin{tabular}{|c|c|c|c|}\hline

\begin{minipage}{4cm}
\vspace{1mm}
\centerline{Nom}
\vspace{1mm}
\end{minipage} &

\begin{minipage}{1.5cm}
\vspace{1mm}
\centerline{Quantit\'e}
\vspace{1mm}
\end{minipage} &

\begin{minipage}{1.5cm}
\vspace{1mm}
Prix unitaire (euro)
\vspace{1mm}
\end{minipage} &

\begin{minipage}{8cm}
\vspace{1mm}
\centerline{Remarques}
\vspace{1mm}
\end{minipage}\\ \hline

\begin{minipage}{4cm}
\vspace{1mm}
Moteur \'electrique \`a broches
\href{http://shop.graupner.de/webuerp/servlet/AA?wgr=821}{Graupner
Micro Speed 6V}
\vspace{1mm}
\end{minipage} &

\begin{minipage}{1.5cm}
\vspace{1mm}
4
\vspace{1mm}
\end{minipage} &

\begin{minipage}{1.5cm}
\vspace{1mm}
5
\vspace{1mm}
\end{minipage} &

\begin{minipage}{8cm}
\vspace{1mm}
Poids: 16 g. Pouss\'ee max.: 60g en lui fournissant
8.5V et 1.2A. Equivalent au moteur
\href{http://www.gws.com.tw/english/product/powersystem/edp.htm}{GW/EDP-50}.
\vspace{1mm}
\end{minipage}\\ \hline

\begin{minipage}{4cm}
\vspace{1mm}
H\'elice \`a pas fixe
\vspace{1mm}
\end{minipage} &

\begin{minipage}{1.5cm}
\vspace{1mm}
4
\vspace{1mm}
\end{minipage} &

\begin{minipage}{1.5cm}
\vspace{1mm}
2
\vspace{1mm}
\end{minipage} &

\begin{minipage}{8cm}
\vspace{1mm}
Les h\'elices \`a pas invers\'e \'etant introuvables on est
oblig\'e d'utiliser quatre \`a pas fixe.
\vspace{1mm}
\end{minipage}\\ \hline

\begin{minipage}{4cm}
\vspace{1mm}
Gyroscope py\'ezo-\'electrique \href{}{GWS PG03}
\vspace{1mm}
\end{minipage} &

\begin{minipage}{1.5cm}
\vspace{1mm}
3
\vspace{1mm}
\end{minipage} &

\begin{minipage}{1.5cm}
\vspace{1mm}
25 sur Ebay
\vspace{1mm}
\end{minipage} &

\begin{minipage}{8cm}
\vspace{1mm}
Ce qui nous int\'eresse, ici, est uniquement le capteur gyro Gyrostar
\href{http://www.murata.com/catalog/s42e3.pdf}{ENC-03}. C'est la
m\'ethode la plus \'economique (en prix et en facililt\'e des
soudures) d'avoir un ENC-03.
\vspace{1mm}
\end{minipage}\\ \hline

\begin{minipage}{4cm}
\vspace{1mm}
Acc\'el\'erom\`etre 2 axes ADXL320
\vspace{1mm}
\end{minipage} &

\begin{minipage}{1.5cm}
\vspace{1mm}
1
\vspace{1mm}
\end{minipage} &

\begin{minipage}{1.5cm}
\vspace{1mm}
5
\vspace{1mm}
\end{minipage} &

\begin{minipage}{8cm}
\vspace{1mm}
C'est le capteur qui indique la verticale. Vu sa petite taille (4x4
mm) il est difficile \`a souder manuellement.
\vspace{1mm}
\end{minipage}\\ \hline

\begin{minipage}{4cm}
\vspace{1mm}
T\'el\'em\`etre GP2Y0D340
\vspace{1mm}
\end{minipage} &

\begin{minipage}{1.5cm}
\vspace{1mm}
1
\vspace{1mm}
\end{minipage} &

\begin{minipage}{1.5cm}
\vspace{1mm}
15
\vspace{1mm}
\end{minipage} &

\begin{minipage}{8cm}
\vspace{1mm}
Passe de l'\'etat 0 \`a 1 lorsqu'un obstacle est \`a moins de 40 cm.
\href{http://info.hobbyengineering.com/specs/SHARP-gp2y0d340_j.pdf}{Datasheet}.
\vspace{1mm}
\end{minipage}\\ \hline

\begin{minipage}{4cm}
\vspace{1mm}
Microcontr\^oleur PIC 16F876A
\href{http://ww1.microchip.com/downloads/en/DeviceDoc/39582b.pdf}{(datasheet)}
\vspace{1mm}
\end{minipage} &

\begin{minipage}{1.5cm}
\vspace{1mm}
2
\vspace{1mm}
\end{minipage} &

\begin{minipage}{1.5cm}
\vspace{1mm}
4
\vspace{1mm}
\end{minipage} &

\begin{minipage}{8cm}
\vspace{1mm}
On peut choisir entre 1 dsPIC ou 2 PIC 16F876A vu que ...
\vspace{1mm}
\end{minipage}\\ \hline

\begin{minipage}{4cm}
\vspace{1mm}
Microcontr\^oleur dsPIC \href{}{30F3011}
\vspace{1mm}
\end{minipage} &

\begin{minipage}{1.5cm}
\vspace{1mm}
1
\vspace{1mm}
\end{minipage} &

\begin{minipage}{1.5cm}
\vspace{1mm}
5
\vspace{1mm}
\end{minipage} &

\begin{minipage}{8cm}
\vspace{1mm}
... le dsPIC a quatre sorties PWM alors que le PIC en
a que deux.
\vspace{1mm}
\end{minipage}\\ \hline

\begin{minipage}{4cm}
\vspace{1mm}
Programmeur (ds)PIC
\vspace{1mm}
\end{minipage} &

\begin{minipage}{1.5cm}
\vspace{1mm}
1
\vspace{1mm}
\end{minipage} &

\begin{minipage}{1.5cm}
\vspace{1mm}
30 sur Ebay
\vspace{1mm}
\end{minipage} &

\begin{minipage}{8cm}
\vspace{1mm}
On peut se construire
son propre programmeur
\href{http://www.jdm.homepage.dk/newpic.htm}{JDM} PIC.
\vspace{1mm}
\end{minipage}\\ \hline

\begin{minipage}{4cm}
\vspace{1mm}
Convertisseur USB/port s\'erie (ou un simple c\^able s\'erie)
\vspace{1mm}
\end{minipage} &

\begin{minipage}{1.5cm}
\vspace{1mm}
1
\vspace{1mm}
\end{minipage} &

\begin{minipage}{1.5cm}
\vspace{1mm}
30
\vspace{1mm}
\end{minipage} &

\begin{minipage}{8cm}
\vspace{1mm}
Pour la communication avec le PC.
\vspace{1mm}
\end{minipage}\\ \hline

\begin{minipage}{4cm}
\vspace{1mm}
Composants \'electroniques
\vspace{1mm}
\end{minipage} &

\begin{minipage}{1.5cm}
\vspace{1mm}

\vspace{1mm}
\end{minipage} &

\begin{minipage}{1.5cm}
\vspace{1mm}

\vspace{1mm}
\end{minipage} &

\begin{minipage}{8cm}
\vspace{1mm}
(AOP, transistors, Mosfet, diodes de roue libre, resistances de
puissances, max232, connectique port s\'erie ...).
\vspace{1mm}
\end{minipage}\\ \hline

\begin{minipage}{4cm}
\vspace{1mm}
Tube de carbone creux (diam\`etre: 6mm; longueur: $>=$
40cm)
\vspace{1mm}
\end{minipage} &

\begin{minipage}{1.5cm}
\vspace{1mm}
1
\vspace{1mm}
\end{minipage} &

\begin{minipage}{1.5cm}
\vspace{1mm}
?
\vspace{1mm}
\end{minipage} &

\begin{minipage}{8cm}
\vspace{1mm}
Servent \`a construire les axes de l'h\'elicopt\`ere.
\vspace{1mm}
\end{minipage}\\ \hline

\begin{minipage}{4cm}
\vspace{1mm}
Plaque d'aluminium (10x10cm)
\vspace{1mm}
\end{minipage} &

\begin{minipage}{1.5cm}
\vspace{1mm}
1
\vspace{1mm}
\end{minipage} &

\begin{minipage}{1.5cm}
\vspace{1mm}
?
\vspace{1mm}
\end{minipage} &

\begin{minipage}{8cm}
\vspace{1mm}
Sert \`a fixer (unifier) les deux axes de l'h\'elicopt\`ere.
\vspace{1mm}
\end{minipage}\\ \hline

\begin{minipage}{4cm}
\vspace{1mm}
Tube rond de laiton 2x0,3-1m
\vspace{1mm}
\end{minipage} &

\begin{minipage}{1.5cm}
\vspace{1mm}
1
\vspace{1mm}
\end{minipage} &

\begin{minipage}{1.5cm}
\vspace{1mm}
?
\vspace{1mm}
\end{minipage} &

\begin{minipage}{8cm}
\vspace{1mm}
Sert \`a construire les pieds qui rentreront dans les tubes de
carbone.
\vspace{1mm}
\end{minipage}\\ \hline

\begin{minipage}{4cm}
\vspace{1mm}
Total :
h\'elico 4 h\'elices
\vspace{1mm}
\end{minipage} &

\begin{minipage}{1.5cm}
\vspace{1mm}
1
\vspace{1mm}
\end{minipage} &

\begin{minipage}{1.5cm}
\vspace{1mm}
170
\vspace{1mm}
\end{minipage} &

\begin{minipage}{8cm}
\vspace{1mm}
C'est le prix
d'un X-UFO
\vspace{1mm}
\end{minipage}\\ \hline

\end{tabular}


\subsection*{Remarques}
Les anciens moteurs \'electrique \`a balais a courant continu
\'etaient
\href{http://shop.graupner.de/webuerp/servlet/AA?wgr=821}{Graupner
  Speed 195}. Ils pesent 7 grammes et leur pouss\'ee max. est de 30g
en leur fournissant ??V et ??A. Equivalent au moteur
\href{http://www.gws.com.tw/english/product/powersystem/edp.htm}{GW/EDP-05}. Ils
ont ete remplace par les Micro-Speed plus rapides.


\section{PIC}
%\section{Trames de donn�es}

copier coller le tableau des trames html.

Expliquer pourkoi ANC sur 8 bits c mieux que 10 bits : -- la precision
on s'en fout (afficher plot zoom sur une courbe) -- allourdi le
npmbres de bits a envoyer.

AJOUTER photo Comm inter composants

DIRE PWM 5khz (vitesse max PIC) mais que si on peut il faut augmenter
la frequence car quand on filtre avec une bobine on a $1/Cs$ et quand
$s$(=5khz) TRES grand le resultat tend vers 0.
