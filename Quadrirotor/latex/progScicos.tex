%=======================================================================%
\newpage
\section{Programmer Scicos}

%====================================================================
\subsection*{La fonction de calcul Scicos}

La fonction de calcul est la fonction C d�finie par l'utilisateur. La
seule contrainte pour quelle soit accept\'ee est qu'elle doit prendre
en param\`etre~:
\begin{itemize}
\item[$\bullet$] un flag {\tt flag} qui indique l'\'etat du bloc
  (parmis les 6 presents : debut, fin de la simulation, execution du
  calcul, ...);
\item[$\bullet$] une structure Scicos {\tt block} contenant le flux de
  donn\'ees.  Les champs les plus interessants sont :
\begin{itemize}
\item {\tt ipar} est un vecteur de parametres de type entier associes au block
  diagramme (definis par la fonction d'interface);
\item {\tt nipar} indique la taille de ce vecteur;
\item {\tt inptr} sont les donnees vectorielles fournies en entree et
  en sortie (a voir comme un bus de donnees en electronique);
\item {\tt insz} et {\tt outsz} sont les tailles des donnees.
\end{itemize}
\end{itemize}

Un exemple simple d'une fonction est :
\begin{verbatim}
void	foobar(scicos_block *block, int flag)
{
   switch (flag)
   {
     case 4:
     /* Code pour l'initialisation de la simulation */
     sciprint("Debut de la simul, j'ai %d param", block->ipar);

     case 1:
     /* Code pour le calcul des sorties */
     block->outptr[i][j] = ... block->inptr[i][j];

     case 5:
     /* Code pour la terminaison de la simulation */

     default:
     break;
   }
}
\end{verbatim}

%====================================================================
\subsection*{La fonction d'interface}
%====================================================================
La deuxi\`eme fonction est \'ecrite en langage Scilab. Elle permet de
d\'efinir comment le bloc doit interagir avec l'IHM, sa g\'eom\'etrie,
son nombre d'entr\'ees, de sorties pour les donnees et les evenements,
la d\'efinition de la fen\^etre de configuration lorsqu'un utilisateur
clique dessus, et le nom de la fonction de calcul.

%====================================================================
\subsection*{Le port serie Posix}
%====================================================================

%====================================================================
\subsection*{Edition de liens avec Scilab}
%====================================================================
L'exemple donne en annexes B decrit comment faire directement dans
l'IHM de Scilab. et d'automatiser la chose pour les utilisations
futures de Scilab.

%====================================================================
\subsection*{Automatiser des t�ches Scilab}
%====================================================================
Compiler une fois pour toute la libserie pour Scilab~:
\begin{verbatim}
    cd chemin_ou_se_trouve_les fichiers
)    exec('builder.sce');
\end{verbatim}

Le fichier {\tt scilab.star} dans le repertoire {\tt \$SCI} permet de
(configurer ??) le Scilab. On peut le modifier directement mais le
plus simple est de creer un fichier {\tt .scilab} dans le dossier {\tt
  .Scilab/version\_de\_scilab}

\begin{verbatim}
    cd chemin_ou_se_trouve_les fichiers
    load lib;
    exec loader.sce;
\end{verbatim}

et eventuellement la ligne :
\begin{verbatim}
    scicos Balancoire13quick.cos;
\end{verbatim}

%====================================================================
\subsection*{Couleur du Multi-plot}
%====================================================================

TODO: mettre beau plot

\subsubsection*{Plot 1}

\begin{tabular}{|p{1.5cm}|p{5cm}|p{2cm}|p{1.5cm}|} \hline
Num\'ero d'entr\'ee & Nom & Couleur & Num\'ero couleur \\ \hline\hline
1 & Consigne inclinaison & Noir & 1 \\ \hline
2 & Acc\'elerometre X & Vert & 3 \\ \hline
3 & PWM Moteur 1 & Bleu & 2 \\ \hline
4 & PWM Moteur 2 & Violet & 6 \\ \hline
\end{tabular}

\subsubsection*{Plot 2}
\begin{tabular}{|p{1.5cm}|p{5cm}|p{2cm}|p{1.5cm}|} \hline
Num\'ero d'entr\'ee & Nom & Couleur & Num\'ero couleur \\ \hline \hline
1 & Courant moteur 1 & Bleu clair & 4 \\ \hline
2 & Courant moteur 1 filtr\'e & Bleu & 2 \\ \hline
3 & Consigne courant moteur 1 & Noir & 1 \\ \hline
4 & Consigne courant moteur 2 & Violet & 6 \\ \hline
5 & Courant moteur 2 filtr\'e & Rouge & 5 \\ \hline
6 & Consigne courant moteur 2 & Noir & 1 \\ \hline
\end{tabular}

\subsubsection*{Plot 3}
\begin{tabular}{|p{1.5cm}|p{5cm}|p{2cm}|p{1.5cm}|} \hline
Num\'ero d'entr\'ee & Nom & Couleur & Num\'ero couleur \\ \hline \hline
1 & Gyroscope Y & Vert & 3 \\ \hline
2 & Gyroscope Y filtr\'e & Rouge & 5 \\ \hline
3 & d\'eriv\'ee acc\'el\'erom\`etre X & Bleu & 2 \\ \hline
\end{tabular}

\subsubsection*{Plot 4}
\begin{tabular}{|p{1.5cm}|p{5cm}|p{2cm}|p{1.5cm}|} \hline
Num\'ero d'entr\'ee & Nom & Couleur & Num\'ero couleur \\ \hline \hline
1 & Altimetre & Noir & 1 \\ \hline
2 & Retard Scilab &  &  \\ \hline
3 & Retard PIC 1 & &  \\ \hline
3 & Mode PIC 1 & &  \\ \hline
\end{tabular}

%====================================================================
%\subsection*{D�termination de la fonction non lin�aire i2PWM}
%====================================================================

%% On envoie une consigne PWM et on mesure le courant du moteur gr\^ace
%% \`a une resistance de puissance (comme l'explique la section
%% \ref{puis}). Comme la tension moyenne et le courant resultant sont lus
%% sur un convertisseur analogique 10 bits d'un PIC 16F876A, les valeurs
%% sont donc comprises entre 0 et 1023.

%% Conversion : PWM = 1024 == tension max.

\begin{tabular}{|*{6}{c|}} \hline
Consigne PWM & Exp. 1 & Exp. 2 & Exp. 3 & Exp. 4 & Exp. 5\\ \hline \hline
1012 & 150 & 163 & 165 &  & 295 \\ \hline
912 & 150 & 161 & 164 &  & 283 \\ \hline
812 & 150 & 158 & 163 & 765 & 280 \\ \hline
712 & 150 & 161 & 158 & 311 & 276 \\ \hline
612 & 150 & 153 & 156 & 282 & 267 \\ \hline
512 & 146 & 148 & 151 & 270 & 257 \\ \hline
412 & 137 & 140 & 143.5 & 254 & 240 \\ \hline
312 & 127 & 125.5 & 128.5 & 230 & 215 \\ \hline
212 & 115 & 108 & 102.5 & 200 & 190 \\ \hline
112 & 80 & 66 & 60 &  & 130 \\ \hline
12  & 20 & 11 & 9.75 &  & 9 \\ \hline
\end{tabular}\\[1mm]

Note~:
\begin{description}
\item[Exp\'erience 1] Moteur Speed 195, r\'esistance de
  puissance 1.8$\Omega$, tension fournie: +6V.
\item[Exp\'erience 2] Moteur Micro Speed N. 1, r\'esistance de
  puissance 1.8$\Omega$, tension fournie: +6V.
\item[Exp\'erience 3] Moteur Micro Speed N. 2, r\'esistance de
  puissance 1.8$\Omega$, tension fournie: +6V.
\item[Exp\'erience 4] Moteur Micro Speed, r\'esistance de puissance
  1$\Omega$, tension fournie: +9V.  L'experience a ete arret\'ee car
  le moteur risquait la destruction.
\item[Exp\'erience 5] Moteur Micro Speed, r\'esistance de
  puissance 1$\Omega$, tension fournie: +8.5V.
\end{description}
