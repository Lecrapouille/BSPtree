%=======================================================================%
\newpage
\section{Usinage des pieces}\label{matos}
%-----------------------------------------------------------------------%

%=======================================================================%
\subsection{Patrons}
%=======================================================================%
Gr�ce � une plaque d'alumiun, il est n�cessaire d'usiner, � partir des
deux patrons donn�s figures ci-dessous, quatre attaches de moteurs et
deux attaches pour la croix.\\[0.1cm]
\begin{minipage}[b]{.45\linewidth}
\centering\epsfig{figure=img/patron_attachemoteur, width=\linewidth}
\caption{Patron des attaches moteurs.}\label{patronattache}
\end{minipage}\hspace{2cm}
\begin{minipage}[b]{.3\linewidth}
\centering\epsfig{figure=img/patron, width=\linewidth}
\caption{Patron de la croix.}\label{patroncroix}
\end{minipage}

\subsubsection*{Patron croix}
Dans le patron fig. \ref{patroncroix}, trois formes g�om�triques sont
visibles~:

%\dessin{img/patron}{0.6}{Patron de la croix.}{patroncroix}
\begin{itemize}
\item[$\bullet$] deux petits triangles (en pointill�) sont des parties
  � �liminer;
\item[$\bullet$] deux grands triangles qui sont des parties � coller
  avec les deux autres triangles de la deuxi�me plaque;
\item[$\bullet$] un rectangle de longueur $L$ arbitraire et de largeur
  $\pi\;D$ (o� $D$ doit �tre le diam�tre des tubes de l'h�licopt�re
  donc, ici, 0.6 mm) va s'enrouller autour d'une des deux tiges/axes
  de l'appareil.
\end{itemize}

A l'aide d'un �tau, les grands triangles sont pli�s le long du segment
du rectangle et on enroulle les rectangles sur les axes de
l'helico. Avec de la colle super forte on colle l'ensemble. Avec des
pinces � linge on sert les triangles afin de bien les coller
ensemble. Le fait de limer la surface des triangles aide � mieux
coller les 2 plaques entre elles (en �liminant la couche qui prot�ge
l'aluminium). La colle prend en un jour.

\subsubsection*{Patron attache moteur}
La ligne verticale de longueur 1.88 ($\pi\;D$)
permet, comme dans la sous section \ref{sectoldattache}, de
s'enrouller autour du tube de carbone. Dans ce cas, les 2 lignes
horizontales se retrouver cote a cote. La nouveaute consiste a avoir
ajouter ces deux lignes horizontales permettent d'entourer le
moteur. Comme elle sont cote a cote, la surface entiere du moteur est
recouverte. D'ou une meilleure fixation de la colle. Une partie
facultative est le cache troue qui se rabat sur le haut du moteur.
