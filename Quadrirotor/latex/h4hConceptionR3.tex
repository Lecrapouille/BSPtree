\documentclass[a4paper, 11pt]{article}%{report}
\usepackage{fullpage}
\usepackage[latin1]{inputenc}
\usepackage[french]{babel}
\usepackage{amsmath}
\usepackage{graphicx}
\usepackage{picins}
\usepackage{hyperref}
%\usepackage{makeidx}
%\usepackage{showlabels}
\usepackage{epsfig}
\hypersetup{
backref=true,    %permet d'ajouter des liens dans...
pagebackref=true,%...les bibliographies
hyperindex=true, %ajoute des liens dans les index.
colorlinks=true, %colorise les liens
breaklinks=true, %permet le retour � la ligne dans les liens trop longs
urlcolor= blue,  %couleur des hyperliens
linkcolor= blue, %couleur des liens internes
bookmarks=true,  %cr�� des signets pour Acrobat
bookmarksopen=true, %si les signets Acrobat sont cr��s, les afficher compl�tement.
pdftitle={Projet H4H}, %informations apparaissant dans
pdfauthor={Quentin QUADRAT},     %dans les informations du document
pdfsubject={Mac OS X}          %sous Acrobat.
}

%%%%%%%%%%%%%%%%%%%%%%%%%%%%%%%%%%%%%%%%%%%%%%%%%%
\newcommand{\dessin}[4]{
\begin{figure}[htb]
\centering
\includegraphics[scale= #2]{#1}
\caption{#3}
\label{#4}
\end{figure}}

%%%%%%%%%%%%%%%%%%%%%%%%%%%%%%%%%%%%%%%%%%%%%%%%%%
\newcommand{\dessinsscaption}[2]{
\begin{figure}[htb]
\centering
\includegraphics[scale= #2]{#1}
\end{figure}}

%%%%%%%%%%%%%%%%%%%%%%%%%%%%%%%%%%%%%%%%%%%%%%%%%%
\newtheorem{remarque}{Remarque} \newcommand{\AAA}{{\mathcal A}}
\newcommand{\BB}{{\mathcal B}} \newcommand{\CC}{{\mathcal C}}
\newcommand{\DD}{{\mathcal D}} \newcommand{\EE}{{\mathcal E}}
\newcommand{\FF}{{\mathcal F}} \newcommand{\GG}{{\mathcal G}}
\newcommand{\HH}{{\mathcal H}} \newcommand{\II}{{\mathcal I}}
\newcommand{\JJ}{{\mathcal J}} \newcommand{\KK}{{\mathcal K}}
\newcommand{\LL}{{\mathcal L}} \newcommand{\MM}{{\mathcal M}}
\newcommand{\NN}{{\mathcal N}} \newcommand{\OO}{{\mathcal O}}
\newcommand{\PP}{{\mathcal P}} \newcommand{\QQ}{{\mathcal Q}}
\newcommand{\RR}{{\mathcal R}} \newcommand{\SSS}{{\mathcal S}}
\newcommand{\TT}{{\mathcal T}} \newcommand{\UU}{{\mathcal U}}
\newcommand{\VV}{{\mathcal V}} \newcommand{\WW}{{\mathcal W}}
\newcommand{\XX}{{\mathcal X}} \newcommand{\ZZ}{{\mathcal Z}}
\newcommand{\bbR}{{\mathbb R}} \newcommand{\bbD}{{\mathbb D}}
\newcommand{\bbO}{{\mathbb O}} \newcommand{\bbS}{{\mathbb S}}
\newcommand{\bbE}{{\mathbb E}} \newcommand{\bbN}{{\mathbb N}}
\newcommand{\bbM}{{\mathbb M}} \newcommand{\bbV}{{\mathbb V}}
\newcommand{\bbC}{{\mathbb K}} \newcommand{\bbF}{{\mathbb F}}
\newcommand{\bbP}{{\mathbb P}}
\newcommand{\www}{{\mathfrak w \;}} \newcommand{\fff}{{\mathfrak f \;}}
\newcommand{\nnn}{{\mathfrak n \;}} \newcommand{\aaa}{{\mathfrak a \;}}
\newcommand{\hhh}{{\mathfrak h \;}}
%%%%%%%%%%%%%%%%%%%%%%%%%%%%%%%%%%%%%%%%%%%%%%%%%%

%\title{Stabilisation d'un h�licopt�re 4 h�lices sur une rotule gr�ce � un ordinateur}
%\title{Stabilisation et pilotage d'un h�licopt�re � quatre h�lices}

\makeindex

\begin{document}
%\pagestyle{empty}
\title{\vskip -2cm
  \Large \sl \bf  Quentin QUADRAT\\
    Login: quadra\_q, UID: 17115, Promo: 2007 \\
    \vspace{0.5cm}\hrule
  \vskip 3.2cm
  {\Large CAHIER DE CONCEPTION -- r�vision \No 3}\\
  \today \\
  \vskip 1cm
  {\LARGE {\textsc{\'Etude et r�alisation d'un mod�le r�duit d'h�licopt�re � quatre h�lices}}}
  \vskip 2cm
  \dessinsscaption{img/titre}{0.8}
}

\date
{
\vspace{1cm}
      \begin{minipage}[b]{.2\linewidth}
\centering\epsfig{figure=img/epita, width=\linewidth}
\end{minipage}\hspace{1cm}
\begin{minipage}[b]{.3\linewidth}
\centering\epsfig{figure=img/gistr, width=\linewidth}
\end{minipage}\\
      %\large EPITA \\
  \large \textbf{�cole Pour l'Informatique et les Techniques
  Avanc�es} \\
}

\maketitle
\strut\thispagestyle{empty}
\setcounter{page}{0}
\newpage

\begin{quote}
\og \emph{La th�orie, c'est quand on sait tout et que rien ne fonctionne. La
pratique, c'est quand tout fonctionne et que personne ne sait
pourquoi. Ici, nous avons r�uni th�orie et pratique : Rien ne
fonctionne... et personne ne sait pourquoi !}\fg
\end{quote}

Albert Einstein
\newpage

\tableofcontents
\newpage

%=======================================================================%
\part*{Introduction}
\addcontentsline{toc}{part}{Introduction}
\section{Pr�sentation du projet}\label{intro}
%=======================================================================%
Ce document pr�sente la r�alisation d'un mod�le r�duit d'h�licopt�re �
quatre h�lices avec son banc d'essai �lectrom�\-canique.  Ce travail a
�t� r�alis� dans le cadre d'un projet de la sp�cialisation temps r�el
de l'EPITA.

Les caract�ristiques principales de l'h�licopt�re sont les suivantes.
Il est constitu� d'une croix en fibre de carbone assembl�e au moyen
d'une pi�ce en aluminium fa\c conn�e � la main dans une plaque de ce
m�tal. Quatre moteurs �lectriques � courant continu avec collecteur
qui sont attach�s sur la croix de fibre de carbone par des attaches en
aluminium fa\c conn�es, elles aussi, � la main. Dans chacun des tubes
creux en carbone sont ins�r�es des pattes en laiton sur lesquelles
peut reposer l'h�licopt�re. Sur la croix est attach�e une carte
�lectronique contenant les capteurs servant � la stabilisation de
l'appareil~: trois gyroscopes un axe, un acc�l�rom�tre deux axes, un
capteur de proximit� commutant � une quarantaine de centim�tre d'un
obstacle.  L'�lectronique de puissance, l'�lectronique de commande et
la source d'�nergie ne sont pas, pour l'instant,
embarqu�es. L'alimentation �lectrique est assur�e par une alimentation
de PC standard 12V, suivi d'un r�gulateur, qu'il a fallu r�aliser,
fournissant du 5V et du 8.5V avec une intensit� maximale de
5A. L'�lectronique de puissance consiste en quatre MOSFET command�s
par des entr�es PWM. L'�lectronique de commande consiste en deux
microcontr�leurs PIC 16F876 de Microchip \cite{microchip} reli�s entre
eux par un bus I$^2$C connect�s � un PC standard par un bus s�rie. Ces
deux microcontr�leurs font l'acquisition des donn�es et envoient au
MOSFET les commandes PWM calcul�es par le PC. La stabilisation est
calcul�e par le PC avec des logiciels de hauts niveaux Scilab-Scicos
\cite{Scilab,Scicos} qui sont les analogues libres du couple
Matlab-Simulink. Pour tester la stabilisation de l'appareil, un banc
d'essai, r�alis� en Lego, permet de maintenir et de limiter les
mouvements de l'h�licopt�re en lui laissant un nombre limit� de degr�s
de libert�.


% Les microcontr�leurs de l'h�licopt�re et
%l'ordinateur communiquent entre eux. Les donn�es �mises des diff�rents
%capteurs de l'h�licopt�re sont re�ues par l'ordinateur. Celui-ci
%effectue les calculs num�riques (loi de commandes, consignes) puis
%renvoie les valeurs au microcontr�leur qui r�alise les commandes de
%stabilisation. Les moyens de calcul embarqu�s permettront dans une
%deuxi�me phase de rendre autonome l'h�licopt�re en lui permettant de
%calculer lui m�me les lois de commande.

%% La stabilisation de l'h�licopt�re est calcul�e
%%   par un ordinateur non embarqu� avec des outils de hauts
%%   niveaux.  La communication entre des microcontr�leurs et
%%   l'ordinateur permet � ce dernier de recevoir les
%%   donn�es �mises des diff�rents capteurs de
%%   l'h�licopt�re et d'effectuer les calculs
%%   num�riques (loi de commandes, consignes) puis de renvoyer les
%%   valeurs au microcontr�leur qui r�alise les commandes de
%%   stabilisation. Les moyens de calcul embarqu�s permettront
%%   dans une deuxi�me phase de rendre autonome
%%   l'h�licopt�re en lui permettant de calculer lui
%%   m�me les lois de commande.

La construction de l'h�licopt�re fait appel � plusieurs sp�cialit�s
de l'ing�nierie � savoir~: -- construction m�canique, -- �lectronique,
-- informatique, -- automatique. Chacun de ces aspects seront discut�s
successivement dans la suite de ce document.


Des projets similaires existent d�j� sous la forme commerciale et sont
connus sous les noms de Draganflyer \cite{draganflyer}, Engager
\cite{engager}, X-UFO \cite{xufo}, Microdrones \cite{microdrones}. Il
existe �galement des projets r�alis�s par des �tudiants ou par des
passionn�s (r�f�rences \cite{xbird} � \cite{josej}).  En g�n�ral ces
h�licopt�res sont radio-command�s ce qui ajoute une difficult�
suppl�mentaire dans la conception mais ce qui simplifie aussi la
stabilisation car l'op�rateur humain est capable d'observer et de
compenser les d�rives des gyroscopes \cite{xbird}.  En g�n�ral ces
h�licopt�res poss�dent deux moteurs tournant dans un sens et deux
autres tournant dans l'autre afin de stabiliser le lacet. Parfois, sur
des mod�les de petites tailles, pour �viter l'inversion du sens de
rotation des moteurs, ce qui implique l'utilisation d'h�lices
propulsives difficiles � trouver, les moteurs sont inclin�s de fa\c
con � cr�er une pouss�e compensant le lacet due � la rotation des
moteurs (voir \cite{josej}). Nous adopterons ce point de vue faute
d'avoir pu trouver des h�lices propulsives de petites tailles.

%%=======================================================================%
%\section{Aper�u g�n�ral d'un h�licopt�re � quatre h�lices}\label{apercu}
%%=======================================================================%
%%\subsection{Qu'est ce qu'un h4h ?}
%%=======================================================================%

%Un h�licopt�re � quatre h�lices est une plate-forme volante comprenant
%une \href{meca}{partie m�canique} et une \href{electro}{partie
%  �lectronique}.

%La partie m�canique a la forme d'une croix (donc deux axes) sur
%laquelle est attach�e quatre moteurs.  En g�n�ral, afin de r�duire au
%maximum le lacet de l'appareil, deux moteurs tournent dans un sens et
%les deux autres dans le sens oppos�. On devra donc utiliser deux types
%d'h�lices. Parfois, sur des mod�les de petites tailles, pour �viter
%l'inversion du sens de rotation des moteurs, on incline les moteurs
%(voir
%\href{http://www.rcgroups.com/forums/showthread.php?s=05bcbf26fe1c60ab6f128e5e38fc88c5&t=297067&pp=15}{le
%  MicroHeli4 de JoseJ sur le forum Rcgroups}).

%La partie �lectronique assure la stabilit� de la plate-forme en
%contr�lant la vitesse des moteurs en fonction des donn�es d'une
%centrale inertielle et des consignes envoy�es par l'utilisateur
%(g�n�ralement par radio-commande). Elle peut �ventuellement
%communiquer avec un ordinateur non embarqu� au moyen d'un port s�rie �
%des fins de d�bogage, de chargement des programmes des
%microcontr�leurs. Voir la section \href{asservi}{Asservissement}.

%=======================================================================%
%\subsection{Description du comportement de l'h�licopt�re}
%=======================================================================%
%% Un h�licopt�re n'est pas un syst�me stable (deux moteurs d'une
%% m�me s�rie n'auront pas forc�ment le m�me comportement), il
%% faut l'asservir, gr�ce � la carte �lectronique coupl�e �
%% l'ordinateur, afin qu'il maintienne son attitude.

%% Si l'on suppose que la dynamique de l'h�licopt�re est
%% id�ale. Pour obtenir les comportements suivants, voici ce que l'on
%% doit faire~:
%% \begin{description}
%% \item[Mouvement stationnaire] Si on consid�re que les quatre moteurs
%%   tournent � la m�me vitesse et que tous les �l�ments sont
%%   identiques (h�lices, dimensions, �quilibre des masses), le
%%   mobile est stable sur ses 3 axes : il vole � plat
%%   et ne tourne pas sur son axe central car, � vitesse �gale pour
%%   les quatre h�lices, les couples de rotation g�n�r�s par
%%   paire de moteurs s'annulent. Pour cela, deux moteurs tournent
%%   dans un sens et les deux autres dans le sens oppos�. On devra donc
%%   utiliser deux types d'h�lices.
%% \item[Monter et descendre] Pour le faire monter ou descendre, il
%%   suffit d'augmenter ou de diminuer la puissance des quatre moteurs
%%   (tous ensemble). Comme tous les rotors tournent � la m�me
%%   vitesse, il n'y a pas de rotation horizontale.
%% \item[D�placement] Pour obtenir un mouvement de tangage ou roulis,
%%   (avancer vers l'avant par exemple), il suffit d'augmenter la vitesse
%%   du moteur arri�re et diminuer celle du moteur avant (dans les
%%   m�me proportions pour conserver la portance intacte) sans
%%   modifier la vitesse des moteurs lat�raux. On fera l'inverse pour
%%   pencher vers l'arri�re. Le roulis est obtenu avec les moteurs
%%   gauche et droit.
%% \item[Tourner] Pour le mouvement de lacet (rotation sur lui-m�me),
%%   il faut augmenter la vitesse d'une paire de moteur et diminuer
%%   d'autant celle de l'autre paire. Le sens du mouvement de lacet
%%   d�pendra du sens de rotation qu'on aura choisi pour les paires de
%%   moteurs.
%% \end{description}

%% Des figures illustrant le comportement g�n�ral de la dynamique
%% d'un h�licopt�re peuvent �tre trouv�es sur les liens
%% \cite{grzflyer, Latour, Noth}. Des articles \cite{helicodesign,
%%   helicopid} d�crivent le mod�le physique de l'appareil ainsi que
%% son contr�le avec les m�thodes (PID et LQG).

%=======================================================================%
\section{Carct�ristiques principales du projet}\label{contraintes}
%=======================================================================%

\begin{description}
\item[Robustesse � son environnement.] L'appareil est
    pr�vu pour fonctionner en int�rieur, avec un minimum
    de perturbation atmosph�rique.

\item[Contraintes de dimension et de poids de
   l'h�licopt�re.]  La plate-forme se veut �tre de
   dimension et de poids le plus r�duit possible (envergure
   inf�rieure a 20 cm, poids en dessous des 200 g). On a choisi de ne
   pas utiliser de cartes d'acquisitions � cause de leur poids et de
   leur prix. L'acquisition des donn�es est r�alis�e par les deux
   microcontr�leur PIC.

%% (type mini cartes m�res
%%    embarqu�es) est prohib�e car elles sont trop
%%    excessives (poids, volume, puissance de calculs) pour ce projet. La
%%    partie �lectronique doit donc �tre fabriqu�e et
%%    adapt�e � l'h�licopt�re. Elle doit
%%    galement avoir une interface de communication avec l'ordinateur
%%    (pour le d�bugage, chargement de programmes).

\item[Communication avec un ordinateur non
  embarqu�.] L'h�licopt�re communique avec un
  ordinateur non embarqu� au moyen d'un port s�rie.  Le
  microcontr�leur envoie les donn�es de la centrale
  inertielle � l'ordinateur.  Ce dernier fait les calculs
  flottants de stabilisation et les communique au
  microcontr�leur embarqu�.

\item[Source d'�nergie.] L'h�licopt�re ne
  dispose pas de batterie, il est aliment� une source
  d'�nergie au sol au moyen de fils �lectriques. Ce qui
  permet une �conomie de poids importante et une autonomie
  beaucoup plus grande mais qui implique l'existence
  des fils reliant l'h�licopt�re au sol. L'embarquement de la source
  d'�nergie �lectrique pose en effet un gros probl�me, car les plus petits moteurs
  consommant de l'ordre de 10W  chacun l'autonomie est souvent tr�s r�duite. Certains
  mod�les du commerce peuvent avoir une autonomie de quelques
  minutes seulement.

%%  {\tt Contrainte de l'ordinateur et de l'IHM.} L'interface homme
%%    machine permet de simuler la plate-forme, puis de la
%%    contr�ler r�ellement tout en visualisant ses
%%    �tats. L'ordinateur sera un mod�le standard (PC,
%%    Macintosh), avec �ventuellement un OS temps r�el dur
%%    ou mou mais il devra �tre suffisamment r�actif pour
%%    assurer son r�le de gestion de la stabilit�.
\end{description}



%\part{Travail r�alis�}\label{apercu}

%=======================================================================%
\section{Avancement du projet}
%=======================================================================%
\subsection{Premi�re soutenance (d�but septembre)}
%=======================================================================%
La premi�re soutenance a pr�sent� l'h�licopt�re avec seulement deux
moteurs sur un banc d'essai qui permettait un seul degr� de libert�~:
le tangage.

L'h�licopt�re avait un de ses axes maintenus par le banc autour duquel
il pouvait tourner. Il se comportait comme une balan�oire (photo
\ref{v1}) o� les deux moteurs commandaient l'inclinaison de la balan\c
coire.  Cette inclinaison �tait acquise par l'acc�l�rom�tre gr�ce � un
PIC, �tait communiqu�e, par le port s�rie, au PC qui calculait la
sortie PWM � envoyer aux moteurs. Cette commande �tait fournie au PIC
par l'interm�diaire du port s�rie et le PIC mettait en forme, sur ses
deux sorties PWM, le signal � envoyer au MOSFET, commandant ainsi deux
moteurs.

Sur ce premier prototype, seul l'acc�l�rom�tre �tait embarqu�.  Les
fils liant la carte d'essai � l'h�licopt�re �taient assez rigides.
Ils se comportait comme des ressorts non-lin�aires, difficilement
mod�lisables, tendant � maintenir l'h�licopt�re dans une position plus
ou moins inclin�e contribuant � la stabilit� de l'appareil.\\[3mm]
\begin{minipage}[b]{.45\linewidth}
\centering\epsfig{figure=img/balancoire1, width=\linewidth}
\caption{Prototype 1}\label{v1}
\end{minipage}\hspace{6mm}
\begin{minipage}[b]{.5\linewidth}
\centering\epsfig{figure=img/balancoire2, width=\linewidth}
\caption{Prototype 2}\label{v2}
\end{minipage}

Les courants dans les moteurs n'�taient pas observ�s, la boucle de
r�gulation �laborait la commande PWM � partir de la seule donn�e
d'inclinaison en calculant un PID r�gl� empiriquement.  La poursuite
des consignes d'inclinaison �tait suivie avec une qualit� jug�e pas
compl�tement satisfaisante.


Un deuxi�me prototype a �t� ensuite r�alis� (photo \ref{v2}). Les fils
ont �t� remplac�s par des fils plus souples et plus longs. Une carte
�lectronique contenait en plus de l'acc�l�rom�tre, un gyroscope un axe
et un capteur de proximit� infrarouge.

Mont� sur le m�me banc, l'h�licopt�re avait deux positions possibles~:
-- l'une stable o� le centre de gravit� est au dessous de l'axe de
rotation, maintenu par le banc; -- l'autre, instable en absence de
pouss�e des moteurs, dans laquelle le centre de gravit� est au dessus
de l'axe de rotation.  Il est beaucoup plus facile de commander
l'h�licopt�re dans la premi�re position au moins dans les situations
de faible pouss�e des moteurs.

Afin d'am�liorer la qualit� de la r�gulation les courants ont �t�
observ�s.  La commande �tait alors r�alis�e gr�ce � deux boucles~: --
une premi�re boucle interne poursuit une consigne de courant en
agissant sur la commande PWM, -- une deuxi�me boucle poursuit
l'inclinaison en envoyant les consignes de courant � la deuxi�me
boucle. Malgr� la sophistication de cette r�gulation, recommand�e par
les sp�cialistes de ces probl�mes pour �largir la bande passante de la
r�gulation donc la r�activit� de la commande, l'am�lioration observ�e
n'a pas �t� tr�s importante. Une des raisons probables, restant �
confirmer par l'exp�rience, est la lenteur de la boucle de courant due
� sa r�alisation dans le PC. Le rem�de �tant de faire la boucle de
courant dans le PIC plut�t que dans le PC.


Apr�s la r�alisation de ce deuxi�me prototype la r�gulation de la
balan\c coire restait donc pas compl�tement satisfaisante, bien que
susceptible de pouvoir maintenir en l'air l'h�licopt�re. Des
oscillations non voulues restaient visibles � l'oeil, mais ne
semblaient pas compromettre la stabilit� de l'appareil.  Au cours de
ces �tapes les deux moteurs, utilis�s sous 6V, de faible puissance,
�taient insuffisants pour pouvoir soulever l'appareil.


%=======================================================================%
\subsection{Deuxi�me soutenance (fin octobre)}
%=======================================================================%
Le travail de la deuxi�me p�riode (photo \ref{v3}) a eu pour but
d'am�liorer la qualit� de la r�gulation et d'essayer de faire voler
l'h�licopt�re en ne lui laissant que deux degr�s de libert�~: -- le
tangage d�crit dans la premi�re partie -- un mouvement vertical dans
une glissi�re de l'axe en fibre de carbone autour duquel peut tourner
l'h�licopt�re. Pour cela il a fallu progresser sur plusieurs points.

Il a fallu obtenir une pouss�e plus importante en changeant les
moteurs. Les nouveaux moteurs consommant plus et fonctionnant avec une
tension diff�rente il a fallu changer l'alimentation.  Le banc d'essai
a du �tre modifi� pour r�aliser le deuxi�me degr� de libert�~: le
mouvement vertical.

Les non-lin�arit�s entre la correspondance courant et consigne PWM ont
�t� �tudi�es en r�gime stationnaire. Elles sont dues aux charges
a�rodynamiques sur l'h�lice et ont �t� compens�es dans le r�gulateur
de la boucle de courant.

L'utilisation de Scicos dans les boucles d'asservissement introduit
des retards importants du fait de~: -- de la vitesse de la liaison
s�rie entre le PIC et le PC, -- des contraintes temporelles sur la
gestion des processus de linux, -- des visualisations graphiques dans
le scope et des temps de calcul de Scicos.  L'influence de ce retard
sur les boucles d'asservissement a �t� analys�e. Pour am�liorer cette
vitesse, un changement d'ordinateur et d'OS a �t� n�cessaire
(cf. section (\ref{asser})). Au cours de ce travail quelques bogues de
Scicos sont apparus et ont �t� signal�s puis r�solus par les auteurs
de Scicos.

Les niveaux de bruit sur la mesure de l'inclinaison et des courants
ainsi que la d�rive de l'acc�l�rom�tre en fonction de l'intensit� du
courant passant dans les moteurs ont �t� grandement am�lior�s en
modifiant les circuits de masse et d'alimentation des composants.

\dessin{img/balancoire3}{0.8}{Prototype 3 et son �lectronique}{v3}

La d�rive basse fr�quence du gyroscope a �t� partiellement filtr�e, il
reste toutefois des bruits importants et inexpliqu�s qui limite pour
l'instant l'utilisation des gyroscopes (contrairement au signal de
l'acc�l�rom�tre qui est maintenant satisfaisant).

La fonction de transfert du comportement m�canique a �t� explicit�e.
Une boucle de r�gulation simplifiant la partie stable da la fonction
de transfert en boucle ouverte a �t� essay�e dans le but d'am�liorer
le gain de boucle tout en assurant une marge de stabilit�
satisfaisante. L'id�e �tait d'am�liorer la r�gulation pour �liminer
des oscillations r�siduelles autour des consignes demand�es. Les gains
obtenus restent m�diocres. Une meilleure mod�lisation �lectrom�canique
sera tent�e pour la troisi�me soutenance.

Malgr� les r�sultats relativement d�cevant de la r�gulation du tangage,
les premiers essais de vols dans la glissi�re ont �t� r�alis�s.
%Mais deux moteurs
%restent insuffisants pour que l'h�licopt�re puisse voler avec une marge suffisante pour pouvoir
%r�guler son altitude. Il faut encore augmenter la pouss�e.


%=======================================================================%
\subsection{Troisi�me soutenance (d�but d�cembre)}
%=======================================================================%
Deux moteurs arrivent � soulever la carcasse de l'h�licopt�re mais
sont en limite de puissance, il n'est donc pas possible
de r�aliser l'asservissement d'altitude. Pour pouvoir voler dans
cette espace � deux degr�s de libert� (tangage, altitude) il faut
les quatre moteurs. Leur mise en oeuvre implique de gros changements.


Il faut modifier le banc d'essai afin qu'il continue � limiter les
mouvements de l'h�licopt�re tout en permettant aux quatre moteurs de
tourner afin d'obtenir une pouss�e suffisante pour pouvoir voler.  Il
faut r�aliser une nouvelle carte �lectronique semblable � la premi�re
et doubler la puissance �lectrique disponible ce qui suppose r�aliser
une nouvelle alimentation capable de supporter 5A sous 8.5V. Voir
photo (\ref{alim}).

Les PIC 16F876A poss�dant seulement deux sorties PWM, l'utilisation de
quatre moteurs impose l'ajout d'un nouveau PIC de m�me type ou le
passage � un processeur plus puissant comme le dsPIC. Si le dsPIC est
beaucoup plus puissant il est aussi plus compliqu� � programmer.  Par
exemple l'assembleur passe de 30 � plus de 80 instructions.  C'est
donc un changement de technologie dont la ma�trise demande un minimum
de temps incompatible avec les �ch�ances � respecter. Il a donc �t�
d�cid� d'utiliser un deuxi�me PIC 16F876A et de faire communiquer les
deux PIC au moyen d'un bus I$^2$C. Voir photo (\ref{i2c}).

\dessin{img/alim}{0.45}{Nouvelle alimentation 8.5V 5A.}{alim}

Il faut doubler la complexit� du sch�ma Scicos avec les probl�mes de
vitesse que cela impose puisque~: -- on demande � la liaison s�rie de
transmettre un flux double de celui n�cessaire aux deux moteurs, -- on
doit faire un calcul et un affichage deux fois plus long.  D'autre
part, il reste � am�liorer la r�gulation de du tangage en faisant un
meilleur mod�le de la fonction de transfert courant-inclinaison.  La
boucle de courant reste aussi am�liorable � condition de la r�aliser
par les PICs au lieu de Scicos. De m�me des filtres anti-repliement de
spectre doivent �tre impl�ment�s au niveau du PIC.

\dessin{img/i2c}{0.5}{Carte � essai pour la communication
  I$^2$C.}{i2c}

De ce vaste programme, seule une faible partie a �t� r�alis�e dans le
temps imparti (le reste sera r�alis� pendant le stage de fin d'ann�e)
� savoir~: -- la nouvelle alimentation a �t� construite, -- la liaison
I$^2$C entre 2 PIC a �t� programm�e mais reste bogu�e (elle fonctionne
dans le cas d'un PIC et d'une EPROM, mais pas entre 2 PIC), -- les
fixations en aluminium pour les deux autres moteurs ont �t� faites et
coll�es sur les moteurs, -- la mod�lisation �lectrom�canique du couple
balan\c coire moteur a �t� �tudi�e et les param�tres correspondants
ont �t� estim�s sur le banc d'essais, -- une nouvelle r�gulation du
tangage a �t� r�alis�e, -- la nouvelle carte d'essai �lectronique a
�t� commenc�e, elle n'a pas �t� connect�e avec l'ancienne afin de
pr�server le prototype � deux moteurs pour la d�monstration.



\dessin{img/balancoire3_2}{1.2}{Derni�re version du prototype.}{picbal1}

%% \section{R�sum� des r�alisations}
%% \subsubsection{Premi�re soutenance}
%% Pour la premi�re soutenance, les points suivants ont �t�
%% r�alis�s :
%% \begin{itemize}
%% \item[$\bullet$] la construction de la partie m�canique constitu�e
%%   d'une croix en carbone et en aluminium, de quatre moteurs Speed-195
%%   Graupner et d'une carte contenant les capteurs (un acc�l�rom�tre
%%   et un gyroscope);
%% \item[$\bullet$] la construction du banc d'essai en Lego Technique;
%% \item[$\bullet$] la construction de la partie �lectronique (non
%%   embarquee) constitu�e d'une partie logique g�rant les
%%   entr�es/sorties et une interface de puissance;
%% \item[$\bullet$] un protocole de communication entre l'ordinateur et
%%   un microcontr�leur via un port s�rie;
%% \item[$\bullet$] la cr�ation du d�but du sch�ma block pour
%%   stabiliser l'h�licopt�re.
%% \end{itemize}

%% \subsection{Deuxi�me soutenance}
%% Pour la deuxi�me soutenance, les points suivants ont �t�
%% r�alis�s :
%% \begin{itemize}
%% \item[$\bullet$] Remplacement des vieux moteurs Speed-195 par des
%%   Micro-Speed 6V Graupner plus puissants. Nouvelles attaches pour
%%   moteurs plus solides.
%% \item[$\bullet$] Modification de la balan�oire pour permettre la
%%   regulation en altitude.
%% \item[$\bullet$] Duplication de l'�lectronique. Mise en place de la
%%   communication I$^2$C entre les deux PIC 16F876A. Correction du
%%   probl�me de parasites dans la masse, g�n�r� par les moteurs en
%%   s�parant au mieux les masses (branchement en �toile).
%% \item[$\bullet$] Am�lioration de la vitesse d'�chantillonnage en
%%   passant d'un Machintosh iBook G4 933 MHz � un PC 1.2 Ghz sur une
%%   Ubuntu patchee pour le temps r�el. Un gain de 4 sur la vitesse
%%   d'�chantillonnage a �t� constate.
%% \item[$\bullet$] Meilleure compr�hension du mod�le physique et des
%%   retards pour la partie automatique (utilisation de feedforward).
%% \end{itemize}

%\section{Apercu des anciennes versions de l'h�licopt�re}
%% \begin{minipage}[b]{.45\linewidth}
%% \centering\epsfig{figure=img/balancoire1, width=\linewidth}
%% \caption{Version 1.}\label{v1}
%% \end{minipage}\hspace{6mm}
%% \begin{minipage}[b]{.5\linewidth}
%% \centering\epsfig{figure=img/balancoire2, width=\linewidth}
%% \caption{Version 2.}\label{v2}
%% \end{minipage}\\[0.5mm]


%Ce document, dans le cadre de l'ann�e de sp�cialisation en temps r�el
%de l'EPITA, pr�sente un projet mettant en oeuvre l'�tude et la
%r�alisation d'un h�licopt�re mod�le r�duit � quatre h�lices (que l'on abr�gera, par la suite, par le mot h�licopt�re ou H4H) et de son banc d'essai �lectrom�canique et logiciel. Le but du projet est d'arriver � stabiliser le vol de l'h�licopt�re.

%Un h�licopt�re � 4 h�lices est une plate-forme volante qui a la forme d'une croix sur laquelle est attach�e, � chacune de ses extr�mit�s, une paire moteur-h�lice. La partie �lectronique de l'h�licopt�re, situ�e au centre de la croix, contr�le la vitesse des moteurs en fonction des consignes envoy�es par l'utilisateur. En g�n�ral, elle poss�de un port de communication avec le PC permettant de d�buger ou de charger des programmes.

%Dans le cadre de ce projet, la stabilisation est calcul�e par un ordinateur non embarqu� avec des logiciels de hauts niveaux. Les microcontr�leurs de l'h�licopt�re et l'ordinateur communiquent entre eux. Les donn�es �mises des diff�rents capteurs de l'h�licopt�re sont re�ues par l'ordinateur. Celui-ci effectue les calculs num�riques (loi de commandes, consignes) puis renvoie les valeurs au microcontr�leur qui r�alise les commandes de stabilisation. Les moyens de calcul embarqu�s permettront dans une deuxi�me phase de rendre autonome l'h�licopt�re en lui permettant de calculer lui m�me les lois de commande.

%%----------------------------------------------------------------------------------------------------------------------------------%
%\section{Contraintes du projet}
%\begin{description}
% \item[Robustesse � son environnement.] L'appareil est pr�vu pour fonctionner en int�rieur, avec des conditions m�t�orologiques id�ales (donc sans contraintes et sans perturbations).
% \item[Contraintes de dimension et de poids l'h�licopt�re.] L'engin se veut �tre de dimension et de poids le plus r�duit possible (ordre d'id�e : envergure en dessous des 30 cm, poids en dessous des 200 g). L'utilisation de cartes commerciales d'acquisitions (type mini cartes m�res embarqu�es) est prohib�e car elles sont trop excessives (poids, volume, puissance de calculs) pour ce projet. La partie �lectronique doit donc �tre fabriqu�e et adapt�e � l'h�licopt�re. Elle doit �galement avoir une interface de communication avec l'ordinateur (pour le d�bugage, chargement de programmes).
% \item[Contrainte de l'ordinateur et de l'IHM.] L'interface homme machine permet de simuler la plate-forme, puis de la contr�ler r�ellement tout en visualisant ses �tats. L'ordinateur sera un mod�le standard (PC, Macintosh), avec �ventuellement un OS temps r�el dur ou mou mais il devra �tre suffisamment r�actif pour assurer son r�le de gestion de la stabilit�.
%\end{description}


%=======================================================================%
\newpage
\part*{R�alisation de la partie m�canique}\label{meca}
\addcontentsline{toc}{part}{R�alisation de la partie m�canique}
%\part{Construction h�licopt�re et du banc d'essai}\label{matos}
%-----------------------------------------------------------------------%
%\section{Contexte}

Ce chapitre explique comment la structure de l'h�licopt�re a �t�
construite. Son \emph{design} a �t� pens� pour que l'h�licopt�re soit
le plus simple possible � fabriquer, qu'il soit de petite taille et
peu co�teux. Une journ�e suffit pour construire la croix, les pattes
et les attaches des moteurs et attendre que la colle prenne.

\dessin{img/h4h2}{0.5}{Prototype 1.}{h4h1}

%=======================================================================%
\section{Construction de la structure en croix}\label{croix}
%=======================================================================%
La structure de l'h�licopt�re est constitu�e de deux tubes
en fibre de carbone de 20 cm de longueur maintenus ensemble
par deux pi�ces en aluminium fa\c conn�es � la main et coll�es
ensemble. Voir photo (\ref{croix}). Les tubes sont creux
(diam�tre 6 mm) afin de pouvoir ins�rer des tiges de laiton qui
serviront de pieds � la structure. Voir photo (\ref{h4h1}).

Les fibres de carbone constituent un mat�riau tr�s l�ger mais qui amortit
peu les vibrations. Il faut donc �tre attentif aux modes propres de la
structure. Dans notre cas, du fait de la petite dimension de l'h�licopt�re,
ces modes propres ne sont pas g�nants.

\dessin{img/croix2}{0.7}{Attache des tubes de carbone.}{croix}

%%\\[0.5cm]
%% \begin{minipage}[b]{.45\linewidth}
%% \centering\epsfig{figure=img/h4h1, width=\linewidth}
%% \caption{La carcasse termin�e de l'h�licopt�re.}\label{h4h1}
%% \end{minipage}\hspace{0.5cm}
%% \begin{minipage}[b]{.5\linewidth}
%% \centering\epsfig{figure=img/croix2, width=\linewidth}
%% \caption{Les deux pi�ces d'aluminium fixant ensemble les deux axes de
%% l'h�licopt�re.}\label{croix}
%% \end{minipage}

%%\\[0.5cm]
%% \begin{minipage}[b]{.45\linewidth}
%% \centering\epsfig{figure=img/h4h1, width=\linewidth}
%% \caption{Helico avec ses pattes.}\label{croixfinale1}
%% \end{minipage}\hspace{0.5cm}
%% \begin{minipage}[b]{.5\linewidth}
%% \centering\epsfig{figure=img/croix, width=\linewidth}
%% \caption{Helico sans ses pattes.}\label{croixfinale2}
%% \end{minipage}

%Les deux axes de l'h�licopt�re doivent �tre solidement attach�s. Pour
%cela, on devra usiner deux pi�ces de longueur arbitraire dans une
%plaque d'aluminium selon le patron \ref{patroncroix} expliqu� dans l'annexe
%\ref{patrons}. Apres usinage, les deux plaques et les deux tubes en
%fibre de carbone sont coll�es ensemble (photo \ref{croixfinale2}).

%=======================================================================%
\section{Construction des attaches des moteurs}\label{moteur}
%=======================================================================%

%% \subsection{Ancienne attache}\label{sectoldattache}
%% L'ancienne version de l'helicoptere possedait des attaches adapt�es
%% pour les moteurs Speed-195 de Graupner. Vu qu'ils n'etaient pas assez
%% puissants pour soulever et stabiliser la carcasse de l'helicoptere
%% (ils etaient au max de leur puissance donc plus de marge de
%% puissance), ils ont ete remplace par des Micro-Speed 6V de
%% Graupner. Les attaches ont du etre repensees. De plus, des problemes
%% de colle trop l�che rendaient trop fragile l'helicoptere.

%% \dessin{img/attache}{0.5}{Ancienne attache pour moteur Speed-195}{oldattache}

%% Le proc�d� � couper un rectangle (1cm x 4cm) d'alumium et de
%% l'enrouller sur les tiges de carbone (avec l'aide de l'�tau). Leur
%% forme ressemblait � un grand om�ga en 3D ($\Omega$, un anneau avec
%% deux pattes horizontales). L'om�ga �tait ensuite coll� sur le
%% moteur comme sur la photo \ref{attache} ci dessus.

%% Comme l'une des tiges repose sur l'autre (et donc est plus 'haute' que
%% l'autre), il faut faire attention (pour l'est�tique) aux longeurs
%% des pattes des om�ga afin que les h�lices soient toutes � la
%% m�me hauteur.

%% \subsection{Nouvelle attache}\label{sectnewattache}

La premi�re soutenance avait pr�sent�e une version de l'h�licopt�re
qui poss�dait des attaches adapt�es aux les moteurs Speed 195 de
Graupner \cite{graupner} (voir photo (\ref{oldattache})), elles se
d�collaient souvent. De nouvelles attaches (voir photo
(\ref{newattache})) adapt�es � la forme cylindrique des moteurs Micro
Speed 6V de Graupner ont �t� r�alis�es. Elles ont une bien meilleure
tenue m�canique que les pr�c�dentes m�me si les nouveaux moteurs
utilis�s sont plus puissants.\\[0.5cm]
%\begin{itemize}
%\item[$\bullet$] Les Speed-195 n'�taient pas assez puissants par rapport aux
%  Micro-Speed 6V et ont donc �t� remplac�s.
%\item[$\bullet$] Les Micro Speed 6V ont une forme cylindrique alors
%  que les Speed-195 sont de forme cubique.
%\item[$\bullet$] La surface pour la colle des anciennes attaches
%  n'�tant pas assez grandes, la colle ne prenait pas et les attaches
%  se d�tachait facilement des moteurs.
%\end{itemize}
%L'annexe \ref{patrons} explique le patron des nouvelles attaches
%\ref{patron_attachemoteur} et comment les construire.
\begin{minipage}[b]{.45\linewidth}
\centering\epsfig{figure=img/attache, width=\linewidth}
\caption{Anciennes attaches pour moteur Speed 195.}\label{oldattache}
\end{minipage}\hspace{2cm}
\begin{minipage}[b]{.4\linewidth}
\centering\epsfig{figure=img/attache2, width=\linewidth}
\caption{Nouvelle attache pour moteur Micro Speed.}\label{newattache}
\end{minipage}

%\dessin{img/attache}{0.5}{Nouvelle attache pour moteur Micro-Speed}{newattache}

%\dessin{img/patron_attachemoteur}{1.0}{Patron des attaches pour moteur Micro-Speed.}{patronattache}

%% \begin{minipage}[b]{.5\linewidth}
%% \centering\epsfig{figure=img/patron_attachemoteur, width=\linewidth}
%% \caption{Patron des attaches pour moteur Micro-Speed.}\label{patronattache}
%% \end{minipage}\hspace{0.5cm}
%% \begin{minipage}[b]{.5\linewidth}
%% \centering\epsfig{figure=img/patron_attachemoteur, width=\linewidth}
%% \caption{attache enroullee prete a coller sur le moteur.}\label{photopatron}
%% \end{minipage}

%% \section{Construction des pieds}\label{pieds}
%% On coupe la tige de laiton en quatre morceaux. Parce que les axes sont
%% superpos�s, on coupe deux tiges plus grandes de laiton que les deux
%% autres (la diff�rence des tailles est le diam�tre des tiges de
%% l'h�lico). Avec un �tau, on les plie.  On peut ins�rer puis
%% coller un tube pour les vis (qui se mettent dans les murs) dans les
%% pattes de laiton afin qu'elles puissent s'ins�rer dans les tiges de
%% carbonne l'h�licopt�re.

%% \dessin{img/h4h1}{0.1}{Helcio avec pattes}{h4h1}

%=======================================================================%
\section{Construction de la carte des capteurs}\label{capt}
%=======================================================================%

Cette carte est une plaque d'essai en epoxy que l'on peut d�tacher de
la croix. Elle contient, pour l'instant, un gyroscope, un
acc�l�rom�tre et un capteur de distance infrarouge (voir
photo~\ref{dessus}). Il reste de la place pour deux gyroscopes et un
AOP avec ses r�sistances et condensateurs.\\[0.5cm]
\begin{minipage}[b]{.45\linewidth}
\centering\epsfig{figure=img/capteurdessus, width=\linewidth}
\caption{Carte des capteurs vue de dessus.}\label{dessus}
\end{minipage}\hspace{6mm}
\begin{minipage}[b]{.51\linewidth}
\centering\epsfig{figure=img/capteurdessous, width=\linewidth}
\caption{Carte des capteurs vue de dessous.}\label{dessous}
\end{minipage}

Cette carte est plac�e au dessus la croix de l'h�licopt�re. Une
deuxi�me carte �lectronique d�tachable portant l'�lectronique de
contr�le et de puissance (PIC, AOP, MOSFET, r�gulateur), plus lourde,
sera plac�e en dessous.  Le centre de gravit� de l'h�licopt�re sera
ainsi plac� en dessous de la structure alors que le centre de pouss�e
des h�lices, lui, est plac�e au dessus. L'h�licopt�re sera stable.  Les
deux cartes s'embo�teront ensemble.

En l'absence de la deuxi�me carte �lectronique, les moteurs sont
retourn�s et la carte des capteurs se trouve en dessous de la
structure.  Ainsi, lorsque l'h�licopt�re est attach� sur le banc
d'essai, son centre de gravit� se trouve ainsi sous l'axe de rotation.
A l'�quilibre stable, les h�lices sont dirig�es vers le haut et
l'h�licopt�re se comporte m�caniquement comme un pendule standard.

%=======================================================================%
\section{Construction du banc d'essai}\label{balanc}
%=======================================================================%

Le but du banc d'essai et de limiter les degr�s de libert� laiss�es
aux mouvement de l'h�licopt�re. Les degr�s libres sont~: -- une
rotation que l'on appellera tangage, -- une translation verticale
pour pouvoir tester une version simplifi�e du vol.

Lorsque le banc d'essai permet le seul tangage, on parlera
de \emph{balan�oire} pour l'ensemnble h�licopt�re-banc d'essai.  Le
banc d'essai a �t� fait en Lego Technique pour des raisons de
simplicit�.  Il doit �tre suffisamment haut pour pouvoir tester le
capteur de proximit� commutant � quarante centim�tre. Elle doit aussi
�tre suffisamment large pour pouvoir l'utiliser avec les quatre
moteurs.\\[0.1mm]
%La forme
%des pieds de la balan�oire n'est pas importante tant qu'elle est assez
%haute. Les trous dans les briques sont assez grands pour laisser
%passer les pieds de l'h�licopt�re qui seront orient�s vers le haut.
%L'ajout de tiges horizontales LEGO sur chacun des coins de la
%balan�oire permet de bloquer les pieds de l'h�licopt�re s'il bascule
%trop.
\begin{minipage}[b]{.55\linewidth}
\centering\epsfig{figure=img/balancoire, width=\linewidth}
\caption{Version 1 -- Un mod�le simple de rotule trois axes a �t� r�alis�e
mais n'a pas encore servie.}\label{balv1}
\end{minipage}\hspace{0.5cm}
\begin{minipage}[b]{.4\linewidth}
\centering\epsfig{figure=img/balancoire3_3, width=\linewidth}
\caption{Version 2 -- Banc d'essai modifi� permettant le mouvement vertical.}\label{balv2}
\end{minipage}

%La version 2 de la balan\c coire a �t� agrandi avec des briques DUPLO
%et mesure maintenant 40 cm de haut. Deux barres verticales de LEGO,
%lisses et parall�le entre elles, ont �t� ajout�es afin d'�tudier la
%r�gulation en altitude de l'h�licopt�re. Le but est de laisser
%l'h�licopt�re coulisser verticalement o� un elastique permet de
%limiter sa chute.


%\include{pic}

\newpage
%=======================================================================%
\part*{R�alisation de l'�lectronique}\label{electro}
\addcontentsline{toc}{part}{R�alisation de l'�lectronique}
%=======================================================================%
\section{Sp�cifications de l'�lectronique}
%=======================================================================%
Les cartes �lectroniques embarqu�es de l'h�licopt�re (conf�re le
sch�ma complet page \ref{h4hschema}) doivent comprendre :
\begin{itemize}
\item[$\bullet$] Une partie contenant tous les capteurs (un
  acc�l�rom�tre, trois gyroscopes, un altim�tre infrarouge),
\item[$\bullet$] Une partie logique contenant les microcontr�leurs,
  la mise en forme des signaux et le port de communication avec
  l'ordinateur,
\item[$\bullet$] Une partie �lectronique de puissance qui alimente
  les quatre moteurs contr�l�s par les microcontr�leurs.
\end{itemize}

\dessin{img/electrov2}{0.7}{Le circuit � embarquer.}{pcirc}

Elles doivent pouvoir g�rer :
\begin{itemize}
  \item[$\bullet$] neuf entr�es analogiques :
    \begin{itemize}
      \item les courants circulant dans les moteurs soit quatre
      entr�es analogiques,
      \item les vitesses angulaires donn�es par les gyroscopes
      soit trois entr�es analogiques,
      \item l'orientation par rapport � la verticale (roulis,
	tangage) donn�e par
	l'acc�l�rom�tre, soit deux entr�es
	analogiques.
      \item L'altitude soit une entr�e num�rique.
    \end{itemize}
  \item[$\bullet$] quatre sorties PWM (Pulse Width Modulation) contr�lant
  la vitesse des moteurs.
  \item[$\bullet$] un port s�rie de communication avec l'ordinateur.
  \item[$\bullet$] un bus I$^2$C de communication entre
    les microcontr�leurs.
\end{itemize}




%=======================================================================%
\section{Les microcontr�leurs PIC}\label{choix}
%=======================================================================%
\subsection{Choix des microcontroleurs}
%=======================================================================%
Les microcontr�leurs de Microchip \cite{microchip} ont �t�
  choisis � cause de leur prix, de la disponibilit� de
  la documentation et de la distribution gratuites des logiciels (PC
  et Linux) pour les programmer et les d�boguer. Parmi les
  microcontr�leurs de Microchip, deux choix sont possibles :
\begin{itemize}
  \item[$\bullet$] Soit utiliser deux PIC 16F876A,
  \item[$\bullet$] Soit utiliser un dsPIC 30F3011.
\end{itemize}

En effet, les PIC 16F876A ont deux sorties PWM et cinq entr�es
analogiques alors que le dsPIC 30F3011 a au moins quatre sorties PWM
et neuf entr�es analogiques.

Le choix s'est port� sur le PIC 16F876A plut�t que sur le dsPIC
30F3011. Etant novice dans ce domaine, ce choix �tait le plus simple
(quitte � devoir re-�crire le programme ASM pour le dsPIC) sachant que
l'excellent cours sur les PIC de Bigonoff \cite{bigonoff} est une aide
pr�cieuse pour la compr�hension des PIC 16F84 et 16F87x. Par contre,
pour le dsPIC, on ne dispose que de la documentation du constructeur
Microchip, qui bien que tr�s riche, est beaucoup moins p�dagogique que
le cours de Bigonoff. Avant d'utiliser des dsPIC, il est pr�f�rable de
ma�triser le fonctionnement des PIC.  Dans une deuxi�me version du
projet, on utilisera un unique dsPIC car, en plus du tr�s grand nombre
d'entr�es/sorties disponibles il dispose d'un multiplieur int�gr� qui
permet de mettre en oeuvre plus facilement les contr�leurs et les
filtres n�cessaires.

On a choisi le 16F876A au lieu du 16F877 pour son meilleur compromis
puissance/poids (format PDIP). Le 16F877 est beaucoup plus lourd (40
pattes contre 28).

Un PIC 16F876A est suffisant pour contr�ler un axe de
l'h�licopt�re. Dans le cas de deux axes � la communication I$^2$C pr�s
les programmes de chaque PIC sont les m�mes.

%=======================================================================%
\subsection{Communication inter-composants}
%=======================================================================%
De nombreuses communications entre les diff�rents composants (capteur,
actionneur, processeur) doivent �tre �tablies.  Le diagramme
(\ref{comm}) d�finit les diff�rents acteurs, les communications �
r�aliser avec leur sens et les ordres temporels � respecter (les
num�ros entour�s).

\dessin{img/comm}{0.5}{Chronologie de la communication inter
  composants}{comm}

\begin{enumerate}
\item Le PIC ma�tre et le PIC esclave lisent les valeurs analogiques
  (sauf l'altim�tre) des capteurs (acc�l�rom�tre, gyroscopes).
\item Scilab envoie au PIC ma�tre les consignes des signaux PWM des
  quatre moteurs.
\item Pendant que le PIC ma�tre envoie au PIC esclave, via le module
  I$^2$C, les consignes PWM des moteurs 3 et 4,  les
  consignes sont envoy�s aux MOSFET commandant  les moteurs 1 et 2.
\item Une fois termin� la r�ception des consignes PWM pour les moteurs 3 et 4
 le PIC esclave les fait suivre aux moteurs et envoie
  les conversions analogiques de ses capteurs au PIC ma�tre.
\item Pendant que le PIC ma�tre envoie � Scicos ses acquisitions
  analogiques et celles du PIC esclave.
\end{enumerate}

\subsubsection{La communication USART avec l'ordinateur}
Le PIC, configur� en mode USART (Universal Synchronous Asynchronous
Receiver Transmitter) communication s�rie asynchrone de type RS232
avec un ordinateur via ses pattes Rx (r�ception) et Tx (transmission)
branch�es sur un port s�rie. La vitesse de transmission est de 19200
bauds, 8 bits de donn�es, parit� paire, 1 bit de stop et aucun
contr�le de flux.  Le composant MAX232 convertit les signaux du port
s�rie (+12V/-12V) en signaux adapt�s au PIC (0V/+5V). Le cours de
Bigonoff \cite{bigonoff} explique en d�tail comment fonctionne
l'ensemble et fournit un programme type � charger sur le
microcontr�leur et un programme PC pour tester la communication s�rie.

\subsubsection{La communication I$^2$C entre les deux PIC}
Pour faire communiquer les deux PIC, deux choix sont possibles~: --
soit utiliser le protocole SPI, -- soit utiliser le protocole I$^2$C.
Bien que la communication SPI soit beaucoup plus simple (chaque PIC
�coute et envoie un octet de donn�e simultan�ment), il a
l'inconv�nient d'utiliser deux pattes analogiques du PIC esclave (�
cause de l'utilisation des ports \emph{Slave Select}). Puisque neuf
entr�es analogiques sont n�cessaires, il n'est pas possible d'utiliser
le mode SPI. Le cours de Bigonoff \cite{bigonoff} explique en d�tail
ce protocole.

La communication I$^2$C est �galement une communication de type s�rie
multi-ma�tres, multi-esclaves fonctionnant avec deux fils (horloge et
donn�es). Seul le protocole mono-ma�tre et multi-esclave nous
int�resse. Les esclaves ont tous un num�ro d'identification
diff�rent. Le PIC ma�tre est l'unique chef d'orchestre, il ne sait
faire que deux actions diff�rentes~: d�signer un esclave pour lui
envoyer des donn�es (on dit qu'il est en mode �criture) et d�signer un
esclave pour recevoir des donn�es (on dit qu'il est en mode
lecture). L'esclave ne peut donc pas choisir l'instant d'envoi
d'une donn�e.

La communication se fait de la fa\c con suivante~: le PIC ma�tre
commence toujours une communication en envoyant un bit de start suivit
du num�ro d'identification d'un esclave puis d'un bit qui indique
s'il va envoyer des donn�es ou que c'est � l'esclave de lui envoyer des
donn�es. L'esclave envoie alors un bit d'acquiescement pour dire qu'il
est pr�sent. Le ma�tre, selon le cas envoie un paquet de donn�es ou
attend que les donn�es arrivent. Lorsqu'une donn�e est re\c cue,
un bit d'acquiescement est envoy�e. Seul le ma�tre d�cide quand se
termine la communication.

Du point de vue du c�blage �lectrique, un bus I$^2$C utilise sur
chaque ligne une r�sistance de rappel (d'au moins 1700 $\Omega$)
branch�e sur le 5V afin d'�viter des
courts circuits. Un bit de niveau bas met au niveau bas la ligne
quelque soit les autres envois. Le cours de
Bigonoff \cite{bigonoff} d�taille le protocole I2C et donne des
exemples d'utilisation que nous avons adapt� � notre application.

% \section{Diagramme de flot donn�es}

%=======================================================================%
\subsubsection{Chronogramme des interruptions}
%=======================================================================%

Les PIC permettent de lancer plusieurs op�rations en parall�le. Les
fins de ces calculs sont signal�es par des interruptions. Les PIC ont
�t� programm�s pour faire tourner en parall�le deux timers,
une conversion analogique, la g�n�ration de deux signaux PWM, la
r�ception et l'�mission d'un octet sur le port s�rie et le
module I$^2$C. Le chronogramme des interruptions est illustr� sur la
figure \ref{chrono}.

\dessin{img/chronointerrup}{0.6}{Chronograme des interruptions du PIC
ma�tre.}{chrono}

Le PIC ma�tre contient deux buffers permettant de communiquer avec
l'ordinateur~:
\begin{itemize}
\item[$\bullet$] un premier pour la r�ception, que l'on appellera
  \emph{BufRec};
\item[$\bullet$] un deuxi�me pour l'�mission, que l'on appellera
  \emph{BufEm}.
\end{itemize}

Chaque r�sultat des conversions analogiques est sauvegard� dans
\emph{BufEm}. Une interruption (repr�sent�e par une fl�che noire
sur le chronogramme \ref{chrono}). Le PIC permet une seule conversion
� la fois. On doit alors changer de patte analogique. Il faut
attendre que~:
\begin{itemize}
\item[$\bullet$] les condensateurs de la nouvelle patte se chargent;
\item[$\bullet$] la conversion analogique se fasse.
\end{itemize}
Comme ces dur�es d�pendent de la tension et de la chaleur du PIC,
on utilise le timer 0 avec une p�riode assez grande pour pouvoir
relancer une nouvelle lecture (interruption en vert sur le chrono).

Scilab envoie alors les consignes PWM par le port s�rie, ce qui cr�e
une interruption (orange sur le chrono). Le PIC ma�tre les lie et les
stocke dans \emph{BufRec}. Il profite de cette interruption pour
envoyer le contenu de \emph{BufEm}.

Les consignes PWM, stock�es dans \emph{BufRec}, sont les nouvelles
valeurs des comparateurs du PIC. Deux comparateurs sont pr�sents et
peuvent fonctionner avec le timer 2. Lorsque la valeur du timer 2
atteint celle du comparateur $x$ , +5V est g�n�r� sur la patte CCP$x$
du PIC jusqu'au d�bordement du Timer 2, qui permet de terminer un
cycle d'un signal PWM, en mettant � la masse les sorties des pattes
CCP. La vitesse de notre signal PWM est de 5 kHz avec un PIC poss�dant
un quartz de 20MHz. Enfin, avant qu'une interruption de d�bordement du
Timer 2 soit lanc�e, quatre interruptions de d�bordement du Timer 0
sont lanc�es.

Le chronogramme des interruptions du PIC esclave est �quivalent au
sch�ma (\ref{chrono}). Les routines d'interruptions sont les m�mes. La
seule diff�rence est l'interruption USART qui est alors remplac�e par
une interruption I$^2$C.


%=======================================================================%
\subsection{Interface de programmation}\label{icsp}
%=======================================================================%
Une technique, o� un microcontr�leur est programm� apr�s qu'il ai �t�
soud� sur une carte est appel� \emph{In Circuit Programming}
(ICP). La programmation et le d�boguage du PIC se fait gr�ce � une
interface � 5 fils, reli�e en s�rie avec une autre carte �lectronique
appel�e programmateur.

Lorsque la pin VPP du PIC est � +12V (au lieu du +5V), le PIC se met
en mode \emph{programmation} et charge le nouveau programme par la
patte ICSPDAT, synchronis� par ISPCLK.

\dessin{img/icsp}{0.45}{Cablage ICSP d'un PIC}{icsp}

Un programmateur PIC que l'on branche sur un port s�rie d'un PC peut
�tre facilement fabriqu� comme l'explique le site \cite{jdm}. D'autres
programmateur, plus �volu�s, permettant le d�bogage en ligne sont
disponibles chez le constructeur et des vendeurs ind�pendants
accessibles sur Ebay. La photo (\ref{pjdm}) et le sch�ma (\ref{sjdm})
montrent mon premier programmeur JDM. Les composants ont �t� plac�s par
rapport au JDM original afin de pouvoir acc�der au PIC directement sur
la carte.\\[0.5mm]
\begin{minipage}[b]{.5\linewidth}
\centering\epsfig{figure=img/jdm_schematic, width=\linewidth}
\caption{Sch�ma �lectronique du programmateur JDM modifi�.}\label{sjdm}
\end{minipage}\hspace{0.5cm}
\begin{minipage}[b]{.5\linewidth}
\centering\epsfig{figure=img/jdm, width=\linewidth}
\caption{Premier programmateur JDM.}\label{pjdm}
\end{minipage}



%\section{Apercu de la carte electronique}\label{aop}
%=======================================================================%
%Voici la carte �lectronique contr�l�e par le PIC ma�tre. Cette carte
%permet de contr�ler un des deux axes de l'h�licopt�re.

%\dessinsscaption{img/h4h_schematic}{0.14}\label{h4hschema}

%PARLER: filtre AOP accelerometre, modifier valeur resistances. Ajouter
%communication inter PIC: i2C parceque SPI on pert 2 analogiques.
%\section{Sch�ma de la carte �lectronique}\label{schema}
%=======================================================================%
\section{Les capteurs}\label{capteur}
%=======================================================================%

%=======================================================================%
\subsection{L'acc�l�rom�tre}
%=======================================================================%
\subsubsection{Le capteur}\label{acc}
%=======================================================================%
Un acc�l�rom�tre deux axes plac� horizontalement au centre de gravit�
(suppos� immobile) de l'h�licopt�re, permet de d�tecter la
verticale. C'est le capteur le plus important dans la phase d'essai o�
le centre de gravit� de l'h�licopt�re sur son banc d'essai est
maintenu immobile (cas de la balan�oire). Un unique acc�l�rom�tre
5G ADXL320 de AnalogDevice est alors suffisant pour
stabiliser l'h�licopt�re.  Malheureusement, les acc�l�rom�tres
disponibles bons march�s existent sous formes de petites tailles
($4 \times 4$ mm) et sont tr�s difficiles � souder sur une carte epoxy
standard.

\dessin{img/accelero}{0.7}{Carte acc�l�rom�tre 1.5cm $\times$ 1.5cm.}{pacc}

L'acc�l�rom�tre choisi donne un signal utile compris entre +1.3V (�
l'horizontale) et +1.7V (� 90 degr�s) qui doit donc �tre recentr� et
normalis� entre 0V et +5V afin d'avoir une pr�cision maximale avec le
convertisseur analogique num�rique 8 ou 10 bits disponibles sur le PIC.

%=======================================================================%
\subsubsection{Amplification du signal}\label{aopaccelro}
%=======================================================================%

Un AOP est un composant �lectronique qui permet d'amplifier un signal
continu ce qui permet au final de cr�er des fonctions math�matiques
(d'o� le mot op�rationnel) telles que la d�riv�e, l'int�grale,
l'addition, le log, des filtres ... Le sites \cite{aop1,aop2,aop3}
�tudient les AOP et expliquent les diff�rents montages possibles. Des
AOP sous la forme de composants � deux sorties (LM358) ou � quatre
sorties (LM324) consommant du +5V sont utilis�s pour la carte de
l'h�licopt�re. Voici le sch�ma caract�ristique (\ref{aop1})~:

%% Dans le cadre de ce projet, seuls les montages AOP permettant de faire
%% des transformations affines nous int�ressent, en effet le filtrage
%% des bruits des signaux des capteurs ou bien l'application d'un PID sur
%% un signal se fait par logiciel sur un ordinateur non embarqu�
%% (voir chapitre \href{asservissement}{asservissement}) (ou directement
%% sur un dsPIC pour une version future). Par contre, pour garder la
%% pr�cision maximale des calculs, les signaux doivent �tre centr�s
%% et normalis�s.


\dessin{img/aop1}{0.5}{AOP en montage soustracteur.}{aop1}

On obtient la sortie Vout = G (V$_1$-V$_2$) avec un gain G =
R$_2$/R$_1$. En pratique, les valeurs des r�sistances R$_1$ et R$_2$
seront choisies de l'ordre de 100 K$\Omega$.
%CENTER(\dessin{helico/schemas/aopsoustracteur2.jpg}{0.8}{toto}{label})
%% L'utilisation d'un AOP en montage soustracteur
%%   \href{helico/schemas/aopsoustracteur.jpg}{figure de droite} est
%%   utilis�.
Pour conna�tre les valeurs des r�sistances R1 et R2, on r�sout le
syst�me suivant, o� $G$ est le gain et $r$ la tension de r�f�rence.

$$G (1.3 - r) = 0$$
$$G (1.7 - r) = 5$$

D'o� : $G = 12.5$ et $r = 1.3$ D'o� les valeurs R$_1$ = x, R$_2$ =
x, R$_a$ = x et R$_b$ = x sur le sch�ma (\ref{aop2}). La sortie
amplifi�e Vout est branch�e directement sur une des pattes
analogiques d'un PIC.

\dessin{img/aop2}{0.5}{AOP en montage soustracteur et son pont diviseur.}{aop2}

%=======================================================================%
\subsubsection{Filtre anti-repliement de spectre}\label{filtreacceler}
%=======================================================================%
La fr�quence d'�chantillonnage de l'ordinateur est limit� � 50Hz
� cause du d�bit de la liaison s�rie.
Les PIC feront eux les acquisitions � une fr�quence de 5kHZ.
Nous devons couper les signaux de fr�quence sup�rieure � 25Hz afin d'�viter de
les retrouver sous forme de signaux parasites dans le signal �chantillonn� (repliement de spectre).

Nous utilisons un filtre anti-repliement de spectre d'ordre 4. Pour
le r�aliser nous le d�composons en un filtre d'ordre 2
r�alis� �lectroniquement (en utilisant l'AOP) et un deuxi�me filtre d'ordre 2 �crit en assembleur dans les PICs.
Le filtre �lectronique assure �galement la coupure � 2.5kHz n�cessaire
� l'acquisition analogique des PICs.

\subsubsection*{Filtre �lectronique}
Pour r�aliser ces filtre des condensateur $C_1$ et $C_2$ sont ajout�s
au circuit de montage l'AOP (\ref{aop2}) ce qui donne le nouveau
sch�ma (\ref{aop3}).

\dessin{img/aop3}{0.5}{AOP en montage soustracteur et avec filtre anti-repliement de spectre.}{aop3}

\begin{itemize}
\item[$\bullet$] Pour trouver la valeur de $C_2$, sachant que
  $\omega=25\times 2\pi$ et $R_2$ est fix� � 250k$\Omega$, il faut r�soudre
  l'�quation suivante~:
  \begin{equation}
    \left(\frac{1}{R_2C_2\omega}\right)^2=\frac{1}{\sqrt{256}}
  \end{equation}


% Cette formule se justifie. Comme $R_2$ et $C_2$ sont en parall�le,
% nous avons la formule suivante : $\frac{R_2}{R_2C_2\omega+1}$. Pour
% $\omega$ grand (les fr�quences que l'on supprime) et que seul le
% d�nominateur nous int�resse, on s'implifie par
% $\frac{1}{R_2C_2\omega}$. Enfin, $\frac{1}{256}$ vient du fait que la
%sortie de l'AOP sera convertie sur 8 bits (8 bits suffisent au lieu de
%10 bits).
%$R_2C_2=\frac{16}{\omega}=\frac{16}{25 \times

On en d�duit~:
$R_2C_2=0.025$ et
$C_2=4/(250\times 10^3\times 25\times 2\pi)=100$nF

\item[$\bullet$] Pour d�terminer $C_1$, le calcul est le m�me sachant
  que la r�sistance de l'acc�l�rom�tre est de 32k$\Omega$~:
  $C_1=0.025/(32\times 10^3) = 800$nF.
\end{itemize}

\subsubsection*{Filtre logiciel}
Suivant le m�me principe, le filtre logiciel du second ordre
$y=(\frac{u}{1+0.025\omega})^2$, s'�crit~:
\begin{align*}
z_{n+1} &= z_n\left(1-\frac{1}{128}\right)+\frac{u_n}{128} \\
y_{n+1} &= y_n\left(1-\frac{1}{128}\right)+\frac{z_n}{128}
\end{align*}
o� $\frac{1}{128}$ est le pas de temps et $u_n$ est le signal
analogique de l'AOP et $y_n$ le r�sultat � envoyer � l'ordinateur.

%=======================================================================%
\subsection{Les gyroscopes}
%=======================================================================%
\subsubsection{Le capteur}\label{gyro}
%=======================================================================%
Il existe maintenant des composants �lectroniques contenant des
gyroscopes capable de mesurer la vitesse angulaire gr�ce �
l'observation des forces de Coriolis sur une barre vibrante.  Ils sont
beaucoup moins co�teux que les gyroscope m�canique qui eux sont
capable de mesurer les variations angulaire d'un mobile par rapport �
une direction fixe obtenue par une toupie tournant � grande vitesse.
Les prix des gyroscopes m�canique �tant trop �lev� pour ce projet,
trois gyroscopes pi�zo-�lectriques un axe Gyrostar (s�ries
ENC) de Murata sont utilis�s.

 Ces gyroscopes donnent un signal utile dans une bande de
fr�quence de l'ordre de l'hertz � quelques dizaines de hertz,
en particulier la mauvaise qualit� du signal � basse fr�quence
induit une d�rive difficile � filtrer. Cette d�rive est tr�s g�nante
pour maintenir l'h�licopt�re dans une position stationnaire
(objectif principal de ce projet). N�anmoins le signal
de vitesse angulaire est de bien meilleure qualit� que la vitesse
obtenue en d�rivant les donn�es venant de l'acc�l�rom�tre

D'autre part certains auteurs affirment
qu'il n'est pas possible
de stabiliser la plate-forme avec les seuls acc�l�rom�tres
� cause des acc�l�rations du centre de gravit� suppos�es
perturber l'information sur la verticale fournie par l'acc�l�rom�tre.
Ces affirmations sont douteuses dans notre cas puisque l'acc�l�rom�tre
�tant suppos� fix� approximativement au centre de gravit� dans une position
perpendiculaire � la pouss�e des moteurs l'acc�l�ration du mobile � une
contribution sur l'acc�l�rom�tre.

A priori, une fois ma�tris� la d�rive du gyroscope, la meilleure fa\c
con de concevoir le r�gulateur est d'utiliser le signal donn� par le
gyroscope et de le recaler par celui donn� par l'acc�l�rom�tre.

\begin{minipage}[b]{.4\linewidth}
\centering\epsfig{figure=img/gyro, width=\linewidth}
\caption{un GWS MPG10 contenant un gyro Murata.}\label{pgyro1}
\end{minipage}\hspace{3cm}
\begin{minipage}[b]{.3\linewidth}
\centering\epsfig{figure=img/gyro1, width=\linewidth}
\caption{La sortie analogique r�cup�r�e par le fil blanc.}\label{pgyro2}
\end{minipage}

Les capteurs Gyrostar sont difficiles � trouver, mais sont le coeur de
gyroscopes utilis�s en a�romod�lisme comme les GWS PG03. Les
PG03 sont con�us pour �tre branch�s entre la t�l�commande et modifier
le signal envoy� aux servomoteurs. Ils disposent donc de
l'�lectronique pour moduler le signal de t�l�commande. Deux choix sont
alors possibles :
\begin{itemize}
\item[$\bullet$] r�cup�rer le signal analogique du Gyrostar avec la
  carte du PG03 (image \ref{pgyro2}),
\item[$\bullet$] utiliser directement la sortie du PG03 (� savoir la
modulation PWM pour servomoteur).
\end{itemize}

Le premier choix �limine la contrainte de vitesse du signal
t�l�command� modul� � 50Hz des PG03 mais n'utilise plus son circuit
d'amplification. Il semble facile de dessouder le Gyrostar du PG03
mais il est moins dangereux de souder un fil sur la patte Out (num�ro
4) du Gyrostar directement sur la carte du PG03 figure
(\ref{pgyro2}). C'est la m�thode utilis�e ici (image \ref{pgyro2}).

%=======================================================================%
\subsubsection{Amplification du signal}\label{aopgyro}
%=======================================================================%
Le document \cite{latour} de l'EPFL \cite{epfl} donne le filtre passe
bande suivant (\ref{fgyro}) pour �liminer les d�rives et filtrer les
bruits du gyroscope.
\begin{itemize}
\item[$\bullet$] Le filtre passe haut coupe les fr�quences en dessous
  de 0.33Hz pour r�duire la d�rive;
\item[$\bullet$] Le filtre passe bas �limine les fr�quences au dessus
  de 588Hz;
\item[$\bullet$] le signal est amplifi� d'un facteur 9.
\item[$\bullet$] le signal est centr� sur VDD/2 (soit +2.5V en entr�e).
\end{itemize}

\dessin{img/filtregyro}{0.5}{Filtre pour gyroscope de l'EPFL.}{fgyro}

Malheureusement le signal de notre gyroscope reste encore
perturb� par un bruit de taille fixe apparaissant de fa\c con al�atoire.
D'autre part une d�rive g�nante reste pr�sente. Il reste encore
du travail � faire pour exploiter au mieux le signal du gyroscope.
%=======================================================================%
\subsection{L'altim�tre infrarouge}\label{alti}
%=======================================================================%

  L'altim�tre infrarouge Sharp GP2Y0D340K est le plus petit
  et le moins cher des capteurs IR. Il commute de 0 �
  1 en pr�sence d'un obstacle � moins de
  40cm. Il se branche sur une patte num�rique du PIC. La
  distance de commutation semble d�pendre de la couleur de
  l'obstacle. Le capteur sera dirig� vers le haut de
  l'h�licopt�re (et non vers le bas) afin qu'un
  utilisateur puisse contr�ler l'altitude de la plate-forme en
  pr�sentant au dessus un obstacle que l'h�licopt�re essaiera de poursuivre.
  Dans la phase d'essai, il est dirig� vers le bas, l'h�licopt�re essaie
  de se maintenir � 40cm du sol.


%=======================================================================%
\section{L'�lectronique de puissance}\label{puis}
%=======================================================================%
Le but de l'�lectronique de puissance est d'alimenter et de commander
les moteurs. Pour des raisons de simplicit�, cette version
d'h�licopt�re utilise uniquement des moteurs �lectriques � collecteur aliment�s
en courant continu (MDC) tournant dans un seul sens. Il suffit donc
de 4 MOSFET pour les commander � partir des commandes PWM envoy�es par les
PICs.

Les moteurs fonctionnent sous 7V et consomment 4A.
Pour pouvoir observer les courants passant dans les moteurs, des r�sistances
de 1 Ohm sont mise en s�rie avec les moteurs. Les courants passant
dans chaque moteur �tant de l'ordre de 1A, on choisit d'alimenter
sous 8.5. Une alimentation capable de fournir 5A sous 8.5V est donc n�cessaire.
D'autre part l'�lectronique de commande (par exemple les PICs) doit
�tre aliment�e en 5V. Ce 5V est obtenu � partir du 8.5V
gr�ce � un r�gulateur L78S05.


%=======================================================================%
\subsection{Sch�mas de base}
%=======================================================================%
La documentation de Microchip \cite{microchip,AN905} \og Brushed DC
Motor Fundamentals\fg montre deux sch�mas ((\ref{highside}) et
(\ref{lowside})) pour contr�ler un MDC selon le type de MOSFET
utilis�.\\[0.5cm]
\begin{minipage}[b]{.35\linewidth}
\centering\epsfig{figure=img/highside, width=\linewidth}
\caption{High side.}\label{highside}
\end{minipage}\hspace{4cm}
\begin{minipage}[b]{.35\linewidth}
\centering\epsfig{figure=img/lowside, width=\linewidth}
\caption{Low side.}\label{lowside}
\end{minipage}

Dans ces deux sch�mas, le signal PWM est envoy� par les pattes CCP1 ou CCP2 (Compare
Capture PWM) d'un PIC. Afin d'obtenir la pr�cision
maximum (10 bits) le signal PWM (cr�neaux 5V/0V) est envoy� avec une fr�quence de
5kHz). En faisant varier le rapport cyclique du
signal PWM (rapport entre le temps de l'�tat haut et le temps de
l'�tat bas), on fait varier la tension au borne du moteur suppos�
approximativement constante gr�ce � la self du moteur et la capacit� rajout�e
filtrant le courant. Le cours
de \cite{bigonoff} explique comment g�n�rer un signal en mode
compare ou en mode PWM.

La r�sistance R$_1$ prot�ge le PIC des surtensions et la r�sistance
R$_2$ emp�che le MOSFET d'�tre passant �
l'initialisation du PIC. La diode, dite de roue libre, qui
doit �tre capable commuter rapidement, prot�ge le MOSFET des surtensions
qui appara�trait en son absence au instant de commutation du signal PWM.
En effet, la bobine accumule l'�nergie magn�tique qui doit pouvoir
se d�charger au moment ou l'alimentation du moteur est coup� par
le MOSFET.

On utilise des MOSFET P plus facile � trouver en bo�tier TO 220.
L'inconv�nient du MOSFET P, dans le sch�ma (\ref{highside}), est la
tension de la g�chette du MOSFET � 8.5V incompatible
avec la tension haute (5V) du PIC. Ce probl�me se r�sout
facilement par l'adjonction d'un transistor ouvert/ferme par
le signal PWM.
possible d'utiliser un MOSFET P, � condition d'utiliser un transistor
NPN et trois r�sistances. Voir sch�ma (\ref{intpuiss1}).

%\dessin{img/moteur1}{0.4}{Sch�ma utilis� pour commander les MOSFET P.}{mot1}

Un condensateur plac� en parall�le au moteur permet de filtrer ses
parasites du moteur.

%=======================================================================%
\subsection{R�le de la \og roue libre \fg}
%=======================================================================%
La tension $u$ aux bornes d'une bobine vaut $L\frac{di}{dt}$. A
l'instant ou le MOSFET coupe le courant $u$ devient un Dirac
n�gatif. Ce qui signifie en pratique qu'une grosse surtension $V_B
>>V_A$ est cr��e avec des risques d'apparition d'arcs �lectriques dans
l'interrupteur ici le MOSFET.  La diode prot�ge le circuit en
renvoyant le courant dans la bobine de B vers A, courant qui va
s'amortir progressivement � cause des r�sistances �lectriques.

\dessin{img/bobdiode}{0.5}{Interruption de courant passant dans une self.}{bobdiode}

%=======================================================================%
\subsection{Mesure du courant dans les moteurs}
%=======================================================================%

%% On impose une vitesse au moteur en fonction d'une tension. Le
%% probl�me est que la bobine d'un moteur se comporte comme un filtre
%% passe bas. Le moteur est un syst�me en boucle ferm�e stable : un
%% signal PWM d'une fr�quence suffisante va g�n�rer un courant
%% stable dans la bobine du moteur. Apr�s plusieurs essais de
%% l'h�licopt�re sur son banc d'essai (mode balan�oire) en utilisant
%% la figure de gauche et un logiciel de calcul num�rique, la
%% stabilisation de l'h�licopt�re n'�tait pas id�ale. Une des
%% raisons est que le contr�le du moteur �tait en boucle ouverte.


%dessin{img/moteur1}{0.4}{Interface de puissance.}{moteur1}

L'asservissement des moteurs se faisant en courant, on doit l'observer.
Le capteur correspondant est constitu�e d'une r�sistance dont on
mesure la tension aux bornes. La loi d'Ohm donne alors le courant
cherch�. On utilise donc le sch�ma suivant (\ref{intpuiss2}).\\[0.5cm]
%\dessin{img/moteur2}{0.4}{Interface de puissance et lecture du courant.}{moteur2}
\begin{minipage}[b]{.35\linewidth}
\centering\epsfig{figure=img/moteur1, width=\linewidth}
\caption{Interface de puissance.}\label{intpuiss1}
\end{minipage}\hspace{3cm}
\begin{minipage}[b]{.45\linewidth}
\centering\epsfig{figure=img/moteur2, width=\linewidth}
\caption{Interface de puissance et AOP.}\label{intpuiss2}
\end{minipage}\\[0.5mm]

La lecture de la tension se fait gr�ce � un AOP en montage soustracteur.
L'inconv�nient de ce montage est la perte de puissance par effet joule
dans la r�sistance. D'autres possibilit�s existe pour observer le courant
en consommant moins de puissance mais ne seront pas discut�s ici.
On aurait pu placer $R_2$ entre le moteur et la masse, ce qui aurait
�vit� d'avoir � faire la soustraction r�alis�e par l'AOP. Ce
montage diff�rentiel peu sensible aux perturbations a �t� choisi
ici  surtout pour �conomiser la pose d'un fil suppl�mentaire
entre la plaque � essais et l'h�licopt�re.

%% L'ajout d'une r�sistance de puissance Rxx
%% (2W, 1$\Omega$) permet �galement de limiter l'intensit� dans le moteur
%% (si l'h�lice se bloque, la force �lectromotrice devient nulle et
%% l'intensit� dans le moteur devient grande ce qui peut
%% l'endommager). Un AOP en montage soustracteur sur la r�sistance de
%% puissance permet de normaliser et centrer la tension
%% �lectromagn�tique. La sortie de l'AOP est branch�e sur une patte
%% analogique d'un PIC. Le chapitre
%% \ref{asservi} de ce document explique
%% comment fonctionne le feed-back.
%% TODO
%% expliquer pourquoi choper la diff�rence de tension (masse) est moins
%% bon que s'utiliser un aop: --> c plus stable : un pariste apprait.
%% TODO
%% expliquer que les moteurs crachent des parasites --> condo 4700uF
%% TODO
%% expliquer que c quand meme bien de mettre un transistor pour decoupler
%% un mosfet.

\section{Alimentation �lectrique}
Comme nous l'avons vu pr�c�demment, les moteurs doivent �tre aliment�s
avec une tension de 8.5V et avec une intensit� maximale de 4A. Le
composant LM338K est capable de fournir 5A. Un
potentiom�tre (R66 sur le sch�ma (\ref{h4hschema})) permet de r�gler la
tension de sortie � 8.5V. Un condensateur C25 de grande capacit�
(0.1F) permet de filtrer les parasites et micro coupures.
Un r�gulateur  LM7805 permet d'obtenir
du 5V (1A) � partir du 8.5V pour alimenter les PIC et les capteurs.
Comme le LM338K est aliment� en 12V (gr�ce  une alimentation PC 12V 9A)
il doit dissiper une puissance allant jusqu'� 14W, un gros radiateur
est n�cessaire. On a choisi un radiateur dont la temp�rature augmente
de 2 degr�s par W. Il ne devrait donc pas d�passer 50 degr�s � la
charge maximale.


\dessin{img/alim}{0.4}{Nouvelle allimentation 8.5V 5A.}{alim2}

\newpage
\pagestyle{empty}
\dessinsscaption{img/h4h_schematic}{0.19}\label{h4hschema}
\newpage
\pagestyle{plain}






\newpage
%=======================================================================%
\part*{R�alisations des asservissements}\label{asser}
\addcontentsline{toc}{part}{R�alisations des asservissements}
%=======================================================================%
Les microcontr�leurs PIC 16F876A embarqu�s sur l'h�licopt�re,
permettent de r�aliser facilement l'acquisition des signaux
analogiques et la g�n�ration des signaux PWM. Par contre ils ne
disposent pas module DSP int�gr� et donc sont mal adapt�s pour faire
du calcul num�rique en ligne.

L'utilisation d'un ordinateur non embarqu� mais connect� par une
liaison s�rie avec les microcontr�leurs permet d'ajouter au syst�me
la puissance de calcul manquante. Il existe des logiciels libres
sp�cialis�s au calcul num�rique � la commande et traitement du signal
comme Scilab permettant de programmer ais�ment les lois de commandes
et de les mettre en oeuvre en temps r�el.
%=======================================================================%
\section{L'ordinateur dans la boucle}
%=======================================================================%
\subsection{Scilab et Scicos}
%=======================================================================%
Il existe deux types de programmes scientifiques~:
\begin{itemize}
\item[$\bullet$]les logiciels
alg�briques faisant essentiellement du calcul symbolique (Maple,
Mathematica, Maxima, Axiom, et MuPad);
\item[$\bullet$]les logiciels de calcul
scientifique faisant essentiellement de l'analyse num�rique (Scilab,
Matlab).
\end{itemize}

Scilab \cite{Chancelier, Scilab} est un logiciel libre pour le calcul
scientifique. C'est un interpr�teur de langage manipulant des
objets typ�s dynamiquement. Il inclut de nombreuses fonctions
sp�cialis�es pour le calcul num�rique organis�es sous forme de
librairies ou de boites � outils qui couvrent des domaines tels que
la simulation, l'optimisation, et le traitement du signal et du
contr�le.

Une des bo�tes � outils les plus importantes de Scilab est
Scicos \cite{Chancelier, Scicos}. Scicos est un simulateur hybride
avec un �diteur graphique de bloc diagrammes permettant de mod�liser
et de simuler des syst�mes dynamiques. Il est particuli�rement utilis�
pour mod�liser des syst�mes o� des composants temps-continu et
temps-discret sont inter-connect�s.

%=======================================================================%
\subsection{Port s�rie sous Scicos}
%=======================================================================%
Bien que Scicos poss�de des moyens de communication avec l'ext�rieur,
il ne poss�dait pas de module tout fait pour communiquer par les port s�rie.
Il a donc fallu rajouter � Scicos
un bloc diagramme qui puisse envoyer et �couter des donn�es sur un
port s�rie.

Ceci est facilement r�alisable, car Scilab permet de faire de
l'�dition de liens sur des fonctions C et de les lier � des blocs
diagrammes Scicos. Pour construire un bloc Scicos, deux fichiers sont
n�cessaires, donc deux fonctions : \emph{fonction de calcul}
et \emph{fonction d'interface}.

En cr�ant la fonction de calcul et d'interface pour le port s�rie,
Scicos devient un oscilloscope num�rique 50 Hz. Il lit les donn�es des
capteurs de l'h�licopt�re sur le port s�rie, les affiche sous forme de
courbes, calcule la loi de commande et envoie les consignes PWM sur le
port s�rie. La vitesse du port s�rie est de 19200 bauds, 8 bits de
donn�es, 1 bit de stop et aucun flot de contr�le (ni mat�riel ni
logiciel).

%=======================================================================%
\subsection{Vitesse de Scicos}
%=======================================================================%
\subsubsection{Le temps r�el sous Scicos}
%=======================================================================%
Une des difficult� de ce projet a �t� d'obtenir et d'assurer
une vitesse suffisante d'ex�cution des calculs des affichages
et des transmissions des donn�es. Ces difficult�s proviennent
des limitations de vitesse du port s�rie, de l'ordonnanceur de Linux,
et de la vitesse d'affichage de Scicos.

Scicos permet de lancer une simulation en  temps
r�el \footnote{en modifiant la valeur {\tt Real Time Scaling} du menu
{\tt <Simulate><Setup>}.}. Une seconde Scilab correspond alors au mieux � une
seconde r�elle. Scicos peut �tre ralenti durant la simulation
d'un buffer graphique
trop petit ou � cause de bloc diagramme �crit en langage Scilab
(bloc {\tt Sciblock}) que Scicos doit interpr�ter\footnote{Les {\tt
Sciblock} sont � �viter. Il vaut mieux utiliser {\tt Mathematical
Expression} ou {\tt C block} qui sont beaucoup plus rapides.}.

Ces restriction sur la fr�quence d'�chantillonnage induit � cause du
th�or�me de Shannon (la fr�quence d'�chantillonnage doit �tre
plus du double de la fr�quence maximale du signal pour pouvoir le restituer
compl�tement) que la bande de fr�quence qui peut �tre trait�e est inf�rieure
� 25 Hz. Bande de fr�quence suffisante pour la r�gulation du tangage
mais insuffisante pour la boucle de courant.

%=======================================================================%
\subsubsection{Vitesse d'�chantillonnage obtenues}
%=======================================================================%
Les vitesses d'�chantillonnage ont �t� obtenues sous Scicos avec un
port s�rie � 19200 bauds~:
\begin{itemize}
\item[$\bullet$] 20 Hz sur un Macintosh iBook G4 cadenc� � 933 MHz.
\item[$\bullet$] 50 Hz sur un PC 1.2 GHz avec un Linux Ubuntu recompil� en
                 mode 1 kHz.
\end{itemize}

%=======================================================================%
\subsubsection{D�finition d'un p�riode et fr�quence Scicos}
%=======================================================================%
Dans l'utilisation du temps de Scicos, il faudra prendre garde � ce
qui est appel� une fr�quence est en fait une pulsation et que les
p�riodes r�f�rent aux �v�nements g�n�r�es par l'horloge (un cycle d'un
signal carr� {\tt square wave generator} n�cessite deux tops d'horloge
pour �tre g�n�r�.

%% \section{Asservissement de l'h�licopt�re}
%% L'approche exp�rimentale a �t� utilis�e afin de
%% d�terminer le comportement de l'h�licopt�re et de le stabiliser. La
%% partie th�orique est bien pour comprendre un comportement de
%% l'h�licopt�re, mais en pratique d'autres ph�nom�nes peuvent intervenir
%% et au final le comportement attendu n'est plus le m�me. Ce chapitre
%% explique comment l'asservissement de cet h�licopt�re fonctionne. Le
%% seul vrai avantage

%=======================================================================%
\section{Asservissement des moteurs}
%=======================================================================%
Il existe deux m�thodes pour contr�ler un moteur~:
\begin{itemize}
\item[$\bullet$] soit l'asservissement en vitesse (alias
  asservissement en tension). Son avantage est que la consigne PWM
  s'obtient lin�airement en fonction de la tension. Son inconv�nient
  est que la charge (qui varie proportionnellement � la vitesse au
  carr� de l'h�lice) du moteur est non lin�aire par rapport � la
  tension.
\item[$\bullet$] soit l'asservissement en couple (alias asservissement
  en courant). Son avantage est que le couple (et donc la charge) est
  une fonction lin�aire du courant. Son inconv�nient est qu'une
  consigne PWM est une fonction non lin�aire du courant.
\end{itemize}

%L'asservissement en courant permet d'obtenir une plus grande bande
%passante que l'asservissement en vitesse.
Dans notre cas, � cause de la lenteur de Scilab qui nous emp�che
de r�aliser une bonne boucle de courant, la sup�riorit� de
l'asservissement en courant sur l'asservissement en
tension n'est pas claire. Dans le futur la boucle de courant
sera r�alis�e dans le PIC. L'avantage de la boucle de courant
devrait appara�tre plus nettement.


%% L'avantage deDans la premi�re m�thode, si on fixe la tension, le
%% moteur la FEM est proportionnelle la vitesse du moteur
%% Dans la deuxi�me, si on fixe le courant, on fixe le couple et le
%% couple et proportionnel
%% tens: les PWM commande direct la tension des mot
%% inconv�nient: charge mot est non lin
%% cour:propo a la charge modilise la baln de facon lineair
%% il faut convertir en tension et les non linearites seront la
%% La deuxi�me bande passante plus grande mais Scicos ...
%% + la portance est proportionnelle a la vitesse au carre. donc lin�aire
%% alors que la vitesse sera non lineaire

%% on va modeliser la fct non lin�aire en statique on suppose que


%=======================================================================%
\subsection{�tude th�orique d'un moteur DC}\label{mot}
%=======================================================================%
Cette section, tir�e d'un cours de EPFL \cite{epfl},
explique le mod�le d'un moteur � courant continu. Un moteur est constitu� de deux parties~:
\begin{itemize}
\item[$\bullet$] une partie tournante appel�e \emph{rotor} qui
  contient une  bobine appel�e \emph{induit}. Cette bobine est
  caract�ris�e par sa r�sistance $R$ et son inductance $L$.
  Du fait de son mouvement elle est le si�ge d'une force contre
  �lectromotrice $e$ proportionnelle � sa vitesse angulaire.
 \item[$\bullet$] une partie fixe appel�e \emph{stator} qui dans
notre cas sera un aimant permanent caract�ris� par un coefficient
appel� \emph{excitation magn�tique} $\Phi$ .
\end{itemize}

La tension au borne de l'induit est alors donn�e par
\begin{align}
u(t) & = R i(t) + L di/dt + e(t),\\
e(t) & = k\;\Phi\;\omega(t).
\end{align}

Le courant dans le rotor dans le champ magn�tique du stator est le
si�ge de force �lectromagn�tique induisant un  couple
�lectromagn�tique $T_e(t)$.
Il est proportionnel au
courant induit $i(t)$. Il vaut~:
\begin{align}
T_e(t) & = k\;\Phi\;i(t).\label{eqcouple}
\end{align}

Notant $J$ l'inertie du moteur sa dynamique s'�crit~:
\begin{align}
J\frac{d\omega}{dt} & = T_e - R_{f}\;\omega(t)-T_{r}.
\end{align}

O�, $T_{r}$ est un couple r�sistant et $R_f$ un couple caract�risant
les frottements. En posant $K_e = K_t = k\Phi$, on obtient le sch�ma
bloc suivant (\ref{mottheo1}) mod�lisant le moteur. Le moteur est un syst�me
en boucle ferm�.

\dessin{img/moteurtheo}{0.6}{Sch�ma th�orique d'un moteur.}{mottheo1}

Les fonctions de transfert des blocs $A$ et $B$ s'�crivent~:
$$H_{A}=\frac{1/R}{1 + s L/R},\quad H_{B=}\frac{1/Rf}{1 + s J/Rf}.$$

Dans notre cas, quatre moteurs Graupner Micro Speed sont utilis�s et
sont aliment� par une tension 7V (courant 1.2A) hach�e par un
signal PWM � 5kHz. L'imp�dance de la bobine du moteur est de 0.2 mH et
sa r�sistance interne est de 2$\Omega$. Le param�tre $k$ n'a pas �t�
identifi�.

%=======================================================================%
\subsection{Asservissement en courant}\label{pidtheocourant}
%=======================================================================%
%\subsubsection{Differentes fa\c con d'asservir un moteur}

La r�gulation th�orique d'un moteur avec un PID se fait selon le
sch�ma (\ref{pidmot1}) suivant.

\dessin{img/pidmoteurtheo1}{0.6}{R�gulation d'un moteur avec un feedback.}{pidmot1}

Le PID de ce diagramme corrige l'{\tt erreur} et la fait converger
vers 0 (si le PID est bien r�gl�). La fonction du bloc {\tt
  Conversion courant vers consigne PWM} se comporte, dans notre cas,
de fa\c con non lin�aire.

Cependant, dans notre cas, nous voulons que la sortie du r�gulateur
PID soit la valeur de la {\tt Consigne de courant}. Pour cela, nous
devons modifier le sch�ma bloc (\ref{pidmot1}) en ajoutant un
feedforward. Le nouveau sch�ma bloc ressemble donc � l'image
(\ref{pidmot2}). La notion de feedforward est tr�s bien expliqu�e sur le
cours d'automatique d'\.Astr\"om \cite{Astrom}, gratuit et disponible
sur son site web.

\dessin{img/pidmoteurtheo2}{0.6}{R�gulation d'un moteur avec un
  feedforward et feedback.}{pidmot2}

\.Astr\"om r�sume les diff�rences entre feedback et feedforward~:\\[0.5mm]
\begin{tabular}{|p{7cm}|p{7cm}|} \hline
Feedback & Feedforward \\ \hline
Est utilis� en boucle ferm�e & Est utilis� en boucle ouverte \\ \hline
R�agit seulement quand il y a des perturbations & R�agit avant que les
perturbations arrivent\\ \hline
Est robuste aux erreurs de mod�lisation & Est non robuste aux erreurs de
mod�lisation \\ \hline
Pose des risques d'instabilit� & Pose aucun risque d'instabilit� \\ \hline
\end{tabular}


Pour att�nuer la perturbation, le feedforward s'utilise de la fa\c con
suivante~:

\dessin{img/feedforward}{0.4}{Att�nuation de la perturbation.}{feed2}

o� la perturbation est totalement �limin�e mais le {\tt process}
$P_1$ doit �tre bien compris puisqu'il doit �tre explicitement invers�
dans le bloc $P_1^{-1}$.

%\section{�tude pratique d'un moteur}\label{mot}

%% Malheureusement, la pratique est beaucoup plus p�nible que la
%% th�orie. En effet~:
%% \begin{itemize}
%% \item[$\bullet$] la fonction qui convertit un courant en une consigne
%%   PWM d�n moteur avec son h�lice est non lin�aire. D'apr�s les
%%   exp�riences, elle a la forme d'une racine carr�e. On la d�montre
%%   dans la section suivante.
%% %  $\left(\frac{i}{60}\right)^7+2i$.
%% \item[$\bullet$] Les moteurs crachent des parasites dans la masse qui
%%   est commune aux capteurs. Leurs sorties sont parasit�es. Le faite
%%   d'ajouter une capacit� en parall�le au moteur �limine ces parasites
%%   mais change l'ordre du moteur. L'ordre du moteur passe de un a deux.
%% \end{itemize}

%=======================================================================%
\subsection{D�terminer la fonction non lin�aire PWM-Courant
en r�gime �tabli}\label{nonlin}
%=======================================================================%
Une s�rie de mesure sur des r�ponses indicielles � permis de d�terminer
le courant en fonction de consignes de PWM. Apr�s inversion de la
fonction on obtient la fonction en statique du bloc {\tt conversion
 courant vers PWM}. On supposera que cette fonction en dynamique se
comporte lin�airement entre les points mesur�s (le r�gulateur PID
jouera ce r�le).

En reprenant l'�tude du moteur section (\ref{mot}). Le circuit induit
d'un moteur en r�gime �tabli se comporte, du point de vue �lectrique,
comme une r�sistance $R$ et d'une FEM en s�rie.
% (cf. secion \ref{mot}). En effet la self agit que dans les r�gimes
%transitoires.
Le couple $C$ de l'�quation (\ref{eqcouple}), en r�gime �tabli, est
proportionnel au courant et s'�crit donc $C=k\;i$.

On trouve dans des documents que le couple (tra�n�e, portance) sur une
h�lice est �galement une fonction d�pendante du carr� de la vitesse de
l'h�lice (donc du rotor $\omega$), d'o� $C=k_1\;\omega^2$. On en d�duit la
vitesse $\omega$ en fonction du courant : $\omega=\sqrt{i\;k/k_1}$.

La tension $u$ aux bornes du circuit induit vaut
$u=Ri+e=Ri+k_2\sqrt{i\;k/k_1}$.
Soit plus simplement~:
$$u=K\sqrt{i}+R\;i$$

Les mesures obtenues figure (\ref{i2p}), lors des exp�riences
confirment la th�orie. Pour obtenir ces mesures, on donne une consigne
de PWM sur 10 bits sur un moteur avec son h�lice et on observe le
courant obtenu.\\[0.5mm]
\begin{minipage}[b]{.48\linewidth}
\centering\epsfig{figure=img/i2p, width=\linewidth}
\caption{PWM (ordonn�es) en fonction du courant (abscisses).}\label{i2p}
\end{minipage}
\begin{minipage}[b]{.5\linewidth}
\centering\epsfig{figure=img/i1i2, width=\linewidth}
\caption{Correspondance des courants entre 2
  moteurs.}\label{diffcourant}
\end{minipage}%\hspace{0.5cm}
%\dessin{img/i2p}{0.4}{PWM (ordonn�es) en fonction du courant (abscisses).}{i2p}

%=======================================================================%
\subsection{Non lin�arit� entre les moteurs}\label{nonlin}
%=======================================================================%
Les quatre moteurs du m�me type ne proviennent pas de la m�me s�rie et
pr�sente des caract�ristiques l�g�rement diff�rentes. Si la rotation d'une h�lice est plus dure, le moteur tourne moins vite
et la FEM induite est plus faible. Soumis � la m�me tension externe le courant
est plus grand.

%\dessin{img/i1i2}{0.4}{Correspondance des courants entre 2 moteurs}{diffcourant}

Afin d'�quilibrer la pouss�e des deux moteurs g�rant un axe de
l'h�licopt�re, une fonction statique �tablissant la correspondance entre
courants des deux moteurs a due �tre d�termin�e.
% (cf annexes\ref{}).
On voit sur la courbe (\ref{diffcourant}), que pour des
courants importants (haut droit), un des deux moteurs sature.

Le graphique (\ref{courant}) a �t� obtenu en simulation r�elle du sch�ma
(\ref{pidmot2}). On envoie une consigne de courant aux deux moteurs
(signal carr� en noir). On voit que les courants (rouge et bleu)
arrivent � suivre les consignes. Ce r�sultat a �t� obtenu
essentiellement avec un feedforward statique compensant la
non-lin�arit�, seul un terme int�gral a �t� ajout� pour assurer erreur
asymptotique nulle.

\dessin{img/courant}{0.6}{Consignes de courant et courant obtenus.}{courant}

%=======================================================================%
\section{Asservissement du tangage}
%=======================================================================%

%=======================================================================%
\subsection{Mod�le de l'h�licopt�re}\label{modelh4h}
%=======================================================================%

On obtient la r�ponse indicielle de l'h�licopt�re (figure (\ref{sysdyna})) en
regardant comment il oscille lorsqu'on le laisse se balancer sur le
banc d'essai.

%% Dans le chapitre \ref{matos} li� � la construction de la m�canique de
%% l'h�licopt�re, on a vu que la carte des capteurs, selon l'endroit o�
%% elle se place, peut rendre l'h�licopt�re naturellement stable (carte
%% plac�e en dessous de la croix) ou instable (carte plac�e au dessus de
%% la croix).

\dessin{img/reponseindic}{0.4}{R�ponse indicielle de
  l'h�licopt�re stable.}{sysdyna}

De cette courbe, on en d�duit la fonction de transfert $H(s)$ qui au
courant associe l'inclinaison de
l'h�licopt�re~:
$$H(s) = \frac{300}{s^2+1.28s+31}$$
Cette formule a �t� obtenue en supposant que la pouss�e �tait proportionnelle
au courant ce qui n'est vrai qu'en r�gime �tabli.

On peut essayer d'am�liorer le mod�le en ajoutant la dynamique du
rotor. En effet la charge est une fonction non lin�aire de la vitesse
du moteur alors que le courant agit sur son acc�l�ration. Une
lin�arisation autour d'un r�gime donn� montre que le comportement
dynamique serait mieux mod�lis� par
$$\frac{300}{(s^2+1.28s+31)(1+ks)}$$
formule dans laquelle il faut estimer $k$.
Apr�s mesure de la fonction de transfert il appara�t qu'elle n'est pas
d'ordre 3 dans la zone de fr�quence utile. Cette fonction de transfert
a �t� alors estim� en envoyant une s�quence de sinuso�des et en
observant le gain de la r�ponse. Une bonne approximation de ce
transfert est~:
$$G(s)=\frac{240+40s}{s^2+1.28s+31}$$

%On en d�duit la fonction de transfert $G(s)$ pour l'h�licopt�re instable~:
%$$G(s) = \frac{12}{s^2+0.7s-25}$$

On v�rifie que le syst�me $G(s)$ est stable car il n'a que des p�les
n�gatifs.

%\dessin{img/balanBF}{0.5}{Simulation de l'h�licopt�re.}{simubf}

%% La r�alit� est tout autre~: les retards accumul�s dans le transition
%% des donn�es entre PIC et Scicos ainsi que la faible fr�quence
%% d'�chantillonnage du � Scilab font qu'il est difficile de stabiliser
%% correctement l'h�licopt�re. La section \ref{black} explique le
%% comportement du mod�le en fonction de la longueur du retard ainsi

%=======================================================================%
%\subsection{Feedforward}\label{feed}
%=======================================================================%
%% Dans ses notes de cours d'automatique, Karl Johan \.Astr\"om
%% \cite{Astrom} r�sume les diff�rences entre feed-back et feed-forawrd.

%% \begin{tabular}{|p{7cm}|p{7cm}|} \hline
%% Feed-back & Fedd-forward \\ \hline
%% Utilis� en boucle ferm�e & en boucle ouverte \\ \hline
%% R�agit seulement quand il y a des perturbations & R�agit avant que les
%% perturbations arrivent\\ \hline
%% Robuste aux ereurs de mod�lisation & Non robuste aux erreurs de
%% mod�lisation \\ \hline
%% Risque d'instabilit� & Aucun risque d'instabilit� \\ \hline
%% \end{tabular}

%=======================================================================%
\subsection{Etude de l'influence n�faste du retard}\label{black}
%=======================================================================%
En simulation, il est facile de stabiliser l'h�licopt�re, mais dans la
r�alit�, les retards accumul�s (transition des donn�es entre PIC et
Scicos, faible fr�quence d'�chantillonnage de Scilab) font qu'il est
difficile de stabiliser correctement l'h�licopt�re.  Etudions
l'influence des retards sur un transfert obtenu pr�c�demment (ancienne
�tude datant de la r�vision 2 de ce rapport)~:
$$G(s)=\frac{192}{s^2+0.7s+25}.$$
%avec un seul moteur gr�ce au sch�ma bloc suivant.

A cause des retards, la fonction de transfert $G(s)$ du mod�le de
l'h�licopt�re stable doit �tre modifi�e~:
$$J(s)=G(s)\exp^{-\tau s}$$

Pour $\tau=0.02$ et en utilisant la formule du d�veloppement limit�
$\exp^{-\tau s}\simeq 1-\tau s\simeq\frac{1-\tau/2 s}{1+\tau/2 s}$,
$J(s)$ s'�crit~:

$$J(s)=G(s)\left(\frac{1-0.005s}{1+0.005s}\right)^2$$

Soit $F(s)$ la fonction de transfert, utilis�e dans le feedforward,
qui inverse le mod�le de l'h�licopt�re (avec plus ou moins
d'exactitude) comme expliqu� dans la section \ref{pidtheocourant}. Si
on suppose que mod�le $G(s)$ est exact, on a alors~:
$$F(s) = J(s)^{-1} = \frac{(s^2+0.7s+25)}{192}$$

On introduit la fonction de transfert du filtre anti-repliement de
spectre vue dans le chapitre concernant l'�lectronique
(chapitre \ref{electro})~:
$$A(s) = \left(\frac{1}{1+0.025s}\right)^4$$

Nous obtenons la fonction de transfert de l'h�licopt�re et
modifi� par un contr�leur (on a ignor� le mod�le d'un bloqueur �chantillonneur
({\tt Sample/Hold}) parce qu'il serait trop compliqu� � utiliser)~:
$$K(s) = A(s)F(s)J(s)$$

En tra\c cant le diagramme de Nyquist et de Black de $K(s)$~:\\[0.5mm]
\begin{minipage}[b]{.5\linewidth}
\centering\epsfig{figure=img/ny1, width=\linewidth}
\caption{Trac� de Nyquist de $K(s)$.}\label{ny1}
\end{minipage}\hspace{0.2cm}
\begin{minipage}[b]{.48\linewidth}
\centering\epsfig{figure=img/black1, width=\linewidth}
\caption{Trac� de Black de $K(s)$.}\label{bl1}
\end{minipage}

Le trac� du lieu de Nyquist de la fonction de transfert $K$ nous montre qu'�
cause du retard nous ne pouvons pas
reboucler ce syst�me avec des gains trop forts (la courbe de Nyquist
risquerait d'entourer le point $(-1;0)$).

En modifiant les diff�rents param�tre comme~: la longueur du retard,
la vitesse �chantillonnage, la pr�cision du mod�le $G(s)^{-1}$ on se
fait une bonne id�e de la valeur des gains possibles du r�gulateur.\\[0.5mm]
\begin{minipage}[b]{.5\linewidth}
\centering\epsfig{figure=img/ny2, width=\linewidth}
\caption{Trac� de Nyquist de $K(s)$ avec une impr�cision de 30\% sur $G(s)^{-1}$.}\label{ny1}
\end{minipage}\hspace{0.2cm}
\begin{minipage}[b]{.48\linewidth}
\centering\epsfig{figure=img/black2, width=\linewidth}
\caption{Black de $K(s)$ avec une impr�cision de 30\% sur
  $H(s)^{-1}$ et gain proport. de 3}\label{bl1}
\end{minipage}
%================================
\subsection{Le contr�leur adopt�}
%===============================
La fonction de transfert identifi�e~: $$G(s)=\frac{240+40s}{S^2+1.28s+31}$$
�tant stable et d'inverse stable, le correcteur adopt� est la somme de~:
$$\frac{31}{(240+40s)(1+0.025)^2}$$ appliqu� au signal de~:
l'acc�l�rom�tre et de $$\frac{s+1.28}{(240+40s)(1+0.025s)^2}$$
appliqu� au signal venant du gyroscope. Le r�sultat de cette
correction est de placer les p�les du syst�me en boucle ouverte
corrig� � -40. Un feedback de gain 1/2 a �t� utilis�. Les r�sultats
obtenus sont donn�s dans la figure (\ref{resul}). Le sch�ma scicos
d�terminant compl�tement le r�gulateur est donn� dans les sch�mas
Scicos (\ref{scipic}), (\ref{scimot1}), (\ref{scimot2}), (\ref{scimain}).

%================================
\subsection{R�sultats obtenus}
%===============================
Dans la figure (\ref{resul}), le premier diagramme (celui du haut)
montre l'inclinaison (en vert), observ�e par l'acc�l�rom�tre, obtenue
en r�ponse � des �chelons de consigne d'inclinaison (en noir) envoy�
par le PC. Les consignes PWM pour les deux moteurs sont de couleur
rouge et bleue.

Le deuxi�me diagramme (celui du milieu) montre les deux consignes de
courants (en noir) et les courants observ�s dans les deux moteurs (en
bleu et rouge).

Le deuxi�me diagramme (celui du bas) montre le signal du gyroscope (en
vert) et la d�riv�e du signal de l'acc�l�rom�tre (en violet).

%================================
\section{Sch�ma bloc Scicos}
%================================
Dans les sch�mas Scicos (\ref{scipid}), (\ref{scipic}),
(\ref{scimot1}), (\ref{scimot2}), (\ref{scimain}), les blocs sont
colori�s pour une meilleure compr�hension.

En jaune, les sch�mas blocs correspondant � la communication avec le
PIC ma�tre. En gris, les consignes. En rose p�le, les signaux arrivant
au plot. En vert, les signaux du gyroscope et de l'acc�l�rom�tre
(cette couleur correspond aussi au courbe obtenue sur la
figure (\ref{resul}). En bleu et rouge, les signaux correspondant aux
courants des moteurs 1 et 2. En violet, les r�gulateurs.

\dessin{img/graphzoom}{0.7}{R�sultats obtenus lors d'une simulation.}{resul}

\dessin{img/scicospid}{0.6}{Super bloc du r�gulateur de l'inclinaison
de l'h�licopt�re.}{scipid}

\dessin{img/scicospic}{0.7}{Super bloc pour la communication avec le
PIC ma�tre.}{scipic}

\dessin{img/scicosmot1}{0.7}{Super bloc r�gulant le moteur 1.}{scimot1}

\dessin{img/scicosmot2}{0.65}{Super bloc r�gulant le moteur 2.}{scimot2}

\dessin{img/scicosmain}{1.05}{Sch�ma bloc principal stabilisant
l'h�licopt�re.}{scimain}

%=======================================================================%
%\section{Asservissement en altitude}
%=======================================================================%

%% %\section{TODO: Feedforward vs. feedback}
%% \dessin{img/echantillon}{0.3}{jjj}{echan}
%% helico + Scilab = simulation PIC + echanti + retard = instable avec
%% echant faible; stable avec plein echanti; PID=0 donc .
%% feedforward --> satbilise mais feedback inutile
%% %\section{Plus d'une fa\c con de representer un PID}
%% Decrire PID normal
%% Decrire PID avec approximation/filtre de chaque morceaux
%% Le PID qui filtre les hautes fonctions (truc d'arnaud)


\newpage
%=======================================================================%
\part*{Conclusion}
\addcontentsline{toc}{part}{Conclusion}
%=======================================================================%

Ce rapport a pr�sent� l'�volution de l'�tude et de la r�alisation d'un
h�licopt�re � quatre h�lices sur une p�riode effective de travail
d'environ quatre mois. Une partie du cahier des charges a �t�
r�alis�e. La balan\c coire un axe a �t� faite assez compl�tement.
Seules les pr�alables � la stabilisation deux axes (r�alisation de
l'alimentation 8.5V pour pouvoir alimenter les quatre moteurs, la
programmation de la liaison I$^2$C des deux PIC en cours de debogage)
ont �t� faits. Par contre des essais de vols avec la balancoire un axe
ont �t� commenc�s (la puissance des deux moteurs, en limite de
puissance, emp�che de mettre en oeuvre la r�gulation d'altitude mais le
capteur fonctionne et l'information d'altitude est envoy�e
correctement au contr�leur).

Les retards dans la r�alisation, en fait pr�vus,
(puisque au d�part le minimum annonc� �tait
la r�alisation de la balan\c coire) sont dus � la sous-estimation de la
difficult� de faire une stabilisation de qualit� de la balan\c coire
un axe. Les difficult�s sont multiples~: -- lenteur de la liaison
s�rie, -- lenteur de Linux-Scicos pour faire du temps r�el,
-- difficult�s � obtenir un mod�le fiable,
-- non lin�arit� du syst�me, -- perturbation �lectrique ou
a�rodynamique, -- pr�cision des capteurs, -- d�rive des capteurs,
-- difficult� de la programmation en assembleur des microcontr�leurs,
-- n�cessit� de commander en courant qui oblige � g�rer plus d'entr�es
que pr�vues et � faire des boucles rapides.

Toutes ces difficult�s ont ralenti le projet mais ont
�t� tr�s formatrices car elles font toucher du doigt les diff�rences
existantes entre une simulation d'un logiciel et la r�alisation
physique d'un prototype.

La construction de cet h�licopt�re a fait appel
� plusieurs sp�cialit�s de l'ing�nierie �
savoir~: -- la construction m�canique, -- l'�lectronique, -- l'informatique,
-- l'automatique. Ensemble de domaines beaucoup trop vaste pour �tre
ma�tris� en peu de temps mais qu'un tel projet permet d'aborder
de fa\c con tr�s agr�able.

Ce projet va continuer d'�voluer, puisque pendant la p�riode du stage
des am�liorations vont �tre apport�es comme~: -- terminer la communication
I$^2$C entre les deux PIC, -- finir de stabiliser l'h�licopt�re
complet, -- apprendre la programmation des dsPIC, -- embarquer
l'asservissement de l'h�licopt�re par un dsPIC, -- utiliser des
modules XBee pour la communication sans fil -- et, pourquoi pas,
remplacer les moteurs par des moteurs brushless.

L'aventure continue donc. Merci pour la cr�ation de la motivation !!!


%\printindex

%=======================================================================%
%% \newpage
%% \part{Annexes}
%% \section{Pr�vision du d�roulement du projet}
Le projet se d�compose en plusieurs �tapes, chaque �tape comprenant
plusieurs phases.  Certaines phases peuvent �tre men�es en parall�le
d'autres demandent la r�alisation de t�ches pr�alables. Une tentative
de planification avec affectation des t�ches est r�alis�e pour simuler
les contraintes d'un projet industriel � d�veloppement
it�ratif. L'absence d'exp�rience rendant tr�s peu pr�cise les
estimations de temps, ce planning ne sera qu'indicatif.

\subsection*{R�alisation d'un demi h�licopt�re}
Le but de cette �tape est d'obtenir un demi h�licopt�re sur un banc
d'essai capable de communiquer avec un ordinateur qui calcule la loi
de stabilisation. Le demi h�licopt�re est constitu� de deux moteurs
li�es par un axe rigide sur lequel sont dispos�s un gyroscopes et un
acc�l�rom�tre � deux degr�s de libert�s. La stabilisation consiste �
maintenir l'axe horizontal attach� au banc en son centre de
gravit�. Le banc permettant un seul degr� de libert� (rotation de
l'axe autour de son centre de gravit� dans un plan fix�.

\begin{enumerate}
\item La premi�re t�che consiste � mod�liser et � simuler la dynamique
  d'un demi d'h�licopt�re (que l'on appelle aussi balan�oire). La
  figure \ref{helicoiter} montre la demi balan�oire. La simulation et
  la conception de la commande est r�alis�e gr�ce � un logiciel de
  calcul num�rique (notre choix s'est port� sur le logiciel Scilab qui
  est l'�quivalent de Matlab mais qui est libre). La commande est du
  type PID, placement de p�les ou LQG. La balan�oire est mod�lis�
  comme un syst�me lin�aire multi entr�es (acc�l�rom�tre, gyroscope)
  multi sorties (moteur/h�lice). Le but de cette t�che est de simuler
  la dynamique de la balan�oire et sa stabilisation � l'horizontale.

\item La deuxi�me t�che consiste � r�aliser la communication entre
  l'ordinateur et un microcontr�leur embarqu� par leur port s�rie. Le
  microcontr�leur r�alise les entr�es-sorties et sous-traite �
  l'ordinateur~:

  \begin{itemize}
  \item les calculs de la loi de commande (fonctionnement normal),
  \item l'identification des param�tres le d�bogage de la loi de
    commande (mode d�veloppement).
  \end{itemize}

  En effet, un PC standard dispose d'une puissance de calcul et d'un
  environnement logiciel sans commune mesure avec ceux disponibles sur
  un microcontr�leur. Cette t�che permet de se doter d'outils
  analogues � ceux disponible dans les laboratoires d'�lectronique
  disposant de g�n�rateurs de signaux, d'oscilloscope, d'analyseur de
  spectre num�riques (10 bits) dans une bande de fr�quence de 0 �
  10kHz. Le laboratoire num�rique est alors la bo�te � outils Scicos de
  Scilab qui est l'analogue de Simulink pour Matlab (�diteur de blocs
  diagrammes).

\item La troisi�me t�che consiste � construire (mat�riellement) la
  demi balan�oire avec le microcontr�leur embarqu�. Le bloc diagramme
  'dynamique' du logiciel de simulation est alors remplac� par la
  'v�ritable' dynamique de la balan�oire acquise gr�ce au
  microcontr�leur embarqu�. Scicos calcule le feed-back en faisant des
  calculs flottants et renvoie le r�sultat au microcontr�leur qui
  impl�mente en ligne le r�sultat. En fin de compte le microcontr�leur
  de la balan�oire fait l'acquisition des donn�es du gyroscope, les
  envoie � l'ordinateur. Ce dernier calcule la commande et renvoie le
  r�sultat au microcontr�leur. Celui-ci envoie par sa sortie PWM la
  commande au moteur (et donc � l'h�lice). La balan�oire doit se
  stabiliser.
\end{enumerate}

\subsection*{R�alisation de l'h�licopt�re complet}

Une r�alisation compl�te du H4H sera faite en utilisant la
m�thodologie mise au point sur la balan�oire. Elle se fera en
plusieurs �tapes.
\begin{enumerate}
\item R�alisation du simulateur de l'h�licopt�re complet avec la
  r�gulation des ses trois axes de libert�s.
\item R�alisation d'une autre balan�oire identique � la premi�re.
  Puis assemblage de l'h�licopt�re.
\item R�alisation du circuit �lectronique assurant la stabilisation du
  lacet en utilisant l'information d'un gyroscope suppl�mentaire.
\item Teste sur le banc d'essai (sur une rotule).
\item R�alisation de l'�lectronique de stabilisation de l'altitude et
  essaie de vol libre.
\end{enumerate}

\newpage
\pagestyle{empty}
\dessinsscaption{img/helicoiter}{0.8}
\newpage
\pagestyle{plain}

%% \newpage
%=======================================================================%
\section{Specification SART}

%=======================================================================%
\subsection{Diagramme pr�liminaire}
%=======================================================================%

\dessin{img/DiagPreleminaire}{0.7}{Diag pr�liminaire SART}{prel}

L'interface homme machine (IHM) permet � l'utilisateur d'envoyer des
consignes au calculateur de la loi de commande et de visualiser sous
forme de graphes les valeurs des �tats du syst�me.

Pour stabiliser l'h�licopt�re, nous avons besoin de cinq acquisitions
provenant de trois types de capteur (qui seront donc les entr�es de
notre asservissement). Une premi�re acquisition analogique est la
projection de l'acc�l�ration sur le plan d�fini par la croix. Elle est
fournie par un acc�l�rom�tre. Trois acquisitions de vitesse angulaire
mesurant le roulis et le lacet et le tangage sont donn�es sous forme
analogique par trois gyroscopes. Enfin, une acquisition num�rique
d'altitude est fournie par un altim�tre commutant de 0 � 1 � une
distance de 40 cm du sol.  Comme l'h�licopt�re poss�de quatre moteurs
nous devons calculer les quatre tensions d�terminant la vitesse des
moteurs (sortie de l'asservissement).

%=======================================================================%
\subsection{Diagramme � plat}
%=======================================================================%

D�taillons le processus {\tt STABILISATION d'un h�licopt�re � 4 h�lices} de
la figure (\ref{prel}). Il se d�compose en trois sous processus : deux
processus de transformation de donn�es ({\tt GERER entr�e/sortie} et
{\tt CALCULER la loi de commande} que l'on num�rotera respectivement
par $1.0$ et $2.0$) et un processus de contr�le ({\tt
  SUPERVISEUR}). Cf. figure (\ref{stab}).

\dessin{img/DiagStabilisation}{0.7}{D�tail du processus {\tt STABILISATION}}{stab}

Le microcontr�leur de l'h�licopt�re va g�rer les entr�es sorties des
cinq capteurs et des quatre moteurs (processus $1.0$). Un ordinateur
externe de l'h�licopt�re va s'occuper du processus $2.0$ du calcul de
la loi de commande (asservissement). Il va retourner au
microcontr�leur soit des vitesses moteurs � mettre sous forme PWM,
soit des consignes. L'utilisateur indique au superviseur le mode de
fonctionnement choisi. En plus des �v�nements marche et arr�t, il
existe deux modes (�v�nements) suppl�mentaires qui sont {\tt mode
  ordi} et {\tt mode autonome}. Ces modes en activant (A) ou inhibant
(I) des sous processus de $1.0$ et $2.0$ vont rendre l'h�licopt�re
ind�pendant (ou d�pendant) de l'ordinateur. La machine � �tat du {\tt
  SUPERVISEUR} est donn�e en section \ref{mef}.

\subsection{Processus de gestion des entr�es sorties}

Les acquisitions analogiques de l'acc�l�rom�tre et des gyroscopes sont
transform�es en valeurs num�riques gr�ce aux convertisseurs analogique
num�rique (CAN) du microcontr�leur. Selon l'�tat courant du {\tt
  SUPERVISEUR} (et donc de la machine � �tat), le micro\-contr�leur va
:
\begin{itemize}
\item[$\bullet$] soit dialoguer avec l'ordinateur, � savoir envoyer
  les valeurs des capteurs ({\tt Entr�es}) � l'ordinateur puis
  recevoir les vitesses des moteurs calcul�es par l'ordinateur;
\item[$\bullet$] soit recevoir les consignes venant de l'ordinateur
  puis r�aliser sa propre loi de commande.
\end{itemize}
Au final le microcontr�leur retourne une vitesse moteur sous forme
d'impulsion PWM (Pulse Width Modulation) qui sera transform�e en
tension exploitable par les moteurs gr�ce une �lectronique de
puissance utilisant des MOSFET (Metal Oxyde Semiconductor Field Effect
Transistor). Cf. figure (\ref{gio}).

\dessin{img/DiagMC}{0.7}{Processus de gestion des entr�es sorties}{gio}

\subsection{Processus d'asservissement}

\dessin{img/DiagOrdi}{0.7}{Processus d'asservissement}{ass}

Un ordinateur avec un logiciel sp�cialis� dans le calcul num�rique, va
effectuer l'asservissement de l'h�licopt�re. Il prend en entr�e les
valeurs des capteurs fournies par le microcontr�leur de l'h�licopt�re
puis retourne les vitesses des quatre moteurs n�cessaires � la
stabilisation. Une IHM (interface homme machine) va permettre �
l'utilisateur de fournir une consigne � l'h�licopt�re, de modifier la
loi de commande, de voir sous forme de graphiques les valeurs des
�tats du syst�me. Cf. figure (\ref{ass}).
\section{Machine � �tats}\label{mef}

Comme nous l'avons dit pr�c�demment, le {\tt SUPERVISEUR} est une
machine � �tats finis avec trois �tats diff�rents : arr�t, mode
autonome et mode ordinateur (c'est � dire d�pendant de l'ordinateur)
et formant un graphe fortement connexe.

L'activation en mode {\tt autonome} va inhiber l'asservissement fait
par l'ordinateur et activer celui du microcontr�leur. L'h�licopt�re
garde, cependant, la possibilit� de communiquer avec l'ordinateur
(recevoir des consignes envoyer les informations fournies par les
capteurs).

Dans le mode {\tt ordi}, la loi de commande est uniquement calcul�e
par l'ordinateur, l'h�licopt�re dialogue avec l'ordinateur mais
d�sactive ses propres calculs d'asservissements. Il envoie �
l'ordinateur les valeurs des capteurs et re�oit la consigne des
vitesses des moteurs. Cf. figure (\ref{mef}).

\dessin{img/DiagMEF}{0.7}{Machine � �tat du {\tt SUPERVISEUR}}{mef}









%% %=======================================================================%
\newpage
\section{Programmer Scicos}

%====================================================================
\subsection*{La fonction de calcul Scicos}

La fonction de calcul est la fonction C d�finie par l'utilisateur. La
seule contrainte pour quelle soit accept\'ee est qu'elle doit prendre
en param\`etre~:
\begin{itemize}
\item[$\bullet$] un flag {\tt flag} qui indique l'\'etat du bloc
  (parmis les 6 presents : debut, fin de la simulation, execution du
  calcul, ...);
\item[$\bullet$] une structure Scicos {\tt block} contenant le flux de
  donn\'ees.  Les champs les plus interessants sont :
\begin{itemize}
\item {\tt ipar} est un vecteur de parametres de type entier associes au block
  diagramme (definis par la fonction d'interface);
\item {\tt nipar} indique la taille de ce vecteur;
\item {\tt inptr} sont les donnees vectorielles fournies en entree et
  en sortie (a voir comme un bus de donnees en electronique);
\item {\tt insz} et {\tt outsz} sont les tailles des donnees.
\end{itemize}
\end{itemize}

Un exemple simple d'une fonction est :
\begin{verbatim}
void	foobar(scicos_block *block, int flag)
{
   switch (flag)
   {
     case 4:
     /* Code pour l'initialisation de la simulation */
     sciprint("Debut de la simul, j'ai %d param", block->ipar);

     case 1:
     /* Code pour le calcul des sorties */
     block->outptr[i][j] = ... block->inptr[i][j];

     case 5:
     /* Code pour la terminaison de la simulation */

     default:
     break;
   }
}
\end{verbatim}

%====================================================================
\subsection*{La fonction d'interface}
%====================================================================
La deuxi\`eme fonction est \'ecrite en langage Scilab. Elle permet de
d\'efinir comment le bloc doit interagir avec l'IHM, sa g\'eom\'etrie,
son nombre d'entr\'ees, de sorties pour les donnees et les evenements,
la d\'efinition de la fen\^etre de configuration lorsqu'un utilisateur
clique dessus, et le nom de la fonction de calcul.

%====================================================================
\subsection*{Le port serie Posix}
%====================================================================

%====================================================================
\subsection*{Edition de liens avec Scilab}
%====================================================================
L'exemple donne en annexes B decrit comment faire directement dans
l'IHM de Scilab. et d'automatiser la chose pour les utilisations
futures de Scilab.

%====================================================================
\subsection*{Automatiser des t�ches Scilab}
%====================================================================
Compiler une fois pour toute la libserie pour Scilab~:
\begin{verbatim}
    cd chemin_ou_se_trouve_les fichiers
)    exec('builder.sce');
\end{verbatim}

Le fichier {\tt scilab.star} dans le repertoire {\tt \$SCI} permet de
(configurer ??) le Scilab. On peut le modifier directement mais le
plus simple est de creer un fichier {\tt .scilab} dans le dossier {\tt
  .Scilab/version\_de\_scilab}

\begin{verbatim}
    cd chemin_ou_se_trouve_les fichiers
    load lib;
    exec loader.sce;
\end{verbatim}

et eventuellement la ligne :
\begin{verbatim}
    scicos Balancoire13quick.cos;
\end{verbatim}

%====================================================================
\subsection*{Couleur du Multi-plot}
%====================================================================

TODO: mettre beau plot

\subsubsection*{Plot 1}

\begin{tabular}{|p{1.5cm}|p{5cm}|p{2cm}|p{1.5cm}|} \hline
Num\'ero d'entr\'ee & Nom & Couleur & Num\'ero couleur \\ \hline\hline
1 & Consigne inclinaison & Noir & 1 \\ \hline
2 & Acc\'elerometre X & Vert & 3 \\ \hline
3 & PWM Moteur 1 & Bleu & 2 \\ \hline
4 & PWM Moteur 2 & Violet & 6 \\ \hline
\end{tabular}

\subsubsection*{Plot 2}
\begin{tabular}{|p{1.5cm}|p{5cm}|p{2cm}|p{1.5cm}|} \hline
Num\'ero d'entr\'ee & Nom & Couleur & Num\'ero couleur \\ \hline \hline
1 & Courant moteur 1 & Bleu clair & 4 \\ \hline
2 & Courant moteur 1 filtr\'e & Bleu & 2 \\ \hline
3 & Consigne courant moteur 1 & Noir & 1 \\ \hline
4 & Consigne courant moteur 2 & Violet & 6 \\ \hline
5 & Courant moteur 2 filtr\'e & Rouge & 5 \\ \hline
6 & Consigne courant moteur 2 & Noir & 1 \\ \hline
\end{tabular}

\subsubsection*{Plot 3}
\begin{tabular}{|p{1.5cm}|p{5cm}|p{2cm}|p{1.5cm}|} \hline
Num\'ero d'entr\'ee & Nom & Couleur & Num\'ero couleur \\ \hline \hline
1 & Gyroscope Y & Vert & 3 \\ \hline
2 & Gyroscope Y filtr\'e & Rouge & 5 \\ \hline
3 & d\'eriv\'ee acc\'el\'erom\`etre X & Bleu & 2 \\ \hline
\end{tabular}

\subsubsection*{Plot 4}
\begin{tabular}{|p{1.5cm}|p{5cm}|p{2cm}|p{1.5cm}|} \hline
Num\'ero d'entr\'ee & Nom & Couleur & Num\'ero couleur \\ \hline \hline
1 & Altimetre & Noir & 1 \\ \hline
2 & Retard Scilab &  &  \\ \hline
3 & Retard PIC 1 & &  \\ \hline
3 & Mode PIC 1 & &  \\ \hline
\end{tabular}

%====================================================================
%\subsection*{D�termination de la fonction non lin�aire i2PWM}
%====================================================================

%% On envoie une consigne PWM et on mesure le courant du moteur gr\^ace
%% \`a une resistance de puissance (comme l'explique la section
%% \ref{puis}). Comme la tension moyenne et le courant resultant sont lus
%% sur un convertisseur analogique 10 bits d'un PIC 16F876A, les valeurs
%% sont donc comprises entre 0 et 1023.

%% Conversion : PWM = 1024 == tension max.

\begin{tabular}{|*{6}{c|}} \hline
Consigne PWM & Exp. 1 & Exp. 2 & Exp. 3 & Exp. 4 & Exp. 5\\ \hline \hline
1012 & 150 & 163 & 165 &  & 295 \\ \hline
912 & 150 & 161 & 164 &  & 283 \\ \hline
812 & 150 & 158 & 163 & 765 & 280 \\ \hline
712 & 150 & 161 & 158 & 311 & 276 \\ \hline
612 & 150 & 153 & 156 & 282 & 267 \\ \hline
512 & 146 & 148 & 151 & 270 & 257 \\ \hline
412 & 137 & 140 & 143.5 & 254 & 240 \\ \hline
312 & 127 & 125.5 & 128.5 & 230 & 215 \\ \hline
212 & 115 & 108 & 102.5 & 200 & 190 \\ \hline
112 & 80 & 66 & 60 &  & 130 \\ \hline
12  & 20 & 11 & 9.75 &  & 9 \\ \hline
\end{tabular}\\[1mm]

Note~:
\begin{description}
\item[Exp\'erience 1] Moteur Speed 195, r\'esistance de
  puissance 1.8$\Omega$, tension fournie: +6V.
\item[Exp\'erience 2] Moteur Micro Speed N. 1, r\'esistance de
  puissance 1.8$\Omega$, tension fournie: +6V.
\item[Exp\'erience 3] Moteur Micro Speed N. 2, r\'esistance de
  puissance 1.8$\Omega$, tension fournie: +6V.
\item[Exp\'erience 4] Moteur Micro Speed, r\'esistance de puissance
  1$\Omega$, tension fournie: +9V.  L'experience a ete arret\'ee car
  le moteur risquait la destruction.
\item[Exp\'erience 5] Moteur Micro Speed, r\'esistance de
  puissance 1$\Omega$, tension fournie: +8.5V.
\end{description}

%% %=======================================================================%
\newpage
\section{Usinage des pieces}\label{matos}
%-----------------------------------------------------------------------%

%=======================================================================%
\subsection{Patrons}
%=======================================================================%
Gr�ce � une plaque d'alumiun, il est n�cessaire d'usiner, � partir des
deux patrons donn�s figures ci-dessous, quatre attaches de moteurs et
deux attaches pour la croix.\\[0.1cm]
\begin{minipage}[b]{.45\linewidth}
\centering\epsfig{figure=img/patron_attachemoteur, width=\linewidth}
\caption{Patron des attaches moteurs.}\label{patronattache}
\end{minipage}\hspace{2cm}
\begin{minipage}[b]{.3\linewidth}
\centering\epsfig{figure=img/patron, width=\linewidth}
\caption{Patron de la croix.}\label{patroncroix}
\end{minipage}

\subsubsection*{Patron croix}
Dans le patron fig. \ref{patroncroix}, trois formes g�om�triques sont
visibles~:

%\dessin{img/patron}{0.6}{Patron de la croix.}{patroncroix}
\begin{itemize}
\item[$\bullet$] deux petits triangles (en pointill�) sont des parties
  � �liminer;
\item[$\bullet$] deux grands triangles qui sont des parties � coller
  avec les deux autres triangles de la deuxi�me plaque;
\item[$\bullet$] un rectangle de longueur $L$ arbitraire et de largeur
  $\pi\;D$ (o� $D$ doit �tre le diam�tre des tubes de l'h�licopt�re
  donc, ici, 0.6 mm) va s'enrouller autour d'une des deux tiges/axes
  de l'appareil.
\end{itemize}

A l'aide d'un �tau, les grands triangles sont pli�s le long du segment
du rectangle et on enroulle les rectangles sur les axes de
l'helico. Avec de la colle super forte on colle l'ensemble. Avec des
pinces � linge on sert les triangles afin de bien les coller
ensemble. Le fait de limer la surface des triangles aide � mieux
coller les 2 plaques entre elles (en �liminant la couche qui prot�ge
l'aluminium). La colle prend en un jour.

\subsubsection*{Patron attache moteur}
La ligne verticale de longueur 1.88 ($\pi\;D$)
permet, comme dans la sous section \ref{sectoldattache}, de
s'enrouller autour du tube de carbone. Dans ce cas, les 2 lignes
horizontales se retrouver cote a cote. La nouveaute consiste a avoir
ajouter ces deux lignes horizontales permettent d'entourer le
moteur. Comme elle sont cote a cote, la surface entiere du moteur est
recouverte. D'ou une meilleure fixation de la colle. Une partie
facultative est le cache troue qui se rabat sur le haut du moteur.

%% %=======================================================================%
\newpage
\part{Annexes C}

%====================================================================
\section{Liste du mat\'eriel}\label{matos}

\begin{tabular}{|c|c|c|c|}\hline

\begin{minipage}{4cm}
\vspace{1mm}
\centerline{Nom}
\vspace{1mm}
\end{minipage} &

\begin{minipage}{1.5cm}
\vspace{1mm}
\centerline{Quantit\'e}
\vspace{1mm}
\end{minipage} &

\begin{minipage}{1.5cm}
\vspace{1mm}
Prix unitaire (euro)
\vspace{1mm}
\end{minipage} &

\begin{minipage}{8cm}
\vspace{1mm}
\centerline{Remarques}
\vspace{1mm}
\end{minipage}\\ \hline

\begin{minipage}{4cm}
\vspace{1mm}
Moteur \'electrique \`a broches
\href{http://shop.graupner.de/webuerp/servlet/AA?wgr=821}{Graupner
Micro Speed 6V}
\vspace{1mm}
\end{minipage} &

\begin{minipage}{1.5cm}
\vspace{1mm}
4
\vspace{1mm}
\end{minipage} &

\begin{minipage}{1.5cm}
\vspace{1mm}
5
\vspace{1mm}
\end{minipage} &

\begin{minipage}{8cm}
\vspace{1mm}
Poids: 16 g. Pouss\'ee max.: 60g en lui fournissant
8.5V et 1.2A. Equivalent au moteur
\href{http://www.gws.com.tw/english/product/powersystem/edp.htm}{GW/EDP-50}.
\vspace{1mm}
\end{minipage}\\ \hline

\begin{minipage}{4cm}
\vspace{1mm}
H\'elice \`a pas fixe
\vspace{1mm}
\end{minipage} &

\begin{minipage}{1.5cm}
\vspace{1mm}
4
\vspace{1mm}
\end{minipage} &

\begin{minipage}{1.5cm}
\vspace{1mm}
2
\vspace{1mm}
\end{minipage} &

\begin{minipage}{8cm}
\vspace{1mm}
Les h\'elices \`a pas invers\'e \'etant introuvables on est
oblig\'e d'utiliser quatre \`a pas fixe.
\vspace{1mm}
\end{minipage}\\ \hline

\begin{minipage}{4cm}
\vspace{1mm}
Gyroscope py\'ezo-\'electrique \href{}{GWS PG03}
\vspace{1mm}
\end{minipage} &

\begin{minipage}{1.5cm}
\vspace{1mm}
3
\vspace{1mm}
\end{minipage} &

\begin{minipage}{1.5cm}
\vspace{1mm}
25 sur Ebay
\vspace{1mm}
\end{minipage} &

\begin{minipage}{8cm}
\vspace{1mm}
Ce qui nous int\'eresse, ici, est uniquement le capteur gyro Gyrostar
\href{http://www.murata.com/catalog/s42e3.pdf}{ENC-03}. C'est la
m\'ethode la plus \'economique (en prix et en facililt\'e des
soudures) d'avoir un ENC-03.
\vspace{1mm}
\end{minipage}\\ \hline

\begin{minipage}{4cm}
\vspace{1mm}
Acc\'el\'erom\`etre 2 axes ADXL320
\vspace{1mm}
\end{minipage} &

\begin{minipage}{1.5cm}
\vspace{1mm}
1
\vspace{1mm}
\end{minipage} &

\begin{minipage}{1.5cm}
\vspace{1mm}
5
\vspace{1mm}
\end{minipage} &

\begin{minipage}{8cm}
\vspace{1mm}
C'est le capteur qui indique la verticale. Vu sa petite taille (4x4
mm) il est difficile \`a souder manuellement.
\vspace{1mm}
\end{minipage}\\ \hline

\begin{minipage}{4cm}
\vspace{1mm}
T\'el\'em\`etre GP2Y0D340
\vspace{1mm}
\end{minipage} &

\begin{minipage}{1.5cm}
\vspace{1mm}
1
\vspace{1mm}
\end{minipage} &

\begin{minipage}{1.5cm}
\vspace{1mm}
15
\vspace{1mm}
\end{minipage} &

\begin{minipage}{8cm}
\vspace{1mm}
Passe de l'\'etat 0 \`a 1 lorsqu'un obstacle est \`a moins de 40 cm.
\href{http://info.hobbyengineering.com/specs/SHARP-gp2y0d340_j.pdf}{Datasheet}.
\vspace{1mm}
\end{minipage}\\ \hline

\begin{minipage}{4cm}
\vspace{1mm}
Microcontr\^oleur PIC 16F876A
\href{http://ww1.microchip.com/downloads/en/DeviceDoc/39582b.pdf}{(datasheet)}
\vspace{1mm}
\end{minipage} &

\begin{minipage}{1.5cm}
\vspace{1mm}
2
\vspace{1mm}
\end{minipage} &

\begin{minipage}{1.5cm}
\vspace{1mm}
4
\vspace{1mm}
\end{minipage} &

\begin{minipage}{8cm}
\vspace{1mm}
On peut choisir entre 1 dsPIC ou 2 PIC 16F876A vu que ...
\vspace{1mm}
\end{minipage}\\ \hline

\begin{minipage}{4cm}
\vspace{1mm}
Microcontr\^oleur dsPIC \href{}{30F3011}
\vspace{1mm}
\end{minipage} &

\begin{minipage}{1.5cm}
\vspace{1mm}
1
\vspace{1mm}
\end{minipage} &

\begin{minipage}{1.5cm}
\vspace{1mm}
5
\vspace{1mm}
\end{minipage} &

\begin{minipage}{8cm}
\vspace{1mm}
... le dsPIC a quatre sorties PWM alors que le PIC en
a que deux.
\vspace{1mm}
\end{minipage}\\ \hline

\begin{minipage}{4cm}
\vspace{1mm}
Programmeur (ds)PIC
\vspace{1mm}
\end{minipage} &

\begin{minipage}{1.5cm}
\vspace{1mm}
1
\vspace{1mm}
\end{minipage} &

\begin{minipage}{1.5cm}
\vspace{1mm}
30 sur Ebay
\vspace{1mm}
\end{minipage} &

\begin{minipage}{8cm}
\vspace{1mm}
On peut se construire
son propre programmeur
\href{http://www.jdm.homepage.dk/newpic.htm}{JDM} PIC.
\vspace{1mm}
\end{minipage}\\ \hline

\begin{minipage}{4cm}
\vspace{1mm}
Convertisseur USB/port s\'erie (ou un simple c\^able s\'erie)
\vspace{1mm}
\end{minipage} &

\begin{minipage}{1.5cm}
\vspace{1mm}
1
\vspace{1mm}
\end{minipage} &

\begin{minipage}{1.5cm}
\vspace{1mm}
30
\vspace{1mm}
\end{minipage} &

\begin{minipage}{8cm}
\vspace{1mm}
Pour la communication avec le PC.
\vspace{1mm}
\end{minipage}\\ \hline

\begin{minipage}{4cm}
\vspace{1mm}
Composants \'electroniques
\vspace{1mm}
\end{minipage} &

\begin{minipage}{1.5cm}
\vspace{1mm}

\vspace{1mm}
\end{minipage} &

\begin{minipage}{1.5cm}
\vspace{1mm}

\vspace{1mm}
\end{minipage} &

\begin{minipage}{8cm}
\vspace{1mm}
(AOP, transistors, Mosfet, diodes de roue libre, resistances de
puissances, max232, connectique port s\'erie ...).
\vspace{1mm}
\end{minipage}\\ \hline

\begin{minipage}{4cm}
\vspace{1mm}
Tube de carbone creux (diam\`etre: 6mm; longueur: $>=$
40cm)
\vspace{1mm}
\end{minipage} &

\begin{minipage}{1.5cm}
\vspace{1mm}
1
\vspace{1mm}
\end{minipage} &

\begin{minipage}{1.5cm}
\vspace{1mm}
?
\vspace{1mm}
\end{minipage} &

\begin{minipage}{8cm}
\vspace{1mm}
Servent \`a construire les axes de l'h\'elicopt\`ere.
\vspace{1mm}
\end{minipage}\\ \hline

\begin{minipage}{4cm}
\vspace{1mm}
Plaque d'aluminium (10x10cm)
\vspace{1mm}
\end{minipage} &

\begin{minipage}{1.5cm}
\vspace{1mm}
1
\vspace{1mm}
\end{minipage} &

\begin{minipage}{1.5cm}
\vspace{1mm}
?
\vspace{1mm}
\end{minipage} &

\begin{minipage}{8cm}
\vspace{1mm}
Sert \`a fixer (unifier) les deux axes de l'h\'elicopt\`ere.
\vspace{1mm}
\end{minipage}\\ \hline

\begin{minipage}{4cm}
\vspace{1mm}
Tube rond de laiton 2x0,3-1m
\vspace{1mm}
\end{minipage} &

\begin{minipage}{1.5cm}
\vspace{1mm}
1
\vspace{1mm}
\end{minipage} &

\begin{minipage}{1.5cm}
\vspace{1mm}
?
\vspace{1mm}
\end{minipage} &

\begin{minipage}{8cm}
\vspace{1mm}
Sert \`a construire les pieds qui rentreront dans les tubes de
carbone.
\vspace{1mm}
\end{minipage}\\ \hline

\begin{minipage}{4cm}
\vspace{1mm}
Total :
h\'elico 4 h\'elices
\vspace{1mm}
\end{minipage} &

\begin{minipage}{1.5cm}
\vspace{1mm}
1
\vspace{1mm}
\end{minipage} &

\begin{minipage}{1.5cm}
\vspace{1mm}
170
\vspace{1mm}
\end{minipage} &

\begin{minipage}{8cm}
\vspace{1mm}
C'est le prix
d'un X-UFO
\vspace{1mm}
\end{minipage}\\ \hline

\end{tabular}


\subsection*{Remarques}
Les anciens moteurs \'electrique \`a balais a courant continu
\'etaient
\href{http://shop.graupner.de/webuerp/servlet/AA?wgr=821}{Graupner
  Speed 195}. Ils pesent 7 grammes et leur pouss\'ee max. est de 30g
en leur fournissant ??V et ??A. Equivalent au moteur
\href{http://www.gws.com.tw/english/product/powersystem/edp.htm}{GW/EDP-05}. Ils
ont ete remplace par les Micro-Speed plus rapides.


\section{PIC}
%\section{Trames de donn�es}

copier coller le tableau des trames html.

Expliquer pourkoi ANC sur 8 bits c mieux que 10 bits : -- la precision
on s'en fout (afficher plot zoom sur une courbe) -- allourdi le
npmbres de bits a envoyer.

AJOUTER photo Comm inter composants

DIRE PWM 5khz (vitesse max PIC) mais que si on peut il faut augmenter
la frequence car quand on filtre avec une bobine on a $1/Cs$ et quand
$s$(=5khz) TRES grand le resultat tend vers 0.

%% %=======================================================================%
\newpage
%=======================================================================%
\part*{Bibliographie}\label{biblio}
\addcontentsline{toc}{part}{Bibliographie}
\begin{thebibliography}{999999999}

\subsection*{Page web}
\bibitem{qq}L'�volution de ce projet peut �tre suivie sur la page web: \\
http://quentin.quadrat.free.fr/index-fr.html


%==========================================================================
\subsection*{Les autres projets h�licopt�res � quatre h�lices}
%==========================================================================
\bibitem{xbird}\emph{http://forum.xbird.org/}

L'indispensable forum en fran\c cais o� tous les passion�s
d'h�licopt�res 4 h�lices viennent pour partager leur connaissance.

\bibitem{grzflyer}\emph{http://perso.wanadoo.fr/grzflyer/}

Site en fran�ais sur la construction d'un draganflyer. Il donne les
caract�ristiques du mat�riel �lectronique et m�canique. L'inconv�nient est
qu'il donne aucune explication concernant la r�alisation du circuit
�lectronique et du mod�le physique.

\bibitem{robovolint}\emph{http://www.irccyn.ec-nantes.fr/hebergement/Robvolint/}

Projet fran\c cais sur la construction d'un h�licopt�re � quatre
h�lices intelligent pour le vol � l'int�rieur de batiments o� l'acc�s
est innaccessible par des humains (accident�, contamin�). Ce projet
est r�alis� par des partenaires tels que l'INRIA, le CNRS, le CEA, ...


\bibitem{adi}\emph{http://homepages.paradise.net.nz/jameskea/}

Site en anglais montrant un projet abouti d'h�licopt�re. La photo
de la couverture de ce rapport provient de ce site.

\bibitem{xfly}\emph{http://perso.orange.fr/pjdag/nico/index.html}

Un autre projet fran\c cais de plateforme volante (non fini). Des
explications tr�s claires y sont donn�es sur l'�lectronique, la
programmation des dsPIC (en C), la m�canique.

\bibitem{epfl}\emph{http://asl.epfl.ch/research/projects/VtolIndoorFlying/indoorFlying.php}
\bibitem{profepfl}\emph{http://asl.epfl.ch/member.php?SCIPER=149618}

Projet �tudi� et r�alis� par des �l�ves et des professeurs de l'Ecole
Polytechnique F�d�rale de Lausanne (EPFL).

\bibitem{helicodesign} Samir Bouabdallah, Pierpaolo Murrieri, Roland Siegwart, \emph{Design and Control of an Indoor Micro Quadrotor}
 \bibitem{helicopid} Samir Bouabdallah, \emph{PID vs. LQ Control
 Techniques Applied to an Indoor Micro Quadrator},

Ce sont les documents importants pour la mod�lisation
physique et la stabilisation d'un h�licopt�re. On peut les t�l�charger
respectivement sur : \begin{itemize}
\item http://asl.epfl.ch/aslInternalWeb/ASL/publications/uploadedFiles/325.pdf
\item http://asl.epfl.ch/aslInternalWeb/ASL/publications/uploadedFiles/330.pdf
\end{itemize}

\bibitem{Latour}Pierre-Olivier Latour, \emph{Computer based control system for a model helicopter},  2002.
\bibitem{Noth}Andr� Noth, \emph{Synth�se et impl�mentation d'un contr�leur pour Micro H�licopt�re � 4 rotors}, F�vrier 2004.
\bibitem{kunze}Marc Kunze, \emph{H�licopt�re indoor}, F�vrier 2003.

Documents r�aliser par des �l�ves de l'APFL (le meilleur est le
premier de la liste). On peut les t�l�charger respectivement sur :
\begin{itemize}
\item http://asl.epfl.ch/research/projects/VtolIndoorFlying/rapports/rapportLatour.pdf
\item http://asl.epfl.ch/research/projects/VtolIndoorFlying/rapports/rapportNoth.pdf
\item http://asl.epfl.ch/research/projects/VtolIndoorFlying/rapports/rapportKunze.pdf
\end{itemize}


\bibitem{josej}{\footnotesize http://www.rcgroups.com/forums/showthread.php?$\backslash$s=05bcbf26fe1c60ab6f128e5e38fc88c5\&t=297067\&pp=15}

Forum anglais, o� le concepteur explique comment il a construit son
h�licopt�re. Cet h�licopt�re est de petite taille et il n'est pas tr�s
stable. Il doit �tre stabiliser avec la t�l�commande.


\bibitem{draganflyer}Site du Draganflyer~: \emph{http://www.rctoys.com/draganflyer5.php}
\bibitem{engager}Site de l'Engager~: \emph{http://hobby.keyence.co.jp/english/saucer.html}
\bibitem{xufo}Site de l'X-UFO~: \emph{http://www.firebox.com/?dir=firebox\&action=product\&pid=1024}
\bibitem{microdrones}Site des microdrones~: \emph{http://www.microdrones.com/}
\bibitem{tribelles}Site des tribelles~: \emph{http://ovirc.free.fr/Tribelle.php}

Sites commerciaux o� l'ont peut acheter son h�licopt�re
radio-command�.  Les microdrones sont s�rement les plus belles
plateforme volantes mais aussi les plus ch�res. Les tribelles sont des
h�licopt�res � trois h�lices.


\bibitem{soucoupe} \emph{http://www.rcgroups.com/forums/showthread.php?t=394375}

Faire voller une soucoupe volante est �galement possible. \og \emph{I want
to believe} \fg


%---------------------------
\subsection*{Logiciels utilis�s}
%\hrule\vspace{0.5cm}
\bibitem{Scilab}La page principale de Scilab : \emph{http://www-rocq.inria.fr/syndex/}

Scilab est un logiciel libre pour le calcul scientifique, c'est
un interpr�teur de langage manipulant des objets math�matiques typ�s
dynamiquement. Il inclut de nombreuses fonctions sp�cialis�es pour
le calcul num�rique organis�es sous forme de librairies ou de
boites � outils qui couvrent des domaines tels que la simulation,
l'optimisation, et le traitement du signal et du contr�le.

\bibitem{Scicos}La page principale de Scicos : \emph{http://www.scicos.org/}

Une des boites � outils les plus importantes de Scilab est
Scicos. C'est un �diteur graphique de bloc diagrammes
permettant de mod�liser et de simuler des syst�mes dynamiques. Il
est particuli�rement utilis� pour mod�liser des syst�mes o\`u
des composants temps-continu et temps-discret sont inter-connect�s.

\bibitem{SynDEx}La page principale de SynDEx : \emph{http://www-rocq.inria.fr/syndex/}

SynDEx est un logiciel de CAO mettant en oeuvre la m�thodologie Ad�quation
Algorithme Architecture (AAA) pour le
prototypage rapide et l'optimisation de la mise en oeuvre
d'applications distribu�es temps r�el embarqu�es. A partir d'un
algorithme et d'une architecture donn�s sous forme de graphe SynDEx
g�n�re une impl�mentation distribu�e de l'algorithme en macro-code m4.

\bibitem{KTechlab}La page principale de KTechlab: \emph{http://ktechlab.org/}
\bibitem{Qucs}La page principale de Qucs : \emph{http://qucs.sourceforge.net/screenshots.html}

KTechlab et Qucs sont des simulateurs de circuits int�gr�s.


%==========================================================================
\subsection*{Documentation des logiciels}
%==========================================================================
\bibitem{Chancelier}Stephen L. Campbell, Jean-Philippe Chancelier and Ramine Nikoukhah, \emph{Modeling and Simulation in Scilab/Scicos}, Springer, 2005.

Ce livre (en Anglais) est un tutoriel sur l'utilisation de Scilab et de Scicos et s'attarde sur ses outils de simulation et de mod�lisation.

\bibitem{scibook}

Ce livre (en Fran�ais) pourvoie un autre tutoriel sur l'utilisation de Scilab et de Scicos en plus simple mais il date un peu.

\bibitem{Yves}Thierry Grandpierre, Christophe Lavarenne, Yves Sorel, \emph{Mod�le d'ex�cutif distribu� temps r�el SynDEx}, INRIA, 1998.

Ce document s'adresse aux concepteurs d'applications distribu�es temps r�el embarqu�es, qui d�sirent optimiser l'implantation de leurs algorithmes de commande et de traitement du signal et des images sur des architectures multiprocesseurs. Il s'adresse ensuite plus particuli�rement aux utilisateurs du logiciel SynDEx qui supporte la m�thodologie AAA.

\bibitem{genex}Yves Sorel, \emph{G�n�ration automatique d'ex�cutifs distribu�s temps r�el embarqu�s optimis�s pour SoC}

G�n�ration automatique d'ex�cutifs avec le logiciel SynDEx.


%==========================================================================
\subsection*{Cours sur l'automatique}
%==========================================================================
\bibitem{Faure}Pierre Faure et Michel Depeyrot, \emph{El�ments d'automatique}, Dunod, 1974.

Livre sur l'automatique, complet mais assez dur pour d�buter.

\bibitem{Astrom} Karl Johan \.Astr\"om  \emph{Control System Design} ME155A.

Karl Johan est un personnage important dans l'histoire de
l'automatique. Ses cours sont t�l�chargeables sur son site
\emph{http://www.control.lth.se/$\sim$kja/} Ils sont complets et tr�s biens
pour d�buter l'automatique (Lecture Note 1).

\bibitem{assermoteur}

\bibitem{AN905} Application Notes 905 de Microchip sur la structure
des moteurs � balais.

%\bibitem{ordo}Andreas Ermedahl, \emph{Schedulability Analysis Assignment}, 2004.
%\bibitem{Ordo}Philipe Baptiste, Emmanuel Neron, Francois Soud \emph{Mod�les et algorithmes en ordonnancement}, Ellipses 2004.

%==========================================================================
\subsection*{Cours sur les microcontr�leurs PIC et sur l'�lectronique}
%==========================================================================
\bibitem{bigonoff}\emph{http://www.abcelectronique.com/bigonoff/}

Le site de Bigonoff contenant tous les cours INDISPENSABLES et
gratuits pour ma�triser la programmation assembleur des PIC de
Microchip (16F84, 16F876, 16F877, 16F87x(A), 18Fxx8). Un grand merci
pour Bigonoff.

\bibitem{microchip}\emph{http://www.microchip.com/}

Microchip fournissent de la documentation technique gratuite
(\emph{Application Notes}) sur des dommaines vari�s (contr�le d'un
moteur, programmation PIC, ...)


\bibitem{Tbot}\emph{http://thomas.cremel.free.fr/wiki/wikiread.php?LaFaq}
FAQ concernant les probl�mes les plus fr�quemment
rencontr�s en robotique et �lectronique (niveau d�butant).

\bibitem{jdm}\emph{http://www.jdm.homepage.dk/newpic.htm}

Sch�ma �lectronique du programmateur JDM permettant de flasher un PIC.


\bibitem{aop1}\emph{http://perso.orange.fr/e-lektronik/LEKTRONIK/C8.htm}
\bibitem{aop2}\emph{http://etronics.free.fr/dossiers/analog/analog50/ampliop.htm}
\bibitem{aop3}\emph{http://licencer.free.fr/AOP.html}

Cours sur les AOP.


\bibitem{serie1}\emph{http://www.masoner.net/articles/async.html}
\bibitem{serie2}\emph{http://www.aurel32.net/elec/port\_serie.php}
\bibitem{serie3}\emph{http://www.easysw.com/~mike/serial/serial.html}

Programmation Linux du port s�rie.

%==========================================================================
\subsection*{Autres}
\bibitem{graupner}Page web des moteurs � courant continu de Graupner~:

\emph{http://shop.graupner.de/webuerp/servlet/AA?wgr=821}
\bibitem{gws}Page web des moteurs � courant continu de GWS~:

\emph{http://www.gws.com.tw/english/product/powersystem/edp.htm}
%==========================================================================
\end{thebibliography}

\end{document}









