%%
%% quadrat_q-rapport.tex for AOSTE in /Users/crapouille
%%
%% Made by Quentin Quadrat
%% Mail   <quentin.quadrat@free.fr>
%%
%% Started on  Thu May  3 21:26:44 2007 Quentin Quadrat
%% Last update Sun Jun 10 00:14:13 2007 Quentin Quadrat
%%

\documentclass[a4paper, 11pt]{report}
\usepackage{fullpage}
\usepackage[latin1]{inputenc}
\usepackage[french]{babel}
\usepackage{amsmath}
\usepackage{graphicx}
\usepackage{picins}
\usepackage{hyperref}
%\usepackage{makeidx}
%\usepackage{showlabels}
\usepackage{epsfig}
\hypersetup{
backref=true,    %permet d'ajouter des liens dans...
pagebackref=true,%...les bibliographies
hyperindex=true, %ajoute des liens dans les index.
colorlinks=true, %colorise les liens
breaklinks=true, %permet le retour ? la ligne dans les liens trop longs
urlcolor= blue,  %couleur des hyperliens
linkcolor= blue, %couleur des liens internes
bookmarks=true,  %cr�� des signets pour Acrobat
bookmarksopen=true, %si les signets Acrobat sont cr��s, les afficher compl�tement.
pdftitle={Projet H4H}, %informations apparaissant dans
pdfauthor={Quentin QUADRAT},     %dans les informations du document
pdfsubject={Mac OS X}          %sous Acrobat.
}

\usepackage{boites}

\makeatletter
\newenvironment{moncadre}{%
  \renewcommand*{\bkvz@right}{}%
  \renewcommand*{\bkvz@top}{}%
  \renewcommand*{\bkvz@bottom}{}%
 \breakbox}{\endbreakbox}
\makeatother 

%%%%%%%%%%%%%%%%%%%%%%%%%%%%%%%%%%%%%%%%%%%%%%%%%%
\newcommand{\dessin}[4]{
\begin{figure}[htb]
\centering
\includegraphics[scale= #2]{#1}
\caption{#3}
\label{#4}
\end{figure}}

%%%%%%%%%%%%%%%%%%%%%%%%%%%%%%%%%%%%%%%%%%%%%%%%%%
\newcommand{\dessinsscaption}[2]{
\begin{figure}[htb]
\centering
\includegraphics[scale= #2]{#1}
\end{figure}}

%%%%%%%%%%%%%%%%%%%%%%%%%%%%%%%%%%%%%%%%%%%%%%%%%%
\newtheorem{remarque}{Remarque} \newcommand{\AAA}{{\mathcal A}}
\newcommand{\BB}{{\mathcal B}} \newcommand{\CC}{{\mathcal C}}
\newcommand{\DD}{{\mathcal D}} \newcommand{\EE}{{\mathcal E}}
\newcommand{\FF}{{\mathcal F}} \newcommand{\GG}{{\mathcal G}}
\newcommand{\HH}{{\mathcal H}} \newcommand{\II}{{\mathcal I}}
\newcommand{\JJ}{{\mathcal J}} \newcommand{\KK}{{\mathcal K}}
\newcommand{\LL}{{\mathcal L}} \newcommand{\MM}{{\mathcal M}}
\newcommand{\NN}{{\mathcal N}} \newcommand{\OO}{{\mathcal O}}
\newcommand{\PP}{{\mathcal P}} \newcommand{\QQ}{{\mathcal Q}}
\newcommand{\RR}{{\mathcal R}} \newcommand{\SSS}{{\mathcal S}}
\newcommand{\TT}{{\mathcal T}} \newcommand{\UU}{{\mathcal U}}
\newcommand{\VV}{{\mathcal V}} \newcommand{\WW}{{\mathcal W}}
\newcommand{\XX}{{\mathcal X}} \newcommand{\ZZ}{{\mathcal Z}}
\newcommand{\bbR}{{\mathbb R}} \newcommand{\bbD}{{\mathbb D}}
\newcommand{\bbO}{{\mathbb O}} \newcommand{\bbS}{{\mathbb S}}
\newcommand{\bbE}{{\mathbb E}} \newcommand{\bbN}{{\mathbb N}}
\newcommand{\bbM}{{\mathbb M}} \newcommand{\bbV}{{\mathbb V}}
\newcommand{\bbC}{{\mathbb K}} \newcommand{\bbF}{{\mathbb F}}
\newcommand{\bbP}{{\mathbb P}}
\newcommand{\www}{{\mathfrak w \;}} \newcommand{\fff}{{\mathfrak f \;}}
\newcommand{\nnn}{{\mathfrak n \;}} \newcommand{\aaa}{{\mathfrak a \;}}
\newcommand{\hhh}{{\mathfrak h \;}}
%%%%%%%%%%%%%%%%%%%%%%%%%%%%%%%%%%%%%%%%%%%%%%%%%%

\begin{document}
%\pagestyle{empty}
\title{\vskip -3cm
  \Large \sl \bf  Quentin QUADRAT\\
    quadra\_q, UID 17115, promo 2007 \\
    \vspace{0.5cm}\hrule\vspace{0.5cm}
      \dessinsscaption{figures/epita}{0.2}
      \large EPITA \\
  \large \textbf{�cole Pour l'Informatique et les Techniques
  Avanc�es} \\
  \vskip 3.5cm
  {\Large RAPPORT DE STAGE}\\
  \vskip 1cm
  {\LARGE {\textsc{D�veloppement d'un train virtuel de CyCabs avec Scilab/Scicos/SynDEx}}}\\
    \vskip 2cm
    1er Janvier, 1er Juillet 2007\\
  \vskip 3.5cm
}

\date
{\vskip -3.5cm
  \vspace{15mm}
  Supervis� par\\
  \vskip 0.25cm
  {\Large \sl \bf Yves SOREL } \\
  INRIA - Domaine de Voluceau - Rocquencourt B.P.105\\
  78153 Le Chesnay Cedex - France\\
  T�l: (1) 39 63 52 60 - email: Yves.Sorel@inria.fr\\
  \vspace{0.5cm}\dessinsscaption{../figures/inria}{0.2}
  \large INRIA \\
  \large \textbf{Institut National de Recherche en Informatique et en Automatique} \\
  \vspace{0.5cm}\hrule
}

\maketitle
\strut\thispagestyle{empty}
\setcounter{page}{0}

\tableofcontents
%%====================================================================
%%====================================================================

%%====================================================================
%%====================================================================
\chapter{Introduction}
%\addcontentsline{toc}{part}{Introduction}
%\markboth{\uppercase{Introduction}}{\uppercase{Introduction}}
\section{Contexte et but du stage}
Ce rapport pr�sente le travail r�alis� dans le cadre du stage de 6 mois de fin
d'�tude pour l'EPITA, intitul� ``D�veloppement d'un train virtuel de CyCabs avec Scilab/Scicos/SynDEx'' et s'est d�roul� � l'INRIA Rocquencourt sous la direction de Yves Sorel,
responsable du projet AOSTE dont l'�quipe travaille sur des mod�les et
des m�thodes pour l'analyse et l'optimisation des syst�mes temps r�el
embarqu�s. L'�quipe AOSTE est en  relation avec l'�quipe IMARA
qui travaille sur des projets de route automatis�e.

L'�quipe AOSTE d�veloppe le logiciel de CAO SynDEx qui met en oeuvre
la m�thodologie AAA pour le prototypage rapide et l'optimisation de la
mise en oeuvre d'applications distribu�es temps r�el embarqu�es. A
partir d'un algorithme et d'une architecture donn�s sous forme de
graphe, SynDEx g�n�re une impl�mentation distribu�e de l'algorithme en
macro-code. La  validation de l'algorithme
a �t�  faite avec les logiciels de mod�lisation/simulation Scilab/Scicos,
d�velopp�s par l'�quipe SCILAB/METALAU.

Le CyCab est un v�hicule �lectrique command� manuellement par un
joystick. Il sert d'application directe au logiciel SynDEx. C'est
ainsi qu'a �t� g�n�r� le code distribu� sur les processeurs embarqu�s
du CyCab pour la conduite manuelle. A pr�sent, le but est d'effectuer
un train de CyCabs li�s �lectroniquement gr�ce � une cam�ra � bas-co�t.

Le but de mon stage a �t� de poursuivre le travail men� sur le suivi automatique
de CyCabs. Il faut  impl�menter un algorithme d'estimation de la distance
entre deux v�hicules bas� sur de la d�tection de contours dans des images. 
Il faut d�terminer l'asservissement longitudinal du CyCab suiveur en
simulation avec le logiciel Scilab. Il faut enfin enfin g�n�rer, gr�ce
� la cha�ne d'outils Scilab/Scicos/SynDEx le code temps r�el,  avec ses
communications et synchronisations distribu�es sur chacun des processeurs embarqu�s. 

Ce rapport se compose de huit parties. Dans la premi�re partie, on
pr�sente l'architecture mat�rielle du CyCab (les deux noeuds � coeur
MPC, le PC embarqu�, reli�s par un bus CAN). Dans la deuxi�me partie, on pr�sente le
syst�me d'exploitation du PC
embarqu� bas�  sur un Linux temps r�el. Dans les parties trois et quatre, nous rappellerons les
�l�ments d'automatique et de traitement d'images utilis�s. Dans les parties
cinq et six, nous pr�sentons les logiciels de d�veloppement~:
Scilab/Scicos et SynDEx. Enfin, dans les parties sept et huit nous
parlons des logiciels g�n�r�s par Syndex pour faire de la conduite manuelle puis automatique.\\[0.5cm]


\section{Travail effectu�}
J'ai du explor� plusieurs domaines~:
\begin{itemize}
\item[$\bullet$] La r�tro-ing�nierie~:

\begin{itemize}
\item En espionnant les �tats du CyCab,
\item En traduisant l'application de conduite manuelle SynDEx vers Scicos et o� j'ai �t� confront� � des probl�mes des erreurs entre les signaux observ�s et simul�s dus � des probl�mes d'arrondis.
\item En traduisant des morceaux de code assembleur en langage Scilab/Scicos.
\end{itemize}

\item[$\bullet$] L'�lectronique~:
\begin{itemize}
\item En dichotomisant les noeuds MPC afin d'isoler le probl�me des pannes des cartes.
\item En faisant  de la  maintenance �lectronique (pannes � d�tecter, fils � ressouder, etc. ) 
\item En espionnant la circulation des donn�es  temps r�el sur un bus CAN,
\item En  choisissant et rempla�ant les cartes du PC embarqu�.
\end{itemize}

\item[$\bullet$] Le traitement d'images~:
\begin{itemize}
\item En r�cup�rant les images d'une cam�ra FireWire,
\item En lisant des documents sur le traitement de l'image.
\item En appliquant des filtres de traitement d'images simples comme Sobel, Laplace, ...
\item En faisant une IHM pour visualiser/d�buger le traitement de l'image servant � l'estimation de
distance entre les v�hicules.
\end{itemize}

\item[$\bullet$]L'automatique~:
\begin{itemize}
\item En d�terminant le r�gulateur  de suivi de v�hicules,
\item En appliquant  le filtrage de Kalman  au suivi de contours.
\end{itemize}

\item[$\bullet$] La programmation~:
\begin{itemize}
\item En d�couvrant le  Linux temps r�el RTAI  (installation, configuration et l'utilisation),
\item En approfondissant ma connaissance du noyau et des modules Linux,
\item En d�couvrant la biblioth�que Xlib.
\item En apprenant � me servir des logiciels Scilab, Scicos et SynDEx.
\end{itemize}
\end{itemize}
\part{Structure d'accueil et contexte}
%%====================================================================
%%====================================================================
%\part{Accueil et contexte}
\chapter{Structure d'accueil}
\section{INRIA Rocquencourt}
Faisant suite � l'IRIA cr�� en 1967, l'INRIA est un
�tablissement public � caract�re scientifique et technologique
(EPST) plac� sous la double tutelle du ministre charg� de la
Recherche et de l'Industrie.

Les missions qui lui ont �t� confi�es sont :
\begin{itemize}
\item entreprendre des recherches fondamentales et appliqu�es,
\item r�aliser des syst�mes exp�rimentaux,
\item organiser des �changes scientifiques internationaux,
\item assurer le transfert et la diffusion des connaissances et du
savoir-faire,
\item contribuer � la valorisation des r�sultats de recherches,
\item contribuer, notamment par la formation, � des programmes de
coop�ration avec des pays en voie de d�veloppement,
\item effectuer des expertises scientifiques.
\end{itemize}


L'objectif est donc d'effectuer une recherche de haut niveau, et d'en
transmettre les r�sultats aux �tudiants, au monde �conomique et
aux partenaires scientifiques et industriels.


L'INRIA accueille dans ses 6 unit�s de recherche situ�es �
Rocquencourt, Rennes, Sophia Antipolis, Grenoble, Nancy et Bordeaux,
Lille, Saclay et sur d'autres sites � Paris, Marseille, Lyon et Metz,
3 500 personnes dont 2 700 scientifiques, issus d'organismes
partenaires de l'INRIA (CNRS, universit�s, grandes �coles) qui
travaillent dans plus de 120 "projets" (ou �quipes) de recherche
communs. Un grand nombre de chercheurs de l'INRIA sont �galement
enseignants et leurs �tudiants (environ 950 pr�parent leur th�se dans
le cadre des projets de recherche de l'INRIA).

Les chercheurs en math�matiques, automatique et informatique de
l'INRIA collaborent dans les cinq th�mes suivants :

\begin{enumerate}
\item syst�mes communicants,
\item syst�mes cognitifs,
\item syst�mes symboliques,
\item syst�mes num�riques,
\item syst�mes biologiques.
\end{enumerate}

Un projet de recherche de l'INRIA est une �quipe de taille
limit�e, avec des objectifs scientifiques et une th�matique
relativement focalis�s, et un chef de projet qui a la
responsabilit� mener et coordonner les travaux de l'�quipe. Toutes
ces �quipes sont tr�s souvent communes avec des �tablissements
partenaires.

Le sujet de ce stage s'inscrit dans les activit�s du projet AOSTE :
Mod�les et M�thodes pour l'Analyse et l'Optimisation des
Syst�mes Temps-R�el Embarqu�s. Ce projet est bilocalis� �
Rocquencourt et Sophia Antipolis. La partie situ�e � Rocquencourt
s'int�resse plus particuli�rement � l'optimisation des
syst�mes distribu�s temps r�el embarqu�s.

\section{Projet AOSTE Rocquencourt}
AOSTE est l'acronyme pour "Modeling Analysis and Optimisation of Systems
with real-Time and Embedded constraints".

Les travaux de l'�quipe concernent quatre axes de recherche~:
\begin{itemize}
\item[$\bullet$] la mod�lisation de tels syst�mes � l'aide de la th�orie
  des graphes et des ordres partiels,
\item[$\bullet$] l'optimisation d'implantation � l'aide~:
\begin{itemize}
\item d'algorithmes d'ordonnancement temps r�el dans le cas
  monoprocesseur,
\item d'heuristiques de distribution et ordonnancement temps r�el
  dans le cas multicomposant (r�seau de processeurs et de circuits
  int�gr�s),
\end{itemize}
\item[$\bullet$] les techniques de g�n�ration automatique de code pour
  processeur et pour circuit int�gr�, en vue d'effectuer du
  co-d�veloppement logiciel-mat�riel,
\item[$\bullet$] la tol�rance aux pannes.
\end{itemize}

Tous ces travaux sont r�alis�s avec l'objectif de faire le lien
entre l'automatique et l'informatique en cherchant � supprimer la
rupture entre la phase de sp�cification/simulation et celle
d'implantation temps r�el, ceci afin de r�duire le cycle de
d�veloppement des applications distribu�es temps r�el
embarqu�es.

Ils ont conduit d'une part � une m�thodologie appel�e
AAA (Ad�quation Algorithme Architecture) et d'autre part � un logiciel de CAO niveau
syst�me pour l'aide � l'implantation de syst�mes distribu�s
temps r�el embarqu�s, appel� \emph{SynDEx}.
%%====================================================================
%%====================================================================
%\chapter{SynDEx}
%expliquer la hierarchie des ope. 1 fct = 1 oper specialisee.
%+ expliquer les 2 types de graphes
%+ ordre partiel ordre total.
%\section{Exemple de scheduling et de macro g�n�ration avec SynDEx}
%image Archi + Algo + schedule + M4
%mettre le vieux dessin de fct de Syn
\chapter{Contexte}
%\section{Syst�me temps r�el}
%\chapter{Syst�me temps r�el}
%==================================================
\section{Syst�mes r�actifs temps r�el embarqu�s}
On s'int�resse dans ce document � la programmation de syst�mes
informatiques pour des applications de commande et de traitement du
signal et des images, soumises � des contraintes temps r�el et
d'embarquabilit� \cite{Yves}.  Dans ces applications, le syst�me commande son
environnement en produisant, par l'interm�diaire d'actionneurs, une
commande qu'il calcule � partir de son �tat interne et de l'�tat
de l'environnement, acquis par l'interm�diaire de capteurs.

Les syst�mes informatiques �tant num�riques, les signaux
d'entr�e acquis par les capteurs, ainsi que ceux de sortie produits
par les actionneurs, sont discr�tis�s
(�chantillonnage-blocage-quantification), aussi bien dans l'espace
des valeurs que dans le temps.  La pr�cision de la commande d�pend
de la r�solution de cette discr�tisation.

R�agir trop tard peut conduire � des cons�quences
catastrophiques pour le syst�me lui-m�me ou son environnement.
Une analyse math�matique utilisant la th�orie de la commande
permet de d�terminer d'une part une borne sup�rieure sur le
d�lai qui s'�coule entre deux �chantillons (cadence), et d'autre
part une borne sup�rieure sur la dur�e du calcul (latence) entre
une d�tection de variation d'�tat de l'environnement (stimulus) et
la variation induite de la commande (r�action).

En plus de ces contraintes temps r�el, l'application est soumise �
des contraintes technologiques d'embarquabilit� et de co�t, qui
incitent � minimiser les ressources mat�rielles (architecture)
n�cessaires � sa r�alisation (l'architecture peut �tre
compos�e de plusieurs processeurs, et de circuits sp�cialis�s,
pour satisfaire les contraintes temps r�el).

%==================================================
% \section{Implantation distribu�e optimis�e}
%
% Bien que les processeurs d'usage g�n�ral soient de plus en plus
% performants, certaines applications de traitement du signal et des
% images n�cessitent une puissance de calcul largement sup�rieure
% � celle actuellement disponible sur les processeurs les plus
% rapides, utilis�s au c{\oe}ur des stations de travail.   C'est
% pourquoi, pour atteindre les objectifs impos�s par ces demandes en
% puissance de calcul, il faut recourir � des calculateurs �
% architecture parall�le. Malheureusement, les contraintes temps
% r�el et d'embarquabilit� sont parfois tellement fortes que les
% processeurs disponibles sur le march� ne suffisent plus. Cela
% conduit donc � utiliser, en compl�ment des processeurs, des
% circuits int�gr�s sp�cialis�s.

Dans ce document, comme nous nous int�ressons uniquement aux
processeurs, plut�t qu'aux circuits int�gr�s sp�cialis�s,
l'implantation de l'algorithme sur l'architecture consiste donc �
traduire (coder) l'algorithme de commande en programmes � charger
dans les m�moires des processeurs pour que ceux-ci les
ex�cutent.

%--------------------------------------------------
\section{Parall�lisation}
Pour une architecture monoprocesseur, l'algorithme serait traduit en
un seul programme, c'est-�-dire cod� en un ensemble d'instructions
ex�cut�es s�quentiellement par le s�quenceur d'instructions du
processeur.  Pour une architecture multiprocesseur, compos�e de
plusieurs s�quenceurs d'instructions op�rant en parall�le, ainsi
que de m�dias de communication leur permettant d'�changer des
donn�es, il faut partitionner l'ensemble des instructions en
fonction du nombre de s�quenceurs d'instructions, allouer un
s�quenceur d'instructions � chacun des sous-ensembles
d'instructions et enfin ajouter des instructions de synchronisation et
de transfert de donn�es, et leur allouer des m�dias de
communication, pour supporter les d�pendances de donn�es entre les
instructions de l'algorithme qui sont ex�cut�es par des
s�quenceurs d'instructions diff�rents.

Un partitionnement simple, par d�coupage lin�aire du programme
s�quentiel en �tapes successives ex�cut�es chacune par un
s�quenceur d'instructions diff�rent, permet rarement une
exploitation efficace du parall�lisme disponible de l'architecture,
car les d�pendances de donn�es entre �tapes sont alors souvent
telles que les s�quenceurs d'instructions passent une partie
importante de leur temps � ex�cuter les instructions de
synchronisation ajout�es pour supporter les d�pendances de
donn�es entre �tapes.  Pour permettre une exploitation plus
efficace du \emph{parall�lisme disponible} de l'architecture, il faut
�tendre l'\emph{ordre total}, d'ex�cution
des instructions du programme s�quentiel monoprocesseur, � un emph{ordre partiel} extrait par une analyse
de d�pendances de donn�es entre instructions, exhibant le emph{parall�lisme potentiel} de l'algorithme, et permettant une
distribution (partitionnement ou ``allocation spatiale'') et un
r�ordonnancement (``allocation temporelle'', limit�e au respect de
l'ordre partiel) individuel des instructions.

La parall�lisation est donc un probl�me d'allocation de
ressources, que l'on d�sire r�aliser de mani�re efficace,
o\`u les ressources sont les s�quenceurs d'instructions et les
m�dias de communication inter-s�quenceurs, et o\`u les t�ches �
allouer � ces ressources sont les instructions de l'algorithme ainsi
que celles de synchronisation et de communication.

%--------------------------------------------------
% \subsection{Optimisation}
% Pour un codage monoprocesseur d'un algorithme donn�, on peut choisir
% n'importe quelle architecture cible multiprocesseur, et pour chaque
% architecture choisie il existe un grand nombre d'implantations
% possibles (c'est-�-dire de distributions et d'ordonnancements des
% instructions, qui respectent l'ordre partiel).  Parmi toutes ces
% implantations possibles, seules sont �ligibles celles dont les
% performances temps r�el (calcul�es � partir des dur�es connues
% d'ex�cution des instructions et des transferts de donn�es
% interprocesseurs) respectent les contraintes temps r�el.  Parmi les
% implantations �ligibles, pour satisfaire les contraintes
% technologiques d'embarquabilit� et de co�t, il faut choisir celle
% qui minimise les ressources mat�rielles (nombre de processeurs, de
% m�dias de communication interprocesseur, et de cellules m�moire).

% Le plus difficile n'est pas de comparer les co�ts de deux
% architectures (il suffit d'en sommer les co�ts des composants), ni
% m�me de v�rifier si une implantation respecte les contraintes
% temps-r�el (ce qui n�cessite un mod�le pr�dictif des
% performances temps-r�el de n'importe quelle implantation possible),
% c'est d'�liminer rapidement les solutions inad�quates, afin
% d'aboutir dans un temps raisonnable au choix d'une bonne implantation.
% Or ce probl�me s'apparente aux probl�mes d'allocation de
% ressources, reconnus NP-complets, et est en g�n�ral de taille
% gigantesque (variant exponentiellement avec le nombre de processeurs
% et d'instructions de l'algorithme), ce qui justifie l'utilisation
% d'heuristiques pour le r�soudre.

% %--------------------------------------------------
% \subsection{Minimisation de l'ex�cutif}
% Tout d'abord, pour minimiser les ressources, on commence par minimiser
% leur gaspillage Pour cela, il faut~:
% \begin{itemize}
% \item d'une part �quilibrer la charge des ressources (m�moires,
%   s�quenceurs d'instructions et m�dias de communication),
%   c'est-�-dire parall�liser au maximum calculs et communications
%   afin de minimiser les dur�es d'inactivit� (attentes)
%   n�cessaires aux synchronisations entre calculs et communications,
% \item d'autre part minimiser les ajouts d'instructions qui prennent
%   les d�cisions d'allocation de ressources (m�moire, s�quenceurs
%   d'instructions et s�quenceurs de communications) afin
%   d'�quilibrer leur charge.
% \end{itemize}

% Pour que le gain apport� par l'optimisation de l'allocation des
% ressources (distribution et ordonnancement des calculs et des
% communications, dont d�coule la distribution des donn�es dans les
% m�moires) ne soit pas p�nalis� par le co�t de l'optimisation
% elle-m�me, il faut que celle-ci soit faite avant l'ex�cution.  Comme
% on dispose alors d'�norm�ment plus de temps que pendant
% l'ex�cution, on peut faire des optimisations plus complexes, plus
% globales et donc probablement plus efficaces.  Aux m�thodes dites
% ``dynamiques'' d'allocation de ressources, qui n�cessitent un
% ex�cutif repr�sentant un volume de code et un temps d'ex�cution
% non n�gligeables pour prendre � l'ex�cution des d�cisions
% d'allocation, on pr�f�rera donc des m�thodes plus ``statiques''
% qui consistent � \emph{synth�tiser} un ex�cutif sur mesure, d'un
% surco�t bien moindre, � partir des d�cisions de distribution et
% d'ordonnancement prises avant l'ex�cution par l'heuristique
% d'optimisation.

% %--------------------------------------------------
% \subsection{Choix de la granularit�}
% Ensuite, comme ce probl�me d'optimisation d'allocation de ressources
% a un espace des solutions � explorer variant exponentiellement avec
% le nombre de processeurs et d'instructions de l'algorithme, il faut en
% r�duire la taille afin que la dur�e d'ex�cution de l'heuristique
% d'optimisation reste acceptable.

% Aussi, comme atomes ou ``grains'' indivisibles de distribution et
% d'ordonnancement, plut�t que de consid�rer des instructions
% individuelles, on consid�rera des agr�gats d'instructions
% pr�ordonnanc�es (correspondant par exemple � des s�quences
% d'instructions issues de la compilation s�par�e de sous-programmes
% FORTRAN ou de fonctions C) qu'on appellera par la suite
% indiff�remment soit \emph{macro-instructions}, soit plus
% g�n�ralement \emph{op�rations}, et des agr�gats de cellules
% m�moire contigu�s (correspondant par exemple � des tableaux ou �
% des structures C) qu'on appellera par la suite soit \emph{macro-registres}
% soit plus g�n�ralement \emph{d�pendances (de donn�es)}.

\section{Synchronisation dans les architectures}
% Tout mod�le �tant ent�ch� d'erreurs dues aux n�cessaires
% approximations simplificatrices, les dates de d�but et de fin
% d'ex�cution des op�rations et des communications, calcul�es par
% le mod�le de pr�diction de performances lors de l'optimisation �
% partir des dur�es d'ex�cution, peuvent �tre suffisament
% diff�rentes lors de l'ex�cution pour que leur ordre relatif (entre
% s�quences parall�les) change.  Pour garantir � l'ex�cution
% l'ordre relatif entre communications et op�rations qui partagent les
% tampons m�moire des donn�es communiqu�es, on ne peut donc se
% contenter des pr�dictions de dates d'ex�cution faites pendant
% l'optimisation, il faut imposer cet ordre relatif par des
% synchronisations explicites dans l'ex�cutif.  En bref, la
% distribution et l'ordonnancement des op�rations et des
% communications sont optimis�s avant l'ex�cution, en fonction des
% dur�es d'ex�cution donn�es, mais ils sont impos�s �
% l'ex�cution ind�pendament des dur�es effectives d'ex�cution.

La synchronisation n'est pas simple entre op�rateurs
(s�quenceurs d'instructions et/ou de communications) car chacun
peut s�quencer ses op�rations ind�pendamment des autres, sauf
dans les deux cas suivants ~:
\begin{enumerate}
\item Lorsque deux op�rateurs requi�rent simultan�ment, pour
  leur micro-op�ration en cours, un acc�s � une m�me ressource
  (partag�e), comme par exemple un bus m�moire, les deux acc�s
  doivent �tre s�quentialis�s, donc l'un des deux op�rateurs
  doit, avant de commencer son acc�s, attendre que l'autre
  op�rateur ait termin� le sien, ce qui rallonge d'autant la
  dur�e d'ex�cution de la micro-op�ration mise en attente~;
%  ces interf�rences entre micro-op�rations, et donc entre
%  macro-op�rations, dues aux arbitrages d'acc�s aux ressources
%  partag�es entre op�rateurs, doivent �tre prises en compte dans
%  le mod�le pr�dictif de performances.
\item Lorsqu'une macro-op�ration consomme en donn�e le r�sultat
  produit par une autre macro-op�ration ex�\-cu\-t�e par un
  autre op�rateur, il faut que ce dernier termine l'ex�\-cu\-tion
  de la macro-op�ration productrice avant que l'autre op�rateur ne
  commence l'ex�cution de la macro-op�ration consommatrice, et
  de plus, comme les algorithmes r�actifs sont par nature
  r�p�titifs, il faut que la macro-op�ration consommatrice soit
  termin�e avant que la macro-op�ration productrice soit � nouveau
  ex�cut�e lors de l'it�ration suivante de la s�quence
  r�p�titive, tout ceci afin que les donn�es ne soient pas
  modifi�es pendant leur utilisation.
\end{enumerate}
Dans les deux cas, des op�rations qui auraient pu �tre ex�cut�es
concurremment doivent �tre ex�cut�es s�queniellement,
mais leur ordre d'ex�cution est sans influence sur le r�sultat
fonctionnel des op�rations dans le premier cas, alors qu'il doit
�tre impos� dans le second cas.

C'est la raison pour laquelle dans le premier cas il n'est pas
n�cessaire de sp�cifier les synchronisations, d'autant plus
qu'elles doivent �tre faites au niveau de la micro-op�ration,
``invisible'' au niveau macroscopique (et m�me au niveau de
l'instruction), et que leurs dates d'occurrence d�pendent des
dur�es relatives d'ex�cution des micro-op�rations ex�cut�es
concurremment.

Par contre dans le second cas, les synchronisations doivent �tre
sp�cifi�es au niveau macroscopique, ins�r�es entre les
macro-op�rations.

Ce niveau ``op�rateur'' de d�composition de l'architecture
correspond � un grain ad�quat pour le probl�me d'optimisation
d'allocation de ressources~: chaque op�rateur s�quence des
macro-op�rations de calcul et/ou de communication, et de
synchronisation.






%%====================================================================
%%==================================================================== 
%\chapter{Logiciel de d�veloppement Scilab}
%\section{Pr�sentation}
\section{Logiciel de mod�lisation et de simulation Scilab/Scicos}
%%====================================================================
%%==================================================================== 

\subsection{Diff�rents types de logiciels scientifiques}
Il existe deux types de programmes scientifiques~: -- les logiciels
alg�briques faisant essentiellement du calcul symbolique (Maple,
Mathematica, Maxima, Axiom, et MuPad), -- les logiciels de calcul
scientifique faisant essentiellement de l'analyse num�rique (Scilab,
MATLAB).

Scilab \cite{Chancelier, Scilab} est un logiciel libre pour le calcul
scientifique.  Scilab est un interpr�teur de langage manipulant des
objets typ�s dynamiquement. Il inclut de nombreuses fonctions
sp�cialis�es pour le calcul num�rique organis�es sous forme de
librairies ou de boites � outils qui couvrent des domaines tels que la
simulation, l'optimisation, et le traitement du signal et du contr�le.

Une des boites � outils les plus importantes de Scilab est Scicos
\cite{Chancelier, Scicos}. Scicos est un �diteur graphique de bloc
diagramme permettant de mod�liser et de simuler des syst�mes
dynamiques. Il est particuli�rement utilis� pour mod�liser des
syst�mes o\`u des composants temps-continu et temps-discret sont
inter-connect�s (syst�mes hybrides) comme le montre la figure (\ref{hybride}).

\dessin{figures/regulateur/hybride}{0.4}{Un syst�me hybride.}{hybride}

Ce syst�me est hybride car~: -- les blocs 3, 5 et 7 sont continus et -- les blocs 4, 6 et 2 sont discrets.

%%%==================================================================== 
%\subsection{De la simulation � l'implantation distribu�e temps r�el}\label{sss}
%%%==================================================================== 

%A FINIR !!!!!!

%Un syst�me dynamique est g�n�ralement compos� d'un \emph{processesus} et de son r�gulateur. Le processus peut �tre vu comme un mod�le physique, par exemple d'une voiture ou un avion. Cette voiture ou avion, poss�de un certains nombres de capteurs et d'actuateurs qui lui permettent de r�aliser certaines t�ches comme tourner les roues, d�placer le v�hicule, ... Un automate permet de\\[0.5cm]
%\begin{minipage}[b]{.45\linewidth}
%\centering\epsfig{figure=figures/syndex, width=\linewidth}
%\caption{Travail de Scicos.}\label{foo1}
%\end{minipage}\hspace{6mm}
%\begin{minipage}[b]{.45\linewidth}
%\centering\epsfig{figure=figures/syndex, width=\linewidth}
%\caption{Travail de SynDEx.}\label{foo2}
%\end{minipage}

%A gauche, les blocs \emph{processus}, \emph{r�gulateur}, \emph{capteurs} et \emph{actuateurs} sont mod�lis�s avec le logiciel Scicos. Une fois que le syst�me fonctionne en simulation (syst�me stable, ...), on r�cup�re les \emph{R�gulateur}, \emph{capteurs} et \emph{actuateurs} on les passent dans le logiciel SynDEx, on en tire un graphe hi�rarchique que l'on appelle \emph{algorithme}. On lui associe un graphe repr�sentant l'architecture mat�rielle (micro-contr�leurs, FPGA, ...). SynDEx r�alisera l'implantation distribu�e temps r�el.

%Un programme Scicos peut �tre traduit en un programme SynDEx gr�ce �
%un traducteur int�gr� dans l'IHM de Scilab, t�l�chargeable sur le site de SynDEx\cite{SynDEx}.

%%==================================================================== 
\subsection{Mise en place de nouveaux blocs Scicos}
%%==================================================================== 

Les blocs pr�d�finis dans les palettes Scicos permettent de construire
des sch�mas tr�s divers et de cr�er des syst�mes dynamiques hybrides
mais dans certains cas on a besoin d'une fonctionnalit� que Scicos ne poss�de pas.
Dans notre cas, nous avons besoin d'obtenir des images � partir d'une cam�ra FireWire.
Ces blocs peuvent �tre d�finis de plusieurs fa�ons (Scilab, C, Fortran), mais dans tous les cas,
Scicos a besoin de deux types de fonctions~: -- une fonction d'interface, presque toujours �crite en Scilab, pour g�rer l'interface avec l'�diteur Scicos,  -- une fonction de simulation r�alisant le
comportement dynamique du bloc.

%\begin{moncadre}
Pour ce stage,  j'ai construit de nouveaux blocs permettant d'obtenir des images � partir d'une cam�ra FireWire et de quelques filtres. J'explique ici comment cr�er les fonctions d'interface et de simulation, qui servent de patron--exemple pour cr�er d'autres blocs Scicos.
%\end{moncadre} 

%%==================================================================== 
\subsubsection*{La fonction d'interface}
%%==================================================================== 

La fonction d'interface d'un bloc d�termine non seulement sa
g�om�trie, sa couleur, le nombre et la taille de ses ports
d'entr�e-sorties, etc., mais aussi les �tats initiaux et ses param�tres.
Elle g�re une fen�tre de dialogue qui permet de ses propri�t�s, ce que
l'utilisateur peut faire en cliquant sur le bloc.

Les fonctions d'interface suivent � peu pr�s toujours le m�me patron~:
\begin{verbatim}
// Fichier: CAMERA_FIREWIRE.sci

function [x,y,typ]=CAMERA_FIREWIRE(job,arg1,arg2)
x=[];y=[];typ=[]
select job

case 'plot' then
  standard_draw(arg1)
case 'getinputs' then
  [x,y,typ]=standard_inputs(o)
case 'getoutputs' then
  [x,y,typ]=standard_outputs(o)
case 'getorigin' then
  [x,y]=standard_origin(arg1)

case 'set' then
  x=arg1;
  graphics=arg1.graphics;
  exprs=graphics.exprs;
  model=arg1.model;
  while %t do
    [ok,height,width,exprs]=getvalue(..
	['Camera FireWire (IEEE 1394)';
	 '';
	 'Donne une image RGB de taille sous';
         'forme de vecteur de taille 3 x height x width'],..
	['Height';
	'Width'],..
	 list('vec',1,'vec',1),exprs)
    if ~ok then break,end //user cancel modification
    graphics.exprs=exprs;
    if ok then
        model.ipar=[height,width]
        graphics.exprs=exprs;
        x.graphics=graphics;
        model.out = 3 * height * width;
        x.model=model;
        break
    end
  end

case 'define' then
  height = 240
  width  = 320
  model  = scicos_model()
  model.sim = list('scicos_camerafirewire',4)
  model.evtin = 1;
  model.out = 3 * 240 * 320;
  model.ipar=[height,width]
  model.blocktype='d'
  model.dep_ut=[%f %t]
  exprs=[string(height); string(width)]
  gr_i=['txt=[''Camera'';''FireWire''];';
    'xstringb(orig(1),orig(2),txt,sz(1),sz(2),''fill'')']
  x=standard_define([4 2],model,exprs,gr_i)
end
endfunction
\end{verbatim}

La figure (\ref{scicoscam}) montre ce que l'on peut obtenir apr�s compilation. Selon la valeur de {\tt job}~: 
\dessin{figures/regulateur/scicos_camera}{0.5}{La fonction d'interface obtenue apr�s compilation.}{scicoscam}

On a utilis� ici les fonctions standard pour dessiner un bloc
rectangulaire avec les cas {\tt plot} (dessiner le bloc), {\tt
  getinputs}, {\tt getoutputs} (retourner les coordonn�es des entr�es
et des sorties).

Les cas {\tt define} et {\tt set} doivent �tre adapt�s. Le premier
d�finit les valeurs initiales des param�tres et le deuxi�me g�re la
fen�tre de dialogue avec l'utilisateur. Les variables {\tt graphics}
et {\tt model} sont des structures repr�sent�es sous forme de
liste. {\tt graphics} contient des informations sur l'aspect du bloc,
comme sa taille, son emplacement, ... et {\tt model} des informations
n�cessaires pour la simulation comme le nom de la fonction de
simulation et son type, le nombre et les tailles de ports
d'entr�es-sorties, les valeurs des �tats, des param�tres, etc.

Dans le cas {\tt define}, nous cr�ons deux variables {\tt height} et
{\tt width} qui d�finissent la taille par d�faut de l'image. {\tt
  model.out} est le port de sortie qui est un vecteur de taille
$3\times240\times320$. Nous avons aucune entr�e {\tt model.in} car
nous d�finissons un capteur. {\tt model.evtin} indique que nous avons
qu'une entr�e d'horloge. {\tt model.blocktype} indique que nous avons
un bloc de type discret. L'�l�ment {\tt [\%f \%t]} indique que ce bloc
ne contient pas de d�pendance directe d'entr�-sortie, mais qu'il est
temps d�pendant.

Dans le cas {\tt set}, nous cr�ons une liste de dialogue qui permet �
l'utilisateur de modifier les variables {\tt height} et {\tt
  width}. Les valeurs de ces deux variables seront affich�es, ainsi
qu'un titre. Lorsqu'on clique sur le bouton {\tt ok} on change la
taille du port de sortie en fonction des nouvelles valeurs de {\tt
  height} et {\tt width}.
%%==================================================================== 
\subsubsection*{La fonction de simulation}
%%==================================================================== 
Le patron de la fonction de simulation en langage C est plus simple
que pour le patron de la fonction d'interface. En effet elle effectue
les t�ches suivantes selon la valeur d'un param�tre {\tt flag}~:
\begin{itemize}
\item[$\bullet$] \emph{initialisation}~: Scicos appelle ce cas une
  seule fois et au tout d�but de la simulation pour lui permettre
  d'initialiser ses �tats initiaux ou ouvrir un le port de la cam�ra.
\item[$\bullet$] \emph{terminaison}~: Scicos appelle ce cas une seule
  fois et � la fin de la simulation pour lui permettre par exemple de
  lib�rer de la m�moire ou de fermer le port de la cam�ra.
\item[$\bullet$] \emph{calcul des sorties}~: la fonction calcule ses
  sorties en fonction des valeurs de ses entr�es et de ses �tats.
\item[$\bullet$] Il existe d'autres cas comme, la mise � jour des
  �tats, calcul des dates des �v�nements de sortie, calcul de la
  d�riv�e de l'�tat continu, ...
\end{itemize}

\begin{verbatim}
/* Fichier: scicos_camera_firewire.c */

#  include <scicos/scicos_block.h>

void		scicos_camera_firewire(scicos_block *block, int flag)
{
  switch (flag)
    {
    case INITIALISATION:
      camera_firewire_open();
      break;
    case TERMINAISON:
      camera_firewire_close();
      break;
    case CALCUL_DES_SORTIES:
      camera_firewire_get_new_image(block);
      break;
    default: break;
    }
}
\end{verbatim}

La valeur de {\tt flag} est mise � jour par Scicos. J'ai cach� la
valeur des identifiants des t�ches en utilisant des {\tt define} m�me si
les d�veloppeurs de Scicos pr�f�rent manipuler directement la valeur
litt�rale de {\tt flag}.

\begin{verbatim}
#define CALCUL_DES_SORTIES      1
#define INITIALISATION          4
#define TERMINAISON             5
\end{verbatim}

On fera attention, au bouton {\tt stop} du menu Scicos qui ne termine
pas la simulation, mais la met en pause. Par cons�quent la fonction
{\tt camera\_firewire\_close} sera appel�e que si on clique ensuite
sur le bouton {\tt restart}.

Nous n'avons pas encore parler de la structure {\tt
  scicos\_block}. Elle contient toutes les informations utiles du
blocs, comme les port d'entr�es, sorties, les param�tres, les �tats~:
\begin{verbatim}
typedef struct {
int nevprt; /* binary coding of activation inputs, -1 if internal ly activated */
voidg funpt; /* pointer: pointer to the computational function */
int type; /* type of interfacing function, current type is 4 */
int scsptr; /* not used for C interfacing functions */
int nz; /* size of the discrete-time state */
double *z; /* vector of size nz: discrete-time state */
int nx; /* size of the continuous-time state */
double *x; /* vector of size nx: continuous-time state */
double *xd; /* vector of size nx: derivative of continuous-time state */
double *res; /* only used for internal ly implicit blocks. vector of size nx */
int nin; /* number of inputs */
int *insz; /* input sizes */
double **inptr; /* table of pointers to inputs */
int nout; /* number of outputs */
int *outsz; /* output sizes */
double **outptr;/* table of pointers to outputs */
int nevout; /* number of activation output ports */
double *evout; /* delay times of output activations */
int nrpar; /* number of real parameters */
double *rpar; /* real parameters of size nrpar */
int nipar; /* number of integer parameters */
int *ipar; /* integer parameters of size nipar */
int ng; /* number of zero-crossing surfaces */
double *g; /* zero-crossing surfaces */
int ztyp; /* boolean, true only if block MAY have zero-crossings */
int *jroot; /* vector of size ng indicating the presence and direction of crossings */
char *label; /* block label */
void **work; /* pointer to workspace if al location done by block */
int nmode; /* number of modes */
int *mode; /* mode vector of size nmode */
} scicos block;
\end{verbatim}

Voici un exemple de fonction pour le calcul des sorties~:
\begin{verbatim}
void    	camera_firewire_get_new_image(scicos_block *block)
{
  int		 i;
  static struct s_device_firewire device;
  static unsigned char rgb[3 * IMAGE_WIDTH * IMAGE_HEIGHT];

  /* block->ipar[0] <==> height */
  /* block->ipar[1] <==> width  */
  camera_firewire_get_frame(&device, rgb);
  for (i = 0; i < 3 * block->ipar[0] * block->ipar[1]; ++i)
    block->outptr[0][i] = rgb[i];
}
\end{verbatim}
%%==================================================================== 
\subsubsection*{Compilation des fonctions d'interface et de simulation}
%%==================================================================== 
Voici un script Scilab qui permet de compiler les fonctions
de simulation et de les lier aux fonctions d'interfaces.
\begin{verbatim}
// Fichier: builder.sce

comp_fun_lst = ['scicos_camera_firewire'];
c_prog_lst   = ['camera_scicos.c'];
prog_list    = strsubst(c_prog_lst, '.c', '.o');
lib_list     = ['libraw1394','libdc1394_control'];

ilib_for_link(comp_fun_lst, prog_list, lib_list, 'c');
genlib('lib_firewire', pwd());
\end{verbatim}

La variable {\tt comp\_fun\_lst} est une liste de cha�ne de caract�res
indiquant les noms des fonctions � utiliser. {\tt c\_prog\_lst} stocke
tous les noms des fichiers C � compiler. {\tt prog\_list} stocke les
noms des fichiers objets et dont l'extension se termine par o.
{\tt lib\_list} stocke les noms de librairies dynamiques n�cessaires �
la compilation. Ces librairies se trouve dans le r�pertoire {\tt
  usr/lib/} et doivent �tre partag�es (pour �tre ouverte lors de
l'ex�cution de la simulation).

Le script suivant permet de lancer la compilation et, si tout va bien,
permet de cr�er une palette nomm�e {\tt myblock.cosf}. Attention, je
n'ai pas bien compris comment cr�er une palette {\tt scicos\_pal}
donc il faut utiliser le menu {\tt Open as palette} pour cr�er une
palette valide avec le fichier {\tt myblock.cosf}.

\begin{verbatim}
exec('builder.sce');
create_palette(pwd());
load lib;
exec loader.sce;
scicos_pal($+1, ["IEEE 1394"; "myblock.cosf']);
\end{verbatim}

\dessin{figures/regulateur/scicos_camera2}{0.5}{Les fonctions d'interface et de simulation obtenues pendant une simulation.}{scicoscam2}
%%==================================================================== 
\subsubsection*{R�sultat}
%%==================================================================== 
Si on doit re-modifier la fonction d'interface ou de
simulation, il faut faire attention � ce que Scilab int�gre bien la
derni�re version, ce qui n'est pas toujours le cas. Par exemple si on
change la fonction d'interface, on doit obligatoirement d�truire le
bloc de la simulation pour en cr�er un nouveau, sinon l'ancienne
version ne laisse pas la place � la nouvelle.

Les figures (\ref{scicoscam}) et (\ref{scicoscam2}) montrent ce que l'on peut obtenir avec des fonctions
d'interface et de simulation.
\section{Logiciel d'implantation SynDEx}
\subsection{M�thodologie AAA}
La m�thodologie d'Ad�quation Algorithme Architecture
est bas�e sur des mod�les de graphes pour sp�cifier d'une part
l'algorithme et d'autre part l'architecture mat�rielle. La
description de l'algorithme permet de mettre en �vidence le
parall�lisme potentiel tandis que celle de l'architecture met en
�vidence le parall�lisme disponible. Cette m�thode consiste en
fait � d�crire l'implantation en terme de transformations de
graphes. En effet, le graphe mod�lisant l'algorithme est
transform� jusqu'� ce qu'il corresponde au graphe mat�riel
mod�lisant l'architecture. L'implantation de l'algorithme sur
l'architecture consiste donc � r�duire le parall�lisme potentiel
au parall�lisme disponible tout en cherchant � respecter les
contraintes temps r�el. Toutes ces transformations effectu�es
avant l'ex�cution en temps r�el de l'application, correspondent
� une distribution et � un ordonnancement des diff�rents calculs
sur les processeurs et des communications sur les liaisons physiques
inter-processeurs. C'est � partir de ces allocations spatiales et
temporelles qu'un ex�cutif va pouvoir �tre g�n�r� et
permettre l'ex�cution de l'algorithme sur l'architecture construite
avec des processeurs. Cependant, pour que cette implantation soit
vraiment efficace, il est n�cessaire de r�aliser une ad�quation
entre l'algorithme et l'architecture. Celle-ci consiste � choisir
parmi toutes les transformations propos�es celle qui optimise les
performances temps r�el. Cette m�thodologie a �t�
concr�tis�e dans un logiciel appel� SynDEx.

{\large
\begin{center}
\begin{tabular}{ccc}
& {\sc Ad�quation}\\
{\sc Algorithme} & $\longleftrightarrow$ & {\sc Architecture}\\
$\downarrow$ & & $\downarrow$\\
parall�lisme & & parall�lisme\\
potentiel & $\stackrel{reduction}{\longrightarrow}$ & disponible\\
$\downarrow$ & & $\downarrow$\\
graphe & & graphe\\
logiciel & $\stackrel{transformation}{\longrightarrow}$ & mat�riel\\
& $\Downarrow$\\
& {\sc ex�cutif}
\end{tabular}
\\
Construit ou compl�ment�\\ � partir d'un Noyau G�n�rique
\end{center}
}


\subsection{SynDEx}\label{syndexsoft}
Comme il a �t� dit en introduction, SynDEx est un outil de
d�veloppement pour l'implantation optimis�e d'algorithmes
respectant des contraintes temps r�el sur des architecture
distribu�es. A partir de graphes flot de donn�es (la description
hi�rarchique d'op�rations est possible) et d'un graphe
d'architecture mat�rielle, des heuristiques sont mises en \oe{u}vre
afin d'en d�duire une distribution et un ordonnancement optimis�
des op�rations satisfaisant les contraintes. L'ad�quation est
r�alis�e en fonction des param�tres des op�rations tels que
temps estim� de calcul ou un imp�ratif sur le type de ressources
(processeurs, media de communication) o\`u l'op�ration doit �tre
ex�cut�e. L'approche y est formelle et fond�e sur la
s�mantique des langages synchrones. Le code issu de l'ad�quation
est un macro-code (m4) qui est ensuite traduit par le macroprocesseur
standard M4 utilisant des noyaux d'ex�cutif (conf�re section \ref{noyauexec}) d�pendant des
processeur sp�cifi�s sur le graphe
d'architecture (figure \ref{oldprincipes}).

\dessin{figures/syndex/syndex}{0.4}{Principe de SynDEx}{oldprincipes}

%\subsection{Mod�les d'algorithmes et d'architectures}
\subsubsection*{Mod�les d'algorithmes}\label{sect_AlgoSimple}
Un algorithme est mod�lis� par un graphe flot de donn�es
�ventuellement conditionn� (il s'agit d'un hypergraphe orient�),
qui se compose de sommets et d'arcs. Un sommet est une op�ration et
un arc un flot de donn�es, c'est-�-dire un transfert de donn�es
entre deux op�rations.

Une op�ration peut-�tre soit un calcul, soit une m�moire
d'�tat (retard), soit un conditionnement ou encore une
entr�e-sortie. Les sommets qui ne poss�dent pas de
pr�d�cesseur sont des interfaces d'entr�e (capteurs recevant les
stimuli de l'environnement) et ceux qui ne poss�dent pas de
successeur repr�sentent des interfaces de sortie (actionneurs
produisant les r�actions vers l'environnement). Dans le cas d'une
op�ration de calcul, la consommation des entr�es pr�c�de la
production des sorties. La figure (\ref{syndexalgo}) donne un exemple de
la description d'un algorithme via SynDEx.\\[0.5cm]
\begin{minipage}[b]{.45\linewidth}
\centering\epsfig{figure=figures/syndex/algo1, width=\linewidth}
\caption{Un algorithme d'un r�gulateur simple sous SynDEx.}\label{syndexalgo}
\end{minipage}\hspace{6mm}
\begin{minipage}[b]{.45\linewidth}
\centering\epsfig{figure=figures/syndex/algo2, width=\linewidth}
\caption{Le m�me r�gulateur mais sous Scicos.}\label{scicosalgo}
\end{minipage}

Les algorithmes peuvent �tre simul�s dans un premier temps avec le logiciel Scilab et son �diteur de sch�ma bloc Scicos \cite{Scilab, Scicos} puis �tre convertis vers un algorithme SynDEx gr�ce � une passerelle \cite{SynDEx}.


\subsubsection*{Mod�les d'architectures}\label{sect_ArchSimple}
Une architecture est mod�lis�e par un graphe dont chaque sommet
repr�sente un processeur ou un m�dia de communication, et chaque
arc repr�sente une connexion entre un processeur et un m�dia de
communications (SAM ou RAM). On ne peut connecter directement deux
processeurs ou deux m�dias. Chaque sommet est une machine
s�quentielle qui s�quence soit des op�rations de calcul pour les
processeurs, soit des op�rations de communications pour les m�dias
de communications.

La figure (\ref{syndexarchi}) montre 5 processeurs de type PC en relation avec un m�dia de communication TCP/IP.
\dessin{figures/syndex/archi}{0.5}{Graphe d'architecture.}{syndexarchi}



\subsection{Heuristique de distribution et d'ordonnancement}

Une fois les sp�cifications de l'algorithme et de l'architecture
effectu�es, il est n�cessaire de r�aliser
l'ad�quation. Celle-ci est charg�e de respecter d'une part l'ordre
des �v�nements v�rifi�s lors de la sp�cification de
l'algorithme et d'autre part les contraintes temps r�el. Pour cela,
est choisie parmi toutes les transformations de graphes possibles,
celle qui optimise les performances temps r�el de l'implantation en
terme de latence. La latence ou temps de r�ponse R est la longueur
du chemin critique du graphe logiciel, dont les sommets sont valu�s
par les dur�es d'ex�cution des op�rations correspondantes y
compris celles des communications inter-processeurs.

\dessin{figures/syndex/exemple}{0.5}{Exemple simple d'algorithme et d'architecture.}{syndexexemple}

Supposons qu'on ait l'algorithme et l'architecture comme montr� sur la figure (\ref{syndexexemple}) et que l'on veuille trouver une distribution de l'algorithme. On suppose que chaque op�ration (t�che) et que chaque communication s'ex�cute en une unit� de temps.

\dessin{figures/syndex/schedul_iter1}{0.5}{Etape 1.}{etape1}

A n'ayant pas de pr�d�cesseur, on doit ex�cuter l'op�ration A en premier. A l'�tape 1, on la place soit sur le processeur $P_1$, soit sur le processeur $P_2$ (figure (\ref{etape1})). Si on choisit la premi�re solution, on peut placer l'op�ration B, soit sur  $P_1$, soit sur le $P_2$ (figure \ref{etape2})). Si on choisit la deuxi�me solution, il faut ajouter une unit� de temps pour la communication. De la m�me fa�on, si on choisit la deuxi�me solution, et que l'on place la t�che C sur un processeur, on peut obtenir la figure (\ref{etape3}). On continuera de la m�me mani�re pour le placement de la t�che D.\\[0.5cm]
\begin{minipage}[b]{.45\linewidth}
\centering\epsfig{figure=figures/syndex/schedul_iter2, width=\linewidth}
\caption{Etape 2.}\label{etape2}
\end{minipage}\hspace{6mm}
\begin{minipage}[b]{.45\linewidth}
\centering\epsfig{figure=figures/syndex/schedul_iter3, width=\linewidth}
\caption{Etape 3.}\label{etape3}
\end{minipage}

Parmi tous ces choix possibles, il faut choisir la meilleure solution, � savoir il faut s�lectionner le meilleur choix parmi les solutions possibles qui vont du choix 1.1.1.1 au choix x.x.x.x . La r�solution de ce probl�me est NP-difficile.
Afin de r�soudre ce probl�me d'optimisation du temps de r�ponse,
une heuristique a �t� d�velopp�e. Il s'agit d'un algorithme
glouton dont chaque �tape alloue une op�ration � un processeur,
route les �ventuelles communications inter-processeurs
c'est-�-dire cr�e des op�rations de communication et alloue
chacune d'elles � une liaison physique. L'ordonnancement des
op�rations de calculs ou de communication est directement d�duit
de l'ordre dans lequel elles sont allou�es.

Cette m�thode consiste donc � faire progresser au long du graphe
une coupe s�parant les op�rations d�j� plac�es sur des
processeurs de celles qui ne le sont pas encore. La progression se
fait en respectant les pr�c�dences du graphe logiciel. De toutes
les op�rations � distribuer sur la coupe et de tous les
processeurs, on choisit la paire qui optimise une fonction locale de
co�t prenant en compte :


\begin{itemize}
\item les diff�rences entre dates locales d'ex�cution au plus
  t�t et au plus tard (schedule flexibility),
\item l'allongement du temps global d'ex�cution : le temps de
  r�ponse (latence),
\item le rythme d'entr�e (cadence),
\item la capacit� m�moire.
\end{itemize}

Afin d'illustrer l'ad�quation, la figure (\ref{schedule})
montre le graphe temporel d'ex�cution d'un algorithme, tir� du tutoriel SynDEx, sur l'architecture de la figure (\ref{syndexarchi}). Le
temps se d�roule de haut en bas avec une colonne par processeur
({\tt root}, ...) ainsi qu'une colonne par m�dia de communication
({\tt bus}). Chaque op�ration de calcul est repr�sent�e
par une bo�te dont la hauteur est proportionnelle � la dur�e
d'ex�cution de l'op�ration. Chaque communication inter-processeurs
est repr�sent�e par une bo�te dont la taille est
proportionnelle � la dur�e de la communication. La communication
commence d�s que l'op�ration qui a fournit la donn�e �
transmettre est termin�e, l'op�ration qui a besoin de la donn�e
transf�r�e commence d�s que la communication est termin�e. La
valeur de la dur�e d'une it�ration du graphe est, quant � elle,
donn�e dans la fen�tre principale de SynDEx.

\dessin{figures/syndex/schedule}{0.6}{Graphe temporel pour l'application du tutorial.}{schedule}

\subsection{Noyaux d'ex�cutif}\label{noyauexec}
\subsubsection{G�n�ration d'ex�cutif}\label{kernel}
Les transformations de graphes mod�lisant le processus
d'implantation de l'algorithme sur l'architecture, permettent de
produire automatiquement des ex�cutifs temps r�el optimis�s,
d�chargeant ainsi l'utilisateur des t�ches fastidieuses de
programmation bas niveau et autorisant du m�me coup une meilleur
concentration sur les probl�mes directement li�s au programme
applicatif.

Un ex�cutif a pour r�le d'allouer les ressources de l'architecture
mat�rielle (unit�s de calcul, de m�moire et communication) au
programme d'application. Les ex�cutifs peuvent �tre class�s en
fonction de leur mani�re d'arbitrer l'allocation des
ressources. Nous avons vu pr�c�demment que cette allocation est
� la fois spatiale (distribution) et temporelle (ordonnancement). Si
les optimisations et les d�cisions que doit prendre l'ex�cutif
sont effectu�es � l'ex�cution, on dit que l'allocation est
dynamique. Par contre, si cela est fait avant l'ex�cution, on dit
que l'allocation est statique (il faut conna�tre les dur�es
d'ex�cution). Dans le cas d'une allocation statique, les ex�cutifs
sont les moins p�nalisants car dans le cas d'une allocation
dynamique, l'ex�cutif consomme une partie des ressources pour
effectuer ses d�cisions d'arbitrage et d'optimisation. C'est
pourquoi SynDEx g�n�re des ex�cutifs statiques. Il y en a autant que de processeurs dans l'architecture.

\dessin{figures/syndex/arbo}{0.5}{Arborescence des ex�cutifs
  SynDEx}{arbo}

Comme les ex�cutifs sont g�n�r�s automatiquement, l'utilisateur n'a
pas d'autre code � �crire que celui de ses op�rations de calculs
et d'entr�e/sortie. Tout le reste, c'est-�-dire la distribution,
l'ordonnancement, les appels des op�rations de calcul et
d'entr�e/sortie, les allocations de la m�moire n�cessaires aux
communications inter-op�rations est g�n�r� automatiquement �
partir des graphes logiciel et mat�riel, des r�sultats de
l'optimisation et d'un noyau g�n�rique. En effet, pour chaque
processeur, l'ex�cutif est constitu� par un assemblage
d'�l�ments d'un noyau g�n�rique d'ex�cutifs (tir�s d'une
biblioth�que syst�me) qui g�re les communications
inter-processeurs.
Chaque ex�cutif g�n�r� par SynDEx est un macro-code ind�pendant de l'architecture. Ce macro-code
est ensuite macro-process� avec des noyaux d'ex�cutifs d�pendant de l'architecture et de l'application, afin d'obtenir du code source, qui apr�s compilation sera  ex�cutable. 

\subsubsection{Arborescence des noyaux d'ex�cutifs}
Les noyaux d'ex�cutifs sous SynDEx sont divis�s en diff�rents groupes
comme le montre la figure (\ref{arbo}). Le travail de ce stage qui a consist� �
aider � la conception des noyaux applicatifs (d�pendants de l'application)  sera
expliqu� plus en d�tail dans la deuxi�me partie de ce rapport.


\part{Architecture mat�rielle et OS}
%%====================================================================
%%==================================================================== 
%\chapter{Architecture du CyCab}
\chapter{Architecture mat�rielle d'un CyCab}
\section{Histoire de la conduite automatique}
%%====================================================================
%%==================================================================== 

Dans le cadre de la route automatis�e, l'INRIA a imagin� un syst�me de
transport original de v�hicules en libre-service pour la ville de
demain. Ce syst�me de transport public est bas� sur une flotte de
petits v�hicules �lectriques sp�cifiquement con�us pour les zones o�
la circulation automobile doit �tre fortement restreinte. Pour tester
et illustrer ce syst�me, un prototype, nomm� CyCab (contraction pour
Cybar Cab), a �t� r�alis� (Figure \ref{CyCab1}).

\dessin{figures/CyCab/CyCab1}{0.5}{Un Cyber Cabi.}{CyCab1}

Les chercheurs de l'INRIA et de l'Inrets (Institut National de
Recherche sur les Transports et leur S�curit�) travaillent depuis 1991
sur de nouveaux moyens de transport intelligent pour la ville. Ils
�tudient en particulier le concept du libre-service et celui de la
voiture automatique. Les premiers r�sultats de recherche ont d�bouch�
sur le projet Praxit�le (1993-1999), qui �tait en exploitation �
Saint-Quentin-en-Yvelines. Les partenaires industriels du projet
�taient CGFTE (la filiale transports publics de Vivendi), Dassault
Electronique, EDF et Renault.
 
Dans le cadre du projet Praxit�le l'INRIA a d�montr� la faisabilit� de
la conduite automatique sous certaines conditions : cr�neau et train
de v�hicule exp�riment� sur une Ligier �lectrique instrument�e � cet
effet.
 
Pour des raisons de l�gislation et de responsabilit� ces automatismes
de conduite n'ont pas pu �tre impl�ment�s sur les Clio �lectriques de
Saint-Quentin-en-Yvelines. Le CyCab a ensuite �t� d�velopp� par
l'INRIA avec l'aide de l'Inrets, de EDF, de la RATP et de la soci�t�
Andruet S.A. pour montrer le potentiel de l'informatique dans la
conduite de v�hicules. Le CyCab est un v�hicule �lectrique � quatre
roues motrices et directrices avec une motorisation ind�pendante pour
chacune des roues et pour la direction. Pour contr�ler et commander
les 9 moteurs du CyCab (4 de traction, 1 de direction et 4 de frein),
une architecture mat�rielle a �t� choisie. Elle est constitu�e de
noeuds intelligents pouvant g�rer les diff�rents moteurs du CyCab et
r�partie autour d'un bus de terrain CAN (Controller Area Network),
tr�s r�pandu dans le monde de l'automobile.
 
Le r�le des noeuds est d'asservir les moteurs en fonction des
consignes de vitesse et de braquage qui transitent sur le bus CAN soit
en provenance de la position du joystick, soit par un programme de
planification de trajectoires. Le noeud doit donc non seulement �tre
capable de fournir la puissance n�cessaire aux moteurs, mais aussi
ex�cuter les boucles d'asservissement de vitesse ou de position. Pour
ce faire il doit prendre en compte un certain nombre d'informations en
provenance des capteurs proprioceptifs : �tat, odom�trie, fins de
course, mesures de temp�rature, de courant, ...

%% Depuis une trentaine d'ann�es, de nombreuses recherches ont �t� effectu�es
%% afin d'automatiser � diff�rents niveaux les voitures et les transports en
%% commun (�tat de l'art dans \cite{SAT}). Un projet ant�rieur de l'INRIA a d�j� 
%% propos� des r�sultats int�ressants sur les trains de v�hicules � accrochage 
%% immat�riel, appel� aussi trains de v�hicules � accrochage virtuel. Mais le 
%% mat�riel utilis� �tait tr�s co�teux (cam�ra tr�s haute r�solution, plusieurs 
%% cartes sp�cifiques telles carte d'acquisition et carte de traitement d'images).
%% Notre but est donc de faire un train de v�hicules sans lien mat�riel au co�t le 
%% plus bas possible.
%% Par la suite et pour simplifier, nous ne parlerons plus que du train de 
%% v�hicules seulement.

Un train de v�hicules est constitu� d'un v�hicule de t�te conduit par un 
chauffeur et d'autres v�hicules automatis�s, chacun suivant celui qui le 
pr�c�de. Ainsi le premier v�hicule est suivi par le deuxi�me qui � son tour est 
suivi par le troisi�me ... C'est donc une procession de v�hicules. Ce type 
d'automatisation a �t� pens� pour les conduites sur autoroute ou dans les 
p�riph�riques. Ce proc�d� a l'avantage de maximiser la vitesse 
des v�hicules ainsi que leur nombre tout en minimisant les accidents.

%% Dans le cadre de la route automatis�e \cite{LARA}, l'INRIA a imagin� un syst�me 
%% de transport original de v�hicules en libre-service pour la ville de demain. 
%% Ce syst�me de transport public est bas� sur une  flotte de petits v�hicules
%% �lectriques sp�cifiquement con\c{c}us pour les zones o� la circulation 
%% automobile doit �tre fortement restreinte. Pour tester et illustrer ce 
%% syst�me, deux prototypes, nomm�s CyCabs, ont �t� r�alis�s dans un premier temps.
%% Le premier est localis� � l'UR (Unit� de Recherche) de Rocquencourt et le 
%% second est localis� � l'UR Rh�ne-Alpes. Leszek Lisowski, ing�nieur-expert 
%% SHARP/PRAXIT�LE, a travaill� durant deux ans sur la conception et la r�alisation
%% des deux prototypes \cite{DTCyCab}. Les deux CyCabs comportent des diff�rences 
%% de conception importantes. Ainsi, autant le premier prototype donnait toute 
%% satisfaction, autant le second posait d'�normes probl�mes de fiabilit� 
%% \cite{RTCyCab}.

%% L'entreprise Robosoft commercialise les CyCabs depuis quelques ann�es et chaque 
%% UR de l'INRIA en poss�de au moins un.

\section{Travail effectu�}
Un des probl�mes que j'ai rencontr� pendant ce stage f�t le manque de documentation. J'ai
ajout� dans ce rapport ma propre documentation. J'ai �galement effectu� une mise � jour mat�rielle et logicielle
du PC embarqu� et fait quelques maintenance sur les noeuds MPC.

%% FIXME Ajouter le fichier EXCEL DES CARTES MPC

 
%%====================================================================
\section{Architecture mat�rielle}
%%====================================================================
%%====================================================================
\subsection{Caract�ristique g�n�rale d'un CyCab}

Il existe diff�rents types de CyCab.  Celui utilis� a les caract�ristiques suivantes :
\begin{itemize}
  \item[$\bullet$] longueur : 1,90 m
  \item[$\bullet$] largeur : 1,20 m
  \item[$\bullet$] poids total avec batteries : 300 kg
  \item[$\bullet$] 4 roues motrices et directrices
  \item[$\bullet$] vitesse th�orique maximale : 20 km/h
  \item[$\bullet$] autonomie : 2 heures d'utilisation continue
  \item[$\bullet$] capacit� d'accueil : 2 personnes
  \item[$\bullet$] conduite manuelle ou automatique.
\end{itemize}

\dessin{figures/CyCab/CyCab2}{0.8}{Architecture d'un CyCab.}{CyCabcoupe}

La figure (\ref{CyCabcoupe}) montre une vue en coupe de
l'architecture du CyCab qui est constitu�e de :
\begin{itemize}
\item[$\bullet$] 1 ensemble de batteries avec un gestionnaire
  automatique de charge (10) et un bouton arr�t d'urgence qui est soit
  de type poussoir (2) soit de type radiocommand� (1).
\item[$\bullet$] 2 cartes �lectroniques (5) et (6) d'acquisition de donn�es
  comprenant chacune un microprocesseur 32 bit Power PC (appel�s
  MPC555). Chaque carte permet de contr�ler 2 roues du CyCab. Nous
  reviendrons plus tard sur l'architecture de ces cartes que l'on
  appellera par la suite \emph{noeuds}.
\item[$\bullet$] 1 PC embarqu� au format rack (taille 2U), plac� sous
  le si�ge (2), poss�dant un processeur Intel cadenc� � 3 GHz, avec un
  Linux temps r�el, RTAI. L'ensemble est aliment� par une tension
  d'entr�e de -48V (350W) et non de 220V. L'�cran est situ� en (3).
\item[$\bullet$] 2 bus CAN ind�pendants~: le bus CAN 0 permet la
  communication entre les 2 MPC555 et le PC embarqu�, alors que le bus
  CAN 1 permet d'ajouter d'�ventuels futurs composants �lectronique.
\item[$\bullet$] 4 moteurs et leurs freins �lectriques (8) et (9)
  contr�l�s par 4 contr�leurs de moteur appel�s Curtis PMC 1227 (9)
  servant d'amplificateurs de puissance pour contr�ler la vitesse des
  roues. La consigne de vitesse est donn�e par une tension de 0 � 5V aux
  Curtis qui fourniront des signaux PWM ad�quats aux moteurs. Les
  Curtis prot�gent les noeuds des contre-courants des moteurs, quand
  par exemple, on les arr�te brusquement.
\item[$\bullet$] 4 d�codeurs incr�mentaux donnant la vitesse des roues (8).
\item[$\bullet$] 1 v�rin de direction �lectrique aliment� par signal PWM (7)
  faisant pivoter les 4 roues.
\item[$\bullet$] 1 encodeur absolu avec sortie SPI et donnant l'angle des
  roues.
\item[$\bullet$] 1 joystick (2) fournissant deux courants indiquant :
  -- la consigne de vitesse des roues, -- la consigne de direction
  des 4 roues.
\item[$\bullet$] Depuis ce stage, le CyCab poss�de une cam�ra type
  webcam (4) se branchant sur un port FireWire du PC embarqu�.
\end{itemize}

Nous verrons, dans les prochains chapitres, comment ces capteurs et
actionneurs sont utilis�s pour faire rouler le CyCab que ce soit pour
de la conduite manuelle ou de la conduite automatique.

%%====================================================================
\subsection{L'architecture du syst�me}
%%====================================================================
La figure (\ref{CyCabuml}), montre en formalisme UML, l'architecture
compl�te du CyCab avec ses moyens de communication, � savoir les deux
noeuds, le PC embarqu� et les deux bus CAN.

\dessin{figures/CyCab/archi}{0.5}{Architecture d'un CyCab.}{CyCabuml}

Comme nous l'avons expliqu� dans le chapitre pr�c�dent, SynDEx va distribuer le programme de conduite sur les 2
noeuds et sur le PC embarqu� qui communiqueront gr�ce au bus CAN 0.  Le
passager (ou acteur en formalisme UML) du CyCab peut communiquer avec
le PC embarqu� (clavier/souris/�cran, USB, FireWire, Ethernet, CAN 1) mais n'a
acc�s aux noeuds que par l'interm�diaire du bus CAN 1.

Comme nous le verrons, chapitre \ref{chaprtai}, une partie du
programme de conduite (manuel ou automatique) va tourner sur un Linux
temps r�el (RTAI ) servant essentiellement de timer 10 ms aux deux noeuds MPC.
La partie LXRT va g�rer les images de la cam�ra, appliquer le
traitement de l'image et communiquer avec RTAI via une m�moire
partag�e. Un programme Linux peut observer le flot de donn�es
pour, par exemple, pouvoir les faire rejouer en simulation.

Comme nous le verrons, chapitre \ref{chaprobucar}, le
programme de conduite s'ex�cute sur les deux noeuds. Il
lit les donn�es fournies par les capteurs (direction des roues,
vitesse des roues, ...).  Il calcule la r�gulation et puis envoie le r�sultat aux diff�rents
actuateurs g�rant les quatre roues du CyCab (traction et direction). La communication se fait
soit par sortie s�rie SPI, soit par lecture analogique ou optique.

%%====================================================================
\subsection{Noeuds � coeur MPC}
%%====================================================================

Les noeuds du CyCab sont constitu�s de quatre parties d�montables, dont
nous allons expliquer l'utilit�~:
\begin{enumerate}
\item une carte m�re,
\item une carte fille s'embo�tant sur la carte m�re,
\item un micro-contr�leur Motorola MPC555,
\item une petite boite, de couleur noire.
\end{enumerate}

\dessin{figures/CyCab/cartes}{0.5}{Carte m�re (� gauche) et fille (�
  droite).}{cartes}

Les entr�es-sorties de la carte m�re (� gauche, figure (\ref{cartes})), sont~:
\begin{enumerate}
\item L'emplacement du MPC555. Par convention avec les autres CyCab de
  l'�quipe IMARA, on donnera l'identifiant 4000 au processeur du
  noeud arri�re et l'identifiant 4001 au processeur du noeud avant.
\item L'emplacement de l'alimentation. Selon l'anciennet� du CyCab, la
  tension d�livr�e par l'alimentation est soit de 24 VDC, soit de
  48 VDC. La plupart des cartes m�res ont �t� modifi�es pour
  fonctionner avec ces 2 types de tension. Elles sont aussi cens�es
  �tre prot�g�es �lectriquement de l'ext�rieur gr�ce � des optocoupleurs
  (composants de couleur blanche sur la figure (\ref{cartes})) mais des
  probl�mes ont �t� d�tect�s � ce niveau.
\item C'est l'entr�e de la boite de couleur noire. Sans cette boite le
  CyCab ne peut d�marrer. Selon les dires de Robosoft, elle permet de
  remettre � z�ro le MPC555 lorsque le bouton arr�t d'urgence est enfonc�.
\item L'entr�e du joystick (uniquement pour le noeud avant du CyCab).
\item Une prise  �lectronique permet de connecter des fils et d'en
  propager d'autres sur l'axe 11. Ces fils sont la masse et un signal 5V.
\item L'entr�e/sortie s�rielle de type SPI vers l'encodeur
  absolu. Seul le connecteur du bas est utilis�.
\item Ne sert pas.
\item Ne sert pas.
\item L'entr�e/sortie CAN 0 permettant la communication avec l'autre
  noeud et le PC embarqu�. Par convention avec les autres CyCab de
  l'�quipe IMARA, la vitesse du bus CAN est de 800 Kbit/s.
\item L'entr�e/sortie CAN 1 permettant de brancher des �quipements
  externes au CyCab (PC portable par exemple).
\item Masse et un signal 5 V.
\end{enumerate}

Je n'ai pas analys� les entr�es/sorties de la carte fille, figure
(\ref{cartes}) � droite. Mais il faut brancher les axes 13, 14 et 15
aux c�bles dont les num�ros correspondent et qui proviennent du
ch�ssis.

Apr�s r�paration d'une paire de noeud, l'application de conduite
manuelle dont nous parlerons plus en d�tail
chapitre \ref{chaprobucar}, s'arr�tait brusquement lorsque les roues
se mettaient � tourner. Robosoft, nous a dit que l'actuateur de
direction des roues g�n�rait de forts champs �lectromagn�tiques (CEM)
qui perturbent la carte m�re. Il nous a conseill� de faire un rappel
de masse, � savoir brancher un fil connectant la masse des capteurs de
direction avant et arri�re � la carcasse du CyCab. L'application du chapitre
\ref{chaprobucar} a re-fonctionn�e depuis.

%%====================================================================
\section{PC embarqu�}
%%====================================================================

Le CyCab contient un PC embarqu�, de forme horizontale (format rack taille 2U), plac� sous le si�ge,
qui communique avec les 2 noeuds via le bus CAN 0. Avant mon stage, le
PC poss�dait un processeur Intel � 233 MHz, un Linux temps r�el
(kernel 2.4, RTAI 2.4 et dont le serveur X ne pouvait se lancer), ce
qui �tait suffisant pour faire tourner l'application de conduite
manuelle (dont nous parlerons plus en d�tail chapitre
\ref{chaprobucar}). Mais sa vitesse est trop faible pour faire
du traitement de l'image, et donc par cons�quent, de faire du suivi
automatique bas� sur une cam�ra.

J'ai remis � jour, l'architecture du PC embarqu�. Il est d�sormais constitu� de~:
\begin{itemize}
\item[$\bullet$] Le m�me ch�ssis aliment� par une tension -48 V.
\item[$\bullet$] Une carte m�re contenant un Pentium 4 cadenc� � 3 GHz et
  512 Mo de m�moire vive.
\item[$\bullet$] Une carte de fond de panier, r�f�rence PCI-5SDA-RS-R30 � 2 slots PCI et 4
  ISA. Une nouvelle version contient 4 slots PCI (conf�re figure (\ref{reiser})).
\item[$\bullet$] Une carte SPI pour la communication CAN.
\item[$\bullet$] Une carte SPI pour la communication IEEE 1394.
\item[$\bullet$] Une carte graphique PCI NVidia GX 5200, la puce graphique
  incluse dans la carte m�re est suffisante pour un affichage confortable.
\item[$\bullet$] Un disque dur IDE � 20 Go.
\end{itemize}

\begin{minipage}[b]{.45\linewidth}
\centering\epsfig{figure=figures/cycab/pci5sda, width=\linewidth}
\caption{Carte riser.}\label{reiser}
\end{minipage}\hspace{2cm}
\begin{minipage}[b]{.2\linewidth}
\centering\epsfig{figure=figures/cycab/cartemere, width=\linewidth}
\caption{Carte m�re.}\label{cartemere}
\end{minipage}

Le syst�me d'exploitation du PC est un Linux temps r�el dont la
distribution est une Debian 4.0 release 0. Le noyau Linux install� par
d�faut par Debian est le 2.6.18. Un noyau
2.6.18-52 a �t� compil� et patch� pour le temps r�el avec RTAI 3.4. Le
chapitre \ref{chaprtai} expliquera plus en d�tail son fonctionnement.

Il est int�ressant de noter que le format de la carte m�re du PC embarqu� n'est pas
identique au format ATX des cartes m�res des PC personnels. Sa largeur est la moiti�
d'une carte m�re d'un PC standard et contient que le processeur et la m�moire vive.
Elle poss�de 2 slots m�les qui lui permettent de s'ins�rer dans une carte de fond de panier,
ou \emph{riser} en anglais, pour pouvoir ajouter des cartes PCI ou ISA. Voir figure (\ref{reiser}).


%%====================================================================
\section{Capteurs et actuateurs}
%%====================================================================
\subsection{Cam�ra FireWire}\label{ieee1394}
%%====================================================================
La cam�ra FireWire que nous utilisons est une Fire-I fabriqu�e par UniBrain (\ref{firei}), sa taille est $35 \times 65$ mm. Sa r�solution maximale est de $640 \times 480$, elle peut donner des images au format monochrome, RGB ou YUV � des fr�quences de 30, 15, 7.5 et 3.5 images par seconde.

\dessin{figures/cycab/firei}{0.6}{Cam�ra UniBrain Fire-I.}{firei}

Nous utilisons les librairies libraw1394 et libdc1394 \cite{1394} pour obtenir les images.
Les configurations que nous avons choisies sont~: format YUV422, taille $320 \times 240$,
7.5 images par seconde. Ces options peuvent facilement �tre modifi�es. 

Les termes FireWire, IEEE 1394 ou i.Link, sont synonymes. L'article \cite{LM69}, nous dit
que ce bus est un bus s�rie plug and play, pouvant supporter jusqu'� 63 p�riph�riques et
permettant diff�rentes topologies alors que l'USB ne permet que des
configurations en �toile. Le d�bit maximum de la norme 1394 le plus
r�pandu est de 400 Mbit/s, ce qui permet de fournir des d�bits plus
�lev�s que celui de la norme USB 2.0.

Il existe 2 modes distincts pour le bus 1394~: -- le mode asynchrone
et -- le mode isochrone. Le mode isochrone est le plus rapide car il
permet l'envoie de paquets de taille fixe � intervalle de temps
r�gulier et donc il n'existe plus d'accus� de r�ception. M�me si le
d�bit th�orique est de 400 Mbit/s, le cadencement des paquets d'une
transmission isochrone fait que le d�bit utile est de 256 Mbit/s.

On fera donc attention � ne pas d�passer ce d�bit maximum si on veut
brancher, par exemple deux cam�ras sur une m�me carte pour faire par
exemple, du traitement de l'image avec des images st�r�o. Pour cela on
jouera sur les param�tres suivants~: -- la taille de l'image, son
format, -- son taux de compression, -- le nombre d'images par seconde.
Par exemple, une image YUV422 de taille $640 \times 480$ � 30 images
par seconde aura un d�bit de 147 Mbit/s. Il n'est donc pas possible de
mettre deux cam�ras sur la m�me carte, il faut les brancher sur deux
cartes diff�rentes.

%%====================================================================
\subsection{Contr�leur Curtis}\label{curtis}
%%====================================================================

La figure (\ref{noeud}) montre une photo de l'avant d'un CyCab sans sa coque. L'arri�re est la r�plique exacte de l'avant. Nous ne voyons pas sur la photo les roues gauche et droit, mais nous voyons leurs suspensions, le v�rin qui permet la direction, le contr�leur PWM qui g�re le v�rin, le contr�leur Curtis g�rant la motricit� des roues, ainsi que le noeud � coeur MPC.

\dessin{figures/cycab/noeud}{0.5}{CyCab sans sa coque, vu de l'avant.}{noeud}

Comme le but de mon stage �tait de faire du suivi longitudinal de CyCab, je me suis uniquement document� sur le contr�leur Curtis PMC 1227. Dont voici le r�sum� du fonctionnement.

Le datasheet du Curtis PMC 1227 \cite{curtis}, nous indique que ce composant est un contr�leur pour moteur �lectrique. Il fournit au moteur une tension comprise entre 24 et 48 V (allant jusqu'� 200 A) et il programmable gr�ce � une sonde. 

La sonde, nous indique que les param�tres suivants on �t� mis.
\begin{center}
\begin{tabular}{|c|c|c|c|} \hline
Nom & valeur (\%) & Nom & valeur (\%)\\ \hline
Throttle & 0 & M1,2 rec decel & 0\\ \hline
Spd limit pot & 100 & M1,2 max speed & 100 \\ \hline
Batt volt & 48.5 & M1,2 min speed & 0 \\ \hline
Mode input A & on & M1,2 main C/L & 60 \\ \hline
Frwrd input & off & M1,2 IR coef & 0 \\ \hline
Reverse input & off & reverse speed &100 \\ \hline
Inhibit in & on & Ramp shape & 50 \\ \hline
main cont & off & Creep speed & 0\\ \hline
em brake drvr & off & Brake DLY & 1.0\\ \hline
push enable in & off & Thrttle type & 0\\ \hline
Thrtl autolocal & off & direct & 0\\ \hline
M1, M2 accel rate & 0 & Thrttle gain & 100 \\ \hline
M1, M2 decel rate & 0 & thrttle deadband & 5.0 \\ \hline
high pedal dis & on & &\\ \hline
\end{tabular}
\end{center}

Avec ses informations et en se rapportant au manuel, on en d�duit la figure (\ref{wigwag}) qui, � gauche, montre la correspondance entre la plage de tension entrant dans le Curtis et � droite, la sortie PWM (signal carr� dont le rapport cyclique est variable) alimentant le moteur DC.

Dans la plage de tension 2.4V et 2.6V, le moteur ne tourne pas, c'est la zone neutre. De 0.58V � 2.4V et de 2.6V � 4.44V la sortie PWM varie lin�airement par rapport � la tension.

\dessin{figures/CyCab/wigwag}{0.5}{Comportement du Curtis.}{wigwag}

%%====================================================================
\subsection{V�rin de direction}
%%====================================================================
On a remarqu� que les roues du CyCab n'�taient pas bien centr�es. Le v�rin doit �tre r�gl�.


%%====================================================================
%\section{Protocole CAN}
%%====================================================================

%%====================================================================
\section{Probl�mes rencontr�s}
%%====================================================================
\subsection{Noeuds MPC}
J'ai eu  un probl�me �lectronique avec les cartes m�res. Lors d'un test, une carte refusait de communiquer. La r�paration de la carte par la soci�t� Robotsoft a dur� 2 mois. La nouvelle carte, ne fonctionnait pas correctement, non plus.

\subsection{PC embarqu�}
Lors du remplacement du PC embarqu�, une barrette m�moire RAM �tait d�fectueuse. 
Il est int�ressant de savoir diagnostiquer ce probl�me. Dans mon cas
elle emp�chait l'installation de certaines distributions Linux et quand bien m�me, l'installation r�ussissait, des messages d'erreur du type {\tt *** glib detected *** corrupted double-linked list : 0xXXXXX} apparaissaient. Une m�thode pour diagnostiquer ce probl�me consiste � installer l'image {\tt memtest} qui se comporte comme une image noyau Linux et donc se lance avec {\tt grub} et teste la m�moire. 

\subsection{Risque potentiel avec le joystick}
Un probl�me grave peut arriver si le joystick se casse. Les plages de valeur des signaux analogiques du joystick sont comprises entre 0 et 5V. Le z�ro est cod� avec la tension 2.5V. Ce centrage est tr�s dangereux car si le joystick a une d�faillance, la tension devient nulle et le contr�leur croit que 
la consigne est au maximum et donc emballe les roues.



%\chapter{Linux, RTAI et LXRT}\label{chaprtai}
\chapter{Syst�me d'exploitation Linux}\label{chaprtai}

\section{Travail effectu�}

J'ai profit� de la mise � jour des cartes �lectroniques du PC embarqu� pour installer un syst�me d'exploitation Linux plus r�cent. Je lui ai appliqu� un patch pour le faire fonctionner avec le temps r�el. 

J'explique ici quelques notions de Linux comme le but du noyau, les modules, la diff�rence entre espace noyau et utilisateur, la diff�rence entre LXRT et RTAI. Dans les annexes \ref{annexes}, j'explique comment installer RTAI et surtout comment configurer et compiler le noyau Linux. Dans le chapitre \ref{qq}, j'explique le travail que j'ai effectu�, � savoir~: corriger un probl�me dans la communication entre LXRT et RTAI.

\section{Rappels}
On d�finit un programme comme �tant une s�quence d'instructions qui doit �tre ex�cut�e dans un certain ordre. Un logiciel ou application est un ensemble de programmes, qui permet � un ordinateur ou � un syst�me informatique d'assurer une t�che ou une fonction en particulier. Un processus est un programme en train de s'ex�cuter. En informatique, on ajoute certaines propri�t�s � la d�finition pr�c�dente. En l'occurrence, un processus se compose d'un texte de programme (du code machine) et d'un �tat du programme (point de contr�le courant, valeur des variables, pile des retours de fonctions en attente, descripteurs de fichiers ouverts, etc.).
Une t�che est un concept abstrait que l'on rapproche d'un processus dans le sens d'un travail ou d'une responsabilit� qui est allou�s et qui doit �tre ex�cut�e.

Le noyau est le composent central de la plupart des syst�mes d'exploitation. Il est responsable de la gestion des ressources du syst�me et de la communication entre le mat�riel et les logiciels utilisateurs (conf�re figure (\ref{kernel})).
 Il fournit la couche d'abstraction la plus basse (conf�re figure~(\ref{layers})) des ressources (comme la m�moire, le processeur et les dispositifs d'entr�e-sortie).  Cela permet d'avoir acc�s aux ressources physiques  par des m�canismes de communication inter processus et par des appels syst�mes. Cette encapsulation du mat�riel lib�re les d�veloppeurs de logiciels de la gestion complexe des p�riph�riques et d'�crire des applications g�n�riques.\\[0.2cm]
\begin{minipage}[b]{.17\linewidth}
\centering\epsfig{figure=figures/rtai/layers, width=\linewidth}
\caption{Abstractions \cite{wiki}.}\label{layers}
\end{minipage}\hspace{1.7cm}
\begin{minipage}[b]{.3\linewidth}
\centering\epsfig{figure=figures/rtai/kernel, width=\linewidth}
\caption{Connexions du noyau \cite{wiki}.}\label{kernel}
\end{minipage} \hspace{1.7cm}
\begin{minipage}[b]{.3\linewidth}
\centering\epsfig{figure=figures/rtai/mono, width=\linewidth}
\caption{Noyau monolithique \cite{wiki}.}\label{mono}
\end{minipage}

Il existe diff�rents types de noyau, comme  par exemple les noyaux monolithiques,  les micro-noyaux, etc. Celui qui nous int�resse ici est le noyau de Linux qui est un noyau monolithique (conf�re figure (\ref{mono})). Sa m�moire est divis�e entre l'espace utilisateur et l'espace noyau.  L'espace utilisateur d�signe la r�gion m�moire d�di�e aux applications, � l'exclusion du noyau lui-m�me, qui fonctionne dans son propre espace m�moire. L'espace noyau est l'endroit o� le code r�el du noyau r�side apr�s son chargement, et o� la m�moire est allou�e pour les op�rations qui prennent place � son niveau. Ces op�rations incluent l'ordonnancement, la gestion des processus et des signaux, des entr�es/sorties assur�es par les p�riph�riques, de la m�moire et de la pagination. 

Cette partition entre espace utilisateur et espace noyau permet d'obtenir une certaine forme de s�curit� : les applications de l'espace utilisateur ne peuvent, par accident ou intentionnellement, acc�der � une zone m�moire ne leur appartenant pas : une telle action d�clenche imm�diatement une trappe du noyau, qui doit envoyer un signal particulier au programme et, g�n�ralement, le terminer. Pour que ce m�canisme fonctionne, il faut que les processeurs disposent d'une unit� de gestion m�moire (MMU) exploitable par le noyau.

Un module est une partie du noyau qui peut �tre int�gr�e pendant le fonctionnement du syst�me d'exploitation. C'est une alternative aux fonctionnalit�s compil�es dans le noyau qui ne peuvent �tre modifi�es qu'en relan�ant le syst�me.

\section{Diff�rences entre modules du noyau Linux et applications Linux}
Il existe des diff�rences entre un module du noyau et une application (outre le fait de leur s�paration de leur espace m�moire \cite{ldd}). Une application (espace utilisateur) accomplit g�n�ralement une t�che unique du d�but jusqu'� la fin de sa vie. Un module (espace noyau) s'enregistre aupr�s du noyau pour servir les futures requ�tes mais sa fonction d'initialisation se termine imm�diatement. En d'autres termes, le but de la fonction d'initialisation \emph{module\_init} est de s'enregistrer au pr�s du noyau afin de pr�parer la future invocation de la fonction du module, puis s'endort. La fonction d'arr�t du module \emph{module\_exit} est invoqu�e juste avant que le module soit d�charg� du noyau. Elle indique au noyau que le service qu'elle fournit ne sera plus actif.

Une autre diff�rence entre module du noyau et une application est que l'arr�t de l'application peut se faire de mani�re \emph{fain�ante} en lib�rant pas ses ressources (m�moire, descripteur de fichier,~...).
L'arr�t d'un module devra se faire correctement sinon les ressources resteront pr�sentes jusqu'au red�marrage du syst�me.

Les d�veloppeurs savent que quand ils d�veloppent une application, ils peuvent appeler des fonctions non connues mais qui le seront au moment de \emph{link} la r�solution des r�f�rences externes vers  les biblioth�ques appropri�es. Par exemple la fonction \emph{printf} est une des fonctions d�finie dans la biblioth�ques \emph{libc}. Un module, quand � lui, est li� seulement avec le noyau, et les seules fonctions qu'il peut appeler sont celles qui sont export�es par le noyau. Il n'y a pas de biblioth�ques avec qui on peut se lier. La fonction \emph{printk}  est une version de la fonction \emph{printf} d�finie dans le noyau et exportable dans un module. Remarquons que \emph{printk} ne supporte pas l'affichage des nombres en virgule flottante.

Les modules du noyau Linux sont g�n�ralement plac�s dans /lib/modules. Ils utilisent l'extension .ko depuis la version 2.6.
Voici un exemple de cr�ation d'un module, tir� de \cite{ldd}~:
\begin{verbatim}
#include <linux/init.h> 
#include <linux/module.h> 
MODULE_LICENSE("Dual BSD/GPL"); 
static int hello_init(void) 
{ 
    printk(KERN_ALERT "Hello, world\n"); 
    return 0; 
} 
static void hello_exit(void) 
{ 
    printk(KERN_ALERT "Goodbye, cruel world\n"); 
} 
module_init(hello_init); 
module_exit(hello_exit); 
\end{verbatim}

Apr�s compilation, on obtient un fichier {hello.ko} que l'on charge dans le noyau avec la commande {\tt insmod ./hello.ko}. On retire le module avec la commande {\tt rmmod hello}.

%\section{Patch du noyau Linux pour le temps r�el}
\section{Linux temps r�el}
\subsection{RTAI}
Le projet RTAI, pour Real Time Application Interface, a pour origine
le d�partement d'ing�nierie a�rospatiale DIAPM de l'�cole
polytechnique de Milan. Pour des besoins internes, Paolo Montegazza du
DIAPM entreprit de d�velopper un produit inspir� de RTLinux dont la
strat�gie est de faire cohabiter le noyau Linux avec un noyau
auxiliaire bas� sur un vrai ordonnanceur temps r�el � priorit�s fixes. Les t�ches temps r�el sont
g�r�es par ce noyau RTAI et le traitement d'autres t�ches est d�l�gu� au
noyau Linux, lui-m�me consid�r� par le noyau temps r�el comme �tant
une t�che de plus faible priorit� (\emph{idle task}. La figure (\ref{diagrtai})
montre cette architecture.

\dessin{figures/rtai/rtai}{0.5}{Architecture d'un Linux temps r�el.}{diagrtai}

En effet, le code de masquage des interruptions a �t� r��crit
toutes les interruptions hard sont initialement trait�es par le noyau temps r�el
et transmises au noyau Linux seulement si l'interruption ne correspond
pas � une tache temps r�el. En bref, les g�n�rations d'interruptions softs
sont laiss�es � Linux. L'ensemble des services tel que~: les ordonnanceurs, les FIFO ou
encore les allocations dynamiques de m�moire est fourni par des
modules du noyau, que l'on charge dynamiquement en utilisant simplement les
commandes standard de Linux~: \emph{insmod} ou \emph{modprobe}.

%Les services de bases sont fournis par plusieurs modules qui
%permettent de faire du temps r�el dur, un ordonnancement totalement
%pr�emptif bas� sur un mod�le de priorit� fixe. Que nous verrons dans
%la section suivante.

%dessin

\subsection{LXRT}
Les d�veloppeurs de RTAI ont r�cemment fait un effort particulier sur une extension appel�e LXRT (LinuX Real Time) permettant de d�velopper des t�ches temps r�el dans l'espace utilisateur et non plus dans l'espace noyau. Ce dernier point est extr�mement int�ressant au niveau de  la facilit� de d�veloppement car nous savons que d�velopper dans l'espace noyau est relativement plus complexe (la m�moire de l'espace noyau ne poss�de pas de
m�canismes de protection contre les acc�s invalides ou encore de
programmes pouvant amen�s � un panne du syst�me. La modification d'une
zone m�moire non d�sir�e cause la corruption g�n�rale du syst�me).

Au niveau des performances,  les r�sultats par rapport au m�me programme d�velopp� dans l'espace noyau sont tr�s honorables et ce bien que LXRT soit encore en cours d'�volution.

\subsection{Ordonnanceurs}
L'ordonnanceur se trouve �tre le coeur de RTAI.  Il fournit � travers
une s�rie de m�canismes les capacit�s temps r�el. En utilisant
l'ordonnanceur RTAI, le processus peut satisfaire des contraintes
temps r�el dures en ex�cutant les t�ches de fa�on d�terministe .

RTAI fournit des services inter et intra espace utilisateur et noyau
temps r�el durs sym�triques. Un tel support passe � travers deux
ordonnanceurs, qui � l'heure actuelle sont nomm�es \emph{rtai\_lxrt} pour LXRT
et \emph{rtai\_sched} pour RTAI. Ils peuvent fonctionner aussi bien dans
l'espace utilisateur que dans celui du noyau et ils se distinguent
seulement par les objets qu'ils peuvent ordonnancer. Cela signifie que~:
-- \emph{rtai\_lxrt} est un co-ordonnanceur GNU/Linux, -- \emph{rtai\_sched}
supporte non seulement le temps r�el dur (pour
l'ensemble des objets Linux ordonnan�able, comme les
processes/threads/kthreads) mais aussi  les propres t�ches noyau
RTAI.

\subsection{Comparaison de Linux et RTAI/LXRT}
Finalement on distinguera 4 types de t�ches selon leurs contraintes temporelles et
l'espace m�moire dans lequel elles vivent~:
\begin{itemize}
\item  les \emph{processus} qui s'ex�cutent dans l'espace utilisateur sans contrainte temporelle dure,
\item  les \emph{modules} qui s'ex�cutent dans l'espace noyau sans 
contrainte temporelle dure,
\item  les  \emph{t�ches RTAI} qui s'ex�cutent dans l'espace noyau
avec contraintes temporelles dures,
\item  les \emph{t�ches LXRT} qui s'ex�cutent dans l'espace utilisateur 
avec contraintes temporelles dures.
\end{itemize}
\begin{center}
\begin{tabular}{|c|p{6cm}|p{6cm}|} \hline
Type de t�che& Avantages & inconv�nients \\ \hline
Processus & $\bullet$ Elle laisse plus de ressources temporelles aux autres t�ches. Car consid�r� par le noyau temps r�el comme �tant
une t�che de plus faible priorit� & $\bullet$ Tr�s souvent pr�empt�e. \\
& $\bullet$ On a acc�s aux fonctions Linux standards comme les
  fonctions (\emph{open}, \emph{write}). & \\
& $\bullet$ Protection contre les violations aux acc�s m�moire (on a un
  segmentation fault plut�t qu'un kernel panic de l'OS) et outil de
  d�bugage.& \\ \hline
%%
T�che LXRT & $\bullet$  Protection contre les violations aux acc�s m�moire (on a un
  segmentation fault plut�t qu'un kernel panic de l'OS) et outil de
  d�bugage. & $\bullet$   Moins r�active que RTAI car un appel syst�me doit
  traverser la couche utilisateur. \\
& $\bullet$ On a acc�s aux fonctions Linux standards comme les
  fonctions (\emph{open}, \emph{write}). & \\ \hline
%%
 Module & $\bullet$  Souplesse dans le d�veloppement de driver, car peut �tre charg�/d�charg� dynamiquement, ce qui �vite de red�marrer � chaque fois le syst�me d'exploitation. & $\bullet$ Il faut charger/d�charger des modules avant et
  apr�s ex�cution du programme. \\ \hline
  %%
T�che RTAI & $\bullet$ Souplesse dans le d�veloppement de driver, car peut �tre charg�/d�charg� dynamiquement, ce qui �vite de red�marrer � chaque fois le syst�me d'exploitation. & $\bullet$ La programmation doit �tre soigneuse,
  car �tant dans l'espace noyau, la m�moire n'est plus prot�g�e~:
  l'acc�s en �criture peut se faire partout et donc tuer l'OS entier. \\
  & Plus r�active que LXRT. & $\bullet$ On n'a pas acc�s aux fonctions Linux standards comme
  les fonctions (\emph{open}, \emph{write}, ... ). \\  \hline
\end{tabular}
\end{center}




\section{Noyau d'ex�cutif SynDEx pour RTAI}

Le logiciel SynDEx poss�de un noyau d'ex�cutif pour RTAI. Comme on l'a expliqu� pr�c�dem\-ment ces ex�cutifs g�n�r�s sont des modules, que l'on charge dans le noyau avec la commande {\tt insmod}.
Dans le cas de la g�n�ration de l'ex�cutif pour la conduite manuelle, on obtient un module appel� {\tt root.ko} que l'on lance avec la commande {\tt insmod}. L'arr�t de l'application, se fait, d'abord en appuyant sur le bouton rouge d'arr�t d'urgence du CyCab, puis en la�ant la commande {\tt make stop}.

Dans le cas de la g�n�ration de l'ex�cutif pour la conduite automatique, c'est plus compliqu� car SynDEx ne poss�de pas de noyau ex�cutif pour LXRT. Il faut compiler et ex�cuter un programme LXRT qui communiquera avec le module {\tt root}. La synchronisation de la communication n'est donc pas g�n�r�e automatiquement.















\part{Conduite manuelle du CyCab}
\chapter{Notions d'automatique}
Dans ce chapitre, des �l�ments d'automatique sont pr�sent�s rapidement
afin de faciliter la compr�hension des chapitres suivants o� des
applications SynDEx et Scicos sont donn�es.

%%====================================================================
\section{Rappel de quelques �l�ments  de l'automatique}
%%====================================================================
Consid�rons un bateau \cite{Faure} ayant un pilote automatique
recevant en permanence le cap actuel $\alpha$ du bateau et le cap
d�sir� $\alpha_c$. En utilisant ces informations le pilote automatique
g�n�re au cours du temps des ordres de positionnement $\epsilon$ du
gouvernail de fa�on � ce que l'erreur de cap $e=\alpha_c -\alpha$ soit
maintenue aussi faible que possible sachant que le bateau re�oit des
perturbations ext�rieures (vent, ...). La figure (\ref{bateau}) montre le sch�ma bloc du mod�le comme nous l'avons montr� dans la section \ref{sss}.

\dessin{figures/regulateur/bateau}{0.6}{Pilote automatique de bateau (P.A.)}{bateau}

Pour ce faire, une loi de commande (bloc P.A.), calculant $\epsilon= f(e)$ en
fonction des informations disponibles, pourrait �tre~:
\begin{itemize}
\item[$\bullet$] par �-coups : $$\epsilon = \begin{cases}
+\epsilon_m & \text{si } e>0\; ,\\
-\epsilon_m & \text{si } e<0\; .
\end{cases} $$
Ca a l'avantage d'�tre simple, mais n'est pas tr�s efficace. Le syst�me va osciller (un coup trop � gauche, un coup trop � droite, ...)

\item[$\bullet$] proportionnelle : $$\epsilon = Ke\;,$$
Cette m�thode � l'inconv�nient de faire appara�tre une erreur asymptotique.

\item[$\bullet$] proportionnelle et int�grale : $$\epsilon = Ke +
  1/\rho\int^te(\alpha)\;d\alpha\;,$$
Le terme int�grale permet d'additionner les erreurs du syst�mes. La sommation des erreurs va continuer jusqu'� ce que la valeur corresponde � la valeur de la consigne. L'inconv�nient du terme int�grale est que la r�ponse � une perturbation devienne plus lente que le seul terme proportionnel.

\item[$\bullet$] proportionnelle et d�riv�e : $$\epsilon = Ke +
  K'de/dt\;,$$ Le terme  d�riv� permet une r�ponse plus rapide lors d'un changement rapide de l'erreur. En th�orie, ce terme aide le syst�me a �tre plus r�actif, mais en pratique, les bruits blancs (vibration, erreur de lecture des capteurs, ...) vont rendre instable le syst�me (on d�rive du bruit). 

\item[$\bullet$] proportionnelle et int�grale et d�riv� etc.
\end{itemize}

C'est la th�orie des asservissements qui permettra, dans un cas
particulier de choisir la loi de commande la mieux adapt�e. Le document AVR221 d'Atmel
explique plus en d�tail et de fa�on tr�s simple, l'impl�mentation d'un r�gulateur PID
(dans le cas temps discret) et la m�thode Ziegler-Nichols pour trouver les bons gains \cite{avr221}
car les syst�mes deviennent instables quand les valeurs des gains sont trop forts.

Dans les paragraphes suivants on va introduire les outils permettant d'analyser
la classe des syst�mes lin�aires temps invariant.

\section{Repr�sentation sous forme de matrice d'�tat}
Un syst�me lin�aire temps invariant, peut s'�crire sous la forme de matrice d'�tat, comme pour  l'exemple suivant, du mouvement d'un pendule invers\'e avec les conditions initiales nulles. On peut �crire le syst�me avec $x_1$ l'angle et $x_2$ sa position, et $u$ l'entr�e u syst�me et $y$ la sortie (l'observation)~:
\begin{align*}
\dot{x_1} & = x_2 \\
\dot{x_2} & = \frac{mgl}{J} \sin{x_1} + \frac{ml}{J} u \cos{x_1}
\end{align*}

On peut \'ecrire $\dot{x_2}= f(x_1, x_2, u)$. Gr\^ace \`a la formule de Taylor  :
$$\dot{x_2} \simeq f(0, 0, 0) + \delta x_1 f_{x_1}'(0, 0, 0) + \delta x_2
f_{x_2}'(0, 0, 0) + \delta u f_u'(0, 0, 0)$$

Ce qui donne :
$$\delta \dot{x_2} \simeq \delta x_1 \left( \frac{mgl}{J} \right) + \delta u \left( \frac{ml}{J} \right)$$
$$\delta \dot{x_1} = \delta x_2$$

Finalement, en changeant de notation $\delta x \rightarrow x, \delta u \rightarrow u, \delta y \rightarrow y$, on obtient le systeme suivant sous la forme matricielle :
\begin{align*}
 \dot{x} = &
\begin{bmatrix}
0 & 1 \\ 
\frac{mgl}{J} & 0\\
\end{bmatrix}
\begin{bmatrix}
 x_1 \\  x_2 \\
\end{bmatrix}
+ \begin{bmatrix}
0 \\  \frac{ml}{J} \\
\end{bmatrix}  u
\\
 y= & \begin{bmatrix}
1 & 0 \\ 
\end{bmatrix}
\begin{bmatrix}
 x_1 \\  x_2 \\
\end{bmatrix}
+ \begin{bmatrix}
0\\
\end{bmatrix}  u
\end{align*}

Soit de la forme g�n�rale~:
\begin{align*}
\frac{dx}{dt} = & \; Ax + Bu \\
y = & \; Cx 
\end{align*}

%%====================================================================
\section{Transform�e de Laplace et transform�e en $z$}
Deux transformations permettent de ramener l'analyse des syst�mes
dynamiques lin�aires � du calcul alg�brique. Ce sont la transform�e de
Laplace et la transform�e en $z$. Elles transforment des fonctions du
temps en une fonction d'une variable dont la partie imaginaire
s'interpr�te en terme de fr�quence.  Ces transformations permettent de
r�soudre les �quations diff�rentielles lin�aire � coefficients
constants.

%%====================================================================
\subsection{Transform�e de Laplace}
\subsubsection*{D�finition}
La transform�e de Laplace est d�finie de la mani�re suivante~: soit
$f(t)$ une fonction du temps d�finie pour $t>0$. Alors~:
$$\LL[f(t)] \equiv F(s) \equiv
\lim\limits_{
  \begin{subarray}{l}
    T \rightarrow \infty \\
    \epsilon \rightarrow 0
  \end{subarray}} \int_{\epsilon}^{T} f(t)\;e^{-st} dt=
\int_{0^{+}}^{\infty} f(t)\;e^{-st}dt\;\;\;\; 0<\epsilon<T$$
o\`u $s$ est un variable complexe d�fini par $s\equiv \sigma + j\omega$.

%%====================================================================
\subsubsection*{ Transform�e de Laplace de d�riv�es}
Montrons que la transform�e de Laplace de la d�riv�e $df/dt$ d'une
fonction $f(t)$ vaut~:
$$\LL\left[\frac{df}{dt}\right] = s \LL f - f(0^{+})$$

En int�grant par partie, on obtient :
$$\left[ e^{-st}f(t)\right]^{\infty}_{0^{+}}+s\int_{0^{+}}^{\infty}fe^{-st}dt$$

Finalement :
$$\LL\left[\frac{df}{dt}\right] = s\LL f - f(0^{+})$$

%%====================================================================
\subsubsection*{Autre exemple}
Une autre formule utilis�e dans l'application du chapitre suivant est
la transform�e de Laplace de la double d�riv�e $d^{2}f/dt^{2}$ d'une
fonction $f(t)$~:
$$\LL\left[\frac{d^{2}f}{dt^{2}}\right] = s^2F(s)-sf(0^{+})-\frac{df}{dt}\Big|_{t=0^{+}}$$

% \subsection{Transform�e inverse de Laplace}
% \subsubsection*{D�finition}
% Soit $F(s)$ la transform�e de Laplace de la fonction $f(t), t > 0$,
% la transform�e inverse de $F(s)$ est~:
% $$\LL^{-1}[F(s)]\equiv f(t) = \frac{1}{2\pi}\int_{c-j\inf}^{c+j\infty}F(s)\;e^{st}ds$$
% o� $c > \sigma_0$ ($\sigma_0$ est un nombre r�el, crit�re de la
% transformabilit� de $f(t)$).

%%====================================================================
\subsection{Transform�e en $z$}
La transform�e en $z$ est utilis�e pour d�crire des signaux en temps
discret.  Soit $\{f(k)\}$ d�note une s�quence de valeur r�elle $f(0),
f(1), f(2), \ldots$ ou bien $f(k)$ pour $k=0,1,2,\ldots$. Alors on
d�finit la transform�e en $z$ par~:
$$\ZZ\{f(k)\}\equiv F(z) = \sum_{k=0}^{\infty}f(k)z^{-k}$$ o\`u $z$
est un variable complexe d�fini par $z\equiv \sigma + j\omega$.

\section{Placement de p�les}
Un syst�me lin�aire temps invariant (LTI) ayant un nombre fini d'�tats
s'�crit g�n�ralement de la fa�on suivante~:
\begin{align*}
\frac{dx}{dt} = & \; Ax + Bu \\
y = & \; Cx 
\end{align*}
O� la $A$ matrice est une matrice $n \times n,$ $B$ une matrice $n
\times p,$ $C$ une matrice $q \times n,$ . Sous block diagramme il se
dessine comme sur la figure \ref{blockdiag}.


\dessin{figures/regulateur/blockdiag}{0.6}{u est l'entr\'ee, y la sortie, x
  l'\'etat (ou m\'emoire)}{blockdiag}

La fonction de transfert d'un syst�me lin�aire temps invariant
mono-entr�e mono-sortie est une fonction rationnelle s'�crivant~:
$$G(s) = C(sI-A)^{-1}B=\frac{N(s)}{D(s)}\;,$$
o� $N$ et $D$ sont deux polyn�mes.

On appelle p�les les z�ros du polyn�me $D(s)$, ils sont aussi les
valeurs propres de la matrice $A$. La pr�sence du p�le $\lambda$
implique que la sortie du syst�me $y(t)$ contient une composante de la
forme $e^{\lambda t}$.  Alors, si $Re(\lambda)>0$ la sortie tend vers
l'infini lorsque $t$ tend vers l'infini. Le syst�me est dit instable.
Un syst�me est asymptotiquement stable si et seulement si toutes les
valeurs propres v�rifient~: $Re(\lambda_i) < 0$.

Consid�rons le syst�me boucl� ($u$ en feedback sur l'�tat) dans lequel
$u=Kx+v$ le nouveau syst�me s'�crit
\begin{align*}
\frac{dx}{dt} = & \; (A+BK)x + Bv \; ,\\
y = & \; Cx \; .
\end{align*}
Le feedback �tant � notre disposition on peut choisir ses coefficients
de fa�on � placer les p�les du syst�me boucl� o� l'on veut. Par
exemple rendre stable par feedback un syst�me instable.

Un th�or�me indique sous quelles conditions sur les matrices $(A,B,C)$
il est possible de placer les p�les du syst�me o� l'on veut.

%%====================================================================
\section{Discr�tisation d'une int�grale}

SynDEx manipule seulement des mod�les temps discret (et donc, pas de
temps continu) et il n'est pas capable de manipuler des boucles
alg�briques implicites. C'est pourquoi, une boucle SynDEx doit
contenir au moins un d�lais ($1/z$).  Par cons�quent, notre
application Scicos qui est un syst�me dynamique temps continu doit
�tre convertie en temps discret pour pouvoir �tre utilis�e dans
SynDEx.

L'�quation diff�rentielle $\dot{x}=u$ est discr�tis�e d'une
fa�on simple en utilisant un sch�ma de Euler. Notons $h$ 
le pas de discr�tisation et $x_0$ une valeur initiale
arbitraire. Le syst�me discret s'�crit ~:
\begin{equation}
x_{n+1}-x_{n}=u h\label{sys2}
\end{equation}
Finalement, le syst�me (\ref{sys2}) est donn� en Scicos (SynDEx)
par la figure. \ref{sciintegdiscr} (\ref{synintegdiscr}). Notons que
la variable $h$, stock�e dans le contexte de Scicos, est utilis�e
dans l'entr�e du gain et dans la d�finition de l'horloge.
Dans SynDEx, $h$ est d�fini en tant que param�tre dans la d�finition d'un gain et
l'horloge est utilis�e directement dans le code source des op�rations.\\[2mm]
\begin{minipage}[b]{.45\linewidth}
\centering\epsfig{figure=figures/regulateur/scicos_integraldiscrete, width=\linewidth}
\caption{Une int�grale discr�tis�e dans Scicos (temps discret).}\label{sciintegdiscr}
\end{minipage}\hspace{10mm}
\begin{minipage}[b]{.4\linewidth}
\centering\epsfig{figure=figures/regulateur/syndex_integraldiscrete, width=\linewidth}
\caption{Une int�grale discr�tis�e dans SynDEx.}\label{synintegdiscr}
\end{minipage}

% On remarquera, que SynDEx ne fonctionne pas si 

%\section{Critere observabilte}
%\section{commandabilit�}
%\section{Kalman filter}
%\section{temps discret temps continu}
%papier PID Atmel

%%====================================================================
%%====================================================================
%\chapter{Logiciel applicatif~: conduite manuelle du CyCab}\label{chaprobucar}
\chapter{Mod�lisation de la conduite manuelle du CyCab}\label{chaprobucar}
%%====================================================================
%%====================================================================
\section{Observation des �tats du CyCab}
%%====================================================================
\subsection{Travail effectu�}
%%====================================================================
Il existe au moins deux applications SynDEx de conduite manuelle du CyCab~:
l'application \emph{Manu}, et l'application
\emph{Robucar}. \emph{Robucar} est l'application d�velopp�e par la
soci�t� Robosoft. Elle est utilis�e par l'�quipe IMARA comme base de
d�veloppement. Elle est plus r�cente que l'application
\emph{Manu}. \emph{Robucar} et \emph{Manu} sont des r�gulateurs,
apr�s leur t�l�chargement sur l'architecture du CyCab (� savoir deux
noeuds MPC et le PC embarqu�) il est possible de conduire le CyCab
avec le joystick.

%\section{L'application originelle SynDEx \emph{Robucar}}

A mon arriv�e, j'avais deux applications SynDEx~: \emph{Robucar} originelle
et l'application \emph{Robucar} modifi�e par Lionel Durand, le
stagiaire pr�c�dent. Dans ce chapitre, on expliquera le contenu et le
fonctionnement du \emph{Robucar} fait par Robosoft. La
reprise du travail de Lionel sera expliqu�e dans la section \ref{lionelfuck}.
Comme nous l'avons expliqu� dans la section \ref{syndexsoft}, un algorithme
sous SynDEx est un ensemble de blocs interconnect�s, mais au premier abord l'application \emph{Robucar} n'est pas facile �
comprendre~: -- les blocs appellent du code assembleur MPC555 (que ce
soit pour les calculs sur la r�gulation, de communication ou de
gestion des capteurs/actuateurs), -- enfin, on ne conna�t pas les plages de
valeur de ces signaux.

Un des buts de mon stage �tait de remplacer certains blocs de r�gulation
de la vitesse des roues qui sont en relation avec le joystick par des blocs r�gulant la vitesse mais � partir d'une
cam�ra. Mais il est difficile de trouver les bons gains si on ne conna�t
pas le domaine des valeurs correspondant � ces blocs.
J'ai donc du traduire l'application SynDEx \emph{Robucar} en une
application Scicos, logiciel de mod�lisation et de simulation, et par la m�me occasion trouver un mod�le physique du CyCab. Ceci a permis de simuler la r�gulation \emph{Robucar} en toute s�curit�, pour moi et pour les stagiaires suivants.

%%====================================================================
\subsection{Observation des �tats}\label{spy}
%%====================================================================
Afin de savoir si la traduction de \emph{\emph{Robucar}} sous Scicos
sera correcte, il est n�cessaire de faire jouer la simulation avec des
valeurs qui correspondent � la r�alit�. Avant de commencer le travail
de traduction, j'ai du trouver un moyen d'espionner les donn�es
du CyCab.

Le premier probl�me rencontr� fut qu'une partie de l'algorithme,
appel� \emph{root}, de l'application Robucar tourne sous RTAI. Passer
par RTAI pour sauver des donn�es dans des fichiers n'est pas ce qui y
a de plus simple. Comme nous l'avons vu dans le chapitre \ref{chaprtai},
fonctionner avec RTAI a un inconv�nient majeur~: celui de ne pas
pouvoir acc�der aux fonctions Unix de gestion des fichiers (comme
\emph{open}, \emph{write}, \emph{fprintf}, ... ).

Une premi�re solution consisterait � transformer le programme
\emph{root} en une application LXRT car on pourrait acc�der aux
fonctions \emph{open}, \emph{write}, \emph{fprintf} interdites sous
RTAI.

Une deuxi�me solution consisterait � cr�er un programme tournant sous LXRT
en parall�le avec RTAI et transmettre les donn�es par m�moire partag�e
et prot�ger les acc�s concurrents par s�maphores nomm�es. Ceci est assez
p�nalisant, car assez lourd a mettre en place et est peu esth�tique.

Une troisi�me solution, la plus simple mais assez douteuse, consiste �
espionner le fichier \emph{kernel.log} par un script shell. Dans
l'application \emph{Robucar} sous SynDEx on ajoute des blocs appelant
la fonction \emph{rt\_printk}, devant tous les blocs de type
actuateurs et capteurs.

Cette fonction du noyau qui permet d'�crire des cha�nes de caract�res
de fa�on �quivalente � \emph{printf} (sauf qu'elle ne permet pas
l'affichage de flottants). Ces cha�nes de caract�res sont d�rout�es
vers l'�cran et/ou un fichier log nomm� \emph{kernel.org}, selon
configuration. Ce fichier peut-�tre int�gralement lu gr�ce � la
commande shell \emph{dmesg}. Le param�tre optionnel '-c' permet
d'effacer ensuite le fichier \emph{kernel.log}. Il faut s'avoir que ce
fichier log est un fichier cyclique qui peut contenir, dans notre cas,
environ 400 lignes. Donc \emph{rt\_printk} �crit dans ce fichier � la
position courante modulo 400.

J'ai �crit un petit script shell qui recopie
le contenu de ce fichier \emph{kernel.log} dans un fichier texte de
taille infinie~:
\begin{verbatim}
#! /bin/bash
while [ 1 ]
do
   dmesg -c >> CyCab.log
done
\end{verbatim}

Comme RTAI peut tourner � 1 KHz et qu'il travaille peu, on peut
supposer qu'il reste suffisamment de temps pour Linux d'ex�cuter ce
script sans trop de probl�me avant que les 10 ms ne s'�coulent.

En faisant attention au nombre d'appel � \emph{rt\_printk}, on ne perd
pas de donn�es par d�bordement des 400 lignes. Comme un fichier log
r�cup�re �galement des messages d'erreur du syst�me, ils vont se
m�langer aux donn�es du CyCab (comme le d�marrage, et arr�t du CyCab,
...). on ajoute un identifiant pour pouvoir les reconna�tre. On
ajoutera le flag \emph{IN=} pour les donn�es sortant des capteurs et
le flag \emph{OUT=} pour les donn�es entrant dans les actuateurs.

Il ne reste plus qu'a lancer ce script, compiler et ex�cuter
l'application \emph{Robucar}. On obtient le fichier CyCab.log

%%====================================================================
\subsection{Traitement des �tats}
%%====================================================================
J'ai �crit un petit script shell qui va filtrer le fichier
\emph{CyCab.log}, supprimer les messages du noyau, garder les donn�es
des signaux et les s�parer en 2 fichiers textes, lisibles par Scicos.
Car par d�faut, Scicos s'attend � lire des donn�es au format
$7(e10.3,1x)$.

\begin{verbatim}
    #! /bin/bash
    function call_scilab
    {
        cat <<EOF > SCISCRIPT.txt
    A=fscanfMat('/afs/inria.fr/rocq/home/aoste/qquadrat/DATA.txt');
    A(:,$) = (A(:,$) - A(1,$)) / 1000;
    write('$1', A, '(7(e10.3,1x))');
    exit
    EOF

        scilab -nw -ns -nb -f SCISCRIPT.txt
        echo "Creation du fichier $1 [OK]."
        rm -fr SCISCRIPT.txt DATA.txt 2> /dev/null
    }

    if [ "$1" = "" ]
    then
        echo "Donner le nom du fichier log CyCab"
        exit 1
    fi

    if [ "`which scilab 2> /dev/null`" = "" ]
    then
        echo "Il faut installer Scilab"
        exit 1
    fi

    rm -fr IN.txt OUT.txt 2> /dev/null

    cat $1 | sed -e '/^IN=/!d;s/[ \t][ \t]*/ /g' | cut -d" " -f2,4,6,8,10,12 > DATA.txt
    call_scilab IN.txt

    cat $1 | sed -e '/^OUT=/!d;s/[ \t][ \t]*/ /g' | cut -d" " -f2,4,6,8,10 > DATA.txt
    call_scilab OUT.txt
\end{verbatim}

La premi�re partie consiste � s�parer les signaux des capteurs des
actuateurs dans 2 fichiers gr�ce aux commande \emph{sed} et
\emph{cut}. Puis on transforme le format des signaux en signaux
compr�hensibles par Scicos gr�ce � un script Scilab.

\dessin{figures/retro/readfrominputfile}{0.4}{Le bloc \emph{Read
  from input file}.}{read}

La lecture des signaux sous Scicos se fait gr�ce � un bloc \emph{Read
  from input file} comme sur la figure (\ref{read}). On branchera la sortie de l'horloge (en rouge) sur
l'entr�e de l'horloge pour le mettre en �chantillonneur bloqueur.
Ceci permet aux signaux de ne pas se d�synchroniser au cas o� des
donn�es viendraient � manquer (pas lues par l'espion). Dans la figure (\ref{read}),
on voit une configuration possible de ce bloc~:
\begin{itemize}
\item[$\bullet$] on indique le num�ro de la colonne o� est stock� le temps.
\item[$\bullet$] on indique les num�ros des colonnes o� sont stock�es
  les donn�es. Ici les colonnes $1\; 2\; 3\; 4\; 5$.
\end{itemize}

Il faut faire attention, car c'est bien les num�ros des colonnes qu'il faut indiquer et non pas la taille du
bus de donn�es, comme on peut le retrouver dans les autres configurations des blocs Scicos.
%%====================================================================
\section{Notions d'assembleur MPC555}
%%====================================================================
La traduction de l'application Robucar SynDEx en un application Scicos, n'est pas la partie la plus compliqu�e car le code assembleur est assez simple � comprendre, nous expliquerons les instructions les plus importantes.

La partie o� j'ai eu le plus de probl�me fut de comprendre pourquoi les signaux obtenus avec la simulation de Scicos ne correspondait pas aux signaux r�els. Il faut rappeler que les MPC555 travaillent sur des entiers 32 bit alors que Scicos travaille avec des nombres � virgule flottantes et que des probl�mes de d�rive sont apparus, dus aux probl�mes d'arrondis des signaux simul�s. Apr�s l'ajout � certains endroits des blocs qui g�rent les arrondis (� savoir les blocs \emph{quantization}) � permis de simuler le travail sur des entiers, car on ne peut indiquer a scicos de travailler sur des entiers.

L'application Robucar est constitu� de blocs appelant des morceaux de code assembleur ressemblant �~:
\begin{verbatim}
(1)    B(lwz r30,$1);
(2)    cmpwi r30,1;
(3)    bge 0f;
(4)    mulli r30,r30,-1;
(5)    0: B(stw r30,$2)
\end{verbatim}

Ce code m�lange assembleur 555 et macro code M4. Dans le chapitre (\ref{syndex}) nous avons expliqu� � quoi servait le langage M4. Nous allons expliqu� les instructions 555.

La premi�re ligne {\tt B(lwz r30,\$1)} est une macro M4 qui affecte au registre $r_{30}$ la valeur contenue dans la variable M4 {\tt \$1}.

Les lignes {\tt (2)} et {\tt (3)} testent si le registre $r_{30}$ est plus petit que $1$. Si c'est le cas, on saute � la ligne {\tt (5)}, sinon on passe � la ligne {\tt (4)}. L'instruction {\tt cmpwi} compare le contenu d'un registre avec un mot (word ou {\tt w}) entier ({\tt i}). L'instruction {\tt bge 0f} signifie~: \emph{Branch into label {\tt 0f} if $r_{30}$ is Greater or Equal to 1}. La ligne {\tt (4)} multiple le contenu du registre $r_{30}$ par $-1$ et affecte le r�sultat dans le registre $r_{30}$. Enfin, la ligne {\tt (5)} permet de sauver le registre $r_{30}$ dans la variable M4 de {\tt \$2}, elle joue le r�le de {\tt return} en langage C.
%%====================================================================
\section{Mod�lisation en Scicos de la conduite manuelle}
%%====================================================================
La figure \ref{robucosmain}) montre le r�sultat de la traduction de l'application \emph{\emph{Robucar}}
en une application Scicos. 

\dessin{figures/retro/Robucos_main}{0.4}{L'application \emph{Robucar} sous Scicos.}{robucosmain}

Dans la figure (\ref{robucosmain}), on peut voir deux superblocs : -- � droite~: le r�gulateur
\emph{Robucar} SynDEx traduit en formalisme Scicos (couleur bleue) et
-- � gauche~: un superbloc \emph{Capteurs \& Actuateurs \& Plot}
permettant de lire les signaux espionn�s des capteurs et des
actuateurs du CyCab, puis de visualiser les signaux obtenus par
simulation sous la forme d'un graphique (couleur grise) et voir s'ils
correspondent � ceux espionn�s.

Nous verrons dans la section (\ref{phys}) que le superbloc \emph{Capteurs
  \& Actuateurs \& Plot} peut �tre remplac� par une fonction de
transfert simulant le fonctionnement des capteurs et actuateurs.
Rappelons (conf�re section (\ref{sss})) qu'il existe un traducteur automatique de bloc Scicos vers des blocs SynDEx et qu'il suffit, dans l'IHM de Scicos de s�lectionner le super bloc \emph{R�gulateur Robucar} pour obtenir une application SynDEx.

%%====================================================================
\subsection{Superbloc \emph{Capteurs, actuateurs, plots}}
%%====================================================================
\dessin{figures/retro/Robucos_plot}{0.6}{Le superbloc g�rant l'affichage et la lecture des signaux.}{robucosplot}

Le contenu du superbloc de gauche (\emph{Capteurs \& Actuateurs \&
Plot}) est repr�sent� figure (\ref{robucosplot}). Dans la section \ref{spy}, nous
avons vu comment obtenir un fichier
ascii contenant la valeur des signaux (ainsi que le temps) observ�s
lors d'un test r�el de d�placement sur un CyCab. Nous allons exploiter
ces donn�es gr�ce aux deux blocs \emph{Capteurs} et \emph{Actuateurs}
qui sont du type \emph{Read from input file} (figure (\ref{read})).

Les signaux observ�s sortant des capteurs sont respectivement :
\begin{enumerate}
\item La conversion analogique du joystick pour la direction des roues,
\item La valeur de l'encodeur absolue pour la direction des roues arri�res,
\item La conversion analogique du joystick pour la vitesse des roues,
\item La valeur moyenne des quatre d�codeurs en quadrature indiquant
  la vitesse moyenne des roues,
\item La valeur de l'encodeur absolue pour la direction des roues avants,
\item Le temps (sortie horloge).
\end{enumerate}

Les signaux observ�s entrant dans les actuateurs sont respectivement :
\begin{enumerate}
\item La consigne de direction des roues avants,
\item La consigne de direction des roues arri�res,
\item La tension de consignes pour la vitesse des roues avants,
\item La tension de consignes pour la vitesse des roues arri�res,
\item Le temps (sortie horloge).
\end{enumerate}

Tous ces signaux observ�s sont multiplex�s avec les signaux obtenus
par simulation puis sont dessin�s sous de formes de graphiques (bloc
\emph{MScope} pour Multi Scopes). Un exemple de graphique obtenu est
figure \ref{plot}. En noir les signaux espionn�s et en rouge et bleu,
les signaux obtenus par simulation.

\dessin{figures/retro/Robucos_sim}{0.6}{Signaux simul�s et r�els.}{plot}
%%====================================================================
\subsection{Coeur du r�gulateur de conduite manuelle}
%%====================================================================
La figure (\ref{robucossuper}) montre le superbloc \emph{R�gulateur Robucar}
SynDEx traduit en formalisme Scicos (superbloc droit de la figure
(\ref{robucosmain})).

\dessin{figures/retro/Robucos_super}{0.6}{\emph{Robucar} main}{robucossuper}

On voit en entr�e les signaux des 5 capteurs pr�c�demment cit�s dans
la figure (\ref{plot}). On voit �galement 5 superblocs color�s en 3
couleurs diff�rentes (vert, jaune et marron). En effet certains blocs
accomplissent presque le m�me travail, d'o� leur regroupement par
couleur~:
\begin{itemize}
\item En vert : des filtres sur les signaux analogiques du joystick,
\item En jaune : la r�gulation sur la vitesse des 4 roues,
\item En rouge : la r�gulation sur la direction des 4 roues.
\end{itemize}

Nous allons d�tailler chacun de ces superblocs.
%%====================================================================
\subsection{Superbloc \emph{Adoucissement Joystick motricit�}}
%%====================================================================
La figure (\ref{robucosjoyavt}) est le contenu du superbloc
\emph{Adoucissement Joystick Avant/Arri�re} montre le filtre sur le
signal du joystick pour la vitesse des roues (motricit�).

\dessin{figures/retro/Robucos_joy_avt}{0.6}{Filtrage du joystick g�rant la motricit�.}{robucosjoyavt}

La plage de valeur du signal du joystick est comprise entre +115 et
+775. On commence par recentrer ce signal en 0 en lui soustrayant
445. C'est le r�le des blocs verts � gauche de l'image.

Ensuite, on filtre le bruit du signal gr�ce aux blocs roses :
\begin{itemize}
\item[$\bullet$] d'abord par une moyenne pond�r�e des 32 derni�res
  valeurs gr�ce � la fonction dynamique $y=f(u,y)$, o� $u$ est
  l'entr�e, $y$ la sortie et $f$ la fonction qui calcule
  $\frac{u+31y}{32}$.
\item[$\bullet$] Ensuite, avec le bloc \emph{Zone Morte}, si le signal
  est compris entre les valeurs -30 et +30, il est r�duit � 0; sinon
  tronqu� d'une valeur 30. Les extremums du signal sont donc
  -300 et +300.
\end{itemize}

Le signal est multipli� par une constante 3.111 (bloc bleu) qui n'est
d'autre qu'une conversion d'une vitesse en une autre vitesse. La
documentation du CyCab nous dit que la vitesse maximum d'un moteur est
de 7330 mm/s, que le diam�tre d'une roue est de 400 mm, que le
r�ducteur moteur a un ratio de 8 et que la r�volution d'une roue fait
2000 impulsions apr�s d�codage en quadrature et que la p�riode de
lecture est 10 ms. Donc la constante est~:

$$\frac{7330 \times 8 \times 2000 \times 0.01 }{400 \pi \times 300} =
3.11$$

Le deuxi�me bloc bleu \emph{R�ducteur} divise uniquement le signal
n�gatif par 4. Je ne suis pas sure mais c'est � cause d'un autre
r�ducteur moteur pour la marche arri�re.

Ensuite, vient le bloc vert \emph{Acc�l�ration progressive} il
prot�ge le moteur des changements de vitesse trop grands, ce qui
emp�che l'apparition de pics de courant importants dans les moteurs,
risquant d'endommager l'�lectronique ou les moteurs. %(en effet le
%moteur se comporte comme une bobine Ldi/dt).

La fonction $f(u_1)$ est de la forme $f(u_1) = (u_1 + 30)^2 / 933$ et
est born�e par deux fonctions~:
\begin{itemize}
\item[$\bullet$] $u_2 - 7 < f(u1) < u_2 + 1.5$ quand $u1 >= 0$.
\item[$\bullet$] $u_2 + 7 > f(u1) > u_2 - 1.5$ quand $u1 < 0$.
\end{itemize}

% La sortie de ce super
% bloc nous un signal consigne de vitesse pour le bloc jaune.
%%====================================================================
\subsection{Superbloc \emph{R�gulation de la vitesse des roues.}}
%%====================================================================
La figure (\ref{robucosroues}) montre le contenu du bloc jaune
\emph{R�gulation Vitesse 4 Roues} de la figure (\ref{robucossuper}).

\dessin{figures/retro/Robucos_roues}{0.6}{R�gulation de la vitesse des roues.}{robucosroues}

Il permet la r�gulation de la vitesse des quatre roues en fonctionnant
de l'observation de la vitesse r�elle des roues et de la consigne de
vitesse qui est la sortie du superbloc de la figure
(\ref{robucosjoyavt}).

Les blocs en jaunes montrent l'erreur (entre la consigne de vitesse
fournie par le joystick et l'observation de la vitesse r�elle des
roues) entre dans un r�gulateur \emph{Proprotionel--Int�gral} puis est
satur�e dans la plage de valeur $[-400; +400]$. Les blocs en bleu font
que le signal s'additionne ou se soustrait avec la constante 2560.

Comme nous l'avons vu dans la section \ref{curtis}, ce signal
correspond � une tension a fournir au Curtis. 400 correspond � 0.4V et 2560
correspondant � 2.5V, la valeur neutre o� les roues ne tournent
pas. Comme les quatre roues doivent tourner toutes dans le m�me sens, et
que les moteurs � gauche du CyCab sont invers�s par rapport � ceux
du c�t� droit, on doit changer le signe (valeur $< 2560$).

%%====================================================================
\subsection{Filtrer le signal du joystick de direction}
%%====================================================================
La figure \ref{robucosjoydir} montre le deuxi�me bloc vert filtrant
le signal du jystick.

\dessin{figures/retro/Robucos_joy_dir}{0.6}{Filtrage du joystick g�rant la direction.}{robucosjoydir}

Ce signal indique une consigne de direction pour les roues. Son
fonctionnement est identique au superbloc \ref{robucosjoyavt}~: m�me
filtre pond�r�, m�me �limination du bruit par troncature. La
diff�rence vient du bloc vert \emph{D�c�l�ration Progressive} qui
permet de contr�ler la direction des roues en fonction de leur vitesse
(pour �viter de tourner trop brusquement � grande vitesse).

La fonction est de la forme $4(1+\frac{400}{40+u_1})$ et comme pour le
bloc \emph{Acc�l�ration Progressive}, on emp�che le signal de sortir
entre deux bandes qui sont des fonctions dynamiques.
%%====================================================================
\subsection{Superbloc \emph{Adoucissement Joystick de direction}}
%%====================================================================
Enfin, les deux derniers blocs en rouge sur la figure
\ref{robucossuper} permettent de r�guler la direction des roues. L'un
des deux est montr� figure (\ref{robucosdir}).

\dessin{figures/retro/Robucos_dir}{0.6}{R�gulation de la direction des roues arri�res.}{robucosdir}

Les blocs bleus permettent de calculer l'erreur entre la consigne de
direction des roues et de leur observation. On applique un r�gulateur
proportionnel int�gral � l'erreur et on sature la valeur entre -900 et
+900. Enfin on envoie au v�rin le sens et une tension permettant de
faire tourner les roues.
%%====================================================================
\section{Note importante sur l'application de conduite manuelle}
%%====================================================================
Il faut savoir que l'application \emph{Robucar} pour fonctionner
correctement en situation r�elle, doit avoir au moins un bloc tournant
sur RTAI. En effet, RTAI sert de timer de p�riode 10 ms. Il permet de
ralentir les deux noeuds du CyCab pour la r�gulation. Supprimer cet
unique bloc aura pour cons�quence une mauvaise r�gulation.

Une autre remarque importante, est que l'on constate qu'au d�marage de la r�gulation,
RTAI tourne � 3 ms au lieu de 10 ms puis revient � 10 ms. Je ne sais pas d'o� vient ce
ph�nom�ne.

%%====================================================================
%%====================================================================
\chapter{Mod�lisation du process CyCab}\label{phys}
%%====================================================================
%%====================================================================
\section{Principe des moindres carr�s}
Maintenant que nous avons traduit le r�gulateur du CyCab  de SynDEx en
Scicos, il est n�cessaire  de se donner un mod�le
du CyCab pour pouvoir tester le r�gulateur. Dans notre cas, on aimerait conna�tre, la fonction
de transfert qui nous donne la vitesse des roues en fonction des sorties du r�gulateur (entr�es du Curtis) c.a.d.  la fonction
de transfert Curtis -- Quadrature Encoder.  Gr�ce aux
donn�es espionn�es lors d'un test sur le CyCab, on ajuste dans un premier 
un syst�me lineaire  temps discret MIMO
d'ordre un par la m�thode des moindres carr�s.
%un syst�mepeut avoir une
%bonne approximation polynomiale de son mod�le physique en utilisant le
%principe des moindres carr�s dont la formule g�n�rale
%est~: $$\min_{p_0, p_1, ..., p_n} \sum^n_{i=0}(y_i-p(x_i))^2$$

%Une forme d'approximation polynomiale simple est de tracer une droite
%qui approxime un nuage de point. Dans notre cas, on veut traiter le
%cas g�n�ral~: ne pas se restreindre � une droite mais � un polyn�me
%d'ordre donn� qui est la meilleure appriximation du nuage de point.
%Le nuage de point est nos donn�es observ�es.

On suppose donc que le mod�le du CyCab est le syst�me,
$x_{n+1}=a x_n + b u_n$  et $y_n=x_n$ o� les entr�es sorties $y_n$ et $u_n$ sont
connus et les param�tres $a$ et $b$ � ajuster de fa�on � minimiser l'erreur
entre le mod�le et les donn�es observ�es.  
Soit~:
$$\min_{a, b}\; \LL(a,b) = \min_{a, b} \sum_{n=0}^N (x_{n+1} - a x_n - b u_n)^2$$
Donc on r�sout le syst�me~:
$\frac{\delta L}{\delta a} = 0$
et
$\frac{\delta L}{\delta b} = 0$
et on trouve~:
$$\frac{\delta L}{\delta a} = -\sum x_nx_{n+1} + b\sum u_nx_n + a\sum x^2_n = 0$$
$$\frac{\delta L}{\delta b} = -\sum u_nx_{n+1} + a\sum u_nx_n + b\sum u^2_n = 0$$
Soit sous forme matricielle~:
$$\left[\begin{array}{c} a \\ b\end{array}\right]=
\left[
\begin{array}{cc}
\sum x^2_n & \sum u_nx_n \\
\sum u_nx_n & \sum u^2_n \\
\end{array}
\right]^{-1}
\left[\begin{array}{c} \sum x_nx_{n+1} \\ \sum u_nx_{n+1}\end{array}\right]
$$

\section{Programme Scilab}
Il ne reste plus qu'� �crire un petit script Scilab pour obtenir le
mod�le de notre CyCab.
\begin{verbatim}
function [W]=Modelise(file_IN, col_in, file_OUT, col_out)
  M   = fscanfMat(file_IN);
  u   = M(:, col_in);

  M   = fscanfMat(file_OUT);
  x   = M(:, col_out);

  N   = min(size(u, 1), size(x, 1));
  i   = [1:1:N-1];

  a11 = sum(x(i)^2);
  a12 = sum(u(i) .* x(i));
  a21 = a12;
  a22 = sum(u(i)^2);
  A   = [a11, a12; a21, a22];

  b11 = sum(x(i) .* x(i + 1));
  b21 = sum(u(i) .* x(i + 1));
  B   = [b11; b21];

  W   = A \ B;
endfunction

//Modele vitesse roues droites
Roue_droite = Modelise("IN.txt", 4, "OUT.txt", 3)

//Modele vitesse roues gauches
Roue_gauche = Modelise("IN.txt", 4, "OUT.txt", 4)
\end{verbatim}
Une version alternative d'�criture de la fonction {\tt Modelise} plus
courte est possible. Elle utilise le produit scalaire~:
\begin{verbatim}
function [W]=Modelise(x, u)
  xx  = x(1:$-1)
  uu  = u(1:$-1)
  A   = [uu * uu', xx * uu'; xx * uu', uu * uu'];
  B   = [[x,0] * [0,x]'; [0,u] * [x,0]'];
  W   = A \ B;
endfunction
\end{verbatim}

%Comme, seule la r�gulation des roues nous int�resse, nous allons le caract�riser le polynomes de la transformation du signal Curtis--Encodeur.
%$$\begin{cases}
%a= \;,\\
%b=\;.
%\end{cases}$$

\section{R�sultat}

Nous en d�duisons la fonction de transfert $F(z)=b/(1/z-a)$ Curtis--Encodeur. 
Nous pouvons, ensuite, modifier notre application en rempla�ant le bloc {\tt Capteurs \& Actutateurs \& Plot} de la figure (\ref{robucosmain}) par le bloc qui repr�sente la fonction de transfert $F(z)$ comme le montre la figure (\ref{ft}).

\dessin{figures/retro/ft}{0.2}{\emph{Robucar} main avec la fonction de transfert Curtis-d�codeur.}{ft}







\part{Conduite automatique du CyCab}
%%====================================================================
%%====================================================================
\chapter{Notions de traitement d'image}\label{chaptrtimage}
%%====================================================================
%%====================================================================

%On d�signe par \emph{technique de traitement d'images} toutes les
%techniques ayant pour but la modification des caract�ristiques
%chromatiques des pixels des images bitmap. Traitement d'images est
%souvent synonyme d'am�lioration des images avec pour but l'obtention
%d'une plus grande lisibilit�. Il n'y a pas cr�ation d'informations,
%mais mise en �vidence de l'information pertinente d�j�
%pr�sente \cite{opgl}.

{\bf Avertissement}~: afin de r�duire le poids de ce document PDF, les images pr�sentes ont du �tre comprim�es.

%%====================================================================
\section{Formats de couleurs utilis�s}
%%====================================================================
Nous allons pr�senter trois formats de couleur qui sont utilis�s pour
le traitement de l'image concernant la d�tection de CyCab~: --
l'image, donn�e par la cam�ra, au format YUV422 qui est une
compression du format YUV, -- le format YUV et enfin, -- le format
RGB. Nous allons d�crire ces trois formats.
%%====================================================================
\subsection{Codage RGB}
Le format RGB (RVB en fran�ais) attribue un octet par composante de
couleur rouge, verte ou bleue par pixel. Ces trois composantes
primaires additives servent dans les �crans d'ordinateur et de
t�l�vision pour reconstituer toutes les couleurs sur l'image
visualis�e. Par exemple, dans un ordinateur la valeur d'une composante
est un entier compris entre 0 et 255 inclus. L'absence de couleur
correspond au noir (valeur 0) alors que le blanc correspond � la
fusion des 3 couleurs (valeur 255). Comme on repr�sente une couleur
sur les trois composantes, on obtient plus de 16 millions de
possibilit�s th�oriques de couleurs diff�rentes, l'{\oe}il humain n'en
voyant que 2 millions.
%%====================================================================
\subsection{Codage YUV}\label{yuv}
D'apr�s l'article \cite{LM69} on sait que l'oeil humain est plus
sensible au signal de luminance qu'aux signaux de couleurs. Comme le
format RGB n'en tire pas partie, on utilisera alors le format YUV qui
a l'avantage de s�parer une composante moyenne des signaux RGB,
appel�e luminance (Y � savoir le noir et blanc), des deux signaux de
couleur (U et V qui correspondent au rouge et bleu) appel�s
chromances. La couleur verte se retrouve alors par combinaison entre
Y, U et V. Figure (\ref{yuv}) montre une photo et la d�composition de ses
composantes Y, U, V. Les images proviennent de  \cite{wiki}.
 
\dessin{figures/filtre/yuv}{0.5}{Une image et ses composantes Y (haut, droit),  U (bas gauche) et V (bas droit).}{yuv}

Du point de vu historique, le codage YUV provient de la transmission
de la vid�o analogique pour la t�l�vision. Le Conseil Sup�rieur de
l'Audiovisuel n'ayant accord� � la t�l�vision couleur qu'une bande
passante de 8 MHz (au lieu de 32 MHz provenant des trois bandes de
couleurs et de la bande de la luminescence), des ing�nieurs ont d�
trouver une m�thode pour comprimer les donn�es des images RGB pour la
t�l�vision hertzienne. Le syst�me est donc compatible avec les
t�l�viseurs noir et blanc.

De nombreuses �quations, tr�s proches les unes des autres, permettent
de passer du codage YUV au codage RGB.

$$ R = Y - 0.0009267 (U-128) + 1.4016868 (V-128)$$
$$ G = Y - 0.3436954 (U-128) - 0.7141690 (V-128)$$
$$ B = Y + 1.7721604 (U-128) + 0.0009902 (V-128)$$
%%====================================================================
\subsection{Codage YUV422}

Ce format est celui fournie par notre camera FireWire, il permet de
compresser le d�bit vid�o tout en ne d�t�rioraient pas trop
l'information contenue. Les chiffres 4, 2 et 2 indiquent le nombre
d'�chantillons de luminance Y et de chrominance U et V pour coder
l'information d'une image YUV de taille $4 \times 4$. Cela permet
d'avoir 2 octets par pixel au lieu de 3.

Par exemple, si on re�oit le flux d'entr�e YUV422~:
$$Y_1\; Y_2 \; ... \; Y_{16} \; U_1 \; U_2 \; ... \; U_4 \; V_1 \; V_2 \; ... \; V_4$$

On en d�duit l'image $4 \times 4$ YUV d�compress�e~:
$$Y_1 U_1 V_1 \; \; Y_2 U_1 V_1 \; \; Y_3 U_2 V_2  \; \;Y_4 U_2 V_2$$
$$Y_5 U_1 V_1 \; \; Y_6 U_1 V_1 \; \; Y_7 U_2 V_2  \; \;Y_8 U_2 V_2$$
$$Y_9 U_3 V_3 \; \; Y_{10} U_3 V_3 \; \; Y_{11} U_4 V_4  \; \;Y_{12} U_4 V_4$$
$$Y_{13} U_3 V_3 \; \; Y_{14} U_3 V_3 \; \; Y_{15} U_4 V_4  \; \;Y_{16} U_4 V_4$$

%$$\left[
%\begin{array}{cc}
%Y_1 & Y_2 \\
%Y_5 & Y_6 \\
%\end{array}
%\right]_{U_1 V_1}\left[
%\begin{array}{cc}
%Y_3 \; & Y_4 \\
%Y_7 \; & Y_8 \\
%\end{array}
%\right]_{U_2 V_2}$$
%$$\left[\begin{array}{cc}
%Y_9 & Y_{10} \\
%Y_{13} & Y_{14} \\
%\end{array}
%\right]_{U_3 Y_3}\left[
%\begin{array}{cc}
%Y_{11} & Y_{12} \\
%Y_{15} & Y_{16} \\
%\end{array}
%\right]_{U_4 Y_4}$$

D'autres formats YUV compress�s comme les formats YUV420, YUV411 dont le d�bit total
de ce dernier est de 1.5 celui de la luminance. Remarquons que le format YUV est
aussi nomm� YUV444.
%%====================================================================
\section{Op�rations de base sur les pixels}
Comme nous l'avons vu dans la section pr�c�dente, une image est un
ensemble de composants chromatiques cons�cutifs (3 pour RGB et YUV; 1
pour BW). En regroupant ces composants on forme des pixels qui sont
ordonnanc�s par leur position $(x, y)$.

%% Comme nous l'avons vu dans la section pr�c�dente, une image est un
%% ensemble de pixel ordonnanc�s par leur position et o� chaque pixel est
%% constitu� de plusieurs composants chromatiques cons�cutifs (3 pour RGB
%% et YUV; 1 pour BW).

On appelle taille d'une image le couple $(L, l)$, le nombre de pixel
en longueur $L$ et en largeur $l$ qui la constituent. Une repr�sentation
m�moire d'une image est un vecteur de $3lL$ pixels pour des images YUV
et RGB et un vecteur de $lL$ pixel pour les images blanches et noires.
Chaque pixel est cod� sur un octet.

On liera les images d'abord de gauche � droite puis de haut en bas.
On notera $p(x,y)$ un pixel de cordonn�es $(x,y) \in (l,L)$.  Par
exemple en format RGB, $p(x,y)$ retournera le triplet $(R(x,y),
G(x,y), B(x,y))$ dont~:
$$p(x,y)=
\left[\begin{array}{cc} R(x,y) \\ G(x,y) \\ B(x,y) \end{array}\right]
=
\left[\begin{array}{cc} 3(yl + x)\\ 3(yl + x) + 1 \\ 3(yl + x) + 2 \end{array}\right]
$$

Il existe des op�rations de base qui manipulent composante par
composante les pixels d'une image. Ils ont la forme suivant ~:
$$p(x,y) = p_1(x,y) \oplus p_2(x,y)$$
$$p(x,y)=
\left[\begin{array}{cc} R(x,y) \\ G(x,y) \\ B(x,y) \end{array}\right]
=
\left[\begin{array}{cc}
\max(0, \min(255, R_1(x,y) \oplus R_2(x,y)) \\
\max(0, \min(255, G_1(x,y) \oplus G_2(x,y)) \\
\max(0, \min(255, B_1(x,y) \oplus B_2(x,y))
\end{array}\right]$$

o� $p_1$ et $p_2$ sont des
pixels de m�me coordonn�es $x,y$ mais provenant de deux images. $p$
est le pixel r�sultant de l'op�ration. La fonction $\oplus$
peut-�tre~: -- une op�ration arithm�tique de base comme, l'addition,
la soustraction (on veillera � ce que la valeur de $p$ reste comprise
entre 0 et 255 en saturant la valeur du r�sultat); -- des op�rations
bool�enne comme l'op�ration \emph{non}, \emph{ou}, \emph{et}, \emph{ou
  exclusif}.
%%====================================================================
\subsection{Masque}
Dans notre cas, concernant le CyCab, seule l'op�ration \emph{et}
binaire (op�rateur \& en langage C) nous int�resse, car il va nous
permettre de masquer certaines zone de l'image. Le site \cite{opgl}
donne des exemples de ces op�rations, dont voici un exemple avec le
\emph{et}.\\[0.1cm]
\begin{minipage}[b]{.3\linewidth}
\centering\epsfig{figure=figures/filtre/p1, width=\linewidth}
\caption{Image 1.}\label{p1}
\end{minipage}\hspace{6mm}
\begin{minipage}[b]{.3\linewidth}
\centering\epsfig{figure=figures/filtre/p2, width=\linewidth}
\caption{Image 2.}\label{p2}
\end{minipage}
\hspace{6mm}
\begin{minipage}[b]{.3\linewidth}
\centering\epsfig{figure=figures/filtre/et, width=\linewidth}
\caption{Et binaire.}\label{et}
\end{minipage}
%%====================================================================
\subsection{Transformation d'une image en niveaux de gris}
La d�tection des contours se fait sur une image noir et blanc par un produit de 
convolution. La premi�re �tape est alors de transformer l'image de type YUV ou 
RGB en provenance de la cam�ra en une image en niveaux de gris.
%%====================================================================
\subsubsection*{Transformation d'une image RGB en une image en niveaux de gris}
Cette transformation est un simple moyenne pond�r� des trois
composantes de couleurs. La formule g�n�rale est donc la suivante,
pour un pixel $p(x,y)$~: $$p(x,y)=\frac{R(x,y)+G(x,y)+B(x,y)}{3}$$
%%====================================================================
\subsubsection*{Transformation d'une image YUV en une image en niveaux de gris}
Comme pr�cis� dans la section \ref{yuv}, la composante Y correspond au niveau de
gris.  Voici cependant la formule qui permet d'obtenir la composante Y
� partir des composantes RGB : $$Y = 0,299 R + 0,587 G + 0,114 B$$

Les coefficients ne sont pas les m�mes (on s'attendait � avoir $0,333$). Cette 
diff�rence ne va pas nous g�ner puisqu'elle correspond � une r�alit� physique
(le vert est plus lumineux que le rouge qui est � son tour plus lumineux que le
bleu) et puisque l'algorithme qui va �tre utilis� n'est pas sensible � ces 
coefficients lin�aires.

On prendra alors dans notre programme la composante Y pour niveaux de gris, tout
simplement.
%%====================================================================
\section{Filtrage matriciel}
Cette technique consiste � appliquer une matrice $M$ de taille $I
\times J$ (g�n�ralement de taille $3 \times 3$ � chaque pixel
$p(x,y)$ de l'image � transformer de taille $m \times n$. Le calcul
est le suivant~:

\begin{equation}
p(x,y) = \sum^{I}_{i=1}\sum^{J}_{j=1} M(i,j)\; p(y+i,x+j)\label{prodmat}
\end{equation}
avec~:
$x \in [1 \; .. \; L-i-1]$ et $y \in [1 \; .. \; l-j-1]$.

%%FAUX !!

On veillera � ce que la valeur du pixel $p$ reste comprise entre 0 et
255 en saturant la valeur. Nous allons voir dans les section futures
qu'elles sont les formes que peut avoir $M$ et quels sont les
r�sultats que l'on obtient.
%%====================================================================
\subsection{Filtres passe-bas}

Nous ne nous attarderons pas sur les filtres passe-bas dont le but est
de r�duire les parasites (bruits de mesure). Ils agissent par moyenne
sur un voisinage et suppriment donc les d�tails.
%%====================================================================
\subsection{Filtre passe-haut de Sobel}

Les filtres passe-haut nous serons utiles pour le CyCab car ils ont
pour but d'augmenter le contraste et de mettre en �vidence les
contours. Les contours sont une discontinuit� locale de l'intensit�
lumineuse. Les techniques permettant de d�tecter un contour sont
bas�es sur l'utilisation de gradients.

La d�tection de contours du CyCab utilise principalement le filtre de
Sobel dont la matrice est~:
\[M_h=\left[
\begin{array}{ccc}
1 & 2 & 1 \\
0 & 0 & 0 \\
-1 & -2 & -1 \\
\end{array}
\right]\]

Puisque ce filtre est bas�es sur l'utilisation de gradients, les
coefficients 1, 2 et 1 ne sont que l'approximation de la d�riv�e.
Ce filtre permet de d�terminer les contours horizontaux. On obtient un
filtre qui d�termine les contours verticaux par rotation de $\pi/2$
de la matrice $M_h$. On obtient alors~:
\[M_v=\left[
\begin{array}{ccc}
1 & 0 & -1 \\
2 & 0 & -2 \\
1 & 0 & -1 \\
\end{array}
\right]\]

Le r�sultat du produit matriciel (formule (\ref{prodmat})) de $M$ par
le pixel $p(x,y)$ contiendra des valeurs n�gatives et positives ce qui
correspondent aux signes du gradient. Comme nous travaillons sur des
octets non sign�, il faut recentrer la plage de valeur � 128. La
visualisation graphique du r�sultat nous donne une image gris�e (ce
qui correspond � 128) avec des parties claires ou fonc�es.\\[0.5cm]
\begin{minipage}[b]{.3\linewidth}
\centering\epsfig{figure=figures/filtre/serie2/debut, width=\linewidth}
\caption{Image RGB.}\label{bw1}
\end{minipage}\hspace{6mm}
\begin{minipage}[b]{.3\linewidth}
\centering\epsfig{figure=figures/filtre/serie2/sobelHS, width=\linewidth}
\caption{Sobel hor. sign�.}\label{sobhs}
\end{minipage}
\hspace{6mm}
\begin{minipage}[b]{.3\linewidth}
\centering\epsfig{figure=figures/filtre/serie2/sobelVS, width=\linewidth}
\caption{Sobel ver. sign�.}\label{sobvs}
\end{minipage}

Par rapport � la formule \ref{prodmat}, on prendra la valeur absolue
de la valeur du r�sultat du pixel $p(x,y)$, puis on divisera le
r�sultat par un coefficient $k$. Cette division permet de supprimer le
bruit (haute fr�quence).

Les images (\ref{bw2}), (\ref{sobh}) et (\ref{sobv}) donne un exemple
de filtrage de Sobel sur les contours. Il est int�ressant de noter que
le logiciel libre \emph{The Gimp} contient une biblioth�que de
d�tection de contours (Sobel (non) sign�, Laplace, ...).\\[0.5cm]
\begin{minipage}[b]{.3\linewidth}
\centering\epsfig{figure=figures/filtre/serie2/debut, width=\linewidth}
\caption{Image RGB.}\label{bw2}
\end{minipage}\hspace{6mm}
\begin{minipage}[b]{.3\linewidth}
\centering\epsfig{figure=figures/filtre/serie2/sobelH, width=\linewidth}
\caption{Sobel horizontal.}\label{sobh}
\end{minipage}
\hspace{6mm}
\begin{minipage}[b]{.3\linewidth}
\centering\epsfig{figure=figures/filtre/serie2/sobelV, width=\linewidth}
\caption{Sobel vertical.}\label{sobv}
\end{minipage}

% Voici un effet sur le coef $k$~:\\[0.5cm]
% \begin{minipage}[b]{.45\linewidth}
% \centering\epsfig{figure=figures/filtre/serie2/sobelh, width=\linewidth}
% \caption{$k=1$.}\label{sobl1}
% \end{minipage}\hspace{6mm}
% \begin{minipage}[b]{.45\linewidth}
% \centering\epsfig{figure=figures/filtre/serie2/sobelh10, width=\linewidth}
% \caption{$k=10$.}\label{sobh10}
% \end{minipage}

%%====================================================================
\subsection{Histogrammes horizontal et vertical}

L'histogramme est, dans notre cas, un vecteur $H$ de hauteur le nombre de lignes
de l'image trait�e. Chaque �l�ment $(i)$ du vecteur correspond � la moyenne des
pixels de la ligne $i$. Voici la formule g�n�rale pour une image~:

$$h(y)=\frac{\sum_{x=1}^{L}p(x,y)}{L}$$
$$v(x)=\frac{\sum_{y=1}^{l}p(x,y)}{l}$$

Les lignes horizontales les plus grandes de l'histogramme horizontal
correspondent le plus souvent aux contours du CyCab.\\[0.5cm]
\begin{minipage}[b]{.45\linewidth}
\centering\epsfig{figure=figures/filtre/serie2/sobelh, width=\linewidth}
\caption{Sobel Horizontal.}\label{bw2}
\end{minipage}\hspace{6mm}
\begin{minipage}[b]{.45\linewidth}
\centering\epsfig{figure=figures/filtre/serie2/histohbw, width=\linewidth}
\caption{Histo horizontal.}\label{sobh}
\end{minipage}

Malheureusement ce n'est pas toujours le cas~: une haie d'arbres bien
taill�e ou la fa�ade d'un b�timent peuvent cr�er une ligne horizontale
suppl�mentaire.\\[0.5cm]
\begin{minipage}[b]{.45\linewidth}
\centering\epsfig{figure=figures/filtre/fuckoff1, width=\linewidth}
\caption{Haie coup�e.}\label{fuck1}
\end{minipage}\hspace{6mm}
\begin{minipage}[b]{.45\linewidth}
\centering\epsfig{figure=figures/filtre/fuckoff2, width=\linewidth}
\caption{Pied du batiment.}\label{fuck2}
\end{minipage}

%%====================================================================
% \section{Rapports invariants}

%%====================================================================
\section{Probl�matique sur le s�lection des raies}

Dans ce chapitre, nous avons vu quelques briques de bases
concernant le traitement de l'image. Nous avons vu qu'un filtre de
Sobel non sign� suivit d'un histogramme horizontale nous permettait
d'obtenir des lignes horizontales  (raies) du contours du CyCab.

Lors de mes tests avec le traitement de l'image que j'ai h�rit�, j'ai
remarqu� trois choses~:
\begin{itemize}
\item[$\bullet$] L'environnement pouvait perturber la d�tection de CyCab, en cr�ant des raies suppl�mentaires suffisamment grandes dans l'histogramme pour perturber l'estimation de la distance qui s�pare les deux CyCab.
\item[$\bullet$] L'algorithme de d�tection re-d�couvrait les positions des raies � chaque
nouvelle image,  sans utiliser la d�tection du pass�.
\item[$\bullet$]  L'algorithme calcule la distance qui s�pare les deux CyCab avec une m�thode bas�e sur le calcul de rapport invariant avec les positions des raies. Si l'id�e d'utiliser les rapports invariants est int�ressant du point de vue th�orique, il est difficile � mettre au point un programme, car il utilise un certain nombre de raies pouvant aller de 2 � 5.
\end{itemize}

Dans le prochain chapitre, nous d�crirons une m�thode am�lior�e de
d�tection de CyCab.

\input{conduiteautomatique.tex}


\chapter{Conclusion}




Malgr� les nombreuses pannes du mat�riel (noeuds MPC, barrette m�moire, mon ordinateur portable),
ce stage a �t� tr�s b�n�fique sur plusieurs plans. J'ai eu la responsabilit� de r�aliser une application
temps r�el compl�te de niveau industriel. J'ai donc du abord� de nombreux domaines~: mod�lisation,
traitement d'image automatique, informatique distribu�, syst�me, programmation dans plusieurs langage, �lectronique etc. Ce stage m'a aussi familiaris� avec les difficult�s de la r�alisation d'un syst�me mat�riel industriel qui n�cessite la coop�ration de nombreux acteurs externes au projet.

Au moment de l'�criture de ce rapport la partie simulation du syst�me complet 
et l'exp�rimentation du traitement d'image est achev� et fonctionne de fa�on satisfaisante.
La campagne d'exp�rimentation du r�gulateur d�bute et devrait pouvoir �tre termin�e
d'ici la fin du stage fin juillet si d'autres pannes mat�rielles ne retardent pas l'atteinte de cet objectif.

La seule r�gulation longitudinale sera s�rement insuffisante pour le suivi effectif des v�hicules.
Le suivi 2D peut �tre r�alis� tr�s rapidement en adaptant la m�thodologie utilis�e ici
(il suffit de suivre des lignes verticales li�s au Cycab et de les maintenir au centre de l'image)
mais il demande encore un peu de travail pour �tre mis en oeuvre effectivement.





%Tout d'abord il m'a permis d'aborder le domaine de l'automobile et sera certainement un atout majeur pour la suite. De plus le cadre dans lequel j'ai travaill� a �t� un excellent moyen d'acqu�rir plus d'autonomie dans mes recherches et de prendre les directions que je souhaitais. J'ai  beaucoup appr�ci� ce stage dans la mesure o� il est � la fois tr�s 
%concret et tr�s instructif. Il m'a permis de m�ler th�orie et pratique et ainsi d'aborder de nombreux domaines tels que~:

%\begin{itemize}
%\item[$\bullet$] la r�tro-ing�nieurie~:

%\begin{itemize}
%\item en espionnant les �tats du CyCab,
%\item en traduisant l'application \emph{Robucar} SynDEx vers Scicos et o� j'ai �t� confront� � des probl�mes de d�rive dans les signaux dus � des probl�mes d'arrondis.
%\item en traduisant des morceaux de code assembleur en langage Scilab.
%\end{itemize}

%\item[$\bullet$] l'�lectronique~:
%\begin{itemize}
%\item en dicotomisant les noeuds MPC afin d'isoler le probl�me des pannes des cartes.
%\item en faisant un peu de maintenance en soudant quelques fils qui ont pris un jeu et en faisant un rappel de masse entre la carcasse et les noeuds ce qui a permis d'�liminer des reset des noeuds � cause des 
%\item en espionnant la circulation des donn�es avec des espionneurs temps r�el CAN,
%\item en remettant � jour les cartes du PC embarqu�.
%\end{itemize}

%\item[$\bullet$] le traitement de l'image~:
%\begin{itemize}
%\item en r�cup�rant les images d'une camera FireWire,
%\item en lisant quelques documents sur le traitement de l'image.
%\item en appliquant des filtres simples comme Sobel, Laplace, ...
%\item en faisant une IHM pour visualiser/d�buger le traitement de l'image.
%\end{itemize}

%\item[$\bullet$] l'automatique~:
%\begin{itemize}
%\item comme avec la r�gulation de la distance entre 2 v�hicules,
%\item faire du suivi de contours.
%\end{itemize}

%\item[$\bullet$] la programmation~:
%\begin{itemize}
%\item de d�couvrir un Linux temps r�el, savoir l'installer, le configurer et l'utiliser,
%\item d'apprendre plus sur le noyau et les modules Linux,
%\item D�couverte de la programmation Xlib.
%\end{itemize}
%\end{itemize}


%%====================================================================
%%====================================================================
\part{Annexes}\label{annexes}
\chapter{Installer Linux Debian et RTAI}
%%====================================================================
%%====================================================================
\section{Installer RTAI}
%%====================================================================
Vu que le hardware du PC embarqu� a �t� mis � jour pendant
ce stage, j'en ai �galement profit� pour mettre � jour l'OS.
Afin de faciliter la t�che des futurs stagiaires, je d�cris ici,
comment installer une distribution Debian, compiler un
nouveau noyau et installer RTAI.

%%====================================================================
\section{Etape 1 : installer Debian}
\subsection*{L'installation de Debian}
%%====================================================================
J'ai opt� pour l'installation d'une Debian 4.0 NetInst release 0, car
Debian est r�put�e pour sa fiabilit�. Elle contient un noyau
2.6.18. Son installation ne pose pas de probl�me, mais il faut lancer
l'installation en mode \emph{expert} (non test� en mode
\emph{expertgui}) car en mode \emph{normal} Debian choisit
arbitrairement un r�seau connect� � un serveur DHCP alors qu'�
l'INRIA, il imp�ratif de poss�der une adresse IPv4 statique.

Vu que j'ai chois�t une distribution l�g�re (150 Mo), l'installation
des logiciels se fait par le r�seau (d'o� le nom de Debian
NetInst). Il faut laisser Debian se connecter � un serveur http pour
installer tous les packages utiles, sinon on n'a pas serveur X.

%Enfin il faut absolument installer \emph{grub}.

%Voici la configuration du CyCab~:
%\begin{verbatim}
%Nom du PC embarqu�: aoste-rtai
%adresse IPv4: 128.93.5.18
%Mask: 255.255.192.0
%Paserelle: 128.93.1.100
%Serveur de nom: 128.93.1.23 et 192.93.2.78
%pwd root: cycab0
%nom d'utilisateur: aoste
%pwd utilisateur: cycab0
%Possibilite de connexion ssh: oui
%Ex: ssh -X aoste@aoste-rtai
%Ex: ssh -X aoste@128.93.5.18
%\end{verbatim}

%%====================================================================
\subsection*{Faire reconna�tre la puce graphique int�gr�e de la carte m�re}
%%====================================================================
Debian reconna�t tr�s bien la carte graphique NVidia, mais pas la
puce graphique Intel de la carte m�re car il faut choisir l'une ou l'autre
dans le BIOS (touche \emph{delete} lors du boot du PC, puis choisir le
menu XX et (d�s)activer XX).

Une fois Debian la Debian install�e,  on �dite le fichier \emph{/etc/X11/xorg.conf}
et on ajoute les lignes suivantes~:
\begin{verbatim}
Section "Device"
    Identifier   "Intel 82865G"
    Driver        "i810"
EndSection
\end{verbatim}

On cherche la {\tt Section "Screen"} et on ajoute la ligne suivante � c�t� de la ligne {\tt Device NVidia}.
\begin{verbatim}
#    Device    "Intel 82865G"
\end{verbatim}

Le {\tt \#} sert � commenter l'utilisation de ce device, car on pr�f�rera l'utilisation de la carte graphique NVidia.

\subsection*{Finir d'installer les logiciels manquants}
Une fois Linux d�marr�, on se logue en tant que
utilisateur \emph{aoste} et on finit d'installer les packages
manquants gr�ce � l'utilitaire \emph{Synaptic}. On installe les
packages suivants~: demon ssh, \emph{gcc-3.4} (on cr�era un lien symbolique
gcc avec la commande \emph{sudo ln -s gcc-3.4 /usr/sbin/gcc}),
\emph{autotools}, \emph{auto-make}, \emph{ncurses}, \emph{emacs}, \emph{libraw1394-dev}, \emph{libdld394-dev}, \emph{glade2-dev}, et pour tester comment ce comportement de RTAI au
stress du CPU)~: \emph{lxdoom}.


%%====================================================================
\section{Etape 2 : compiler un nouveau noyau}
%%====================================================================
Sur le site de RTAI \cite{rtaiorg} on t�l�charge la version 3.4 de
RTAI dans le r�pertoire \emph{/usr/src}, on la d�compresse et on cr�er
un lien symbolique \emph{rtai} par les commandes~:
\begin{verbatim}
$ cd /usr/src/
$ tar jxvf rtai-3.4.tar.bz2
$ ln -s rtai-3.4 rtai
\end{verbatim}

La version actuelle du noyau Debian est une 2.6.18 or il ne
semble pas exister de patch RTAI pour 2.6.18 bien qu'il existe un
patch pour 2.6.17 ou pour 2.6.19 ! On t�l�charge sur le site
\cite{kernelorg} la version 2.6.16-52.
\begin{verbatim}
$ cd /usr/src/
$ tar jxvf linux-2.6.16-52.tar.bz2
$ ln -s linux-2.6.16-52 linux
\end{verbatim}

La derni�re �tape est utile. On lance le configure graphique (il faut
avoir installer la libncurses), je n'ai pas tester les autres
alternatives (gtk, ...)
\begin{verbatim}
$ make menuconfig
\end{verbatim}

Apr�s avoir s�lectionner les options du noyaux, comme on l'explique
dans la section suivante (\ref{choixkern}), on compile les sources et on
installe l'image du noyau et les modules~:
\begin{verbatim}
$ make
$ sudo make install
$ sudo make modules_install
\end{verbatim}

Je n'ai rencontr� aucun probl�me de compilation. Donc, si tout va bien
on retrouve, dans le r�pertoire \emph{/boot/}, la nouvelle image du noyau
\emph{vmlinux-2.6.16-52-rtai}. C'est tout � fait normal qu'il n'y a pas de fichier
\emph{initrd-2.6.16-52-rtai}. On va
updater \emph{grub} par~:
\begin{verbatim}
$ update-grub
\end{verbatim}
Ou bien manuellement par~:
\begin{verbatim}
$ emacs /boot/grub/menu.lst
\end{verbatim}

%%====================================================================
\section{Etape 3 : red�marrer avec le nouveau noyau install�}
%%====================================================================
On red�marre la machine et au d�marrage du \emph{grub}, on choisit le
nouveau noyau 2.6.16-52-rtai. Si, le d�marrage �choue, c'est que les
options du noyaux n'ont pas �t� mises comme il faut, notamment les deux
importantes~: -- par d�faut le syst�me de fichier ext3 est mis en tant
que module, je l'ai inclus enti�rement dans le noyau, et -- ne pas
mettre la gestion des bus SPI en module. Normalement, apr�s ces deux
v�rifications, Linux se lance sans erreur. Sinon, il faut
activer/d�sactiver d'autres options du \emph{configure} et recompiler le noyau.

%%====================================================================
\section{Etape 4 : compiler RTAI}
%%====================================================================
On revient au dossier de rtai, on lance un configure graphique (un peu
comme pour le kernel), on compile et on installe RTAI.
\begin{verbatim}
$ cd /usr/src/rtai
$ make menuconfig
$ export PATH=PATH:/usr/realtime/bin
$ make
$ sudo make install
$ sudo make modules_install
\end{verbatim}
On sauvegardera la variable \emph{PATH} dans le \emph{.bashrc}.

%%====================================================================
\section{Etape 5 : tester RTAI au stress}
%%====================================================================
Maintenant, on doit v�rifier que RTAI fonctionne correctement, en utilisant le
test {\tt latency} fournis avec les sources RTAI.

\begin{verbatim}
$ cd /usr/realtime/testsuite/kern/latency/
$ ./run
\end{verbatim}

Il faut �galement lancer des applications gourmandes en CPU et qui
vont stresser le syst�me, par exemple une application 3D. Ces
applications "latency killers" causent des retards non pr�dictibles et
sont incompatibles avec le concept de temps r�el. Il faut v�rifier que
la colonne {\tt overuns} reste toujours � 0, sinon il faut enlever les
modules susceptibles de causer des retards par la commande {\tt
  insmod} et/ou supprimer des options du noyau puis recompiler le
noyau et RTAI (faire un {\tt make distclean} dans les sources RTAI).

%%====================================================================
\section{Etape 2prim : choisir les options pour la compilation du
  noyau}\label{choixkern}
%%====================================================================
Je conseille d'utiliser mon fichier {\tt .config} pour activer les
bonnes options. J'indique ici, les options que j'utilise pour mon
kernel 2.6.16.52 patch� avec RTAI 3.4.

\begin{itemize}
\item[$\bullet$] \emph{Code maturity level options} on s�lectionne "Prompt for development ..."
\item[$\bullet$] \emph{General setup} on met le nom {\tt -rtai} � "Local version".
\item[$\bullet$] \emph{Loadable module support} on s�lectionne "Enable module support",
"Module unloading" et "Automatic kernel module loading". D�selectionner "Module versioning support".
\item[$\bullet$] \emph{Processor type and features} Mettre
  "Subarchitecture Type" � {\tt PC-Compatible} et {\tt Processor
    family} � Pentium 4. Selectionner "Preemption Model (Preemptible
  kernel (Low-Latency Desktop))". Mettre "High Memory Support" �
  4GB. D�selectionner "Use register arguments (EXPERIMENTAL)", "kexec
  system call (EXPERIMENTAL)", "kernel crash dumps (EXPERIMENTAL)",
  "Symetric multi-processing support" et "Local APIC support on
  uniprocessors". Mettre "Timer frequency" le plus �lev� possible �
  savoir 1 Khz.
\item[$\bullet$] \emph{Power Management options} Enlever "ACPI Support". D�selectionner
"APM BIOS Support" et "CPU Frequency scaling".
\item[$\bullet$] \emph{Bus options} Activer "PCI support" mais
supprimer "PCI Express support" et "ISA support" car le PC embarqu�
n'en poss�de pas ou n'en utilise pas. Supprimer "PCI hotplug Support".
\item[$\bullet$] \emph{Networking} enlever "Amateur Radio support" et
tout ce qui concerne le "Bluetooth" et "802.11".
\item[$\bullet$] \emph{Device Drivers}
\begin{itemize}
\item \emph{Generic driver options} garder les options par d�faut.
\item \emph{Memory Technology Devices (MTD)} � enlever.
\item \emph{Parallel port support} Peu �tre activ� ou d�sactiv�.
\item \emph{Plug and Play support} garder par d�faut.
\item Enlever \emph{SCSI device support}, \emph{SPI support},
  \emph{I2C support}, \emph{USB support}, \emph{Sound} et
  \emph{Multi-device support (RAID and LVM)}. Activer \emph{IEEE 1394
    support}.
\end{itemize}
\item[$\bullet$] \emph{File systems} Activer "Second extended fs
  support" et "Ext3 extended attributes".
\item[$\bullet$] \emph{Instrumentation Support} d�sactiver tout.
\item[$\bullet$] Garder les autres options par d�faut.
\end{itemize}

%%====================================================================
\section{Etape 4prim : choisir les options pour la compilation de RTAI}\label{choixrtai}
%%====================================================================

Le choix des options pour RTAI est plus simple que le choix des
options du noyau.
\begin{itemize}
\item[$\bullet$] \emph{Menu General} Mettre "Installation directory" �
  {\tt /usr/realtime} et "Kernel source directory" � {\tt
    /usr/src/linux}.
\item[$\bullet$] Mettre \emph{Number of CPUs} � 1.
\end{itemize}

%%====================================================================
%\section{Revenir au vieux PC embarqu�}
%%====================================================================
%Au cas o� la nouvelle version du PC embarqu� ne marcherait pas comme
%il faut, voila les diff�rentes �tapes pour retrouver l'ancienne
%configuration (vieux PC embarqu� et vieux Linux) qui ont fait plus
%leur preuve.
%
%La carte FireWire a pu etre enlevee pour le nouveau PC.
%Lancer Linux (RTAI par defaut)
%adresse\_public ou ifconfig 128.93.xx xxx
%
%Depuis bureau AOSTE ssh -X root@128.93.xx xxx


%\chapter{Administration Linux}
%ssh demon
%mettre en place une clef ssh
%parler que ssh -X root@xxx est cool mais devrait etre banni en l'interdisant dans la conf
%mettre un utilisateur sudo
%Les infos sur l'archi cat /proc/cpuinfo

%Lancer/supp/enlever les modules RTAI, ...

%\chapter{Arborescence des sources}

%genere la doc doxygen

%Presenter l'arborescence des sources:
%++ihm
%++robucars
%++robucos

%configurer la compil du programme IHM
%./configure --camera=...


















%%====================================================================
%%====================================================================
\listoffigures
\addcontentsline{toc}{part}{Tables des figures}
\newpage


\addcontentsline{toc}{part}{Bibliographie}
\begin{thebibliography}{999999999}

%%====================================================================
\subsection*{Logiciels}
%%====================================================================
\bibitem{SynDEx}La page principale de SynDEx : http://www-rocq.inria.fr/syndex/

\bibitem{Scilab}La page principale de Scilab : http://www-rocq.inria.fr/syndex/

\bibitem{Scicos}La page principale de Scicos : http://www.scicos.org/

\bibitem{scicosLM} Il existe 20 articles de prise en main avec les logiciels Scilan/Scicos. Ils sont t�l�chargeables sur {\tt http://www.saphir-control.fr/articles/Articles\_lm.htm}

%%====================================================================
\subsection*{Automatique et Ordonnancement}
%%====================================================================

\bibitem{Yves}Thierry Grandpierre, Christophe Lavarenne, Yves Sorel, \emph{Mod�le d'ex�cutif distribu� temps r�el SynDEx}, INRIA, 1998.

\bibitem{ordo}Andreas Ermedahl, \emph{Schedulability Analysis Assignment}, 2004.

\bibitem{Chancelier}Stephen L. Campbell, Jean-Philippe Chancelier and Ramine Nikoukhah, \emph{Modeling and Simulation in Scilab/Scicos}, Springer, 2005.

\bibitem{Faure}Pierre Faure et Michel Depeyrot, \emph{El�ments d'automatique}, Dunod, 1974.

\bibitem{Astrom} Karl Johan \.Astr\"om  \emph{Control System Design} ME155A.

\bibitem{avr221} \emph{Discrete PID controller} ATMEL, {\tt http://www.atmel.com/dyn/resources/prod\_documents/doc2558.pdf}

\bibitem{Astromweb}La page principale de Karl Johan \.Astr\"om : http://www.control.lth.se/$\sim$kja/

\bibitem{Ordo} Philipe Baptiste, Emmanuel Neron, Francois Soud \emph{Mod�les et algorithmes en ordonnancement}, Ellipses 2004.

%%====================================================================
\subsection*{Traitement de l'image}
%%====================================================================

\bibitem{icare} {\tt http://www-sop.inria.fr/icare/personnel/malis/software/ESMapplications.html} Quelques applications bas�es sur du traitement de l'image, dont un algorithme de suivi de CyCab.

\bibitem{beti} {\tt http://perso.enst.fr/~maitre/BETI/} Biblioth�que d'Exemples de Traitement des Images.

\bibitem{opgl} {\tt http://iup3gmi.univ-fcomte.fr/IG/TraitementImages/TraitementImages.htm} Site d'introduction sur le traitement d'image.

\bibitem{wiki} Certaines images servant d'exemple dans ce document, proviennent du site {\tt http://www.wikipedia.org/}

%%====================================================================
\subsection*{Bus IEEE 1394}
%%====================================================================
\bibitem{LM69}  \emph{Acquisition vid�o au moyen d'une cam�ra IEEE 1394}, Renaud
Dardenne et Marc Van Droogenbroeck. Linux Magazine N 69, F�vrier 2005, page 54 -- 59.

\bibitem{1394} {\tt http://www.linux1394.org/} 

%%====================================================================
\subsection*{Linux, RTAI, kernel}
%%====================================================================
\bibitem{rtaiorg} Le site officiel de RTAI {\tt https://www.rtai.org/}

\bibitem{captainat} Site fournissant de la documentation et des exemples de programmes RTAI et LXRT {\tt http://www.captain.at/rtai.php}

\bibitem{kernelorg} O� t�l�charger un noyau Linux {\tt http://www.kernel.org/pub/linux/kernel/v2.6/}

\bibitem{Simone} \emph{RTAI-Lab tutorial: Scilab, Comedi, and real-time control}, Roberto Bucher, Simone Mannori, Thomas Netter. {\tt https://www.rtai.org/RTAILAB/RTAI-Lab-tutorial.pdf}

\bibitem{rtai2} Cours de RTAI {\tt http://www.courseforge.org/courses/fr/rtai1/}

\bibitem{rtai3} RTAI 3.4 User manual {\tt https://www.rtai.org/index.php?module=documents} {\tt \&JAS\_DocumentManager\_op=downloadFile\&JAS\_File\_id=46}

\bibitem{ldd} Le livre \emph{Linux Device Drivers 3rd edition}, Jonathan Corbet, Alessandro Rubini et Greg Kroah-Hartman,  O'Reilly 2005.

%%====================================================================
\subsection*{Mat�riel}
%%====================================================================

\bibitem{aplus} Les cartes du PC embarqu� sont command�es chez le fournisseur Aplus {\tt http://www.aplus-sa.com/fr/produit.asp}.

\bibitem{curtis} Datasheet et manuel du contr�leur Curtis PMC 1227 {\tt http://curtisinst.com/index.cfm?fuseaction=cDatasheets.dspListDS\&CatID=1}

\end{thebibliography}

\end{document}

%%====================================================================
%%====================================================================




\section{Revenir au vieux PC embarque}
Presenter l'arborescence des sources:
++ihm
++robucars
++robucos

Algo general CyCab

Demarer camera

attendre quelle chauffe

avec IHM virer le decors, selectionner les 2 lignes horizontale qui determinent le cycab

lancer la comm avec RTAI

mesurer hauteur cycab et en deduire une consigne de vitesse

envoyer cons vit aux MPC

observer vitesse cycab

agrandir rectangle

mesurer hauteur


J'ai passe 4 mois de merde a piger les trucs suivants :
++ 1 mois a comprendre : -- RTAI et LXRT (communication, car RTAI interdit utilisation fichier)
-- appli Hello World sur MPC impossible (pas moyen de debug)
-- appli robucar ibittable (assembleur) pas de simulation Scicos pour voir la plage des valeurs (connais pas les capteurs)
-- code degueulasse des stagiaires (Sobel) et deadlock comm RTAI LXRT + mauvaise comprehension du contexte au depart (block Syndex 'Camera' ==> nom trompeur) + mauvaise documentation des stagiaires.

++ 1 mois pour faire un premier regulateur simple (PID) tester dans le truc reel enviter de tout faire planter le cycab quand on ne connait pas les plages de valeur (pas de simulation Scicos) ==> casser la carte ==> ca ma refroidit ==> perte de confiance ==> perte de temps.

++ 1 mois a faire de la retro ingenieurie -- comment lire les donnees venant de RTAI car ecriture fichier interdit (comprehension de l'archi du CyCab (capteurs et leur plage de valeur)) --> Scicos. Erreure grossiere de lecture des donnees (donnees manquantes 30ms) --> comprehension probleme de design (PC embarque == Timer de synchron des MPC) + lecture code assembleur + pas de doc + probleme arrondis (float vs int) dans scicos ==> derive

++ 1 mois pour netoyer le code (IHM pou gagner en vitesse et moyen de debugage) ==> probleme de vitesse du processeur 233Mhz (1 detection / s) comment gagner un peu de vitesse.

++ 1 mois pour l'arrondi ==> 5 mai : TOUJOURS RIEN FAIT

demain je me fait la COMMANDE DE LA CERTE MERE + 2 lassots + film camera + vire conversion YUV2RGB
samedi je finis ca + rapport + bug Scicos deceleration + COEF Cutis
dimanche je regle histoire agrandissement rectangle SUR LE FILM

1 semaine pour regler tous les details agrandissement rectangle et preparation test reels

APRES A DIEU VA

\section{CyCab}
\subsection{caracteristiques} Copier Coller
\subsection{Archi}
2 noeuds MPC 555
1 noeud gere 2 roues grace a 1 controleur moteur (Curtis MCP + donner la reference MPC)
1 linux RTAI sur un processeur x86 a 233MHz
1 joystick donnant la consigne de direction des 4 roues et de vitesse (analogique)
1 decodeur quad pour la vitesse des roues
1 moteur pour la direction des roues (alimente par un PWM)
1 camera firewire (donner ref Libraw 1394)
1 bus CAN
1 ecran

 un diag UML "system"

Photo CyCab demonte, CyCab monte.

Photo d'un noeud MPC (recuperer le PDF de Patrice avec les numeros)

Diagramme de Lionel RTAI --mem partagee -- regul --  MPC

Dire faire attention aux differentes alimentation (25V et 50V).

Dire RTAI = Timer 10 ms.

\subsection{Robucar.m4x}

Retro-ingenieurie SynDex --> truc final --> Scicos --> simulation

Afficher les hyper block SCICOS + simulation + Tableau couleur des plots + comment espionner CyCab (dmesg)

Expliquer traduction ASM MPC --> block diag + probleme de precision floattant vs. entier

Expliquer RTAI == timer 10 ms mais au demerrage a 3 ms.

%%====================================================================
%%====================================================================
\section{Chaine de CyCab}
\subsection{Principe general}
Camera bas cout

Expliquer le principe h/h' = L'/L = alpha (formule) avec h connu + exemple sur 2 photos : une
petite et l'autre grande et afficher les coef.

Expliquer le principe de ma regulation avec la simulation :

%%====================================================================
\subsection{Regulation de la distance entre les deux voitures}


%%====================================================================
\section{Camera : capteur de distance}
\subsection{Libraw 1394}

Recuperer le cours de Linux Mag et citer ref + ref site oueb libraw1394

\subsection{Gestion camera : Linux TR embarque}

Expliquer RTAI et LXRT

\subsection{Traitement de l'image}

Expliquer les couleurs RGB YUV

Expliquer les filtres Laplace Sobel

Detection de contours histo H --> distance

\subsection{IHM pour le debug}

\section{Annexes}

Expliquer comment compiler/executer mon programme ./configure --camera profiler RTAI IHM xv geios

Expliquer comment espionner Robucar.m4x et comment transformer les fichiers en fichier Scilab

\end{document}
















