\documentclass[a4paper, 10pt]{article}
\usepackage[french]{babel}
\usepackage[latin1]{inputenc}
\usepackage{amstext}
\usepackage{amsmath}
\usepackage{graphicx}
\usepackage{longtable}
\usepackage{rotating}
\usepackage{fancyhdr}
\usepackage{amssymb}
\usepackage{picins}
\usepackage{multicol}
%\usepackage{xy}
%\xyoption{all}

%%
\setlength{\textwidth}{18cm}
\setlength{\hoffset}{-3cm}
\setlength{\voffset}{-4cm}
\setlength{\textheight}{28cm}
%%

\newcommand{\dessin}[4]{
\begin{figure}[htb]
\centering
\includegraphics[scale= #2]{#1}
\caption{#3}
\label{#4}
\end{figure}}
%%

\newcommand{\yinifnt}{\fontencoding{OT1}\fontfamily{yinit}\fontsize{40}{60}\selectfont}
\newtheorem{remarque}{Remarque} \newcommand{\AAA}{{\mathcal A}}
\newcommand{\BB}{{\mathcal B}} \newcommand{\CC}{{\mathcal C}}
\newcommand{\DD}{{\mathcal D}} \newcommand{\EE}{{\mathcal E}}
\newcommand{\FF}{{\mathcal F}} \newcommand{\GG}{{\mathcal G}}
\newcommand{\HH}{{\mathcal H}} \newcommand{\II}{{\mathcal I}}
\newcommand{\JJ}{{\mathcal J}} \newcommand{\KK}{{\mathcal K}}
\newcommand{\LL}{{\mathcal L}} \newcommand{\MM}{{\mathcal M}}
\newcommand{\NN}{{\mathcal N}} \newcommand{\OO}{{\mathcal O}}
\newcommand{\PP}{{\mathcal P}} \newcommand{\QQ}{{\mathcal Q}}
\newcommand{\RR}{{\mathcal R}} \newcommand{\SSS}{{\mathcal S}}
\newcommand{\TT}{{\mathcal T}} \newcommand{\UU}{{\mathcal U}}
\newcommand{\VV}{{\mathcal V}} \newcommand{\WW}{{\mathcal W}}
\newcommand{\XX}{{\mathcal X}} \newcommand{\ZZ}{{\mathcal Z}}
\newcommand{\bbR}{{\mathbb R}} \newcommand{\bbD}{{\mathbb D}}
\newcommand{\bbO}{{\mathbb O}} \newcommand{\bbS}{{\mathbb S}}
\newcommand{\bbE}{{\mathbb E}} \newcommand{\bbN}{{\mathbb N}}
\newcommand{\bbM}{{\mathbb M}} \newcommand{\bbV}{{\mathbb V}}
\newcommand{\bbC}{{\mathbb K}} \newcommand{\bbF}{{\mathbb F}}
\newcommand{\bbP}{{\mathbb P}}
\newcommand{\www}{{\mathfrak w \;}} \newcommand{\fff}{{\mathfrak f \;}}
\newcommand{\nnn}{{\mathfrak n \;}} \newcommand{\aaa}{{\mathfrak a \;}}
\newcommand{\hhh}{{\mathfrak h \;}}
%%

\title{Note sur le Cours d' Automatique de K.J. Astrom}
\author{QUADRAT Quentin}

\begin{document}
\maketitle
\tableofcontents
\newpage

\section{Formule de Taylor}
Formule de Taylor avec un param\`etre $x = x_0 + \delta x$ :
$$f(x) = \sum_{n \geq 0}  \left( \frac{f^{(n)}(x_0) (x-x_0)^n}{n!} \right)$$

Formule de Taylor dans le cas vectoriel:
\begin{align*}f(x,y) & = f(x_0,y_0) + \left(f'_x(x_0,y_0)(x-x_0)+f'_y(x_0,y_0)(y-y_0) \right) \\
& + \frac{1}{2} \left(
f''_x(x_0,y_0)(x-x_0)^2+f''_y(x_0,y_0)(y-y_0)^2 + f''_{x,y}(x_0,y_0) (x-x_0)(y-y_0) \right)
\end{align*}
\section{Lin\'earisation}

Exemple de lin\'earisation de l'\'equation du mouvement d'un pendule invers\'e (conditions initiales nulles):
\begin{align*}
\dot{x_1} & = x_2 \\
\dot{x_2} & = \frac{mgl}{J} \sin{x_1} + \frac{ml}{J} u \cos{x_2}
\end{align*}

On peut \'ecrire $\dot{x_2}= f(x_1, x_2, u)$. Gr\^ace \`a la formule de Taylor  :
$$\dot{x_2} \simeq f(0, 0, 0) + \delta x_1 f_{x_1}'(0, 0, 0) + \delta x_2
f_{x_2}'(0,
0, 0) + \delta u f_u'(0, 0, 0)$$

Ce qui donne :
$$\delta \dot{x_2} = \dot{x_2} \simeq 0 + \delta x_1 \left( \frac{mgl}{J}
\cos{0} - \frac{ml}{J} 0 \sin{0} \right) + \delta
u\left( \frac{ml}{J}\cos{0} \right)$$

Ce qui donne :
$$\delta \dot{x_2} \simeq \delta x_1 \left( \frac{mgl}{J} \right) + \delta u \left( \frac{ml}{J} \right)$$

De m\^eme on a :
$$\delta \dot{x_1} = \delta x_2$$

Finalement, on obtient le systeme suivant \ref{sys}:
\begin{align*}
\delta \dot{x_1} = & \; \delta x_2 \label{sys}\\
\delta \dot{x_2} = & \; \delta x_1 \left( \frac{mgl}{J} \right)+ \delta
u \left( \frac{ml}{J} \right) \\
\delta y = & \; \delta x_1
\end{align*}

Ou bien sous la forme matricielle :
\begin{align*}
\delta \dot{x} = &
\begin{bmatrix}
0 & 1 \\ 
\frac{mgl}{J} & 0\\
\end{bmatrix}
\begin{bmatrix}
\delta x_1 \\ \delta x_2 \\
\end{bmatrix}
+ \begin{bmatrix}
0 \\  \frac{ml}{J} \\
\end{bmatrix} \delta u
\\
\delta y= & \begin{bmatrix}
1 & 0 \\ 
\end{bmatrix}
\begin{bmatrix}
\delta x_1 \\ \delta x_2 \\
\end{bmatrix}
+ \begin{bmatrix}
0\\
\end{bmatrix} \delta u
\end{align*}

\dessin{blockdiag}{0.85}{u est l'entr\'ee, y la sortie, x l'\'etat (ou m\'emoire)}{blockdiag}

En changeant de notation $\delta x \rightarrow x, \delta u \rightarrow u, \delta y \rightarrow y$, le syst\`eme s'\'ecrit bien sous la forme (cf. block diagramme \ref{blockdiag}) :
\begin{align*}
\frac{dx}{dt} = & \; Ax + Bu \\
y = & \; Cx + Du
\end{align*}

Avec $A$ une matrice $n \times n,$ $B$ une matrice $n \times p,$ $C$ une matrice $q \times n,$
et $D$ une matrice $q \times p$.

Une propri\'et\'e amusante : on peut emboiter les 4 matrices.
$$\begin{array}{|cc|c|} \hline
0 & 1 & 0 \\
\frac{mgl}{J} & 0 & \frac{ml}{J} \\ \hline
1 & 0 & 0 \\ \hline
\end{array}$$

\subsection{R\'esolution de syst\`eme d'equa diff avec Scilab}
\begin{verbatim}
function xprim=f(t, x)
xprim(1)=x(2);
xprim(2)=(m*g*l)/j*x(1) + (m*l)/j*sin(100*t);
endfunction

m=10; g=10; l=1; j=10; x1_0=0.1; x2_0=0.;
t0=0; tmax=0.1; dt=0.001; t=t0:dt:tmax;

pendule=ode([x1_0;x2_0], t0, t, f);
\end{verbatim}


\section{Syst\`emes lin\'eaires temps invariant}

\subsection{Exercice 1 de lin\'earisation}

Soit $\dot{x} = 2x,$ sachant $x(0) = x_0.$ Que vaut $x(t) ?$

Solution :
$$x(t) = a \; e^{2t}$$
$$\dot{x}(t) = 2a \; e^{2t}$$

Pour trouver $a$, on utilise la condition initiale de l'\'ennonc\'e, soit :
$$x(0) = a = x_0$$

\subsection{Exercice 2 de lin\'earisation}
Soit $\dot{x} = 2x + u,$ sachant $x(0) = x_0.$ Que vaut $x(t) ?$

Solution :

Pour $u=0$ on a : $x(t) = a(t)\; e^{2t}$


En d�rivant on a :
$$\dot{x}(t) = 2a \; e^{2t} + \dot{a}(t) \; e^{2t} = 2x(t) + u(t) = 2a(t) \; e^{2t}+u(t)$$

En se simplifiant :
$$\dot{a}(t)=u(t)\; e^{2t}$$

D'o\`u :
$$a(t)=\int_0^t u(s) \; e^{-2s}ds + a_0$$
$$x(t) = a_0 \; e^{2t} + \int_0^t u(s)(\; e^{2(t-s)}ds$$

En utilisant la condition initiale de l'\'ennonc\'e, on a :
$$x(0) = x_0 = a_0$$

Finalement :

$$x(t)=x_0\; e^{2t}+\int_0^t u(s)\; e^{2(t-s)}ds$$

\subsection{Une d\'emonstration sur les exp de matrices}

Montrons que si $A$ est diagonalisable, nous avons : $e^{At}=Pe^{\Lambda t}P^{-1}$

Soit $A$ diagonalisable, alors $A$ s'\'ecrit sous la forme : $A=P\Lambda P^{-1}$

On rappelle que : $A^n = P \Lambda^n P^{-1}$

D'o\`u :
\begin{align*}
e^A & = \sum_{n\geq0}\frac{A^n}{n!}\\
e^A & = \sum_{n\geq0}\frac{P\Lambda^n P^{-1}}{n!}\\
e^A & =P\left( \sum_{n\geq0}\frac{\Lambda^n}{n!} \right) P^{-1}\\
e^A & =Pe^{\Lambda}P^{-1}
\end{align*}

\subsection{Scilab et les exp de matrice}
\begin{verbatim}
    -->A=[2,1;0,2]
     A  = 
    !   2.    1. !
    !   0.    2. !

    -->exp(2)
     ans  = 
    7.3890561

    -->expm(A)
     ans  = 
    !   7.3890561    7.3890561 !
    !   0.           7.3890561 !
\end{verbatim}

\subsection{Exercice 1 sur les exp de matrices}

$$A=\begin{bmatrix}
0 & 1\\ 1 & 0 
\end{bmatrix} $$

Polyn\^ome caract\'eristique :
$$\begin{vmatrix}
-\lambda & 1\\ 1 & -\lambda
\end{vmatrix} = \lambda^2 - 1$$

D'o\`u : $A = P\Lambda P^{-1}$ avec :
$$\Lambda=\begin{bmatrix}
1 & 0\\ 0 & -1 
\end{bmatrix}$$

Puisque $\Lambda$ est diagonale on \'ecrire :
$$e^{\Lambda t} = e^{\begin{bmatrix}
t & 0\\ 0 & -t
\end{bmatrix}}=\begin{bmatrix}
e^t & 0\\ 0 & e^{-t} 
\end{bmatrix}$$

\subsection{Exercice 2  sur les exp de matrices}
Soit :
$$A=\begin{bmatrix}
\alpha & 1 & 0 \\
0 & \alpha & 1 \\
0 & 0 & \alpha \\
\end{bmatrix}$$

Montrer que :
$$e^{A t}=
\begin{bmatrix}
e^{\alpha t} & t e^{\alpha t} & t^2 e^{\alpha t} \\
0 & e^{\alpha t} & t e^{\alpha t} \\
0 & 0 & e^{\alpha t} \\
\end{bmatrix}$$

On calcule :
$$I + At + \frac{1}{2} (At)^2 + \frac{1}{3!} (At)^3 =
\begin{bmatrix}
1 + \alpha t + \frac{\alpha^2 t^2}{2} + \frac{\alpha^3 t^3}{6} & t + \alpha t^2 + \frac{1}{2} \alpha^2 t^3 & \frac{1}{2}(t^2 + \alpha t^3) \\
0 & 1 + \alpha t + \frac{\alpha^2 t^2}{2} + \frac{\alpha^3 t^3}{6} & t + \alpha t^2 + \frac{1}{2} \alpha^2 t^3 \\
0 & 0 & 1 + \alpha t + \frac{\alpha^2 t^2}{2} + \frac{\alpha^3 t^3}{6} \\
\end{bmatrix}$$

Finalement :
$$e^{A t}=
\begin{bmatrix}
e^{\alpha t} & t e^{\alpha t} & t^2 e^{\alpha t} \\
0 & e^{\alpha t} & t e^{\alpha t} \\
0 & 0 & e^{\alpha t} \\
\end{bmatrix}$$

\section{Mod\`ele d'entr\'ee-sortie}

\subsection{R\'eponse impulsionelle}
Trouver la r\'eponse impulsionelle $y(t)=\int_0^tg(t-s)u(s)ds$ du syst\`eme \ref{sys}.

Condition initiale : $x(0)=0$.

\begin{align*}
y(t) & = \begin{bmatrix}
1 & 0 \\
\end{bmatrix}
\int_0^t e^{\begin{bmatrix}
0 & 1 \\
\frac{mgl}{J} & 0
\end{bmatrix}(t-s)}
\begin{bmatrix}
0 \\ \frac{ml}{J} 
\end{bmatrix} u(s)ds \\
& =\int_0^t
\begin{bmatrix}
1 & 0 \\
\end{bmatrix}
e^{\begin{bmatrix}
0 & 1 \\
\frac{mgl}{J} & 0
\end{bmatrix}(t-s)}
\begin{bmatrix}
0 \\ \frac{ml}{J} 
\end{bmatrix} u(s)ds
\end{align*}

Donc finalement on en d\'eduit :
$$g(t) = \begin{bmatrix}
1 & 0 \\
\end{bmatrix}
e^{t\begin{bmatrix}
0 & 1 \\
\frac{mgl}{J} & 0
\end{bmatrix}}
\begin{bmatrix}
0 \\ \frac{ml}{J} 
\end{bmatrix} $$

\subsection{Transform\'ee de Laplace et fonction de transfert}

Trouver la fonction de transert $Y(s) = G(s)U(s)$ de la r\'eponse impulsionelle $y(t)=\int_0^tg(t-s)u(s)ds$ et en d\'eduire la stabilit\'e du syst\`eme.

On rappelle le syst\`eme \ref{sys} (avec $\frac{ml}{J} = \frac{mgl}{J} = 1$ pour faire simple) et $x_1(0) = 0$ et $x_2(0) = 0$ :
\begin{align}
\dot{x_1} & = x_2 \\
\dot{x_2} & = x_1 + u \\
 y & = x_1
\end{align}

$$\LL(x_2)=X_2(s)=\int_0^{\infty} e^{-st} x_2 ds$$
$$\LL(\dot{x_2})=s X_2(s) - x_2(0)$$
$$\LL(\dot{x_1})=s X_1(s)$$

De (2) on a : $s X_1(s) = X_2(s)$

De (3) on a : $s X_2(s) = X_1(s) + U(s)$

De (1) on a : $Y(s) = X_1(s)$

D'o\`u : $s^2X_1 = sX_2  = X_1 + U$ puis $X_1(s^2-1) = U$

Finalement : 
$$Y=X_1=\frac{U}{s^2-1}$$

et donc, la fonction de transfert s'\'ecrit :

$$G=\frac{Y}{U} = \frac{1}{s^2-1}$$

En regardant les p\^oles ($1$ et $-1$) on en d\'eduit que le syst\`eme est instable.

\subsection{Transform\'ee de Laplace de la d\'eriv\'ee}
Montrons que $\LL\frac{\delta f}{\delta t} = s \LL f - f(0)$

Par une IPP, on pose $u = e^{-st}$ d'o\`u $u' = -se^{-st}$ ainsi que : $v'=f'(t)$ d'o\`u $v = f$

Donc :
$$\left[ e^{-st}f(t)\right]^{\infty}_0+s\int_0^{\infty}fe^{-st}dt$$

Finalement :
$$\LL\frac{\delta f}{\delta t} = s\LL f - f(0)$$

\subsection{Transform\'ee de Laplace de l'int�grale}

Montrons que $\LL\int_0^t f(\tau)d\tau = \frac{1}{s} \LL f $

Par une IPP, on pose $u = \int_0^t f(\tau)d\tau$ d'o\`u $u' = f$ ainsi que : $v'=e^{-st}$ d'o\`u $v = -\frac{1}{s}e^{-st}$

Donc :
$$\left[ -\frac{1}{s}e^{-st} \int_0^{t} f(\tau)d\tau  \right]_0^{\infty} + \frac{1}{s} \int_0^{\infty} e^{-st} f(t)dt$$

Finalement :
$$\LL\int_0^t f(\tau)d\tau = \frac{1}{s} \LL f $$

\section{P\^oles et z\'eros}

Montrons que la fonction de transfert $G(s)$ d'une LTI s'\'ecrit : $G(s) = C(sI-A)^{-1}B$

On part du syst\`eme suivant (avec pour hypoth\`ese $x(0)=0$) :
\begin{align*}
\dot{x} = & \; Ax + Bu \\
y = & \; Cx
\end{align*}

Par transform\'ees de Laplace :
\begin{align*}
sX - x(0) & = Ax + Bu \\
y & = Cx
\end{align*}

D'o\`u : 
$$(sI-A)X=BU$$
$$X=(sI-A)^{-1}BU$$
$$Y=C(sI-A)^{-1}BU$$

Finalement la fonction de transfert :
$$G=\frac{Y}{U} = C(sI-A)^{-1}B$$

\section{Contr\^olabilit\'e -- observabilit\'e}

\subsection{Contr\^olabilit\'e}
Le syst\`eme peut \^etre transf\'er\'e de tout \'etat  vers tout autre autre \'etat ssi la matrice de controlabilit\'e $W_c$ est de rang plein.
$$W_c=\begin{bmatrix}
B & AB & \ldots & AB^{n-1}
\end{bmatrix}$$

Dans le cas du pendule invers\'e (avec $\frac{mgl}{J} = \frac{ml}{J} = 1$) : 
$$W_c=\begin{bmatrix}
B & AB
\end{bmatrix}$$

Calculons :
$$B =\begin{bmatrix}
0 \\ 1
\end{bmatrix}$$
$$AB =\begin{bmatrix}
1 \\ 0
\end{bmatrix}$$

Donc:
$$W_c=\begin{bmatrix}
0 & 1 \\
1 & 0
\end{bmatrix}$$

Comme le rang de $W_c$ est $2$. Le syst\`eme est contr\^olable.

\subsection{Observabilit\'e}

L'\'etat peut \^etre calcul\'e \`a  partir de la sortie si la matrice d'observabilt\'e $W_o$ est de rang plein.
$$W_o=\begin{bmatrix}
C \\ CA \\ \vdots \\ CA^{n-1}
\end{bmatrix}$$

Dans le cas du pendule invers\'e :
$$W_o=\begin{bmatrix}
C \\ CA
\end{bmatrix}$$

Calculons :
$$C =\begin{bmatrix}
0 & 1
\end{bmatrix}$$
$$CA =\begin{bmatrix}
0 & 1
\end{bmatrix}$$

Donc:
$$W_o=\begin{bmatrix}
1 & 0 \\
0 & 1
\end{bmatrix}$$

Comme le rang de $W_o$ vaut $2$. Le syst\`eme est observable.

\section{Forme diagonale}

Rappel : la fonction de transfert du pendule invers\'e $G(s)=\frac{1}{s^2-1}$

Elle s'\'ecrit aussi :
$$G = \frac{1}{(s-1)(s+1)} = \frac{a}{s-1} + \frac{b}{s+1}$$

Avec : $a=1/2$ et $b=-1/2$.

\section{Contr\^oleur}

La forme compagnon peut \^etre obtenue (en \'evitant les changements de bases) gr\^ace \`a :

xxx FAURE xxx

Par un changement de base $z=Fx$, d'o\`u :
$z=f_1 x_1+f_2 x_2$, d'o\`u :
 
 \begin{align*}
\dot{z} & = FAF^{-1}z + TBu = Az + Bu
\end{align*}

\begin{align*}
\dot{z} & = (A+BF)x + Bu\\
& = \begin{bmatrix}
0 & 1 \\
1 & 0
\end{bmatrix}
\begin{bmatrix}
x_1 \\
x_2
\end{bmatrix}
+ \begin{bmatrix}
0 \\ 1
\end{bmatrix}
\begin{bmatrix}
f_1 & f_2
\end{bmatrix}
\begin{bmatrix}
x_1 \\ x_2
\end{bmatrix}\\
& = \begin{bmatrix}
0 & 1\\
1+f_1 & f_2 \\
\end{bmatrix}
\begin{bmatrix}
x_1 \\ x_2
\end{bmatrix}
\end{align*}

On voudrait que le polyn\^ome caract\'eristique ai les p\^oles $-2$ et $-2$ \`a savoir qu'il s'ecrive sous la forme :

$$(s+2)^2=s^2+4s+4 = -s (f_2-s)-1-f_1 = s^2 -s f_2 -1 -f_2$$

On en deduit $f_1=-5$ et $f_2=-4$.

Donc

\subsection{Scilab}
\begin{verbatim}
    sl=syslin('c', [0, 1; -10, -10], [0; 10], [1, 0]);
    t=0:0.05:20;
    a=csim("step",t,sl);
    plot(a);

    deff('u=input(t)','u=sin(t)')
    t=0:0.05:20;
    a = csim(input,t,sl)
\end{verbatim}

\section{Observateur}

Soit :
$$A = \begin{bmatrix}
0 & 1 \\
1 & 0
\end{bmatrix}\!\!,\;
B  = \begin{bmatrix}
0 \\ 1
\end{bmatrix}\!\!,\;
C  = \begin{bmatrix}
1 & 0
\end{bmatrix}$$

On calcule :
$$A-KC = \begin{bmatrix}
0 & 1 \\
1 & 0
\end{bmatrix} -
\begin{bmatrix}
k_1 \\ k_2
\end{bmatrix}
\begin{bmatrix}
1 & 0
\end{bmatrix} =
\begin{bmatrix}
-k_1 & 1 \\
1-k_2 & 0
\end{bmatrix}
$$

Le polyn\^ome caract\'eristique de $A-KC$ s'\'ecrit :
$$P(\lambda) = (k_1 + \lambda) \lambda -1 + k_2 = \lambda^2 + \lambda k_1 -1 + k_2$$

On veut que $P$ soit de la forme $(\lambda + 10)^2 = \lambda^2 + 20\lambda + 100$.
On a donc : $k_1 = 20$ et $k_2 = 101$.

On r\'eutilise l'expression de $U=FX+v$ avec $f_1 = 4$ et $f_2 = 5$ (cf. section pr\'ec\'edente).

D'o\`u :
\begin{align*}
\dot{X} &= \begin{bmatrix}
0 & 1 \\
1 & 0
\end{bmatrix}
\begin{bmatrix}
x_2 \\ x_1
\end{bmatrix}
+
\begin{bmatrix}
0 \\ 1
\end{bmatrix}
\left(
\begin{bmatrix}
-5 & -4
\end{bmatrix}
\begin{bmatrix}
\hat{x_1} \\ \hat{x_2}
\end{bmatrix}
+ v
\right)\\
\dot{\hat{X}} &= \begin{bmatrix}
0 & 1 \\
1 & 0
\end{bmatrix}
\begin{bmatrix}
x_2 \\ x_1
\end{bmatrix}
+
\begin{bmatrix}
20 \\ 101
\end{bmatrix}
(x_1 - \hat{x_1})
+
\begin{bmatrix}
0 \\ 1
\end{bmatrix}
\left(
\begin{bmatrix}
-5 & -4
\end{bmatrix}
\begin{bmatrix}
\hat{x_1} \\ \hat{x_2}
\end{bmatrix}
+ v
\right)\\
y &=\begin{bmatrix}
1 & 0
\end{bmatrix}
\begin{bmatrix}
x_1 \\ x_2
\end{bmatrix}
\end{align*}

Matriciellement l'observateur est :
\begin{align*}
\begin{bmatrix}
X \\ \hat{X}
\end{bmatrix} &=
\begin{bmatrix}
0 & 1 & 0 & 0 \\
1 & 0 & -5 & -4\\
20 & 0 & -20 & 1\\
101 & 0 & -106 & -4
\end{bmatrix}
\begin{bmatrix}
x_1 \\ x_2 \\ \hat{x_1} \\ \hat{x_2}
\end{bmatrix} +
\begin{bmatrix}
0 \\ 1\\ 0 \\ 1
\end{bmatrix} v \\
y &= \begin{bmatrix}
1 & 0 & 0 & 0
\end{bmatrix}
\begin{bmatrix}
x_1 \\ x_2 \\ \hat{x_1} \\ \hat{x_2}
\end{bmatrix}
\end{align*}

XXXX calculer $\dot{\tilde{X}}$ 

\subsection{Scilab}
\begin{verbatim}
    sl=syslin('c', [0, 1, 0, 0; 1, 0, -5, -4; 20, 0, -20, 1; 101, 0, -106, -4],
                   [0; 16/5; 0; 16/5], [1, 0, 0, 0]);
    t=0:0.05:20;
    a=csim("step", t, sl);
    plot(a);
    nyquist(sl, 0.01, 100);
\end{verbatim}

\section{Passer d'une fonction de transfert \`a un syst\`eme d'\'etats directement observable}
\subsection{Exercice 1}

Soit :
$$Y(s)=\frac{s-1}{s^2+2} U(s)$$

On le r\'eecrit diff\'eremment : $$\frac{(s^2+2)}{s-1} Y= U$$

On pose :
$$X_1 = \frac{s Y}{D} \text{ et } X_2 = \frac{Y}{D}$$

On obtient les \'egalit\'es suivantes :
$$sX_2 = X_1$$
$$sX_1 + 2X_2 = U$$

Soit :
$$
\begin{tabular}{cccc}
$sX_2$ = & & $-2 X_2$ & $+ U$ \\
$sX_1$ = & $X_1$ & & \\
$Y$ = & $X_1$ - $X_2$
\end{tabular}
$$

On obtient les matrices :
$$A = \begin{bmatrix}
0 & -2 \\ 1 & 0
\end{bmatrix}\!\!,\;
B  = \begin{bmatrix}
1 \\ 0
\end{bmatrix}\!\!,\;
C  = \begin{bmatrix}
1 & -1
\end{bmatrix}$$

\subsection{Exercice 2}

Soit :
$$Y(s)=\frac{s+1}{s^3+s^2+1} U(s)$$

On le r\'eecrit diff\'eremment : $$\frac{s+1}{s^3+s^2+1} Y = U$$

On pose :
$$X_1 = \frac{s^2 Y}{D} \text{ et } X_2 = \frac{s Y}{D} \text{ et } X_3 = \frac{Y}{D}$$

On a :
\begin{align*}
U &= sX_1 + X_1 + X_3\\
X_1 &= sX_2\\
X_2 &= sX_3\\
Y &= X_2 + X_3
\end{align*}

D'o\`u : 
$$
\begin{tabular}{ccccc}
$sX_1$ = & $-X_1$ & & $-X_3$ & $+ U$ \\
$sX_2$ = & $X_1$ & & & \\
$sX_3$ = & & $X_2$ & &  \\
$Y$ = & & $X_2 $ & $+X_3$ &
\end{tabular}
$$

Finalement, on obtient les matrices :
$$A = \begin{bmatrix}
-1 & 0 & -1 \\
1 & 0 & 0 \\
0 & 1 & 0
\end{bmatrix}\!\!,\;
B  = \begin{bmatrix}
1 \\ 0 \\ 0
\end{bmatrix}\!\!,\;
C  = \begin{bmatrix}
0 & 1 & 1
\end{bmatrix}$$

\section{Passer d'une fonction de transfert \`a un syst\`eme d'\'etats directement contr\^olable}
\subsection{Exercice 1}

Soit :
$$Y(s)=\frac{s+1}{s^2+1} U(s)$$

On le r\'eecrit diff\'eremment : $$(s^2+1) Y = (s+1)U$$

$Y$ observe $X_1$, d'o\`u : $Y=X_1$

On le r\'eecrit diff\'eremment : $$s^2 X_1 + X_1 =  s U + U$$

On pose :
$$sX_1 = \frac{-1}{s} X_1 + U + \frac{1}{s} U = U + \frac{1}{s} (U - X_1)$$

Puis :
$$X_2 = \frac{1}{s} (U - X_1)$$

D'o\`u le syst\`eme :
$$
\begin{tabular}{cccc}
$sX_1$ = & & $X_2$ & $+ U$ \\
$sX_2$ = & $-X_1$ & & $+ U$ \\
$Y$ = & $X_1$ & &
\end{tabular}
$$

Finalement, on obtient les matrices :
$$A = \begin{bmatrix}
0 & 1 \\
-1 & 0
\end{bmatrix}\!\!,\;
B  = \begin{bmatrix}
1 \\ 1
\end{bmatrix}\!\!,\;
C  = \begin{bmatrix}
1 & 0
\end{bmatrix}$$

\section{Nyquist}

\subsection{Theoreme}

Le diagrame de Nyquist permet de determiner graphiquement si un systeme est stable.

Un modele est stable si le point de coordonnees ($-1$, $0$) est entoure dans le sens direct un nombre de fois \'egal au nombre de poles instables de la fonction de transfert. 

Plus la courbe passe pres du point $-1$ moins le modele est stable par rapport aux erreurs de modelisations.

\subsection{Scilab et diagramme de Nyquist}

\begin{verbatim}
    s=poly(0,'s');
    h=syslin('c',1/(s^2+s+1))
    nyquist(h,0.01,100);

    t=0:0.05:20;
    a=csim("step",t,h);
    plot(a);
\end{verbatim}

\subsection{Tracer \`a la main un diagramme de Nyquist}

Soit la fonction de transfert $G(s) = \frac{1}{s+1}$

$$G(i\omega) = \frac{1}{1+i\omega} = \frac{1-i\omega}{1+\omega^2} = x(\omega) + i y(\omega)$$

Avec : $x = 1/(1+\omega^2)$ et $y = -1/(1+\omega^2)$

En mettant $y/x = -\omega$ on a : $x = x^2+y^2$

Ou equivalent :
$$(x-1/2)^2 + y^2 = 1/4$$

C'est l'\'equation du cercle de coordonn\'ees $(1/2, 0)$ et de rayon $1/2$.

\subsection{Exemple}

\begin{verbatim}
   s=poly(0,'s');
   slin=(s+1)/(s^2+5*s+1-0.001*s^5);
   Plant=syslin('c',slin);

   t=0:0.1:50;
   a=csim('step',t,Plant);
   plot(a);

  // poles et zeros
  plzr(Plant);

  //
  // PID
  //
  // k: gain
  // Ti: proportionnelle integrale
  // Td: proportionnelle derivee
  k=10; Ti=0.1; Td=0;

  // controleur
  con=k+1/(Ti*s)+s*Td;
  //boucle fermee
  bf=con*slin/(1+con*slin);
  bflin=syslin('c',bf);
  aa=csim('step',t,bflin);
  plot(aa);

  //nyquist
  h=syslin('c',con*slin);
  nyquist(h, 0.03,10);
\end{verbatim}

En conclusion: XXX

\section{Equation de la programmation dynamique}

On prend l'exemple de barrage o\`u $a(t)$ est l'apport d'eau dans le barrage. La quantit\'e d'eau est $x(t)$. Le barrage \'ecoule $u(t)$ d'eau (cf. Fig \ref{barrage}).

On cherche \`a maximiser $\sum_{n=0}^{N} E_t$ o\`u l'\'energie $E_t = h(x_t)u_t$.

\dessin{barrage}{0.85}{$a_t$ l'apport, $x_t$ le stock (ou quantit\'e), $u_t$ ce qui d\'ecoule}{barrage}

A l'instant $t+1$ la quantit\'e d'eau est : $x(t+1) = x(t) + a(t) - u(t)$.

On calcule la Fonction Valeur (ou Fonction de Bellman) \`a l'instant $t$ :
$$V(x, t)=  Max \left\{
\begin{array}{l|r}
\sum_{n=t}^N E_t & X_t = x
\end{array}
\right\}$$

On veut \'ecrire $V(x,t)$ en fonction de $V(x, t+1)$.  Pour cela, on prend une d\'ecision $u$ et $x$ sur 1 p\'eriode. On connait le co\^ut instantan\'e, c'est : $h(x)u$.  A l'instant $t+1$, j'arrive \`a l'\'etat : $V(x, t+1) = x+a(t)+u$. Par hypoth\`ese de r\'ecurrence, on connait le co\^ut jusqu'\`a la fin : c'est $V(t+1)$. On r\'esume ceci par le diagramme suivant (Fig. \ref{progdyna}).

\dessin{progdyna}{0.85}{Principe de la prog dyna : faire une r\'ecurrence du futur vers le pass\'e}{progdyna}

Donc :
$$V(x,t) = Max\left\{ h(x) u + V(x+a(t)+u, t+1)\right\}$$
\newpage

\section{Pontriagyn}

On cherche \`a minimiser $f(x)$ sachant $g_1(x) = 0, g_2(x) = 0$.

Si j'ai des contraintes, je dualise le syst\`eme en utilisant le lagrangien. Sinon j'int\'egre les donn\'ees directement.

Pour dualiser, il faut resoudre le systeme d'equation aux derivees partielles du Lagrangien egales a $0$.
 
On rappelle que la formule du Lagrangien $\LL(x, \lambda)$ vaut :
$$\LL(x_1, \ldots, x_n, \lambda_1, \ldots, \lambda_n) = f(x_1, \ldots, x_n) + \lambda_1 g_1(x_1, \ldots, x_n) + \ldots + \lambda_n g_n(x_1, \ldots, x_n).$$

Donc le systeme a resoudre est :
\begin{align*}
\frac{\delta \LL}{\delta x_i} & = 0\\
\frac{\delta \LL}{\delta \lambda_i} & = 0 \Rightarrow g_i(x) = 0
\end{align*}

\subsection{Exercice 1}

Minimiser $x_1^2+x_2^2$ sachant $x_1+x_2 = 1$

Le Lagrangien vaut :
\begin{align*}
\LL(x, \lambda) & = f(x) + \lambda g(x)\\
\LL(x, \lambda) & = x_1^2+x_2^2 + \lambda (x_1 + x_2 -1)
\end{align*}

Le systeme a resoudre est :
\begin{align*}
\frac{\delta \LL}{\delta x_1} & = 0 \Rightarrow 2x_1 + \lambda = 0 \Rightarrow x_1 = -\lambda / 2\\
\frac{\delta \LL}{\delta x_2} & = 0 \Rightarrow 2x_2 + \lambda = 0 \Rightarrow x_2 = -\lambda / 2\\
\frac{\delta \LL}{\delta \lambda} & = 0 \Rightarrow x_1 + x_2 - 1 = 0 \Rightarrow -\lambda - 1 = 0\\
\end{align*}

Finalement on a: $\lambda = -1, x_1 = 1/2, x_2 = 1/2$.

\subsection{Exercice 2}

Minimiser $\int^1_0(x(t)^2+u(t)^2)dt$ sachant $\dot{x(t)} = u(t)$

Par definition, le Lagrangien vaut ($C$ est le contour) :
$$\LL(x,u,\lambda) = \int^T_0\left(C(x, u) + \lambda (\dot{x} - f(x, u)\right) dt$$

Avec, dans notre cas : $C(x, u) = x^2 + u^2$ et $f(x, u)= \dot{x} - u.$

On calcule : $\LL(x+\delta x,u + \delta u,\lambda + \delta \lambda)  - \LL(x,u,\lambda)$


$$= \int_0^T\left(x+\delta x)^2 + (u+\delta u)^2 + (\lambda + \delta \lambda) (\dot{x+\delta x} - u - \delta u) - (x^2 + u^2 + \lambda (\dot{x} - u)\right) dt$$
$$= \int_0^T\left( 2x\delta x + 2 u \delta u + \delta \lambda (\dot{x} - u) + \lambda (\dot{\delta x} - \delta u) \right) dt$$
$$ = \int_0^T \left( (\delta x 2 x + \lambda \dot{\delta x}) + (2u - \lambda) \delta u + (\dot{x} - u) \delta \lambda \right) dt $$

On integre par partie $\int_0^T \delta x 2 x + \lambda \dot{\delta x}$, et on obtient :
$$\int_0^T \left( \delta x 2 x - \dot{\lambda} \delta x \right) dt+ \left[ \lambda \delta \lambda \right]^T_0
=\lambda(T) \delta x(T) - \lambda(0) \delta x(0) = \int_0^T \left( (2x - \dot{\lambda}) \delta x \right) dt$$

Donc :
$$ = \int_0^T \left( (2x - \dot{\lambda}) \delta x + (2u - \lambda) \delta u + (\dot{x} - u) \delta \lambda \right) dt $$

On veut que pour tout $\delta x, \delta u, \delta \lambda$ soit egale a zero. Donc on obtient le systeme suivant :
\begin{align*}
\dot{x} - u & = 0\\
2 u - \lambda & = 0\\
2x - \dot{\lambda} & = 0
\end{align*}

\subsection{Exercice 3}

Minimiser $\int_0^T\left( (x^2+u^2) \right) dt + \left( x(t) \right)^2$ sacahant $\dot{x} = x + u$.

On rappelle: $C=x^2+u^2$ et $f=x+u$

$$\int_0^T = \left( (x+\delta x)^2 + (u+\delta u)^2 + (\lambda + \delta \lambda) (\dot{x} + \dot{\delta x} - x-\delta x - u -\delta u) - x^2 -u^2 -\lambda (\dot{x} - x -u) \right) dt + 2x(T)\delta x(T)$$ 


$$=\int_0^T\left( 2x \delta x + 2u \delta u + \lambda \dot{\delta x} - \lambda \delta x - \lambda \delta u + \delta \lambda (\dot{x} - x -u)\right) dt + 2x(T)\delta x(T)$$ 

Apres l'IPP :
$$ = \int_0^T \left( (2x-\lambda -\dot{\lambda}) \delta x + (2u - \lambda)\delta u + (\dot{x} - x -u) \delta \lambda \right) dt + \left( 2x(T)+ \lambda(T) \right) \delta x(T)$$

Finalement on a le systeme :
\begin{align*}
2x - \lambda - \dot{\lambda}& = 0\\
2 u - \lambda & = 0\\
\dot{x} - x -u & = 0 \\
\lambda(T) & = -2x(T)
\end{align*}
\subsection{Exercice 4} \label{ex4}

Sachant $x(0) = 0$ et $x(1) =1$ on veut : $Min_{x(.)}\left\{ J(x)= \int_0^T\left(x^2+\dot{x}^2 \right) dt \right\}$

\begin{align*}
\delta J  &= J(x+\delta x) - J(x) \\
&= \int_0^T \left( (x+\delta x)^2 + (\dot{x} + \dot{\delta x})^2 \right) dt \\
&= \int_0^T \left( 2x\delta x + 2\dot{x} \dot{\delta x}\right) dt
\end{align*}


Apr\'es int\'egration par partie:
$$\left[ \dot{x}\delta x \right]^1_0 + \int_0^1\left( -2\ddot{x}\delta x + 2x \delta x\right) dt$$

A cause de la classe de fonctions que l'on prend ($x(0) = 0$ et $x(1)=1$) toutes les fonctions v\'erifient cette propri\'et\'e : elles passent par ces 2 points.

Donc: $\dot{x}(1)\delta x(1) - \dot{x}\delta x(0) = 0$.

Pour tout $\delta x(.), \delta J$ doit valoir $0$. D'o\`u : $2x - 2\ddot{x} =0$

Pour trouver les solutions de cette \'equation diff du deuxi\`eme ordre ce fait en deux temps :

\begin{itemize}
\item  La solution est du type (sans la constante) est : $e^{\lambda t}$.
D'o\`u : $-\lambda e^{\lambda t} + \lambda e^{\lambda t}  = 0$
Ce qui ce simplifie par $-\lambda^2 + 1 =0$
\item La solution est du type avec la constante est : $A e^{\lambda t} + B  e^{\lambda t}$

Pour $t=0$ on a : $A=-B$.

Pour $t=1$ on a : $A e^{1} + B e^{-1} = 1$

On en d\'eduit $A$ et $B$.
\end{itemize}

\subsection{Rappel sur comment faire une IPP}

$$\begin{array}{ccc}
&& G\\
&& \Uparrow\\
\int_a^b & f & g \\
& \Downarrow& \\
& \dot{f} &
\end{array} = \left[ f G\right]^b_a - \int_a^b \dot{f}G$$

Dans le cas de l'exercice \ref{ex4} on a :

$$\begin{array}{ccc}
&& \delta x\\
&& \Uparrow\\
\int_0^1 & \dot{x} & \dot{\delta x} \\
& \Downarrow& \\
& \ddot{x} &
\end{array} = \left[ \dot{x} \delta x \right]^1_0 - \int_0^1 \ddot{x} \delta x$$

\section{Application Scicos, SynDEx}

Une voiture $V_2$ de position $x_2(t)$ suit une autre $V_1$ de position $x_1(t)$ et essaie de garder une distance $l$ entre elles. La voiture $V_1$ acc\'el\`ere ou ralentit al\'eatoirement. Les sorties $y_1(t)$ et $y_2(t)$ sont respectivement la vitesse de $V_1$ et la distance entre les deux voitures. La voiture $V_2$ contr\^ole l'acc\'el\'eration de $V_1$.

On a le syst�me suivant :
\begin{align*}
\ddot{x_1} &= k_1 u\\
\ddot{x_2} &= k_2(x_1-x_2-l)\\
y_1 &= \dot{x_1} \\
y_2 &= x_1 - x_2 -l
\end{align*}

Pour simplifier le syst\`eme on fait un changement de variable: $z=x_2+l$, puis on renomme $z$ en $x_2$. On obtient alors le syst\`eme suivant :
\begin{align*}
\ddot{x_1} &= k_1 u\\
\ddot{x_2} &= k_2(x_1-x_2)\\
y_1 &= \dot{x_1} \\
y_2 &= x_1 - x_2
\end{align*}

\newpage
\section{Etude compl�te d'un pendule invers� sur un chariot}
\dessin{penduleinversechariot}{0.8}{Mod�le}{pendulechariot}

\subsection{Le mod�le}
\subsubsection{Les �nergies}
L'�nergie cin�tique du chariot : $$\EE_c^{chariot} = 1/2 M \dot{x}^2.$$

L'�nergie potentielle du chariot : $$\EE_p^{chariot} = u x.$$

L'�nergie cin�tique du penduke : $$\EE_c^{pendule} = 1/2 m \left(\left(\frac{d}{dt}(x+l \sin\theta)\right)^2 + \left(\frac{d}{dt}({l \cos \theta})\right)^2\right),$$

$$= 1/2 m (\dot{x}^2 + l^2 \dot{\theta} + 2\dot{x}\dot{\theta} l \cos \theta).$$

L'�nergie potentielle du pendule : $$\EE_p^{pendule} = mgl \cos \theta$$

\subsubsection{Principe de la moindre action}

TODO

On obtient le syst�me suivant :
\begin{align*}
& (M + m) \ddot{x} + ml \frac{d}{dt}(\dot{\theta} \cos\theta) = u\\
& l \ddot{\theta} + \frac{d}{dt}(\dot{x} \cos \theta) + \dot{x} \dot{\theta} \sin\theta = g\sin\theta
\end{align*}

Soit :
\begin{align*}
& (M + m) \ddot{x} + ml \left(\ddot{\theta} \cos\theta - \dot{\theta}^2 \sin \theta \right) = u\\
& l \ddot{\theta} + \ddot{x} \cos \theta = g\sin\theta
\end{align*}

\subsubsection{Lin�arisation}
En supposant que $\theta$ est petit on a :

\begin{align*}
& \ddot{x} + l_2 \ddot{\theta} = f\\
& ml^2 \ddot{\theta} + ml \ddot{x} \cos \theta = mgl\sin\theta
\end{align*}

Avec $l_1 = \frac{Ml}{M + m}, l_2 = \frac{ml}{M+m}, f= \frac{u}{M+m}$

\end{document}

















 